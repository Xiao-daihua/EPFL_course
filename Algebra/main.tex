% main.tex - 科研项目学习笔记模板
% !TeX root = main.tex
%%%%%%%%%%%%%%%%%%%%%%%%%%%%%% DOCUMENT 
\documentclass[12pt]{article}

%%%%%%%%%%%%%%%%%%%%%%%%%%%%%% PACKAGES

% 中文支持(XeLaTeX 编译)
\usepackage[UTF8]{ctex}
\usepackage{xeCJKfntef} 

% \setCJKmainfont{HanziPen SC}
% \setmainfont{HanziPen SC}


% 页面设置
\usepackage[a4paper, left=20mm, right=20mm, top=15mm, bottom=15mm]{geometry}

\PassOptionsToPackage{dvipsnames,svgnames,x11names}{xcolor}
\usepackage{xcolor}


% 数学环境及符号
\usepackage{amsmath, amssymb, amsfonts, amsthm,amsopn}
\usepackage{tensor}              % 张量指标管理
\usepackage{mathtools}           % amsmath增强
\usepackage{physics}             % 物理公式快捷命令
             % Dirac符号
\usepackage{bbold}               % 数学黑体
\usepackage{dsfont}              % 另一种数字体
\usepackage[mathscr]{eucal}     % 花体字母
\usepackage{tensor}              % 张量指标管理
\usepackage{simpler-wick}       % Wick记号
\usepackage{mathrsfs}            % 另一种花体字母

% 颜色与图形相关
\usepackage{graphicx}           % 插图支持
\usepackage{float}              % 浮动体控制
\usepackage{tikz}               % 绘图库
\usetikzlibrary{math}           % tikz数学扩展
\usepackage{geometry}
% 表格与列表
\usepackage{makecell}           % 表格多行换行
\usepackage{multicol}           % 多栏排版
\usepackage{colortbl}           % 表格颜色
\usepackage{enumitem}           % 列表自定义

% 其他辅助
\usepackage{framed}             % 有边框环境
\usepackage{tcolorbox}          % 灵活盒子环境
\tcbuselibrary{breakable}       % 盒子内容分页
\usepackage{thmtools}           % 定理环境管理
\usepackage{thm-restate}        % 定理重述
\usepackage{showlabels}         % 显示标签,调试用(完成后可注释)
\usepackage[normalem]{ulem}     % 下划线、删除线
\usepackage{hyperref}           % 超链接(最后加载)
\usepackage{cleveref}           % 智能引用(紧跟hyperref)
\usepackage{soul}

% 自定义宏包
\usepackage{macros}

% 一个中文可以高亮的包
\usepackage{cjkhl}
\definecolor{lightblue}{rgb}{.8,.8,1}

%%%%%%%%%%%%%%%%%%%%%%%%%%%%%% 自定义命令
\newcommand{\tml}{Teichmüller space}
\newcommand{\hil}{Hilbert space}
\newcommand{\mtc}{Modular Tensor Category}

%%%%%%%%%%%%%%%%%%%%%%%%%%%%%% BEGINNING OF THE DOCUMENT

\begin{document}

\title{\boldmath Lie Algebra in Particle Physics Learning Note}
\author{X. D. H.}

\maketitle

\begin{abstract}
  这是我在EPFL第一学期physics project期间阅读Lie Algebra in Particle Physics一书的学习笔记。主要内容包括Lie Algebra的基础知识,张量算符与Wigner-Eckart定理,Isospin物理背景,SU(3)与根权空间等内容。
\end{abstract}

\tableofcontents
\section{Week 1 Reading}\label{sec:Chapter1Reading} % (fold)
\input{week1reading.tex}
% section Chapter1Reading (end)
\newpage
\section{Week 2 Reading}\label{sec:Week 2 Reading} % (fold)
\subsection{Take home messages for Week 2}
Note for week 2 reading of "Lie Algebras in Particl Physics" by Howard Georgi.

\subsubsection{对称群以及其表示}

\hlr{对称群以及其表示}

对称群的定义上我们改变的是【位置】!!我们现在有n个位置,然后我们把这n个位置上面的物体进行交换。比如:$ (1,2,3) $我们是把第一个位置上的东西移动到第二个,再移动到第三个位置上面。第二个位置的东西移动到第三个再移动到第一个位置上面。第三个位置的东西移动到第一个位置上面再移动到第二个位置上面。

这个等价的一种说法是把$ (1,2,3) $可以画出一个换位图:$ x_1 \to x_2 \to x_3 \to x_1 $这个记号的意思是,$ x_i $是变换前第$ i $个位置上面的物体。然后这个图的意思是$ x_i \to x_j $是把$ x_j $的数值赋值给$ x_i $【没错,赋值的顺序和定义是相反的,就是个傻逼定义。】

\textbf{对很多文章此书上的定义极其不清晰我也不理解。但是我保证上面的理解方式是唯一并且正确的。}


对于所有元素我们可以使用$ k_i $个i-Cycle 进行描述,其中$ \sum k_i \times i = n$。

我们还可以个构造一个表示,Defining representation。我们定义为,参考这个图$ x_1 \to x_2 \to x_n ... $ ,定义一堆基 $ \ket{i} $ 我们根据这个图把$ x_j\to x_k $理解为这个群作用在这个基$ j $上面得到$ k $。也就是$ D \ket{j} = \ket{k} $这样的一个表示。

\bigskip

\hlr{对称群Irr Rep的构建}

我们寻找的对称群的conjugation class。我们会发现每一个cycle structure(比如,比如4,3,3,1也就是$ (\#,\#,\#,\#)(\#,\#,\#)(\#,\#,\#)(\#) $这样子的)对应着一个conjugation class。也就是对于一个有$ k_j $个j-cycle的对称群的conjugation class我们有:
\begin{equation}
  \frac{n!}{\prod_jj^{k_j}k_j!}
  \label{eq:numberofconjugationcalss}
\end{equation}
这么多个元素!!为此我们可以使用Young Tableaux进行表示每一个conjugation class。也就是一个Young Tableaux代表着一个Irreduacible Representation。然后根据Young diagram给出一个不等价不可约表示的基。这里不仔细叙述了,因为并不重要。


\bigskip 

\subsubsection{Continuous Groups and Lie Algebras}

\hlr{连续的群结构}

一个连续群我们可以理解为可以构建一个$ \mathbb{R}^n \to G $的映射。也就是$ g(\alpha) $,通过一个$ \mathbb{R}^n $之中的参数来label群元素。我们一般选取$ g(0) = e $。

\rmk{务必注意,我们的群元素是由实数进行label的$ \alpha \in \mathbb{R}^n $}

\defi{Generator

  对于这个群我们可以在表示的意义下定义生成元,表示作为线性映射(在某种意义下的矩阵),可以进行单位元附近的tayler展开:
  \begin{equation}
    D(d\alpha)=1+id\alpha_aX_a+\cdots 
    \label{eq:reptaylo}
  \end{equation}
  我们定义,单位元处的一阶导数就是群的生成元:
  \begin{equation}
    X_a\equiv -i\left.\frac{\partial}{\partial\alpha_a}D(\alpha)\right|_{\alpha=0}
    \label{eq:generatordef2}
  \end{equation}
}
\begin{itemize}
  \item 如果表示是一个Unitary的表示那么生成元是Hermite的算符 
\end{itemize}
\tip{Generator}{
  Generator到底是什么,我们发现其实就是表示线性映射构成的线性空间的一个元素。注意哦!这个线性空间不是每一个元素都有群元素的意义的,我们从来没有要求表示是一一映射什么的。

  但是我们之后会发现,Generator这些元素构成的线性子空间,存在特殊的代数结构。
}

\begin{itemize}
  \item 如果我们的群选择一个合适的参数化方式,我们可以把群元素的表示写作指数映射的形式:
    \begin{equation}
      D(\alpha)=\exp(i\alpha_aX_a)
      \label{eq:expparametrization}
    \end{equation}
    我们一般都使用这个“合适的参数化形式”。但是请注意这样的写法是默认考虑一切都在生成元附近的,如果太远是不可以的。
\end{itemize}


\tip{本书的表示关系}{
  本书之中所有表示的符号都是右乘在ket上面的:
 \begin{equation}
   D(g)\ket{i} = \ket{j}G_{ji}  \quad \text{or} \quad \bra{i}D(g) = G_{ij}\bra{j}
  \label{eq:repket}
 \end{equation} 
}

\bigskip 

\hlr{李代数结构}

\imp{李代数的结构}{
  我们研究如果一个连续群被使用很好的参数化进行exp parametrization的形式,根据群的乘法定义等价于生成元构成的线性空间存在一个李代数结构。

  也就是存在一个Lie bracket $ [\cdot,\cdot] $ 使得这个线性空间成为一个李代数。人话翻译就是这个线性空间除了一般的加法和数乘结构之外还存在一个commutator结构,并且我们可以使用基的commutator进行计算:
  \begin{equation}
    \begin{aligned}[X_a,X_b]&=if_{abc}X_c.\\f_{abc}&=-f_{bac}\end{aligned}
    \label{eq:liebracket}
  \end{equation}
  这个对易关系我们称为群的\textbf{李代数结构},以及$f_{abc}$为\textbf{结构常数}。
}

\begin{itemize}
\item 我们note $ f_{abc} $在选定一系列合理的参数化之后是一个常数。【之后我们会发现,他按照张量变换】
  \item 群的表示可以自动给出一个李代数的表示,对于所有群表示的李代数,只要参数化一样,结构常数都是一样的。但是表示不同结构常数也必须是一样的!!同时代数表示可以通过指数化给出一个群的表示捏。
  \item 对于Unitary的表示来说,$ f_{abc} $是一个实数。所以说\textbf{只要这个群【存在一个】Unitary表示,这个群的结构常数就是实数!!}
  \item 李代数满足Jacobi Identity:
    \begin{equation}
      [X_a,[X_b,X_c]]+\text{cyclic permutations }=0\mathrm{~.}
      \label{eq:jacobiid}
    \end{equation}
  还有一种写法就是:
  \begin{equation}
    [X_a,[X_b,X_c]]=[[X_a,X_b],X_c]+[X_b,[X_a,X_c]]. 
    \label{eq:anotherjacobi}
  \end{equation}
\end{itemize}

\tip{不同的参数化对于结构常数的影响}{
  我们发现如果我们选取不同的参数化方式,结构常数会发生变化。比如说我们把$ \alpha_a \to \beta_a = f(\alpha_a) $,这个时候根据生成元的定义生成元会发生变化:
  \begin{equation}
    X_a \to X'_a = \frac{\partial \beta_b}{\partial \alpha_a}X_b
    \label{eq:genchange}
  \end{equation}
  这个时候结构常数会发生变化:
  \begin{equation}
    f_{abc} \to f'_{abc} = \frac{\partial \beta_a}{\partial \alpha_i}\frac{\partial \beta_b}{\partial \alpha_j}\frac{\partial \alpha_k}{\partial \beta_c}f_{ijk}
    \label{eq:structureconstantchange}
  \end{equation}
  但是请注意,这个变化是一个类似于张量变换的关系,并不是任意变化!!所以我们可以把结构常数理解为一个张量。并且其实第三个指标应该是上指标才对!!
}

\bigskip 

\hlr{ Adjoint Representation} 

虽然对于一个李群表示可以帮我们自动生成一个李代数表示。根据commutator的结构我们有:
\[
  f_{bcd}f_{ade}+f_{abd}f_{cde}+f_{cad}f_{bde}=0. 
\]
这个告诉我们其实把 structural constant 可以理解为一个线性映射。
\defi{ Adjoint Representation

  我们定义一个线性映射:
  \begin{equation}
    (T_a)_{bc} = -if_{abc}
    \label{eq:adjointrepdef}
  \end{equation}
  这个线性映射满足李代数的结构,并且是一个表示。我们称之为 Adjoint Representation。其实这个表示空间就是同构与李代数的空间。
}
我们不难验证其实:$ [T_a,T_b]=if_{abc}T_c $
\begin{itemize}
  \item 这个表示的维度其实就是李代数的维度「也是李群的维度」也就是「确定一个李群元素需要的指标的维数」
  \item 对于【至少存在一个Unitary表示的群】来说,给出的Adjoint Representation是一个pure imaginary representation。
\end{itemize}

\hlr{Killing Form}

Adjoint Represenation构成了一个线性映射(其实就是李代数到自己的自同态线性映射$ End(g) $)的线性空间。这个线性空间上面我们可以进行一些操作,比如基的变换【其实这个就等价于上面讨论的我们换一个参数化】

同时我们可以定义两个元素的“内积”,其实是一个双线性form:
\defi{ Killing Form

  对于Adjoint Representation所有的Adjoint Rep这样的映射所在的线性空间。我们定义一个双线性形式:
  \begin{equation}
    g_{ab} = \text{Tr}(T_aT_b)
    \label{eq:killingform}
  \end{equation}
  我们称之为 Killing Form。
}

我们发现这是一个对称的双线性形式,并且是非退化的【但是不一定是正定的,甚至可以是不定的】。并且我们会发现如果我们改变参数化或者等价的就是对于$ X_a $进行一个线性变换。那么得到的killing form的变换是:
\begin{equation}
  X_a\to X_a^{\prime}=L_{ab}X_b
  \label{eq:transformation}
\end{equation}
我们会发现结构常数按照之前讨论的:
\begin{equation}
  f_{abc}\to f_{abc}^{\prime}=L_{ad}L_{be}f_{deg}L_{gc}^{-1}
  \label{eq:structureconstanttransformation}
\end{equation}
【注意第三个指标其实是逆变的】所以Adjoint Representation的变换是:
\begin{equation}
  [T_a]\to[T_a^{\prime}]=L_{ad}L[T_d]L^{-1}
  \label{eq:adjreptrans}
\end{equation}
所以Killing form的变换是:
\begin{equation}
  \mathrm{Tr}(T_aT_b)\to\mathrm{Tr}(T_a^{\prime}T_b^{\prime})=L_{ac}L_{bd}\mathrm{Tr}(T_cT_d)
  \label{eq:killingformtrans}
\end{equation}
我们意识到Killing form的变换如同一个二阶张量而Adjoint Representation的变换如同一个一阶矢量算符。而结构常数的变换如同一个(2,1)阶张量。我们不妨选择一些合适的$ L_{ab} $线性组合保证$ g_{ab} $ Killing Form 是对角化的。




\bigskip
\hlr{ Compact Lie Algebras}

\textbf{Compact Lie Algebra}: 我们会发现对于一些特殊的代数称为compact Lie Algebra:killing form是正定的。也就是所有的特征值都是正数,所以我们总可以进行一个$ T_a' = L_{ab} T_b $线性变换导致线性变换之后的基满足:
\begin{equation}
  \mathrm{Tr}(T_aT_b)=\lambda\delta_{ab}
  \label{eq:compactkilling}
\end{equation}

\begin{enumerate}
  \item 对于这样的基之下「这样的参数化之下」,我们的结构常数是\textbf{全反对称的}也就是可以写作:
\begin{equation}
  f_{abc}=f_{bca}=f_{cab}=-f_{bac}=-f_{acb}=-f_{cba}.
  \label{eq:fullyantisymmetric}
\end{equation}
\item 并且Adjoint Representation是一个Unitary Representation。以及生成元$ T_a $都是Hermite的。
\end{enumerate}

\YL{一个大坑!有机会可以学习一下流形视角下面的李群和李代数!!}

\bigskip 
\hlr{ Simple and Semi-simple Lie Algebras}

\textbf{Invariant Subalgebra: }也就是存在一个子空间,和所有Algebra元素commute都是子空间之中的元素。

\begin{itemize}
  \item Inv subalgebra 进行 exp操作得到的必然是一个invariant subgroup
\end{itemize}

\textbf{Simple Lie Algebra: }不存在非平凡invariant subalgebra的Lie Algebra。

\textbf{Simple Group: }也就是可以通过Simple Lie Algebra生成的Lie Group。

\thm{
  The Adjoint Representation of a Lie Algebra is irreducible when 满足\cref{eq:compactkilling} And it is a Simple Lie Algebra. 
}

\textbf{Abelian Invariant Subalgebra: } 就是存在一个元素 $ X $和所有代数元素都对易的【其实就是centre】

这样的子代数给出了表示就是和$ \langle T_a,T_b \rangle = g_{ab} = 0 \times \delta_{ab}$的这些0本征值的元素。因为这些元素的Adjoint表示都是$ T_a  = 0$。 所以我们的Adjoint 表示不能够包含这些代数元素的信息捏。

\textbf{Semi-simple Lie Algebra: }不存在Abelian Invariant Subalgebra的Lie Algebra【这样的大叔可以写成Simple Lie Algebra的直和】

\bigskip
\hlr{Lie Group \& Lie Algebra作用在量子态和算符上面}

一组合理的算符和坐标的变换法则是:
\begin{equation}
  |i\rangle\to|i^{\prime}\rangle=e^{i\alpha_aX_a}|i\rangle.  
  \label{eq:ketransform}
\end{equation}
\begin{equation}
  O\to O^{\prime}=e^{i\alpha_aX_a}Oe^{-i\alpha_aX_a}.
  \label{eq:operatortrasn}
\end{equation}
并且我们可以验证这样的变换后新的算符在新的基下面的分量和旧的算符在旧的基下面的分量是一样的。这很协变了((


\imp{表示作用在张量算符上}{
  我们意识到,表示作用在一个量子态上面就是直接乘上去。我们该怎么理解表示作用在算符上面呢?唯一的理解方式是,不要理解表示单独作用在算符上面永远理解表示作用在【被算符作用的量子态上】:
  \begin{equation}
    O|i\rangle\to e^{i\alpha_aX_a}O|i\rangle=e^{i\alpha_aX_a}Oe^{-i\alpha_aX_a}e^{i\alpha_aX_a}|i\rangle=O^{\prime}|i^{\prime}\rangle.
    \label{eq:actionrepoponops}
  \end{equation}
  只是这个关系我们可以等效的表达为【算符和量子态进行分别变换】并且结果正好是【分别变换然后相乘】:
  
\begin{equation}
  |i\rangle\to|i^{\prime}\rangle=e^{i\alpha_aX_a}|i\rangle.  
  \label{eq:ketransform}
\end{equation}
\begin{equation}
  O\to O^{\prime}=e^{i\alpha_aX_a}Oe^{-i\alpha_aX_a}.
  \label{eq:operatortrasn}
\end{equation}

同样子的我们可以定义李代数作用在算符和量子态上面。这个作用的物理意义是,算符或者量子态在无限小李群变换时候的变换:
\begin{equation}
  \begin{aligned}
  &X_a|i\rangle, \quad -\langle i|X_a\\ 
 &[X_a,O]
\end{aligned}
  \label{eq:operatorliealgebra}
\end{equation}
这也是形式化的定义。但是我们考虑一个李代数作用在算符作用的量子态上面的时候,结果【正好是分别作用然后相加】!!

\hlr{注意:李群的作用作用完是相乘;李代数的作用作用完是相加!!}
}
\bigskip 
\subsubsection{SU(2) Algebra}

一个 Lie 代数基本上由其结构常数所决定:
\begin{equation}
  [J_j,J_k]=i\epsilon_{jkl}J_l
  \label{eq:su2algebra}
\end{equation}
为了构建表示,我们尽可能多地对角化某个算符(通常取 $J_3$),并由此定义升降算符:
\begin{equation}
  \begin{aligned}
    [J_3,J^\pm] &= \pm J^\pm \\
    [J^+,J^-] &= 2J_3 
  \end{aligned}
  \label{eq:rep}
\end{equation}
于是得到表示矩阵元为
\begin{equation}
  \begin{aligned}
  \langle j,m'|J_3|j,m\rangle &= m\,\delta_{m'm} \\ 
  \langle j,m'|J^+|j,m\rangle &= \sqrt{(j-m)(j+m+1)}\,\delta_{m',\,m+1} \\ 
  \langle j,m'|J^-|j,m\rangle &= \sqrt{(j+m)(j-m+1)}\,\delta_{m',\,m-1}
\end{aligned}
\label{eq:repsu2}
\end{equation}

\bigskip 
\hlr{约化一个 SU(2) 表示的方式}

\begin{enumerate}
  \item 寻找 $J_3$ 的最大本征值对应的本征矢,记作 $|j,j\rangle$;
  \item 使用升降算符作用,生成整个不可约表示;
  \item 在正交补空间中继续寻找下一个不可约表示,并用同样的方法构造;
  \item 重复步骤 3,直到将整个表示完全约化。
\end{enumerate}

\bigskip 
\hlr{代数的表示张量积约化}

角动量耦合本质上就是对张量积表示进行上述方法的约化。
\imp{我们怎么理解张量积约化的记号}{
  在这一部分我们使用了一个记号:
  \begin{equation}
    |3/2,3/2\rangle=|1/2,1/2\rangle|1,1\rangle
    \label{eq:tensorproductnotation}
  \end{equation}
  这个记号并不是一个等式,是一个【定义式】。他告诉我们这两个态的张量积可以当作某一个不等价不可约表示【子空间】的heighest weight state。\hlr{总之,记住这是一个记号定义,而不是把一个表示空间变成另一个表示空间。}
}

\imp{约化表示的正交性关系}{
我们发现对于一个巨大的奇奇怪怪的可约化的表示空间的时候,我们约化之后会有海量正交性关系,我们下面进行解释:

\begin{enumerate}
  \item 不同weight的量子态必然正交:这个是由Hermite算符的特征保证的,因为对于SU2来说$ J_3 $是一个Hermite算符。实际上我们研究的一般都是Unitary Rep,所以我们选择的生成元都是Hermite的。
  \item 同样weight但是不同primary给出的descendent都是正交的:这是因为我们的约化操作保证的,我们约化是选择的正交的空间进行构造的!
\end{enumerate}

}

\subsection{Explanation and Questions for Week 2}

\question{ 本书之中我们经常使用 $ \ket{i} $ 作为一个表示的基,这到底是什么意思?}

我觉得本书之中这样的使用其实意思就是$ \ket{i} $的意思就是$ (0,...,1,...0) $ 其中1出现在第i个位置上面。因为算符 $ D_a(g) $ 作用在这个基上面给出的矩阵就是这样的 $ \bra{i} D_a(g) \ket{j} = G $

\imp{李代数的定义讨论}{
  我们说明了连续群的生成元,根据群的乘法关系,如果写作exp parametrization的形式需要存在commutative algebra的结构。

  但是我们发现我们生成元构成的代数不仅仅是一个commutative algebra而且是一个Lie algebra。更还有乘法关系,也就是一个李代数的包络代数的结构。
}

\question{ 生成元本身自带哪些数学结构?(sec:2.2)}

请注意我们的生成元的定义是:
\begin{equation}
  X_a\equiv-i\left.\frac{\partial}{\partial\alpha_a}D(\alpha)\right|_{\alpha=0}
  \label{eq:generatordef}
\end{equation}
也就是选定某一个对于群表示的参数化之后,生成元就是表示对于这个参数的导数。并且为了方便讨论,我们一般使用指数参数化的形式。也就是我们使用这样的一组$ \{ \alpha_a\} $进行参数化,使得$ e $附近的群元素可以写作:$ D(\alpha) = exp(i \alpha_a X_a) $。

\begin{enumerate}
  \item 加法/数乘结构: 
所有的生成元都是线性映射【因为表示是这么定义的】,某两个线性空间之间的线性映射其实构成了一个线性空间,赋予「数乘+加法」结构。根据tayler expansion的性质,我们知道这样组合的生成元仍然是生成元。
\item 乘法结构:
同时,因为我们考虑的表示一般是一个线性空间自己到自己的线性映射。所以,生成元一般可以自然赋予乘法的结构。但是我们并不知道这样的乘法结构之后的物体是否仍然是一个生成元。
\item 对易结构:
只是commutation algebra的结构告诉我们,如果$ X_a $是选定某个指数参数化之后的生成元,那么$ [X_a, X_b] $也是一个对于指数参数化生成元。
\end{enumerate}

\question{ 请告诉我一些线性代数的基本知识球球了!!}

\textbf{实对称矩阵:} 可以进行对角化,并且所有本征值都是实数。但是正负并不确定。显然是Hermite的。

\textbf{纯虚反对称矩阵:}显然也是Hermite的,并且本征值都是纯虚数。

\question{ Adjoint representation 到底由什么决定?}

虽然这本书之中我们讨论的都是连续群的表示和代数的表示。但是 Adjoint representation 是由代数和群乘法结构决定的。这个和任何表示无关,这是一个由代数本身结构决定的自然的表示。这个表示选择的基就是 代数 这个线性空间!!


\imp{抽象的李代数定义以及本书之中的定义}{
  本书之中李代数定义为连续群表示的生成元(单位元附近的导数)。所以自然是表示空间自己到自己的线性映射空间。

  但是,李代数的定义是不依赖于表示的!虽然我们可以通过表示和生成元定义拥有李代数结构的线性映射构成的线性空间。但是李代数结构本身与表示无关并且可以通过抽象的方式进行定义。

  Adjoint Representation并不是由表示决定的而是李代数的抽象结构决定的。
}

\textbf{我们可以用Caley-Hamilton theorem来计算矩阵的函数!!并且通过本征值求解确定系数!!}

\thm{
  Hamilton-Cayley Theorem 

  对于一个$ n\times n $矩阵$ A $,它的特征多项式为:
  \begin{equation}
    f(A)=\sum_{k=0}^{n-1}c_kA^k. 
    \label{eq:expansionmat}
  \end{equation}
}
我们可以使用这个定理来计算矩阵的函数,比如说指数映射。我们求一下A的本征向量,并且作用在两边,我们就可以给出一个系数方程组。反解方程组就可以得到系数。








% section Week 2 Reading (end)

\newpage
\section{Extra: Formal Formalism of Lie Algebra}\label{sec:Extra: Formal Formalism of Lie Algebra} % (fold)
本补充章节我们准备讨论李代数的形式化表达,以及李群李代数和流形之间的关系。同时更重要讨论各种【李代数之间的关系,以及复化!】

\subsection{Lie Group和Lie Algebra定义}

在Georgi的书之中,我们先定义了李群是所有的能够用实数$ \mathbb{R}^n $进行标定的连续群。并且给出了李群把表示的概念,并且说明李群可以被矩阵群faithfully的表示。

然后,我们说明李群的任意表示至少局域的可以(对于simple connected and compact lie group全局的可以,由于georgi仅仅考虑simple connected and compact李群所以他其实并没有强调这个适用性)用explonential map进行标记。

通过选择一个物理上常用的explonential map,我们可以通过李群的表示写成下面的形式:
\begin{align}
  D(g) = e^{i \alpha_a T^a}
\end{align}
其中$ T^a $是李群的表示的生成元「就是一堆矩阵」。然后我们发现了根据群的乘法结构,这些矩阵构成了一个有Lie bracket的代数结构的实线性空间,我们把这些矩阵构成的实线性空间称为这个李群对应的李代数。

但李群和李代数的定义其实可以并不依赖于表示抽象的给出,并且这样的抽象的定义对于推广和理解表示有更多的好处。我们可以直接定义李群和李代数:

\YL{请补充,没写完}

\subsection{su(2)李代数数学,物理定义}

\hlr{数学上su(2)李代数的定义}

首先我们定义什么是SU(2)群:
\defi{
  SU(2)群

  SU(2)群是所有行列式为1的2阶酉矩阵构成的群。
}
对于这个群,我们知道是一个李群。很自然能够推导出它的李代数:
\defi{
  su(2)李代数

  su(2)李代数是所有满足$ X^\dagger+X=0 $并且$ \mathrm{tr}(X)=0 $的2阶矩阵构成的李代数。是一个【实向量空间】
  \begin{align}
    \mathfrak{su}(2)=\{\left.X\in M_2(\mathbb{C})\mid X^\dagger=-X,\mathrm{~Tr}(X)=0\right.\}.
  \end{align}
  这个代数空间我们可以很自然的选择一组基是:
  \begin{align}
    X_i=\frac{i}{2}\sigma_i,\quad[X_i,X_j]=\epsilon_{ijk}X_k.
  \end{align}
}  
所以我们知道su(2)李代数可以理解为所有的反厄米矩阵并且迹为0的2阶矩阵构成的李代数。这是一个实向量空间。对于切空间,我们可以指数映射到群上面。这个指数映射的形式是:
\begin{align}
  U=e^{tX}
\end{align}

\bigskip
\hlr{物理上su(2)李代数的定义}
\bigskip

\hlr{!!!!!!但是!!!!!!}问题是,这样子基都是anti-Hermite的矩阵,但是物理上我们习惯使用Hermite矩阵作为代数元素!!所以物理人定义了一个trick,我们偷偷修改了指数映射的形式是:
\begin{align}
  U=e^{i t H}
\end{align}
然后顺手把基也改成了Hermite矩阵:
\begin{align}
  J_i=\frac{1}{2}\sigma_i = -i X_i,\quad[J_i,J_j]=i\epsilon_{ijk}J_k.
\end{align}
这样子,我们也“如”consistent的得到了一个李代数,并且这个李代数按照modify之后的指数映射映射到SU(2)群上面。并且\textbf{我们动了指数映射,动了基的形式,但是没有动数域!!}所以写作Hermite的形式李代数的元素依旧是对于三个基的实数线性组合捏。

\rmk{但是这么搞总是有bug的

比如:我们写出来$ [J_i,J_j] = i \epsilon_{ijk} J_k $的时候,也就是说仿佛对易出来了一个复数线性组合的生成元,不在代数里面了呃呃呃。并且对易子本身就是anti-Hermite的东西,我们强行让对易子等于一个Hermite的空间之中的物体是完全不可能的。

我们对此的解决方式是:相当于「重新定义」了对易子,对于Hermite的李代数元素使用-i[,]作为李括号。这样子就没有问题了。
}

\bigskip
\hlr{为什么物理学家可以这么乱搞}

我觉得一个原因是,物理上相比于代数表示更重要的是群的表示。因为我们物理的坐标变换是由群来描述的。所以,虽然我们的代数结构没有流形和切空间的interpretation这么严谨,也不是特别简洁,但是只要我们能够通过这种方式得到正确的群表示就行了。


\subsection{复化以及半单李代数表示}

\bigskip
\hlr{su(2)李代数的复化}

当然,我们也可以把su(2)李代数复化成一个复向量空间。这个复化的过程就是把系数域从实数域扩展到复数域。也就是说我们允许基的线性组合系数是复数而不是实数。而不是像是之前强行变Hermite一样,那个不叫复化。我们现在是允许所有复数域上面的线性组合。

对于复化之后的代数【由于物理的定义太不严谨了,我们一般复化是用数学家的定义】我们称之为$ sl(2,\mathbb{C}) $下面给出定义:
\defi{$ sl(2,\mathbb{C}) $定义

就是所有的二维traceless复矩阵构成的李代数:
\begin{align}
  \mathfrak{sl}(2,\mathbb{C})\equiv\{X\in M_2(\mathbb{C})\mid\mathrm{Tr}(X)=0\}
\end{align}
}

\bigskip
\hlr{复化以及su(2)表示}

复化最重要的作用之一就是帮我们求表示。有的时候代数的表示并不好求,我们就先求复化后的代数元素的复线性组合的表示。然后我们再把这个线性组合回去,就可以得到原来代数的表示。

一个最重要的例子就是su(2)的表示,我们不好直接求,那么就先求复化后的产生湮灭算符的表示:
\begin{align}
  J_\pm = J_1 \pm i J_2
\end{align}
这个表示很好求,我们求出来之后再把$ J_1,J_2 $表示回去就可以了。

\bigskip
\hlr{复化以及Cartan Weyl Basis}

我们在代数之中讨论su(2)的Cartan子代数是$ J_3 $,然后升降算符是$ J_\pm $。但是我们发现这个升降算符其实并不是代数里面的元素,因为它是复线性组合的。也就是说我们只能在复化之后的代数里面讨论Cartan Weyl Basis:
\begin{align}
  H=2J_3,\quad E_+& =J_+,\quad E_-=J_-\\ 
  [H,E_\pm]=\pm2E_\pm&,\quad[E_+,E_-]=H.
\end{align}
我们使用这个basis求的其实是$ sl(2,\mathbb{C}) $的表示,然后再把表示回去就可以了。不仅仅是su(2)李代数,其他的李代数我们也可以复化然后使用Cartan Weyl Basis进行表示的求解。所以复化给了我们generally 一个求表示的好方法。

同时,我们也需要记住,cartan Weyl Basis是复化之后的代数里面的东西。我们只能在复化之后的代数里面讨论root和weight。所以我们说到$ A_i, D_i $这些代数的时候,我们说的是复化之后的代数。只是,所有的单李代数都可以形成这个复李代数实化后的结果。


\subsection{实化以及不同李代数关系}


\bigskip
\hlr{复李代数的实化}

我们可以复化,当然也可以再进行实化。对于$ sl(2,\mathbb{C}) $李代数我们可以进行实化得到两个不同的实李代数:
\begin{align}
  \mathfrak{su}(2) = \{a_1 i \sigma_1 + a_2 i \sigma_2 + a_3 i \sigma_3 \mid a_i \in \mathbb{R}\}\\
  \mathfrak{sl}(2,\mathbb{R}) = \{b_1 \sigma_1 + b_2 i \sigma_2 + b_3 \sigma_3 \mid b_i \in \mathbb{R}\}
\end{align}
这两个实李代数都是$ sl(2,\mathbb{C}) $的实化。并且这两个实李代数是完全不一样的。一个是compact的,一个是non-compact的。  
su(2)相当于是把$ sl(2,\mathbb{C}) $挑选出其中所有anti-Hermite的元素构成的实李代数;而$ sl(2,\mathbb{R}) $是挑选出所有实矩阵构成的实李代数。



% section Extra: Formal Formalism of Lie Algebra (end)

\newpage
\section{Week 3 Reading: Tensor Operators and Wigner-Eckart Thm}\label{sec:Week 3 Reading} % (fold)
week 3 reading的内容是第四章,张量算符。
\subsection{Take home messages for Week 3}

\hlr{张量算符的概念}

我们已经知道表示作用在一个表示空间的向量上面的时候就是直接作用;对偶向量还需要$ \times -1 $;作用在算符上面是对易子。并且如果一个表示作用在算符作用的向量上面,可以理解为分别作用再【求和】。
\rmk{注意,对于李群来说是分别作用再相乘,对于李代数(可以理解为生活在指数上面)则是分别作用再相加。}

\defi{张量算符

  张量算符是一组算符$ \{ O^s_i\} $其中s是某一个代数表示的标记。当这个代数的元素作用在张量算符上面的时候满足:
  \begin{equation}
    [J_a,O_\ell^s]=O_m^s\left[J_a^s\right]_{m\ell}.
    \label{eq:tensoroperatordefi}
  \end{equation}
}
\rmk{
  注意,和张量算符有关的有两个表示空间;一个是张量算符本身作用的表示空间。一个是张量算符被某个李代数元素作用之后按照变换的表示空间$ s $!!
}


一个张量算符的例子,就是我们的位置算符。如果我们选择量子力学Hilbert Space作为表示空间。并且我们已知轨道角动量算符是SU(2)代数在Hilbert空间上的表示,动量和位置算符是Heisenberg 代数的表示。根据这些算符的关系定义式:$ J_a=L_a=\epsilon_{abc}r_bp_c $我们可以推导出来:
\begin{equation}
  \left[J_{a},r_{b}\right]=\epsilon_{acd}\left[r_{c}p_{d},r_{b}\right]=-i\epsilon_{acd}r_{c}\delta_{bd}=-i\epsilon_{acb}r_{c}=r_{c}[J_{a}^{\mathrm{adj}}]_{cb}
  \label{eq:tensorposition}
\end{equation}


\bigskip 
\hlr{怎么把张量算符变到标准基之中}

我们上面发现这个算符是按照adjoint representation进行变换而不是一个标准的不等价不可约表示进行变换。我们可以【对张量算符进行线性组合】构造一个按照标准的SU(2)三维的不等价不可约表示变换的一套张量算符。

\thm{张量算符变换表示s的相似变换

  如果我们希望某一套线性组合张量算符按照一个表示进行相似变换之后的表示进行变换。我们考虑相似变换:
  \begin{align}
    SJ_{a}^{D}S^{-1}=J_{a}^{s}
  \end{align}
  那么我们新的一套张量算符是:
  \begin{align}
    O_\ell^s=\Omega_y\left[S^{-1}\right]_{y\ell}\quad\mathrm{~for~}\ell=-s\mathrm{~to~}s
  \end{align}
}

上面的定理是对于一般情况的。但是对于SU(2)我们还有一个简单的方案。对于SU(2)来说,我们希望变换成为标准表示$ J_3 $作用在这些张量算符必然给出一个张量算符的weight的数值。不妨找一个线性组合被$ J_3 $作用之后正好是某一个数值乘上这个算符。再进行升降算符,就可以给出一整套张量算符。


\bigskip
\hlr{张量算符作用表示空间以及Wigner-Eckart定理【SU(2)代数作为例子】}

我们现在研究一个新的表示空间。也就是张量算符作用在不等价不可约表示的空间上面。这个空间也构成了一个代数表示,并且同时这个空间的行为很像是一个张量积表示:
\begin{equation}
  \begin{aligned}J_{a}&O_{\ell}^{s}\left|j,m,\alpha\right\rangle\\&=[J_{a},O_{\ell}^{s}]\left|j,m,\alpha\right\rangle+O_{\ell}^{s}J_{a}\left|j,m,\alpha\right\rangle\\&=O_{\ell^{\prime}}^{s}\left|j,m,\alpha\right\rangle[J_{a}^{s}]_{\ell^{\prime}\ell}+O_{\ell}^{s}\left|j,m^{\prime},\alpha\right\rangle[J_{a}^{j}]_{m^{\prime}m}\end{aligned}
  \label{eq:tensoropeonspace}
\end{equation}  
显然这个空间也可以进行约化成为等效的不等价不可约表示。我们进行约化:
\begin{enumerate}
  \item \textbf{第一步:定义Heighest Weight State} 我们把$ O^s_s\ket{j,j} $强行规定是 $ k_J \ket{J,J} $,其中$ J = s+j $ 。
  \item \textbf{第二步:构建整个表示} 将$ J^\pm $算符作用上去,构建整个表示【这一步务必注意系数!!我们不能把$ J^- $作用一下的态就定义成$ \ket{J,J-1} $我们需要乘上一个系数】【这个注意特别关键,因为我们张量算符作用的态是没有合理的内积定义的,我们不能够归一化或者找正交的态】
  \item \textbf{第三步,寻找下一个Heighest Weight State} 我们寻找在$ J_3 $下面本征值次大,并且是系数恰好满足$ J^+ \ket{J-1,J-1} = 0$的态,起名字为$ k_{J-1} \ket{J-1,J-1} $
  \item 重复上面的步骤,直到我们已经找到了找全了所有的Heighest Weight State。
\end{enumerate}
\rmk{一个最重要的和之前约化张量积表示的不同是:算符作用空间【没有内积的定义】。所以我们不能归一化,更不能够定义正交。我们只能根据上面的规则起名字。但是,一个问题是:起出来的名字对于每一个表示来说,都有一个自由的系数$ k_J $。这个系数由下面的一些东西决定:
\begin{itemize}
  \item 张量算符具体是什么
  \item 三个表示是什么【张量算符表示,张量算符所作用的表示空间,约化后的表示空间】
  \item 其他可能的物理自由度$ \alpha , \beta ...$ 
\end{itemize}
但这个自由度和表示内部的基坐标$ m $是无关的!!!!

}

一个我们上面的起名字的直接导致的结果就是:
\begin{align}
  \sum_\ell O_\ell^s\left|j,M-\ell,\alpha\right\rangle\left\langle s,j,\ell,M-\ell\mid J,M\right\rangle=k_J\left|J,M\right\rangle
\end{align}
对比一下对于张量积表示我们有:
\begin{align}
  \sum_\ell\left|s,\ell\right\rangle\left|j,M-\ell\right\rangle\left\langle s,j,\ell,M-\ell\mid J,M\right\rangle=\left|J,M\right\rangle
\end{align}
这里我并没有$ k_J $系数,因为我们可以定义$ \braket{J,J}{J,J} = 1 $在张量积空间。但是对于张量算符作用空间,我没没有内积的定义。
\imp{only name}{
  我们牢记这些约化都是「重命名」。也就是说对于约化张量积表示,我们是在重命名一个「张量积空间」的向量;自然是有内积结构的。而对于张量算符作用空间,我们是在重命名一个「张量算符作用空间」的向量;这个空间没有内积结构。
}

\rmk{一个可能的疑惑是,我们为什么会有一步是需要$ J^- $作用之后还乘上一个系数??这是因为我们需要保证重命名后的向量完全的满足是一个不等价不可约表示的向量,除了可能多一个global的系数!!}

\thm{Wigner-Eckart定理

这个定理其实就是这个约化的直接结果,我们考虑形式话的“内积”。也就是张量算符表示空间在一个重命名之后的分量,我们有:
\begin{equation}
  \begin{aligned}&\left\langle J,m^{\prime},\beta\right|O_{\ell}^{s}\left|j,m,\alpha\right\rangle\\&=\delta_{m^{\prime},\ell+m}\left\langle J,\ell+m|s,j,\ell,m\right\rangle\cdot\left\langle J,\beta\right|O^{s}\left|j,\alpha\right\rangle\end{aligned}
  \label{eq:wignerckart}
\end{equation}
}
\imp{这个定理写了什么}{
  这个定理是形式话的书写!!但是我们不妨就把这个理解成物理的两个表示之间的「跃迁」。具体看下面的讨论。
}


\bigskip
\hlr{Wigner-Eckart定理理解计算【特别是计算细节】}

一个书中的辅助理解的例子。没啥意思。


\bigskip
\hlr{张量算符约化}

如果一组算符按照可约表示进行变换,显然我们可以约化成为按照不等价不可约表示进行变换的算符。这个过程基本上和张量积表示约化是一样的。一个bug是我们怎么找Heighest Weight Operator??

唯一的方法就是找到所有满足 $ [J^+,O] = 0 $的算符组合。然后再通过下降算符构造。【依然注意系数问题,注意我们下降算符作用在一个基上面给出的不是下降一个weight的基,而是下降一个weight的基乘上一个系数!!】

最后我们看看根据我们原先算符的数量知道有没有约化全!


\bigskip
\hlr{张量算符的乘积}

没啥意义,就是对易子的乘法规则! 

\imp{提示用$ J^- $进行表示构造}{
  一定一定一定要考虑$ J^- $自己表示的系数!!!就是$ \sqrt{\displaystyle\frac{1}{2}(j... m...)} $这样的东西!!!
}


\subsection{Explanations and Questions for Week 3}

\question{ 为什么我们认为角动量算符是SU2 代数的表示?}

因为我们实验给出的理论是角动量算符满足和SU2代数一样的对易关系并且是作用在Hilbert Space上面的算符。所以我们自然认为角动量算符是SU2代数的表示。包括其他动量算符什么的也是一样的道理我们说这些算符是Poincare 代数的表示。
\qed 

\question{ 我们在讨论Wigner-Eckart定理的时候,到底为什么会有仅仅与表示有关的量$ k_J $?}

是因为我们在把一个张量积表示decompose称为很多不等价不可约表示的直和的时候。我们一旦找到了一个【Heighest Weight State】。那么我们可以直接通过升降算符进行构造整个表示。

通过升降算符进行构造的过程是全部被群结构进行定义的。唯一可能出现出现的差别就是我们认为$ J^+ $作用在Heighest weight states上面是0。我们对于量子态的张量积我们可以定义归一化的量子态。但是对于算符作用在量子态上面,的情况,我们并不清楚是否有合理的归一化。所以left一个和表示以及算符相关的系数。
\qed

\question{ 考虑乘上算符之后构成更大的表示空间的量子态的关系到底是什么意思?(sec:4.4)}

对于Wigner-Eckart定理有一个经典的核心写法:
\begin{equation}
  \sum_\ell O_\ell^s\left|j,M-\ell,\alpha\right\rangle\left\langle s,j,\ell,M-\ell\mid J,M\right\rangle=k_J\left|J,M\right\rangle 
  \label{eq:wignereckartth}
\end{equation}

我们需要仔细讨论一下这个式子的意思。表面上左边是一个spin-1/2表示空间的向量作用一个张量算符。右边是一个spin0-3/2表示空间的向量,这很困惑,因为一个向量空间的向量不可以作用一个算符然后变成另一个向量空间的向量。

但是这个式子需要理解为一个定义式子,这个定义式子是基于一个发现:
\defi{张量算符作用空间

张量算符作用在一个表示空间的量子态上面。在generator 的作用在的行为等价于一个张量积表示空间:
\begin{equation}
  \begin{aligned}&J_aO_\ell^s|j,m,\alpha\rangle\\&=[J_a,O_\ell^s]|j,m,\alpha\rangle+O_\ell^sJ_a|j,m,\alpha\rangle\\&=O_{\ell^{\prime}}^s\left|j,m,\alpha\right\rangle[J_a^s]_{\ell^{\prime}\ell}+O_\ell^s\left|j,m^{\prime},\alpha\right\rangle[J_a^j]_{m^{\prime}m}\end{aligned}
  \label{eq:tensorpro}
\end{equation}
也就是我们考虑一个张量算符作用在一个ket构成的线性空间数学结构在李代数表示论下等价于另一个更大的表示空间——我们理解这个更大的表示空间是【张量算符作用空间】
}
下面我们把这个更大的空间进行约化。所以我们写出来式子\cref{eq:wignereckartth}。这个式子的意义是,一个【定义式】,我们把这个更大的空间之中的一个向量「左边」,起名作为一个新的不等价不可约表示的的Heighest weight state「右边」
\qed

\question{为什么张量算符作用在表示空间的这个进行约化需要有$ k_J $系数,一般的张量积表示没有?}

其实是相对而言的。我们显然可以把这个系数融入到CG系数的。我们知识因为人类先研究的张量积表示,我们合理的定义CG系数,并且保证量子态归一化是合理的。于是我们保证张量积是没有$ k_J $的系数的。

但是真实的张量算符作用,我们永远不能保证作用之后的量子态是不是归一的。因为算符的期望值显然不一定是1否则所有张量算符都是trivial的了!!
\qed 

\question{这里我们的$ k_J $系数到底由什么的性质决定捏?}

我们$ k_J $是在构造Heighest weight state的时候产生的,所以我们讨论这个时候可能make a difference的参数:$ s, j, J, O^s, \alpha $都可能对其有影响。
\qed

\question{Wigner-Eckart定理的书写形式是什么意思,从数学上到物理上是什么意思??}

\hlr{数学上进行理解:}

数学上我们写出来:
\begin{equation}
  \begin{aligned}&\left\langle J,m^{\prime},\beta\right|O_\ell^s\left|j,m,\alpha\right\rangle\\&=\delta_{m^{\prime},\ell+m}\left\langle J,\ell+m|s,j,\ell,m\right\rangle\cdot\left\langle J,\beta\right|O^s\left|j,\alpha\right\rangle\end{aligned}
  \label{eq:wethm}
\end{equation}
我们认为左边式子$  \left\langle J,m^{\prime},\beta\right|O_\ell^s\left|j,m,\alpha\right\rangle $的意思是,一个等价于约化后的表示$ \ket{J,m'} $的向量和张量算符作用空间的另一个向量进行内积。内积结果是一个CG系数乘以一个只和表示,张量算符,其他自由度相关的系数。

\bigskip
\hlr{物理上Hilbert Space以及可观测量的理解:}

如果我们把张量算符理解为可观测量,把左右的braket理解为Hilbert Space上面的量子态。我们可以把左边的数值理解为一个【跃迁的强度】!也就是在一定的微扰之后一个角动量跃迁到另一个角动量的强度。

正是张量算符作用在某一个表示空间之后会变成一个更大的表示空间的性质让我们可以进行这样看起来很非法的计算!!

\line

下面我们用这样的思路看一些例子:

例子1: 
$ \left\langle1/2,1/2,\alpha\right|r_3\left|1/2,1/2,\beta\right\rangle=A $注意,我们数学上bra和ket并不属于同样的一个线性空间。ket是属于$ SU(2) $的spin -1/2表示空间;但是bra是属于$ r^1 \ket{1/2,m} $这样的张量算符作用空间的,并且是这个空间进行约化之后的一个不等价不可约表示的Heighest weight state。

例子2:
$ \bra{1/2,1/2,\alpha}r_{+1}\left|1/2,1/2,\beta\right\rangle $这个式子必然是0。因为我们知道我们使用的标准约化方法的定义是需要保证不同表示的量子态之间完全正交的。$ \ket{1/2,1/2} $是约化后spin-1/2表示的Heighest weight state。$ r_{+1}\ket{1/2,1/2} $是约化后spin-3/2表示的Heighest weight state。两个Heighest weight state必然正交。所以我们有结论:$ \bra{1/2,1/2,\alpha}r_{+1}\left|1/2,1/2,\beta\right\rangle  = 0$

\qed 



% section Week 3 Reading (end)

\newpage
\section{Week 4 Reading: Isospin and Physics}\label{sec:Week 4 Reading} % (fold)
\subsection{Take Home Messages}

\YL{[个人感觉这一章不是特别重要,懒得补充了]}


\hlr{特殊讨论:关于Qij以及张量算符约化的关系!!}

大概就是相当于构成了一组表示((在某个表示空间的子空间上面作用((

我们可以把这个表示空间decompose成为这些不可约表示的直和。


\bigskip
\hlr{特殊讨论:关于自旋轨道耦合研究的基!}

对于自旋轨道耦合,我们的微扰Hamiltonian的形式是: $ \Delta H = k L \cdot S $所以这个时候$ S_z, L_z $都不是守恒量了(并不和Hamiltonian commute)。

但是,我们可以很巧妙的选择角动量耦合的基$ \ket{j,m_j} $作为我们的基。因为这个基是$ J^2,J_z $的本征态,并且$ J^2 $和$ J_z $都和Hamiltonian commute。所以我们可以把Hamiltonian在这个基下对角化。



\subsection{Questions and Thoughts}




% section Week 4 reading (end)

\newpage
\section{Week 5 Reading: Root and Weight \& SU(3)}\label{sec:Week 5 Reading} % (fold)

这里我主要follow Georgi的教材,但是也会参考很多不同的思路进行辅助理解本章之中的内容。我们首先补充一些方便理解的数学知识,然后我们主要还是fol Georgi的思路进行讨论。

\subsection{数学补充知识}

\bigskip
\hlr{Adjoint Representation}

之前我们知道,如果我们存在Lie Algebra的一组基$ X_a $我们可以为这样一组基根据我们的structural constant $ f_{abc} $定义一个表示:
\begin{align}
  \left[T_a\right]_{bc}=-if_{abc}\mathrm{~.}
\end{align}
但是这样的理解是有局限性的,我们仅仅能够理解basis dependent然后需要进行一些特殊设计才能够仔细研究表示空间。但是对于Adjoint representation我们还有一个更好的定义方式。


\defi{Adjoint Representation

  对于一个Lie Algebra $ \mathfrak{g} $,我们定义一个映射$ \mathrm{ad} $从$ \mathfrak{g} $到$ \mathrm{End}(\mathfrak{g}) $:
  \begin{align}
    \mathrm{ad}_X(Y)=[X,Y]\mathrm{~for~}X,Y\in\mathfrak{g}\mathrm{~.}
  \end{align}
  显然这个映射是线性的,并且满足:
  \begin{align}
    \mathrm{ad}_{[X,Y]}=[\mathrm{ad}_X,\mathrm{ad}_Y]\mathrm{~.}
  \end{align}
  所以这个映射是一个Lie Algebra的表示,我们把这个表示叫做Adjoint Representation。
}
\begin{itemize}
  \item 首先我们验证这个是一个表示。
    \begin{align}
      \mathrm{ad}_{[X,Y]}(Z)=[[X,Y],Z]=[X,[Y,Z]]-[Y,[X,Z]]=[\mathrm{ad}_X,\mathrm{ad}_Y](Z)\mathrm{~.}
    \end{align}
    \item 其次我们需要知道这个等价于之前的描述。
      \begin{align}
        \mathrm{ad}_{X_a}(X_b)=[X_a,X_b]=if_{abc}X_c\mathrm{~.}
      \end{align}
      所以我们有:
      \begin{align}
        \left[\mathrm{ad}_{X_a}\right]_{cb} =if_{abc} = - i f_{acb} = [T_a]_{bc}\mathrm{~.}
      \end{align}
      这和之前的定义是一样的。其中第一步我们挑出了作用b然后变成c的系数,这个和一般的矩阵定义是一样的。
\end{itemize}



\bigskip
\hlr{Killing Form的定义回顾}

之前我们已经知道对于选定一组基$ X_a $我们可以定义其上的基的一个bilinear form也就是:
\begin{align}
  g_{ab}=\mathrm{Tr}(T_aT_b)\mathrm{~.}
\end{align}
下面我们希望使用上面的Adjoint Representation的定义来重新定义这个bilinear form。并且这个form其实是李代数之间的。毕竟我们的Adjoint Rep是一个表示。

\defi{Killing Form

  对于一个Lie Algebra $ \mathfrak{g} $,我们定义其上的一个bilinear form:
  \begin{align}
    \gamma(X,Y)=\mathrm{tr}(\mathrm{ad}_X\mathrm{ad}_Y)\mathrm{~for~}X,Y\in\mathfrak{g}\mathrm{~.}
  \end{align}
  如果我们选定一组基$ X_a $,那么我们可以定义其上的矩阵:
  \begin{align}
    \gamma_{ab}=\gamma(X_a,X_b)=\mathrm{tr}(\mathrm{ad}_{X_a}\mathrm{ad}_{X_b})\mathrm{~.}
  \end{align}
  我们把这个bilinear form叫做Killing Form。
}

这个定义和上面的是完全一样的,我们可以计算:
\begin{align}
  ad_{X_a}ad_{X_b}(X_c)=ad_{X_a}([X_b,X_c])=[X_a,[X_b,X_c]]
\end{align}
然后我们展开计算并且对于作用前后的c和d取等并求和「tr的定义」然后就可以回到之前的结果。


\subsection{Root and Weights (Georgi)}
这里我们按照Georgi的思路讨论一下李代数的root和weight的定义。我们主要讨论单复李代数。我们回顾对于SU(2)的理解,我们给出表示的操作其实是这样的:
\begin{itemize}
  \item 选择一个Cartan子代数「对于SU(2)来说就是$ J_3 $」
  \item 选择这个Cartan子代数的一个本征基「对于SU(2)来说就是$ \ket{j,m} $」
  \item 复化并构造升降算符「对于SU(2)来说就是$ J^\pm $」
  \item 使用升降算符构造整个表示「对于SU(2)来说就是从$ \ket{j,j} $开始使用$ J^- $」
\end{itemize}
下面把这样的一个技术的概念推广到一般的单李代数上面。我们发现这个操作给出了两个好处:1. 对于一般的单复李代数的分类; 2. 对于一般的单李代数的表示构造。我们不想再讨论复化所以我们其实默认讨论的是单复李代数。

\subsubsection{Cartan子代数以及weight}

\hlr{Cartan子代数的定义}

对于一个单复李代数的任意的不可约表示$ D $我们总能找到一些generator(也就是李代数的一组基)他们之间互相对易,并且物理人喜欢用Hermite的算符所以还要求他们是hermite的「后面会发现hermite能给出很好的结构,比如root的正负关系」。所以我们定义:
\defi{Cartan子代数

  对于一个单复李代数$ \mathfrak{g} $的一个不可约表示$ D $,我们可以坐标变换然后各种组合搞出一组【最多的】generator $ H_i $ 满足:
  \begin{align}
    [H_i,H_j]=0\mathrm{~for~all~}i,j=1\mathrm{~to~}m\mathrm{~.}
  \end{align}
  并且这些$ H_i $是hermite的。我们把这样一组generator叫做Cartan子代数。
}
对于这个子代数有一些值得注意的地方:
\begin{enumerate}
  \item 注意,cartan子代数在某个表示上并不是“唯一的”。他们只是结构上是“共轭的”。但是在代数内部位置是不一样的!
  \item 我们称呼这个子代数的维度为李代数的秩「rank」。这是一个仅仅和李代数本身有关的数字
  \item Cartan子代数对于实李代数并不一定存在,比如数学上的SU(2)全是anti-Hermite的,所以我们永远搞不出一个hermite的Cartan子代数。所以我们这个讨论仅仅适用于单复李代数。
\end{enumerate}

\hlr{Cartan子代数的归一化}

对于这个子代数我们其实上面的条件并没有完全fix这些生成元的样子。因为我们可以对生成元进行rescale以及线性组合。这个并不利于我们的一些计算所以我们现在要rescale,并且线性组合一下这些生成元fix Cartan子代数的归一化是:
\begin{align}
  \mathrm{Tr}(H_iH_j)=k_D\delta_{ij}\mathrm{~for~}i,j=1\mathrm{~to~}m\mathrm{~.}
\end{align}
其中$ k_D $是一个和表示有关的常数。我们可以通过rescale H来fix这个常数。

\rmk{之前有讨论这样的rescale操作不是对于所有代数都能做的,仅仅对于compact lie algebra才行。因为compact lie algebra的killing form是正定的所以我们可以把killing form看成是一个内积然后进行Gram-Schmidt正交化。然后我们知道对于一般表示的tr其实是正比于killing form的,所以我们也可以进行正交化。}

\bigskip
\hlr{Weight的定义}

对于一个Cartan子代数由于互相对易,所以我们可以选择某个不可约表示空间$ D $一组共同的本征态来同时对角化这些$ H_i $。我们把这些本征态叫做weight state,并且我们把这些本征值的组合作为一个向量叫做weight:
\defi{Weight State and Weight

  对于一个单复李代数$ \mathfrak{g} $的一个不可约表示$ D $,我们选择一个Cartan子代数$ \{H_i\} $。那么我们可以找到一组共同本征态$ \ket{\mu} $满足:
  \begin{align}
    H_i\ket{\mu,D}=\mu_i\ket{\mu,D}\mathrm{~for~}i=1\mathrm{~to~}m\mathrm{~.}
  \end{align}
  我们把这样的本征态叫做weight state,并且把这些本征值的组合作为一个向量$ \mu=(\mu_1,\mu_2,\cdots,\mu_m) $叫做weight。(m是这个李代数的rank)
}  

这里我们显然能看到Hermite的一些好处:1. weight是实数;2. weight state可以归一化并且正交。


\subsubsection{Adjoint Representation and Root}

\hlr{Adjoint Representation的表示空间}

我们之前定义一个李代数的Adjoint Representation是,选定任意一个表示的某一组基$ X_a $然后定义:
\begin{align}
  [T_a]_{bc}=-if_{abc}\mathrm{~.}
\end{align}
显然这个定义在一个表示空间,并且表示空间的维度和李代数的维度是一样的。所以我们可以使用李代数本身的元素标记这个表示空间的基础,并且我们根据李代数的一些性质可以赋予这个表示空间一些特殊的数学结构。
\thm{Adjoint Representation的表示空间

  对于Adjoint Representation的表示空间我们使用李代数的元素来标记这些基,$ T^a \to \ket{T^a} $。这个基有下面的性质:
  \begin{enumerate}
  \item 这个基满足线性关系也就是:
    \begin{align}
      \ket{c_a T^a+ b_b T^{b}} = c_a \ket{T^a} + b_b \ket{T^b}\mathrm{~for~any~}c_a\in\mathbb{C}\mathrm{~.}
    \end{align}
    \item 这个基的元素的变换规则是:
      \begin{align}
        T^a\ket{T^b}=\ket{[T^a,T^b]}=-if_{abc}\ket{T^c}\mathrm{~.}
      \end{align}
    \item 这个基的内积可以定义为:
      \begin{align}
        \braket{T^a}{T^b}=\frac{1}{k_D}\mathrm{Tr}(T_aT_b)=\frac{1}{k_D}g_{ab}\mathrm{~.}
      \end{align}
      其中$ k_D $是之前定义Cartan子代数的时候的常数,但是现在我们选择的$ D $是Adjoint Representation【之后我们记作: $ \lambda  = k_D$】这个内积是正定的。
  \end{enumerate}
}
在这样的一组基下面我们可以把Adjoint Reprentation的矩阵元素写作这个样子:
\begin{align}
  X_a|X_b\rangle=|X_c\rangle\langle X_c|X_a|X_b\rangle=|X_c\rangle\left[T_a\right]_{cb}=-if_{acb}|X_c\rangle
\end{align}

\rmk{上面的赋予的结构是合理的这是一个定理。但是我们并没有予以证明。如果从基的观点其实不好证明。但是如果我们使用basis independent的观点看Adjoint Representation上面的定理就是完全显然的,甚至我们根本不需要定义这个内积结构,因为完全没有必要。}


\bigskip
\hlr{Adjoint Representation的Cartan子代数}

对于Adjoint Representation我们显然也可以选择一个Cartan子代数,然后选择一组共同本征态来对角化。同时这个Cartan子代数已经是normalize之后的样子。

我们记录Ajoint Representation的Cartan子代数是$ \{ H_i \} $并且我们不难发现这些Cartan子代数的元素对应的基满足下面的关系:
\begin{align}
  H_i|H_j\rangle=|[H_i,H_j]\rangle=0
\end{align}
并且内积有:
\begin{align}
  \langle H_i|H_j\rangle=\lambda^{-1}\operatorname{Tr}\left(H_iH_j\right)=\delta_{ij}
\end{align}


\bigskip
\hlr{Adjoint Representation下面对角化以及Root}

同样的我们选择一组共同本征态来对角化这些$ H_i $。我们把这些本征态我们记作$ |E_\alpha\rangle $,并且我们把这些本征值的组合作为一个向量叫做root:
\defi{Root State and Root

  对于一个单复李代数$ \mathfrak{g} $的Adjoint Representation,我们选择一个Cartan子代数$ \{H_i\} $。那么我们可以找到一组共同本征态$ |E_\alpha\rangle $满足:
  \begin{align}
    H_i|E_\alpha\rangle=\alpha_i|E_\alpha\rangle\mathrm{~for~}i=1\mathrm{~to~}m\mathrm{~.}
  \end{align}
  我们把这样的本征态叫做root state,并且把这些本征值的组合作为一个向量$ \alpha=(\alpha_1,\alpha_2,\cdots,\alpha_m) $叫做root。(m是这个李代数的rank)

  对于这些本征态,我们一般会选择是正交归一的,所以我们选择的归一化是:
  \begin{align}
    \langle E_\alpha|E_\beta\rangle=\lambda^{-1}\operatorname{Tr}\left(E_\alpha^\dagger E_\beta\right)=\delta_{\alpha\beta}\left(=\prod_i\delta_{\alpha_i\beta_i}\right)
  \end{align}
}


\bigskip
\hlr{从Adjoint Rep的基到李代数的基【Cartan-Weyl Basis】}

显然对角化操作之后,我们得到了一组新的李代数的adjoint representation基也就是$ \{ \ket{H_i}, \ket{E_\alpha} \} $,并且这些基对应着一组Lie Algebra本身的基:$ \{ H_i , E_\alpha\} $。【Cartan-Weyl Basis】

我们发现,在Adjoint Representation的构造其实帮助我们给出了一组李代数的基【与表示无关捏】并且给出了这组基根据Adjoint Representation的构造规则满足下面的对易关系。我们总结一下这些对易关系

\begin{enumerate}
  \item Cartan子代数之间的对易关系:
\begin{align}
  [H_i,H_j]=0\mathrm{~for~all~}i,j=1\mathrm{~to~}m\\
\end{align}
  \item Cartan子代数和root state之间的对易关系:
    \begin{align}
      [H_i,E_\alpha]=\alpha_iE_\alpha
    \end{align}
    \item Root State的共轭「这个时候Hermite的好处又体现出来喽」:
      \begin{align}
        \left[H_i,E_\alpha^\dagger\right]=-\alpha_iE_\alpha^\dagger, \quad E_\alpha^\dagger=E_{-\alpha}
      \end{align}
    \item 归一化关系。我们选择的归一化是:
      \begin{align}
        \lambda^{-1}\operatorname{Tr}\left(E_\alpha^\dagger E_\beta\right)=\delta_{\alpha\beta}
      \end{align}
    \item Root State之间的对易关系:
      \begin{align}
        [E_\alpha,E_\beta]&=N_{\alpha,\beta}E_{\alpha+\beta}\mathrm{~if~}\alpha+\beta\mathrm{~is~a~root}\\
        [E_\alpha,E_\beta]&=0\mathrm{~if~}\alpha+\beta\mathrm{~is~not~a~root~and~}\alpha\neq-\beta\\
        [E_\alpha,E_{-\alpha}]&=\alpha_i H_i
      \end{align}
\end{enumerate}


\subsubsection{Cartan-Weyl Basis对于一般表示}
\label{sec:cartan-weyl-general-rep}

\bigskip
\hlr{Root State in General Representation}

我们知道对易关系是和表示无关的。下面我们考虑Cartan-Weyl Basis的元素作用在一般表示空间上面的weight state的行为,我们发现:
\begin{align}
  H_iE_{\pm\alpha}|\mu,D\rangle=\left[H_i,E_{\pm\alpha}\right]|\mu,D\rangle+E_{\pm\alpha}H_i|\mu,D\rangle=(\mu\pm\alpha)_iE_{\pm\alpha}|\mu,D\rangle.
\end{align}
所以我们知道,H相当于weight state的weight进行作用,而$ E_{\pm\alpha} $相当于把weight state的weight进行平移$ \pm\alpha $。所以我们称之为升降算符。


\bigskip
\hlr{半单复李代数的 su(2) 子代数}

我们发现对于每一个root $ \alpha $,我们都可以构造出一个su(2)子代数,构造方法如下:
\begin{align}
  E^\pm\equiv|\alpha|^{-1}E_{\pm\alpha}, \quad E_3\equiv|\alpha|^{-2}\alpha\cdot H.
\end{align}

并且通过这个子代数我们可以帮助证明一个很重要的结论:
\thm{Root Vector Uniquely Determined

  对于一个单复李代数的任意一个root $ \alpha $,对应的【唯一一个】$ E_\alpha $李代数元素。
}


\thm{Root Vector 不可乘

  如果$ \alpha $是一个root那么除了$ -\alpha $之外所有的$ \alpha $的倍数都不可能是root!
}

\subsubsection{Root Vector的图像表达}
\label{sec:root-vector-geometry}

\bigskip
\hlr{Master Formula}

我们经过观察考虑一个表示$ D $的weight $ \mu $,然后我们使用一个root $ \alpha $对应的su(2)子代数进行分析。我们发现这个weight在这个su(2)子代数下面的表现是一个有限维不可约表示。所以我们会有:
\begin{itemize}
  \item 先不停的使用$ E_{-} $降低weight直到最低点,我们记这个最低点的weight为$ \mu - q\alpha $,并且满足:
    \begin{align}
      E_{-}|\mu - q\alpha,D\rangle=0\mathrm{~.}
    \end{align}
  \item 然后我们不停的使用$ E_{+} $提升weight直到最高点,我们记这个最高点的weight为$ \mu + p\alpha $,并且满足:
    \begin{align}
      E_{+}|\mu + p\alpha,D\rangle=0\mathrm{~.}
    \end{align}
\end{itemize}
我们知道如果考虑“Height Weight”以及“Lowest Weight”那么这个表示的话,他们作用一下$ E_3 $会给出:
\begin{align}
  &\frac{\alpha\cdot(\mu+p\alpha)}{\alpha^2}=\frac{\alpha\cdot\mu}{\alpha^2}+p=j.\\ 
  &\frac{\alpha\cdot(\mu-q\alpha)}{\alpha^2}=\frac{\alpha\cdot\mu}{\alpha^2}-q=-j.
\end{align}
我们消除$ j $然后得到一个很重要的公式:
\thm{Master Formula

  对于某个表示下面的任意一个weight $ \mu $以及任意一个root $ \alpha $满足下面的关系。并且p,q是非负整数:
\begin{align}
\frac{\alpha\cdot\mu}{\alpha^2}=-\frac{1}{2}(p-q).
\end{align}
}
  

\bigskip
\hlr{Root Vector之间的夹角几何关系}

我们会观察到一个有意思的结果,根据对于su(2)子代数的研究,如果我们恰好考虑两个root $ \alpha,\beta $,然后分别使用master formula进行分析,我们会发现:
\begin{align}
  &\frac{\alpha\cdot\beta}{\alpha^2}=-\frac{1}{2}(p-q).\\ 
  &\frac{\beta\cdot\alpha}{\beta^2}=-\frac{1}{2}(p^{\prime}-q^{\prime}).
\end{align}
我们把这两个式子相乘然后得到:
\thm{
  Root Vector之间的夹角关系

  对于任意两个root $ \alpha,\beta $,他们之间的夹角满足下面的关系:
\begin{align}
  \cos^2\theta_{\alpha\beta}=\frac{(\alpha\cdot\beta)^2}{\alpha^2\beta^2}=\frac{(p-q)(p^{\prime}-q^{\prime})}{4}.
\end{align}
其中的$ p,q $都是非负的整数。
}
所以我们知道这个夹角只能是一些特定的数值。下面给出一些例子:
\begin{figure}[H]
  \centering
  \includegraphics[width=0.5\textwidth]{assets/angleroot.png}
  \caption{Root Vector之间的夹角只能是一些特定的数值}
  \label{fig:angleroot}
\end{figure}
所以我们可以图像化的画出来root vector的图像。




\subsection{Examples: SU(3) Lie Algebra (Georgi)}

\hlr{su(3)李代数定义}

我们定义SU(3)群是所有$ 3 \times 3 $ Unitary矩阵构成的$ det = 1 $的群。数学上我们从切空间的定义可以知道su(3)李代数可以有的数学定义;当然我们因为只关注群表示,代数表示就是生成群表示的工具,所以物理人使用Hermite生成元来定义。

\rmk{和su(2)一样的,我们物理人因为只关心群表示所以会把代数给modify成为一个数学上并不严谨的Hermite的东西。}
下面给出定义:
\defi{
  su(3) Lie Algebra

  \begin{itemize}
    \item 数学定义:
      \begin{align}
        su(3)=\{X\in M_{3\times3}(\mathbb{C})|X^\dagger=-X,\mathrm{Tr}(X)=0\}\mathrm{~.}
      \end{align}
    \item 物理定义,我们定义为所有的$ 3 \times 3 $ Hermite Traceless 矩阵构成的实向量空间。对于这个空间我们可以选择一组特别合理的基,也就是Gell-Mann矩阵:
      \begin{align}
        su(3)=\{T^a=\frac{1}{2}\lambda^a|a=1\mathrm{~to~}8\}\mathrm{~.} 
      \end{align}
  \end{itemize}
}

\bigskip
\hlr{Gell-Mann矩阵}

Gell-Mann矩阵前三个元素基本上就是SU(2)的Pauli矩阵的扩展,然后后面五个矩阵是为了补全这个空间。并且这个基下面定义的表示空间的Trace是张这样的:
\begin{align}
  \mathrm{Tr}(T_aT_b)=\frac{1}{2}\delta_{ab}\mathrm{~.}
\end{align}


\bigskip
\hlr{Cartan 子代数以及Root Vector}

下面我们考虑Gell-Mann矩阵所在表示空间的root structure。
\begin{enumerate}
  \item  我们发现这个空间上用Gell-Mann矩阵表示之中正好有两个矩阵$ T_3, T_8 $是对角的还构成了一个Cartan子代数。所以我们重命名为$ H_1, H_2 $.
  \item 我们对角化这两个矩阵,然后共同本征向量对应的本征值也就是这个表示的weight $ \mu_1 = (1/2, \sqrt{3}/6),  \mu_1 = (-1/2, \sqrt{3}/6), \mu_1 = (0, -\sqrt{3}/3) $
  \item 我们知道三个weight的差距就是root vector的数值所以我们可以给出root,并且类比给出升降算符就是root state。
\end{enumerate}

\begin{figure}[H]
  \centering
  \includegraphics[width=0.5\textwidth]{assets/rootvecsu3.png}
  \caption{SU(3)的root vector}
  \label{fig:rootvecsu3}
\end{figure}

这给出了一个系统的求解一个代数root system的方法。
\begin{itemize}
  \item \textbf{step1: 给出表示 }选择一个表示空间,找到这个表示空间下面的Cartan子代数
  \item \textbf{step2: 确定weight  }对角化这个Cartan子代数,找到weight。
  \item \textbf{step3: 寻找root }分析哪些weight vector的差值是root!!这是最核心的一步,操作方法是:
    \begin{itemize}
      \item 首先我们计算root应该有多少个,也就是代数的生成元个数减去cartan子代数的个个数。李代数的dimension - rank就是root的个数。
      \item 然后考虑adj rep也就是各个generator的对易关系看看能不能找到root state,然后确认哪些才是root vector!
    \end{itemize}
\end{itemize}

这个方法的好处是,有的时候很好算。并且可以通过任意表示出发计算得到,但是问题也很明显。我们很难看出某个weight的差值是不是root,如果weight的差值特别多的话。

\YL{[问一下有没有什么更好的方法!!但感觉没有呃呃呃]}


\subsection{SU(3) Lie Algebra (Ramond)}

这本书使用了和Georgi不太一样的convention。主要体现在对于$ E_\alpha $和$ H_i $的归一化的定义上面,所以root vector长度和对易关系可能差一些killing form的乘法。

\YL{回头补充吧!目前觉得并不特别重要}




\subsection{Questions and thoughts}



\question{我们使用的所有$ H,E $什么的到底是李代数还是表示?怎么做Hermite Conjugation?}

我们这里使用的这些算符【全部都是表示】!我们写出来就是“在某个表示空间上”的意思。

因为李代数可以被矩阵进行faithful表示,所以我们不妨写在一个矩阵空间上面,然后理解为在这个表示空间上的各种各样的矩阵。

\qed

\question{对于单李代数,我们的Cartan子空间一定可以找到正交的基吗?也就是满足:}
\begin{align}
  \mathrm{Tr}\left(H_iH_j\right)=k_D\delta_{ij}\mathrm{~for~}i,j=1\mathrm{~to~}m
\end{align}

当然可以!因为我们可以把Killing form看成是一个m维空间的"双线性形"【对于compact lie algebra就是一个内积】,然后我们使用Gram-Schmidt正交化就可以了!

\qed


\bigskip
\question{我们定义内积的$ \lambda $的系数到底是怎么定义的??为什么能消掉那么多的数字???(6.18);(6.10)能保证同时消去吗??}

可以!!其实我们干的就是rescale H和rescale所有的E的结果!
\qed


\bigskip
\question{我们知道所有的simple lie algebra都可以通过root给出一堆su(2)子代数,那么能不能说所有的simple lie algebra是su(2)的直和呢?这样是不是就不是simple了?}

注意!确实我们可以把simple lie algebra的root system分解成一堆su(2)的root system的直和!但是我们不能说simple lie algebra是su(2)的直和!因为我们不能说这些su(2)子代数之间是互相对易的,这些su(2)子代数并不是“互相独立的”而是交错纠缠的。所以这些simple lie algebra也是simple的!
\qed


\bigskip
\question{我们这里研究的su(3)是一个复代数还是实数?}

其实如果我们要研究根系的话必须研究$ sl(3,\mathbb{C}) $。因为只有复代数才有根系结构。georgi的书之中我们自动偷偷进行一个复化了!在这里:
\begin{align}
  \begin{aligned}&\frac{1}{\sqrt{2}}\left(T_1\pm iT_2\right)=E_{\pm1,0}\\&\frac{1}{\sqrt{2}}\left(T_4\pm iT_5\right)=E_{\pm1/2,\pm\sqrt{3}/2}\\&\frac{1}{\sqrt{2}}\left(T_6\pm iT_7\right)=E_{\mp1/2,\pm\sqrt{3}/2}\end{aligned}
\end{align}

\qed

% section Week 5 Reading (end)

\newpage 
\section{Extra 2: Simple Lie Algebra and Representation}\label{sec:Extra 2: Mathematical Approach to Root and Weight} % (fold)

这里我们希望引入上下指标并且允许Killing form是非对角矩阵。同时我们允许使用Killing form来提升和降低指标。并且我们把adjoint rep就定义成李代数空间上的表示。在这样的数学架构下,我们研究root and Weight的结构,以及半单李代数的分类。

这里我们follow 大黄书的Simple Lie Algebras章节的内容。




% section Extra 2: Mathematical Approach to Root and Weight (end)

\newpage
\section{Week 6-7 Reading: Simple Root}\label{sec:Week 6-7 Reading: Simple Roote} % (fold)
上一章之中,我们通过给定了一个李代数发现可以通过一套系统的方法找到所有【simple, complex】李代数内部蕴含的一个结构,也就是root system。我们使用了下面几个步骤:
\begin{itemize}
  \item 通过对易关系找到Cartan子代数(记得归一化)
  \item 通过Cartan子代数在Adjoint representation下的表示找到其共同本征态,也就是root state(记得归一化);同时找到本征值也就是root vector $ \alpha $。 
  \item root vector一一对应了李代数之中的一个generator其结构可以画在以Cartan子代数为坐标轴的空间之中,形成root diagram。
\end{itemize}
但是我们显然可以提出两个问题
\begin{itemize}
  \item root vector并非互相独立,我们能不能找出一个极小的object包含其全部信息。
  \item 我们能不能通过一个操作,通过这个包含全部信息的object重新generate出整个李代数。
\end{itemize}
本章我们主要回答这两个问题。预告一下结论:
\begin{itemize}
  \item 所有root的信息可以通过一个矩阵Cartan Matrix表达。
  \item 通过Cartan Matrix我们可以generate出整个李代数。
\end{itemize}

\subsection{从Simple Root到Cartan Matrix}

\subsubsection{Preliminary知识}

\bigskip
\hlr{回顾root system的重要性质}

我们回顾之前【根据每一个root vector对应的su(2)子代数】对于root system提出了两个重要的性质要求,以及一个特殊定理。我们列举在下面,具体参考\cref{sec:cartan-weyl-general-rep,sec:root-vector-geometry}
\begin{itemize}
  \item \textbf{性质0: }对于单复李代数每一个root,可以给出一个su(2)子代数,构建方法是:
    \begin{align}
      E^{\pm}\equiv|\alpha|^{-1}E_{\pm\alpha},\quad E_{3}\equiv|\alpha|^{-2}\alpha\cdot H.
    \end{align}
  \item \textbf{性质1:} 单复李代数的root vector一一对应一个generator 
  \item \textbf{性质2:} 对于一个root vector $ \alpha $, $ k \alpha $如果也是root vector,那么$ k=\pm 1 $。
  \item \textbf{Master Formula:} 对于任意两个root vector $ \alpha,\beta $,定义$ \beta $在$ \alpha $方向上的投影数:
    \begin{align}
      \frac{2(\alpha\cdot\beta)}{\alpha^2} = q-p\in\mathbb{Z}\mathrm{~,}
    \end{align}
    并且这个整数代表了在$ su(2)_\alpha $子代数的表示之中$ E_\beta $的weight。也就是$ q-p $,其中$ q $代表距离最低weight的步数,$ p $代表距离最高weight的步数。
\end{itemize}
我们下面会不断反复使用这几个概念和定理。



\bigskip
\hlr{Positive Weight的概念}

根据性质2,我们知道root vector只有一半是独立的,我们不妨定义一组positive的root vector,只要他们 $ times  -1$也就可以generate出另外一半的root vector。我们定义:

\defi{
  Positive Weight

  对于某一个李代数的表示$ D $,我们给出一个固定的Cartan子代数basis $ \{H_i\} $【注意,这组基我们还需要固定前后顺序】。在这个基础上,我们定义一个weight $ \mu = (\mu_1,\mu_2,\ldots,\mu_r) $是positive的如果其第一个非零分量是正的。

}

有了positive的概念之后我们就可以定义两个weight之间比大小了!我们定义:
\defi{
  Comparing Weights

  如果两个weight $ \mu $和$ \nu $满足$ \mu - \nu $是positive的,那么我们说$ \mu > \nu $。
}

positive root还告诉我们一件事情,也就是对于性质0之中提到的su(2)子代数,我们可以定义raising and lowering opertor
\begin{itemize}
  \item 如果$ \alpha $是positive root,那么$ E_\alpha $是raising operator,$ E_{-\alpha} $是lowering operator。
\end{itemize}

\subsubsection{Simple Root的定义与性质}

\bigskip
\hlr{Simple Root的定义}

我们依旧发现positive root之中也会存在线性相关。我们可以再找到其中线性无关的部分我们定义为simple root。我们定义:
\defi{
  Simple Root

  对于某一个李代数的root system,我们定义一组positive root $ \{\alpha_i\} $是simple root如果其不能被其他positive root线性表示出来。
}

下面我们研究Simple Root的一些性质,根据这些性质我们最终发现一条路径可以通过simple root generate出整个李代数。


\rmk{
  为了方便我们使用这样的记号,对于Simple Root
\begin{enumerate}
  \item 我们使用$ \alpha^i $来标记第i个root vector【谨记确定了i指标$ \alpha^i $是一个vector】
  \item 同时使用$ E_{\alpha^i} $来标记对应的generator。 我们使用$ \ket{\alpha^i} $来标记这个generator对应的root state。
  \item 对应的su(2)子代数我们使用$ E^{\pm}_i, E_{3 i} $来标记,这个代数记作$ su(2)_i $。任意weight state对于这个子代数的p,q我们记作
\end{enumerate}
}

\bigskip
\hlr{Simple Root的重要性质}

\lmm{
  Simple Root仅可加和

 对于任意两个simple root $ \alpha, \beta $,我们知道$ \alpha - \beta $以及$ \beta - \alpha $都不是root vector 。
}
这意味着我们从simple root组合出来所有的root【仅仅能用相加】。

\lmm{Simple Root的p,q计算

  对于任意两个simple root $ \alpha_i, \alpha_j, i \neq j $可以使用master formula计算出对于$ \alpha_j $的$ p_i, q_i $:
    \begin{align}
     &\alpha_j : \quad q_i = 0, \quad p_i = \frac{2(\alpha_i\cdot\alpha_j)}{\alpha_i^2} \\ 
     &\alpha_i: \quad q_j = 0, \quad p_j = \frac{2(\alpha_i\cdot\alpha_j)}{\alpha_j^2}
    \end{align}
    同时我们也可以通过性质2知道:
    \begin{align}
      &\alpha_i: \quad q_i = \displaystyle\frac{2(\alpha_i \cdot \alpha_i)}{\alpha_i^2} = 2, \quad p_i = 0 \\
      &\alpha_j: \quad q_j =\displaystyle\frac{2(\alpha_j \cdot \alpha_j)}{\alpha_j^2} =  2, \quad p_j = 0
    \end{align}
}
这个性质还有一个推论
\begin{itemize}
  \item 对于任意两个simple root $ \alpha_i, \alpha_j $,我们的夹角和相对长度可以通过内积计算。我们给出记号对于$ su(2)_i $来说,$ \alpha_j $对应的$ p = p_i $;对于$ su(2)_j $来说,$ \alpha_i $对应的$ p' = p_j $。
\end{itemize}

\lmm{
  Simple Root之间的夹角和相对长度限制

  我们发现Simple Root的夹角可以知道:
  \begin{align}
    \cos\theta_{ij} = \frac{(\alpha_i \cdot \alpha_j)}{|\alpha_i||\alpha_j|} = - \displaystyle\frac{1}{2} \sqrt{p p'},  \quad \displaystyle\frac{\alpha_j^2}{\alpha_i^2}  = p/p' \mathrm{~,}
  \end{align}
  所以所有Simple Root的夹角在$ [\pi/2, \pi] $之间,并且只能选取:
  \begin{align}
    \theta_{ij} = \left\{\pi/2, 2\pi/3, 3\pi/4, 5\pi/6 \right\}\mathrm{~,}
  \end{align}
}

下面我们研究simple root的数量以及线性无关的性质,我们发现:
\lmm{
Simple Root作为Cartan子代数空间的Basis【仅仅适用于Simple Lie Algebra】

  Simple Root的数量等于李代数的rank也就是Cartan子代数的数量。

  Simple Root 构成了Cartan子代数空间的一组Basis!!
}

作为基我们显然可以把所有Root 用Simple Root进行展开,这个展开我们发现下面的定理:

\lmm{
  Positive Root可以唯一的使用Simple Root线性展开,【并且展开系数是正整数】
  \begin{align}
    \phi = \sum_i k^i \alpha_i, \quad k^i \in \mathbb{Z}^{+}\mathrm{~,}
  \end{align}
}

\bigskip
\hlr{Root State的两种表示【Dynkin Coefficient表示】}

我们考虑不再使用$ \phi $这样的root vector来表达root state。我们可以选择另外的两种表示方法:

\begin{itemize}
  \item \textbf{表示方法1: 使用$ k_i $计数表示}
\end{itemize}
这个表示很自然。我们考虑一个root state $ \ket{\phi} $,我们知道这个root state可以唯一的使用simple root进行线性展开:
\begin{align}
  \phi = \sum_i k^i \alpha_i, \quad k^i \in \mathbb{Z}^{+}\mathrm{~,}
\end{align}
因此我们可以使用$ \{k^i\} $这个数组唯一的表示root state。

其数学意义是:这个root state是通过堆叠多少个simple root得到的。

这个表达方式等价于我们使用 $ \alpha_i $ 作为一组basis。

\begin{itemize}
  \item \textbf{表示方法2: 使用$ q_i-p_i $数组表示「Dynkin Coefficient」}
\end{itemize}
我们其实还有组数组可以用来表示这个root state并且拥有更concrete的意义。我们一步步构造:
\begin{enumerate}
  \item 对于root state $ \{ \ket{\phi}\} $,我们考虑其在$ su(2)_i $子代数下$ 2E_{3i} $的weight。根据master formula我们知道这个weight是:
    \begin{align}
      \frac{2(\alpha_i \cdot \phi)}{\alpha_i^2} = q_i - p_i\mathrm{~,}
    \end{align} 
\item 由于Simple root构成了一组完备基础,我们可以$ \{q_i - p_i\} $这个数组唯一的表示root state $ \phi $。
\end{enumerate}

其数学意义是:标记了root在每一个simple root 对应的su(2)子代数下的weight。

后面我们知道,这个等价于使用fundamental weight basis。作为基进行的表示。

\rmk{
  我们注意Dynkin Coefficient对于于任意weight state都可以定义,并且都是整数!!这是因为master formula是对于任何表示都是适用的!!
}

\rmk{
  一个特别容易犯的错误是,注意虽然$ q_i-p_i $是内积$ \alpha_i^\vee $和$ \phi $的结果,但是$ q_i - p_i $并不是$ \phi $在$ \alpha_i^\vee $ basis下的展开系数!!!因为这个不是正交归一basis!
}


\subsubsection{Cartan Matrix表达Simple Root}



\bigskip
\hlr{Cartan Matrix表达Simple Root}

我们之前讨论了Simple Root的两种表达形式。我们不妨使用这两种表达形式表达一下这个系统的Simple Root。我们发现:
\begin{itemize}
  \item 第一种表达方式:对于一个simple root $ \alpha_i = \delta^j_i \alpha_j $所以$ k^j = \delta^j_i $。 
  \item 第二种表达方式:对于一个Simple root $ \alpha_i $根据Master Formula:
    \begin{align}
      q_j - p_j = \frac{2(\alpha_j \cdot \alpha_i)}{\alpha_j^2} \mathrm{~,}
    \end{align}
\end{itemize}
使用第二种表达方式下的Simple Root我们发现可以把所有Simple Root的表示写作一个矩阵形式。也就是Cartan Matrix:
\defi{
  Cartan Matrix

  对于一个Simple Lie Algebra的所有Simple Root $ \{\alpha_i\} $,我们定义Cartan Matrix $ A $为:
  \begin{align}
    A_{ij} = \frac{2(\alpha_i \cdot \alpha_j)}{\alpha_j^2} \mathrm{~,}
  \end{align}
  【注意我们是把j指标写在分母上面!】
}
我们讨论怎么从这个矩阵之中读出代数的信息。首先我们分析这个矩阵元素的特质意味着什么:
\begin{itemize}
  \item Cartan Matrix的对角线元素都是2。说明其实$ E_{3 j} \ket{\alpha_j} = \ket{\alpha_j} $。同时我们知道对于自己,$ p_i = 0 $因为$ \alpha_i+\alpha_i $不是root,所以我们知道,所有simple root在自己对应的su(2)子代数下的一个spin-1 表示的highest weight state。
  \item Cartan Matrix的非对角线元素都是非正整数0,-1,-2,-3。这其实对应simple root之间的夹角只能是$ \pi/2, 2\pi/3, 3\pi/4, 5\pi/6 $并且比例需要是满足一定要求的。
  \item Cartan矩阵本身是一个可逆矩阵!
\end{itemize}
下面我们讨论矩阵每一行的意义:
\begin{itemize}
  \item Cartan Matrix每一行代表对于$ \ket{\alpha_i} $这个root state的$ \{ q_j - p_j\} $也就是对于$ su(2)_j $的$ 2E_{3j} $的weight。
\end{itemize}

\bigskip
\hlr{Cartan矩阵用于系数变换}

Cartan矩阵可以作为两种表示方式的转换工具。我们发现:
   \begin{align}
     A_{ij} = (\alpha_i \cdot \alpha_j^\vee) \mathrm{~,}
   \end{align} 
   所以对于任意的Weight State,我们之前讨论的两种表示方式可以通过Cartan Matrix进行转换:
   \begin{align}
      (q_j - p_j) = A_{ij} k^i \mathrm{~,}
    \end{align}




\bigskip
\hlr{Dynkin Diagram图像化表达}

Simple Root的一个角度性质就是:Simple Root之间的夹角只能取有限个值。我们可以使用Dynkin Diagram来表示Simple Root之间的关系。我们定义:
\defi{Dynkin Diagram

  对于一个Simple Lie Algebra的所有Simple Root,我们使用一个圆圈来代表Simple Root,并且使用连线来代表Simple Root之间的夹角关系。规则如下:

\begin{figure}[H]
  \centering
  \includegraphics[width=0.4\textwidth]{assets/dynkindiag.png}
  \caption{Dynkin Diagram的绘制规则}
  \label{fig:dynkindiag}
\end{figure}
}
Dynkin Diagram可以表达夹角结构,但是并不能够表达Simple Root的长度关系。并不能完全还原Simple Root system。还需要补充一个模长具体长度的信息。

\subsection{从Cartan Matrix重构李代数}

我们找到了一个包含全部Simple Root信息的Cartan Matrix。那么我们能不能通过这个矩阵重构出整个李代数呢?答案是肯定的。下面我们展示这个过程。

\subsubsection{Cartan矩阵重构Root System}

重构的第一步,我们需要找出所有的生成元,也就是找到Root System。我们这里是用G2 Lie Algebra作为例子进行说明,其Cartan Matrix是:
\begin{align}
 C(G2) = \begin{pmatrix}2&-1\\-3&2\end{pmatrix}
\end{align}

\bigskip
\hlr{从Cartan 矩阵重构Simple Root}

这一步特别简单,我们发现Cartan矩阵每一行就是对应了$ \{ \alpha_i^\vee \} $basis下的Simple Root表示。所以我们直接把Cartan矩阵的每一行拿出来标记simple root:
\begin{align}
  \alpha_i \to \left( A_{i1}, A_{i2}, \ldots, A_{ir} \right)\mathrm{~,}
\end{align}

对于G2李代数我们发现有两个simple root,分别用$ q_i - p_i $进行标记结果是:

\begin{figure}[H]
  \centering
  \includegraphics[width=0.5\textwidth]{assets/G2simpleroot.png}
  \caption{G2李代数的Simple Root表示}
  \label{fig:g2simpleroot}
\end{figure}

\bigskip
\hlr{从Simple Root重构所有root system}

下面我们利用Simple Root重构整个Root System。我们基本思路就是从Simple Root向上一点点堆叠simple root。并且每一步要验证还不能不能再向上堆:
\begin{itemize}
  \item \textbf{Step 1 列出level 1的simple root: } 

    我们使用下面的图示列出level 1的simple root:
\begin{figure}[H]
  \centering
  \includegraphics[width=0.5\textwidth]{assets/G2simpleroot.png}
  \caption{G2李代数的Simple Root表示}
  \label{fig:g2simpleroot}
\end{figure}
\item \textbf{Step 2 通过分析给出这个level 1每一个root的q值} 

  我们考虑G2李代数level 1的q值:
  \begin{itemize}
    \item 对于$ \alpha_1 $,我们知道$ \alpha_1 +\alpha_1 $不是一个root,所以$ p_1 = 0 $,因此$ q_1 = (q_1-p_1)+0 = 2 $;同时我们知道$ \alpha_1 - \alpha_2 $不是root,所以$ q_2 = 0 $: 对于$ \alpha_1 $我们得到$ (q_1,q_2) = (2,0) $。
    \item 对于$ \alpha_2 $,我们知道$ \alpha_2 +\alpha_2 $不是一个root,所以$ p_2 = 0 $,因此$ q_2 = (q_2-p_2)+0 = 2 $;同时我们知道$ \alpha_2 - \alpha_1 $不是root,所以$ q_1 = 0 $: 对于$ \alpha_2 $我们得到$ (q_1,q_2) = (0,2) $。
  \end{itemize}
\item \textbf{Step 3 通过q值计算出来p值}

  我们现在已经知道这两个simple root的$ q_i - p_i $以及$ q_i $,所以我们可以计算出$ p_i $:
  \begin{align}
    &\alpha_1: \quad p_1 = 2 - 2 = 0, \quad p_2 = 0 - (-1) = 1, \quad (p_1,p_2) = (0,1) \\
    &\alpha_2: \quad p_1 = 0 - (-3) = 3, \quad p_2 = 2 - 2 = 0, \quad (p_1,p_2) = (3,0)
  \end{align}
\item \textbf{Step 4 通过p值判断下一个level有哪些root}

  我们回顾p值的物理意义,也就是$ (E^+_i)^{p_i+1} \ket{\phi} = 0 $也就是还能再用这个raising operator向上升多少次。
  \begin{itemize}
    \item 对于$ \alpha_1 $,我们知道$ (p_1,p_2) = (0,1) $,所以只能使用$ E^+_2 $向上升1次。我们得到下一个root state是:
      \begin{align}
        \alpha_1 + \alpha_2 
      \end{align}
      \item 对于$ \alpha_2 $,我们知道$ p = (3,0) $,所以只能使用$ E^+_1 $向上升3次。我们得到下一个root state是:
      \begin{align}
        \alpha_1 + \alpha_2 
      \end{align}
  \end{itemize}
\item \textbf{Step 5 计算下一个level的q-p值}

  我们知道有公式:
  \begin{align}
    (q_j-p_j)=A_{ij}k^i,
  \end{align}
  所以,上面分析给出了$ k^i $所以我们可以自然的给出q-p。其实就相当于叠加上一个simple root的q-p。
  \begin{align}
    &\alpha_1 + \alpha_2: \quad (q_1 - p_1, q_2 - p_2) = (2,-1) + (-3,2) = (-1,1) 
  \end{align}

  这两个simple root向上只能生成同样的一个root,画在图上就是:
\begin{figure}[H]
  \centering
  \includegraphics[width=0.5\textwidth]{assets/level2.png}
  \caption{G2李代数level 2 root的生成}
  \label{fig:level2}
\end{figure}

\end{itemize}
下面我们同样的方法进行分析level 2的root的q的数值,通过p = q - (q - p)计算出p值,然后通过p值决定能不能继续向上。直到\textbf{某一个level所有的root的p值都是0},说明不能再向上升了。我们把所有的root都找出来了!

最终找到的G2李代数的root system如下图所示:
\begin{figure}[H]
  \centering
  \includegraphics[width=0.5\textwidth]{assets/g2fullroot.png}
  \caption{G2李代数的完整Root System}
  \label{fig:g2fullroot}
\end{figure}


\bigskip
\hlr{使用su(2)子代数辅助光速重构root system}

上面的过程是一个完备的过程,但是一个小技巧会加速这个过程。我们观察root的q-p取值,并且分析q值。我们可以意识到。

$ \alpha_2 $的q-p是$ (-3,2) $,同时q是$ (0,2) $。q第一个是0 说明是在$ su(2)_1 $子代数这是lowest weight state。$ q_1-p_1 = -3$ 说明这个state是一个spin-3/2表示的最低weight state。

因此我们自然可以向上作用三个$ E^+_1 $,直接得到三个state。

同理也可以这样的分析其他root state。这样就可以大大加速root system的重构过程!!



\subsubsection{从Root System重构李代数}


\bigskip
\hlr{从root system给出生成元对易关系}

通过我们之前分析我们知道,对于任何一个root state我们都是一个simple root的$ su(2)_i $子代数的一个weight state。也就是说:
\begin{align}
  E^+_{i} \ket{\alpha_j} = \sqrt{(j+m+1)(j-m)/2} \ket{\alpha_j + \alpha_i} 
\end{align}
其中$ j = (q_i + p_i)/2, m = (q_i - p_i)/2 $。
同时我们知道有关系:
\begin{align}
  E_{i}^+ \ket{\alpha_j} = |\alpha_i|^{-1} E_{\alpha_i} \ket{\alpha_j} = \ket{[E_{\alpha_i}, E_{\alpha_j}]} \mathrm{~,}
\end{align}
我们对比这两个式子可以发现正好给出了$ [E_{\alpha_i},E_{\alpha_j}] $的对易关系的表达式。我们可以使用这个方法计算出positive root之间很复杂的对易关系。

这个时候如果我们再在左边作用上negative root。然后重复使用Jacobi Identity,我们就可以计算出所有的对易关系!!最终我们就重构出了整个李代数的对易关系!!

\bigskip
\hlr{G2代数的例子}

我们依旧使用G2代数作为例子。我们已经重构出了G2代数的root system。我们现在使用上面的方法计算出所有的对易关系。

这个代数的两个simple root分别对应的su(2)子代数的上升算符是:
\begin{align}
  E_1^+=E_{\alpha^1}\quad E_2^+=\frac{1}{\sqrt{3}}E_{\alpha^2}
\end{align}
我们计算产生第一个level的对易关系是:
\begin{align}
  \begin{aligned}
  |[E_{\alpha^1},E_{\alpha^2}]\rangle=&\sqrt{\frac{3}{2}}\left|3/2,-1/2,1\right\rangle\\
  = &\sqrt{\frac{3}{2}}\left|E_{\alpha^1+\alpha^2}\right\rangle
  \end{aligned}
\end{align}
第二个level的对易关系是:
\begin{align}
  \begin{aligned}
  |[E_{\alpha^1},[E_{\alpha^1},E_{\alpha^2}]]\rangle =&2\sqrt{\frac{3}{2}}|3/2,1/2,1\rangle\\ 
  = &\sqrt{6}\left|E_{2\alpha^1+\alpha^2}\right)
  \end{aligned}
\end{align}
然后我们可以得到很多类似的对易关系,完整的写出来是:
\begin{align}
  &E_{\alpha^1+\alpha^2}=\sqrt{\frac{2}{3}}\left[E_{\alpha^1},E_{\alpha^2}\right]\\&E_{2\alpha^1+\alpha^2}=\sqrt{\frac{1}{6}}\left[E_{\alpha^1},[E_{\alpha^1},E_{\alpha^2}]\right]\\&E_{3\alpha^1+\alpha^2}=\frac{1}{3}\left[E_{\alpha^1},[E_{\alpha^1},[E_{\alpha^1},E_{\alpha^2}]]\right]\\&E_{3\alpha^1+2\alpha^2}=\frac{\sqrt{6}}{9}\left[E_{\alpha^2},[E_{\alpha^1},[E_{\alpha^1},[E_{\alpha^1},E_{\alpha^2}]]]\right]
\end{align}
对于这些对易关系,我们可以通过和negative root对易然后一直使用Jacobi Identity来计算出所有的对易关系!!最终我们就重构出了整个G2李代数的对易关系!!

比如:
\begin{align}
  [E_{-\alpha^1},E_{\alpha^1+\alpha^2}]=\sqrt{\frac{2}{3}}\left[E_{-\alpha^1},[E_{\alpha^1},E_{\alpha^2}]\right]
\end{align}







\subsection{从Root System结构重构表示}

我们前面讨论了从Cartan Matrix重构李代数的方法。但是其实上,重构的就是所有的root system也就可以理解为李代数的Adjoint Representation。其实,类似的方法我们可以重构出李代数的任意表示。下面我们进行general的讨论。

\subsubsection{fundamental Weight and Representation}

\bigskip
\hlr{Highest Weight State}

对于Adjoint Rep也就是root system因为Cartan子代数和Simple Root对应的态已知,所以我们由其进行向上构建。但是对于一般的表示,我们完全不知道这个表示里面的这些态。我们不妨先假设一个极端的态然后朝着下构建。我们定义:

\defi{Highest Weight State

对于一个有着Simple Root $ \alpha_i, i \in \{1,..., m\} $的李代数。我们定义,某一个表示$ D $之中的highest weight state,是Cartan子代数的本征态满足:
\begin{align}
  E_{\alpha^j}|\mu\rangle=0\mathrm{~}\forall j
\end{align}
}


\bigskip
\hlr{使用Dynkin Coefficient标记Weight State}

为了方便构造表示,我们会需要一个类似于$ (j,m) $的记号来标记weight state。Weight太难看出位置和结构了!观察会发现,对于Weight State来说$ (j,m) $其实就是我们之前的$ \{ q_i - p_i\} $。也就是Dynkin Coefficient。所以我们使用这个记号来标记weight state。

\rmk{
  我们区分一下Weight和Dynkin Coefficient用来表示Weight State的区别:
  \begin{itemize}
    \item Weight: 是root向量空间用Cartan子代数为基展开的系数。对应着每个Cartan子代数的本征值。
    \item Dynkin Coefficient: 对应着每个su(2)子代数的$ 2E_{3i} $的本征值。是内积$ \alpha_i^\vee $的结果。永远是整数!!!!
  \end{itemize}
}

\bigskip
\hlr{构建表示基本思路}

我们现在给出构建表示的基本思路,下一章会用su(3)具体进行讨论。
\begin{itemize}
  \item \textbf{Step 1: 确定highest weight state}

    我们首先假设一个highest weight state $ \ket{\mu} $,并且使用Dynkin Coefficient $ \{ q_i - p_i \} $进行标记。
  \item \textbf{Step 2: 计算p值} 

    对于Highest Weight State我们知道$ p_i = 0 $,所以我们可以计算出$ q_i $。得知这个态能被哪个su(2)的下降算符下降几次。

  \item \textbf{Step 3: 构建下一个level的weight state}

    我们使用$ E_{-\alpha_i} $作用在highest weight state上面,构建出下一个level的weight state。【其实就是减去能下降的$ \alpha_i $的Dynkin Coefficient】

    \textbf{!!注意!!要注意归一化!!!呀呀呀!!} 
\end{itemize}
如此重复就可以构建出整个表示的weight state!!得到表示空间自然就可以构建表示了!


\subsubsection{fundamental weight and fundamental representation}

对于一个表示,我们知道对于一个Weight State来说有两种表示方法:
\begin{itemize}
  \item 使用weight vector本身进行表示
  \item 使用Dynkin Coefficient进行表示
\end{itemize}
那么我们自然会问,使用weight vector的时候我们用的是Cartan Subalgebra的basis。那么使用Dynkin Coefficient的时候我们用的是什么basis呢?答案是fundamental weight basis!!

\bigskip
\hlr{Fundamental Weight的定义}

\defi{
  Fundamental Weight

  对于一个Simple Lie Algebra的Simple Root $ \alpha_i, i \in \{1,...,r\} $,我们定义其对应的fundamental weight $ \mu_i $满足:
  \begin{align}
    \frac{2(\alpha_j \cdot \mu_i)}{\alpha_j^2} = \delta_{ij} \mathrm{~,}
  \end{align}
}


\rmk{
  我们注意是用的fundamental weight basis而不是$ \alpha_i^\vee $的basis,给出系数是点乘$ \alpha_i^\vee $的结果。出现这个情况是因为$ \alpha_i^\vee $不是正交归一basis!!
}

\bigskip
\hlr{Fundamental Weight其他}

我们根据定义很自然发现,任意的一个weight在fundamental weight下面可以进行展开得到:
\begin{align}
  \mu=\sum_{j=1}^m\ell^j\mu^j
\end{align}
其中系数$ \ell^j $可以通过内积计算出来:
\begin{align}
  \ell^j=\frac{2(\alpha_j\cdot\mu)}{\alpha_j^2}=q_j - p_j
\end{align}
也就是说,Dynkin Coefficient实际上就是fundamental weight basis下的展开系数!!

我们定义fundamental representation为highest weight state是fundamental weight的表示。

\bigskip
\hlr{Dynkin Coefficient的意义}

我们终于可以总结一下,Dynkin Coefficient作为对于任何root和weight state的表示方法的意义了。
\begin{itemize}
  \item Dynkin Coefficient $ \{ q_i - p_i \} $表示了任意Weight State的Weight在fundamental weight basis下展开的系数。我们根据其定义知道,以及master formula可以推导出:【其必须是整数】
  \item Dynkin Coefficient描述了一个Weight State对于所有Simple Root对应的su(2)子代数的 $ 2 E_{3i} $。也就描述了这些Weight State作为这些su(2)子代数的表示的位置。
    \item Dynkin Coefficient暗含了的p,q的意义是:
      \begin{itemize}
        \item p 表示了这个Weight State在$ su(2)_i $子代数下距离highest weight state还有多少步。
        \item q 表示了这个Weight State在$ su(2)_i $子代数下距离最低weight state还有多少步。
      \end{itemize}

      \item Simple root的Dynkin Coefficient就是Cartan Matrix!
      \item \textbf{Dynkin Coefficient的可加性:}我们知道Dynkin Coefficient是fundamental weight basis下面的weight vector的展开。所以,对weight vector的线性组合也是dynkin coefficient的线性组合!!
\end{itemize}


\subsection{Questions and thoughts}

\question{对于su(3)代数,为什么我们在使用su(2)子代数的表示重构代数的时候可以直接认为$ \ket{1/2,1/2} $态是$ \ket{E_{\alpha_1+\alpha_2}} $呢}

因为我们在定义$ \ket{E_\alpha} $这个态的时候已经确定了归一化需要满足:
\begin{align}
  \langle E_\alpha|E_\beta\rangle=\lambda^{-1}\operatorname{Tr}\left(E_\alpha^\dagger E_\beta\right)=\delta_{\alpha\beta}\left(=\prod_i\delta_{\alpha_i\beta_i}\right)
\end{align}
只要我们follow这个归一化,那么我们这个归一化好之后的态必须对应$ \ket{1/2,1/2} $这样的态。
\qed


\bigskip
\question{为什么fundamental weight必然是一个highest weight of fundamental representation ??P112}

可以理解为Georgi在梦里写的话,梦到哪句写哪句(((
\qed

\bigskip
\question{给定一个表示下面的李代数我们可以通过weight的差的方式给出所有的root,但是肯定会更多!!所以?怎么才能找到哪些是真正的root???}

目前感觉只有在Adjoint rep下面尝试进行对角化这一种操作手段捏。
\qed

\bigskip
\imp{李代数的直和以及李群的直积}{
  我们有的时候会搞混直积和直和的概念。我们下面给出解释

  \YL{[回头补充吧!基本上就是李群的直积对应李代数的直和]}
}

% section Week 6-7 Reading: Simple Roote (end)

\newpage
\section{Week 8: Representation of Lie Algebra}\label{sec:Week 8: Representation of SU(3)} % (fold)
\subsection{例子:Fundamental Repesentation of su(3)}

在研究一般表示之前我们以su(3)的fundamental representation作为例子来具体的说明求表示的方法。
\rmk{
  为了方便标记我们使用$ \alpha_i = (\ ,\ ) $来表示root vector以及weight vector的坐标。我们使用$ \alpha_i = [\ ,\ ] $来表示其Dynkin Coefficient。
}

\bigskip
\hlr{su(3)基础知识}

我们回顾一下su(3)的simple root以及Cartan Matrix。
\begin{align}
  \alpha^1=(1/2,\sqrt{3}/2), \quad \alpha^2=(1/2,-\sqrt{3}/2)
\end{align}
其Dynkin Coefficient是:
\begin{align}
  \alpha_1 = [2,-1],\quad \alpha_2=[-1,2]
\end{align}
于是我们可以计算出Fundamental Weights:
\begin{align}
  \mu^1=(1/2,\sqrt3/6),\quad \mu^2=(1/2,-\sqrt3/6)
\end{align}

\bigskip
\hlr{Fundamental Representation of su(3)}

所以我们清楚这个理论有两个fundamental representation。分别以$ \mu^1 $以及$ \mu^2 $作为highest weight。我们先研究$ \mu^1 $对应的表示。

我们计算其Dynkin Coefficient:
\begin{align}
  \mu^1 = [1,0]
\end{align}
由于对于highest weight来说$ p = [0,0] $所以我们计算出来:$ q = p+ \mu_i $所以$ q = [1,0] $也就是说第一个指标可以下降一次,使用$ E^-_{\alpha_1} $。下降得到:
\begin{align}
  \mu_1 - \alpha_1 = [-1,1] 
\end{align}
我们使用这种方法继续下降,直到不能下降为止。我们得到下面的diagram:
\begin{figure}[H]
  \centering
  \includegraphics[width=0.3\textwidth]{assets/10rep.png}
  \caption{SU(3)的fundamental representation (1,0)的weight diagram}
  \label{fig:10rep}
\end{figure}
我们称之为$ (1,0) $表示。

类似的,我们可以得到$ \mu_2 $作为highest weight对应的$ (0,1) $表示,其weight diagram如下:

\begin{figure}[H]
  \centering
  \includegraphics[width=0.3\textwidth]{assets/01rep.png}
  \caption{SU(3)的fundamental representation (0,1)的weight diagram}
  \label{fig:01rep}
\end{figure}
\YL{[请参考Georgi的教材查看这两个表示更具体的信息。]}

\subsection{表示空间结构与Weight}

我们这种从highest weight出发构建表示的方法对于一般表示适用,并且我们可以更好的理解表示空间的结构。

\bigskip
\hlr{Weight与State互相对应}

\begin{itemize}
  \item \textbf{Highest Weight唯一性:}

    对于不可约表示来说highest weight是唯一的。否则我们可以通过两个highest weight得到两个直和子表示。
\end{itemize}

\begin{itemize}
  \item \textbf{Weight \& State对应:}如果一个weight存在那么至少有一个state与之对应但不一定唯一。
    \begin{itemize}
      \item \textbf{不同Weight对应正交的态}

        显然他们是orthogonal的,因为他们的$ H_i $本征值不一样。而Cartan子代数都是Hermite的。
      \item \textbf{某一个Weight对应态}

        并不一定只有一个,我们对于同一个weight子空间的维度称为multiplicity。我们知道,每一个给到一个weight的路径给出了一个state。所以multiplicity最多等于路径的数量。
\begin{itemize}
  \item 如果一个weight的构造路径是唯一的:那么其对应的态是唯一的。multiplicity = 1
    \item 如果一个weight的构造路径不是唯一的:那么其对应的态的multiplicity可能大于1.
\end{itemize}
    \end{itemize}
\end{itemize}

\bigskip
\hlr{判断两个State是不是线性相关}

对于一个weight如果有两个不同的路径构造出来,那么我们可以通过计算norm来判断这两个路径构造的态是不是线性相关,我们只要计算:

\begin{align}
  \langle A\mid B\rangle\cdot\langle B\mid A\rangle=\langle A\mid A\rangle\cdot\langle B\mid B\rangle
\end{align}
我们可以通过反复使用对易关系给normal order来计算内积结果。

\subsection{Weyl Group}

这一部分Georgi完全再说梦话,请当作没看见过。


\subsection{Conjugate Representation}

\bigskip
\hlr{Conjugate Representation的定义}

对于一个具体的表示$ D $,我们定义其conjugate representation $ \overline{D} $。我们发现对于这个表示之中的生成元进行一个操作之后并不改变对易关系,也就是:
\begin{align}
  \left[-T_a^*,-T_b^*\right]=if_{abc}(-T_c^*)
\end{align}
所以我们定义变换后的生成元$ \overline{T_a} = -T_a^* $,那么这些生成元就构成了一个新的表示,称为conjugate representation。

对于表示和conjugate表示之间存在下面的关系:
\begin{itemize}
  \item 如果$ \mu $是一个表示中间的weight,那么$ -\mu $是其conjugate表示中的weight。
  \item 表示$ D $的highest weight $ \mu_{hw} $,那么其conjugate表示的lowest weight是$ -\mu_{lw} $
    \item Dynkin Coefficient的关系:如果$ D $的Dynkin Coefficient是$ (p,q) $,那么其conjugate表示的Dynkin Coefficient是$ (q,p) $
\end{itemize}


\subsection{Questions and thoughts}

\question{为什么su(3)的fundamental rep的highest weight state是和root正交的那些?}

这个是定义捏。我们回顾对于Fundamental Representation定义为,设置所有的fundamental weights $ \mu_i $是highest weight然后构建出来的representation!!

\qed 

\question{
为什么我们从一般Highest weight出发构建表示只能作用negative root对应的lowering operator?
}
书中说是因为如果有一个Positive root对对应的raising,会对易到最右边然后是0了。这个说法特别不严谨,会引起混淆。

正确的表达应该是,如果有一个positive root对应的raising operatoe存在在下面的series之中:
\begin{align}
  E_{\phi_1}E_{\phi_2}\cdots E_{\phi_n}|\mu\rangle
\end{align}
那么我们可以通过对易的方法,将其变为negative root lowering作用的state的线性组合!!
\qed


\question{
  到底怎么使用Weyl Group判断一个weight是不是uniquely对应一个state也就是其multiplicity为1?
}

书中完全没提,但是Georgi可能脑抽了觉得自己讲了。

在大黄书的510页存在这样的一个定理:
\thm{
  如果一个weight $ \lambda $的Multiplicity是1,那么对于任意的Weyl Group元素$ w \in W $,$ w(\lambda) $的Multiplicity也是1。
}
证明思路,也就是Weyl Group是李代数的自同构。所以作用在表示空间也应该是不改变空间的结构的。应该是不改变multiplicity的。

Georgi的脑子里应该就是想这个定理,只是他忘写了((


\bigskip
\question{Weyl Reflection作用在下面这种图上面张什么样子捏?}

\begin{figure}[H]
  \centering
  \includegraphics[width=0.5\textwidth]{assets/weightdiag.png}
  \caption{SU(3)的(2,0)表示的diagram}
  \label{fig:weightdiag}
\end{figure}
在正文之中,我将给出一些表示Weyl Group的orbit用来参考!!


\bigskip
\question{脑子懵了!张量积表示是怎么定义的?}

已经知道两个表示,其张量积表示定义为:
\begin{itemize}
  \item \textbf{表示空间:}即为原表示空间的张量积
  \item \textbf{李代数作用在表示空间:}张量积表示,李代数会分别作用在两个表示空间,然后取和。也就是我们的表示矩阵是:
    \begin{align}
D(X) = D_1(X) \otimes I + I \otimes D_2(X)
    \end{align}
\end{itemize}
有几个东西我们需要区分:
\begin{itemize}
  \item 某一个代数的表示的张量积表示
  \item 两个不同代数的直和代数的表示!!
\end{itemize}
对于张量积表示来说,我们可以知道一个state的weight,根据上面的定义就是两个张量积起来的weight的和。使用Dynkin Coefficient表示的话就是两个Dynkin Coefficient的逐项相加。

\bigskip
\question{对于一般highest weight的方法构建的表示的归一化是怎么选取的?}

\YL{[书中完全没有讨论,回头再看看吧!书中focus weight state的结构!]}

% section Week 8: Representation of SU(3) (end)

\newpage 
\section{Week 9: Tensor Method}\label{sec:Week 9: Tensor Method} % (fold)
\subsection{Tensor Methods for SU(3)}

\YL{[本章内容如果参考Georgi会学废,并且很多东西细思极恐,因此不做整理]}


\subsection{Questions and Thoughts}

% section Week 9: Tensor Method (end)

\newpage
\section{Scratch book}\label{sec:Scratch book} % (fold)
\input{sb.tex}
% section Scratch book (end)

\end{document}
