本补充章节我们准备讨论李代数的形式化表达,以及李群李代数和流形之间的关系。同时更重要讨论各种【李代数之间的关系,以及复化!】

\subsection{Lie Group和Lie Algebra定义}

在Georgi的书之中,我们先定义了李群是所有的能够用实数$ \mathbb{R}^n $进行标定的连续群。并且给出了李群把表示的概念,并且说明李群可以被矩阵群faithfully的表示。

然后,我们说明李群的任意表示至少局域的可以(对于simple connected and compact lie group全局的可以,由于georgi仅仅考虑simple connected and compact李群所以他其实并没有强调这个适用性)用explonential map进行标记。

通过选择一个物理上常用的explonential map,我们可以通过李群的表示写成下面的形式:
\begin{align}
  D(g) = e^{i \alpha_a T^a}
\end{align}
其中$ T^a $是李群的表示的生成元「就是一堆矩阵」。然后我们发现了根据群的乘法结构,这些矩阵构成了一个有Lie bracket的代数结构的实线性空间,我们把这些矩阵构成的实线性空间称为这个李群对应的李代数。

但李群和李代数的定义其实可以并不依赖于表示抽象的给出,并且这样的抽象的定义对于推广和理解表示有更多的好处。我们可以直接定义李群和李代数:

\YL{请补充,没写完}

\subsection{su(2)李代数数学,物理定义}

\hlr{数学上su(2)李代数的定义}

首先我们定义什么是SU(2)群:
\defi{
  SU(2)群

  SU(2)群是所有行列式为1的2阶酉矩阵构成的群。
}
对于这个群,我们知道是一个李群。很自然能够推导出它的李代数:
\defi{
  su(2)李代数

  su(2)李代数是所有满足$ X^\dagger+X=0 $并且$ \mathrm{tr}(X)=0 $的2阶矩阵构成的李代数。是一个【实向量空间】
  \begin{align}
    \mathfrak{su}(2)=\{\left.X\in M_2(\mathbb{C})\mid X^\dagger=-X,\mathrm{~Tr}(X)=0\right.\}.
  \end{align}
  这个代数空间我们可以很自然的选择一组基是:
  \begin{align}
    X_i=\frac{i}{2}\sigma_i,\quad[X_i,X_j]=\epsilon_{ijk}X_k.
  \end{align}
}  
所以我们知道su(2)李代数可以理解为所有的反厄米矩阵并且迹为0的2阶矩阵构成的李代数。这是一个实向量空间。对于切空间,我们可以指数映射到群上面。这个指数映射的形式是:
\begin{align}
  U=e^{tX}
\end{align}

\bigskip
\hlr{物理上su(2)李代数的定义}
\bigskip

\hlr{!!!!!!但是!!!!!!}问题是,这样子基都是anti-Hermite的矩阵,但是物理上我们习惯使用Hermite矩阵作为代数元素!!所以物理人定义了一个trick,我们偷偷修改了指数映射的形式是:
\begin{align}
  U=e^{i t H}
\end{align}
然后顺手把基也改成了Hermite矩阵:
\begin{align}
  J_i=\frac{1}{2}\sigma_i = -i X_i,\quad[J_i,J_j]=i\epsilon_{ijk}J_k.
\end{align}
这样子,我们也“如”consistent的得到了一个李代数,并且这个李代数按照modify之后的指数映射映射到SU(2)群上面。并且\textbf{我们动了指数映射,动了基的形式,但是没有动数域!!}所以写作Hermite的形式李代数的元素依旧是对于三个基的实数线性组合捏。

\rmk{但是这么搞总是有bug的

比如:我们写出来$ [J_i,J_j] = i \epsilon_{ijk} J_k $的时候,也就是说仿佛对易出来了一个复数线性组合的生成元,不在代数里面了呃呃呃。并且对易子本身就是anti-Hermite的东西,我们强行让对易子等于一个Hermite的空间之中的物体是完全不可能的。

我们对此的解决方式是:相当于「重新定义」了对易子,对于Hermite的李代数元素使用-i[,]作为李括号。这样子就没有问题了。
}

\bigskip
\hlr{为什么物理学家可以这么乱搞}

我觉得一个原因是,物理上相比于代数表示更重要的是群的表示。因为我们物理的坐标变换是由群来描述的。所以,虽然我们的代数结构没有流形和切空间的interpretation这么严谨,也不是特别简洁,但是只要我们能够通过这种方式得到正确的群表示就行了。


\subsection{复化以及半单李代数表示}

\bigskip
\hlr{su(2)李代数的复化}

当然,我们也可以把su(2)李代数复化成一个复向量空间。这个复化的过程就是把系数域从实数域扩展到复数域。也就是说我们允许基的线性组合系数是复数而不是实数。而不是像是之前强行变Hermite一样,那个不叫复化。我们现在是允许所有复数域上面的线性组合。

对于复化之后的代数【由于物理的定义太不严谨了,我们一般复化是用数学家的定义】我们称之为$ sl(2,\mathbb{C}) $下面给出定义:
\defi{$ sl(2,\mathbb{C}) $定义

就是所有的二维traceless复矩阵构成的李代数:
\begin{align}
  \mathfrak{sl}(2,\mathbb{C})\equiv\{X\in M_2(\mathbb{C})\mid\mathrm{Tr}(X)=0\}
\end{align}
}

\bigskip
\hlr{复化以及su(2)表示}

复化最重要的作用之一就是帮我们求表示。有的时候代数的表示并不好求,我们就先求复化后的代数元素的复线性组合的表示。然后我们再把这个线性组合回去,就可以得到原来代数的表示。

一个最重要的例子就是su(2)的表示,我们不好直接求,那么就先求复化后的产生湮灭算符的表示:
\begin{align}
  J_\pm = J_1 \pm i J_2
\end{align}
这个表示很好求,我们求出来之后再把$ J_1,J_2 $表示回去就可以了。

\bigskip
\hlr{复化以及Cartan Weyl Basis}

我们在代数之中讨论su(2)的Cartan子代数是$ J_3 $,然后升降算符是$ J_\pm $。但是我们发现这个升降算符其实并不是代数里面的元素,因为它是复线性组合的。也就是说我们只能在复化之后的代数里面讨论Cartan Weyl Basis:
\begin{align}
  H=2J_3,\quad E_+& =J_+,\quad E_-=J_-\\ 
  [H,E_\pm]=\pm2E_\pm&,\quad[E_+,E_-]=H.
\end{align}
我们使用这个basis求的其实是$ sl(2,\mathbb{C}) $的表示,然后再把表示回去就可以了。不仅仅是su(2)李代数,其他的李代数我们也可以复化然后使用Cartan Weyl Basis进行表示的求解。所以复化给了我们generally 一个求表示的好方法。

同时,我们也需要记住,cartan Weyl Basis是复化之后的代数里面的东西。我们只能在复化之后的代数里面讨论root和weight。所以我们说到$ A_i, D_i $这些代数的时候,我们说的是复化之后的代数。只是,所有的单李代数都可以形成这个复李代数实化后的结果。


\subsection{实化以及不同李代数关系}


\bigskip
\hlr{复李代数的实化}

我们可以复化,当然也可以再进行实化。对于$ sl(2,\mathbb{C}) $李代数我们可以进行实化得到两个不同的实李代数:
\begin{align}
  \mathfrak{su}(2) = \{a_1 i \sigma_1 + a_2 i \sigma_2 + a_3 i \sigma_3 \mid a_i \in \mathbb{R}\}\\
  \mathfrak{sl}(2,\mathbb{R}) = \{b_1 \sigma_1 + b_2 i \sigma_2 + b_3 \sigma_3 \mid b_i \in \mathbb{R}\}
\end{align}
这两个实李代数都是$ sl(2,\mathbb{C}) $的实化。并且这两个实李代数是完全不一样的。一个是compact的,一个是non-compact的。  
su(2)相当于是把$ sl(2,\mathbb{C}) $挑选出其中所有anti-Hermite的元素构成的实李代数;而$ sl(2,\mathbb{R}) $是挑选出所有实矩阵构成的实李代数。


