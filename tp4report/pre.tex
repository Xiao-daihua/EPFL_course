\subsection{Lightning Review of BCFT} 

I will first give a very brief review of fundamental concepts and tools in Boundary Conformal Field Theory (BCFT). For a more detailed introduction, one can refer to my CFT and BCFT note \cite{Xiao-daihua_Phys_Learning_CFT_Primary_BCFT_2025, Xiao-daihua_Phys_Learning_CFT_Minimal_CFT_2025}, click \href{https://github.com/Xiao-daihua/Phys_Learning/blob/main/CFT/Primary_BCFT/main.pdf}{here}, and the following references \cite{cardyBoundaryConditionsFusion1989, northeYoungResearchersSchool2025, recknagelBoundaryConformalField}, these three references are found to be most helpful for my understanding of BCFT.

\subsubsection{Gluing Condition}

A Boundary CFT (BCFT) is a conformal field theory on a manifold with boundary induced from a parent CFT. It inherets the operator algebra data of the parent CFT, while having extra data due to the presence of boundaries. The simplest case is to consider a BCFT on the upper half plane (UHP) $\{z=x+iy|y\geq 0\}$, where the real axis $y=0$ serves as the boundary. 

To preserve conformal symmetry at the boundary, we need to impose the so-called gluing condition on the energy-momentum tensor $T(z)$ and $\bar{T}(\bar{z})$ at the boundary:
\defi{
  \textbf{Gluing Condition}

  At the boundary $y=0$, the energy-momentum tensor satisfies:
  \begin{align}
    T(z) = \bar{T}(\bar{z}) \quad \text{at } y=0.
  \end{align}
}
With this gluing condition, we can prove that the conformal symmetry is "partially" preserved, which leads to the existence of a single Virasoro algebra instead of two independent ones:
\thm{
  \textbf{Symmetry Algebra of BCFT}

  In a BCFT defined on the upper half plane with the gluing condition $T(z) = \bar{T}(\bar{z})$ at the boundary $y=0$, the symmetry algebra is generated by a single Virasoro algebra with generators:
  \begin{align}
    L_n^H = \frac{1}{2\pi i} \oint_{C} dz \, z^{n+1} T^H(z) 
  \end{align}
  where $ T^H(z) $ is the analytical continuation of $ T(z) $ and $ \bar{T}(\bar{z}) $. Satisfying:
  \begin{align}
    [L_n^H, L_m^H] = (n-m) L_{n+m}^H + \frac{c}{12} n(n^2-1) \delta_{n+m,0}
  \end{align}
}

\subsubsection{Boundary States Formalism}

There exist various ways of defining a BCFT from a parent CFT on a manifold by imposing different types of boundary conditions at the boundary. A tool of labeling and classifying these boundary conditions is the boundary state formalism. We define the boundary states as;
\defi{
  \textbf{Boundary State}

  Boundary states $\ket{\alpha}$ and $\ket{\beta}$ are defined as states living on
  the boundary of the parent CFT placed on a chopped cylinder, and they satisfy:
  \begin{itemize}
    \item The cylinder has the same geometry as a compactified BCFT on a strip,
          with length $L$ and circumference $\beta$.
    \item The partition function of the parent CFT on a cylinder with boundary
          states $\ket{\alpha}$ and $\ket{\beta}$ at its two ends is equivalent to
          the partition function of the BCFT on a strip with boundary conditions
          $\alpha$ and $\beta$ at the two ends. Mathematically:
  \end{itemize}
  \begin{align}
    Z_{\alpha\beta} = \bra{\alpha}\,
           \tilde{q}^{\frac{1}{2}(L_0 + \bar{L}_0 - c/12)}\,
         \ket{\beta} = \operatorname{Tr}_{\mathcal{H}_{\alpha\beta}}
      \left( q^{\,L_0^{H} - c/24} \right),
  \end{align}
  where $q = e^{-\pi \beta / L}$ and $\tilde{q} = e^{-4\pi L / \beta}$.Here $L_0$ and $\bar{L}_0$ are the Virasoro generators of the parent CFT, while $L_0^{H}$ is the Virasoro generator in the BCFT.
}
We can use the boundary states as a state in the Hilbert space of the parent CFT to encode the boundary conditions of the BCFT. To be consistent with the gluing condition, the boundary states must satisfy the following Ishibashi condition:
\thm{
  \textbf{Ishibashi Condition}

  Boundary states $\ket{\alpha}$ should satisfy:
  \begin{align}
    (L_n - \bar{L}_{-n}) \ket{\alpha} = 0, \quad \forall n \in \mathbb{Z}.
  \end{align}
}
A solution to the Ishibashi condition can be constructed from each primary state $\ket{h}$ of the parent CFT, leading to the definition of Ishibashi states:
\defi{
  \textbf{Ishibashi State}

  For each primary state $\ket{h}$ of the parent CFT, the corresponding Ishibashi state $\ket{h}\rangle$ is defined as:
  \begin{align}
    \ket{h}\rangle = \sum_{N} \ket{h;N} \otimes \overline{\ket{h;N}},
  \end{align}
  where $\{\ket{h;N}\}$ is an orthonormal basis of the Verma module built on the chiral primary state $\ket{h}$. 
}
We can prove that the Ishibashi States satisfy the Ishibashi Conditions.


\subsubsection{Cardy's Condition}

Consider the Parent CFT as a diagonal CFT. Indeed the main topic of the note Ising CFT is a diagonal CFT. There exist another consistency condition for the boundary state:
\thm{
  \textbf{Cardy's Condition}

  For two boundary states $ \ket{\alpha} $ and $ \beta $ they should satisfy that:
  \begin{align}
    \sum_h\langle\alpha|h\rangle\rangle\langle\langle h|\beta\rangle S_h^{h^{\prime}}=n_{\alpha\beta}^{h^{\prime}} \quad \text{or equivalently} \quad  \langle a|h^{\prime}\rangle\rangle\langle\langle h^{\prime}|b\rangle=\sum_hS_h^{h^{\prime}}n_{ab}^h.
  \end{align}
  where $ S^{h'}_h $ is the modular S matrix of the parent CFT and $ n_{\alpha\beta}^{h^{\prime}} $ are non-negative integers representing the multiplicities of the representation with highest weight $ h' $ in the open channel partition function $ Z_{\alpha\beta} $.
}
Proof: see Cardy's original paper \cite{cardyBoundaryConditionsFusion1989}. 

If we assume that the boundary states are linear combination of Ishibashi States, we can also write the Cardy's condition a relation of expansion coefficients:
\thm{\label{thm:cardycondition}
  \textbf{Cardy's Condition (coefficient constrain)}

  Consider the boundary states written as:
  \begin{align}
    \left|a\right\rangle=\sum_{i}B_{i}^{a}\left|i\right\rangle\rangle
  \end{align}
  The Cardy's condition says that:
  \begin{align}
    \sum_{ij} B_i^{a *}B_i^bS_{i}^j\chi_j(q)=\sum_in_{ab}^i\chi_i(q)\quad \Rightarrow \quad \sum_{i} B_i^{a *}B_i^bS_{i}^j=n_{ab}^j
  \end{align}
  where $ \chi_i $ is the character of the Virasoro representation. 
}

A solution to the Cardy's condition is given by comparing it with the Verlinder formula. These solutions are called the Cardy States:
\defi{
  \textbf{Cardy States}

  Cardy states are linear combination of Ishibashi states with coefficients given by:
  \begin{align}
    |a\rangle=\sum_{j}\frac{S_{a}^{j}}{(S_{0}^{j})^{1/2}}|j\rangle\rangle
  \end{align}
}
Proof: taking $ n_{ab}^i $ as the fusion matrix $ N^a_{bc} $ and Cardy states into Cardy's condition, we can see that the Cardy's condition \cref{thm:cardycondition} is exactly the Verlinder formula.


\subsubsection{Elementary Boundary States}

Looking at the Cardy's condition we might see that, if a state $ \ket{a} $ satisfy the condition then $ \mathbb{N}_+\ket{a} $ must also satisfy this condition, for we only insist $ n^i_{ab} $ to be non-negative integers. 

However, we usually only consider the so-called elementary boundary states that can not be decomposed into a sum of other boundary states. Thus we have the definition:
\defi{
  \textbf{Elementary Boundary States}

  For a parent CFT, a induced BCFT have following elementary boundary states $ \{\ket{a_i}\} $ that shall satisfy the following normalization and orthogonality condition: 
  \begin{align}
    n_{a_ia_j}^0 = \delta_{ij}
  \end{align}
}
The normalization is physical, this is because for a field theory, we assume to have a unique vaccum state, thus if $ n_{ab}^0 \geq 2 $ we will have multiple vacuum states in the open channel, which is unphysical, and shall be interpret as putting two identical theory togethor. Our goal is to find all elementary boundary states of a BCFT induced from a parent CFT.

Moreover, we can see that 
\thm{
 Cardy states are elementary boundary states. 
}
This is because, cardy's state is a solution when considering $ n_{ab}^i = N_{ab}^i $ which is the bulk fusion matrix. And we know that according to the 2 point function of bulk primary fields, the vacuum representation only appear once in the fusion of two primary fields when they are the same, written in operator algebra as:
\begin{align}
  \phi_a \times \phi_b = \mathbb{1} \delta_{ab} + \cdots
\end{align}

\subsection{Ising CFT}

Ising CFT as the simplist unitary minimal model $ M(3/4) $ with central charge $ c=1/2 $. Here I will briefly review some basic data of Ising CFT that will be used in the following sections. For a more detailed introduction to Ising CFT, one can refer to \cite{difrancescoConformalFieldTheory1997}.

\subsubsection{Ising CFT Data}
Ising CFT is the simplest A-series unitary minimal model $M(3,4)$ with central charge $c=\tfrac{1}{2}$. 
It has three primary fields:
\begin{itemize}
  \item Identity operator $\mathbb{1}$ with conformal weights 
    $h_{\mathbb{1}}=\bar{h}_{\mathbb{1}}=0$, 
  \item Spin operator $\sigma$ with conformal weights 
    $h_{\sigma}=\bar{h}_{\sigma}= \tfrac{1}{16}$
  \item Energy operator $\epsilon$ with conformal weights 
    $h_{\epsilon}=\bar{h}_{\epsilon}=\tfrac{1}{2}$.
\end{itemize}
They satisfy the folowing operator algebra (fusion rules):
\begin{align}
  &\sigma \times \sigma = \mathbb{1} + \epsilon, \\
  &\sigma \times \epsilon = \sigma, \\
  &\epsilon \times \epsilon = \mathbb{1}.
\end{align}

\subsubsection{Virasoro Characters of Ising CFT}

Charaters of a Virasoro Representation is defined as :
\defi{
  \textbf{Character}

The character $\chi_h(q)$ of a Virasoro representation with highest weight $h$ is defined as:
\begin{align}
  \chi_h(q) = \operatorname{Tr}_{\mathcal{H}_h} \left( q^{L_0 - c/24} \right),
\end{align}
}
For Ising CFT the characters of the three primary fields are given by:
\begin{align}
  &\chi_0(q)=\frac{1}{2}q^{-\frac{1}{48}}\left(\prod_{n=0}^\infty\left(1+q^{\frac{1}{2}+n}\right)+\prod_{n=0}^\infty\left(1-q^{\frac{1}{2}+n}\right)\right),\\
&\chi_{1/2}(q)=\frac{1}{2}q^{-\frac{1}{48}}\left(\prod_{n=0}^\infty\left(1+q^{\frac{1}{2}+n}\right)-\prod_{n=0}^\infty\left(1-q^{\frac{1}{2}+n}\right)\right),\\
&\chi_{1/16}(q)= q^{\frac{1}{24}}\prod_{n=1}^\infty\left(1+q^n\right)
\end{align}


\subsubsection{Modular S Matrix and Transformation of Characters}
The modular S Matrix of Ising CFT is given by:
\begin{align}
  S^i_j=\begin{pmatrix}
    \frac{1}{2} & \frac{1}{2} & \frac{1}{\sqrt{2}} \\
    \frac{1}{2} & \frac{1}{2} & -\frac{1}{\sqrt{2}} \\
    \frac{1}{\sqrt{2}} & -\frac{1}{\sqrt{2}} & 0
  \end{pmatrix}
\end{align}
with the column and row order $ \mathbb{1},\epsilon,\sigma $. The characters form a representation of the modular group with the transformation:
\begin{align}
  \chi_i\left(q\right)=\sum_j S_{i}^j\chi_j\left(\tilde{q}\right).
\end{align}
if we take $ q = e^{-\frac{\pi \beta}{L}} $ and $ \tilde{q} = e^{-\frac{4\pi L}{\beta}} $, we have:
\begin{align}
  &\chi_0(\tilde{q})=\frac{1}{2}\left(\chi_0(q)+\chi_{1/2}(q)\right)+\frac{1}{\sqrt{2}}\chi_{1/16}(q),\\
  &\chi_{1/2}(\tilde{q})=\frac{1}{2}\left(\chi_0(q)+\chi_{1/2}(q)\right)-\frac{1}{\sqrt{2}}\chi_{1/16}(q),\\
  &\chi_{1/16}(\tilde{q})=\frac{1}{\sqrt{2}}\left(\chi_0(q)-\chi_{1/2}(q)\right).
\end{align}
as well as the inverse transformation:
\begin{align}
  &\chi_0(q)=\frac{1}{2}\left(\chi_0(\tilde{q})+\chi_{1/2}(\tilde{q})\right)+\frac{1}{\sqrt{2}}\chi_{1/16}(\tilde{q}),\\
  &\chi_{1/2}(q)=\frac{1}{2}\left(\chi_0(\tilde{q})+\chi_{1/2}(\tilde{q})\right)-\frac{1}{\sqrt{2}}\chi_{1/16}(\tilde{q}),\\
  &\chi_{1/16}(q)=\frac{1}{\sqrt{2}}\left(\chi_0(\tilde{q})-\chi_{1/2}(\tilde{q})\right).
\end{align}
\rmk{
  For Ising CFT the modular S matrix is self-inverse, i.e. $ S = S^{-1} $.
}

\subsection{Ising BCFT}
For a BCFT induced from Ising CFT, we can construct its boundary states by solving the Ishibashi condition and the Cardy's condition.

\subsubsection{Ishibashi States of Ising BCFT}

The Ishibashi states of Ising CFT can be constructed from its three primary states:
\defi{
  \textbf{Ising Ishibashi States}

  Ising CFT has three Ishibashi states:
  \begin{align}
    &|0\rangle\rangle=\sum_{N}|0;N\rangle\otimes\overline{|0;N\rangle},\\
&|1/2\rangle\rangle=\sum_{N}|1/2;N\rangle\otimes\overline{|1/2;N\rangle},\\
&|1/16\rangle\rangle=\sum_{N}|1/16;N\rangle\otimes\overline{|1/16;N\rangle}.
  \end{align}
}

\subsubsection{Cardy States of Ising BCFT}

 Using the modular S matrix of Ising CFT, we can write down the Cardy states by definition:
\defi{
  \textbf{Ising Cardy States}

  Ising BCFT exist three Cardy states:
  \begin{align}
    &\left|+\right\rangle=\frac{1}{\sqrt{2}}|0\rangle\rangle+\frac{1}{\sqrt{2}}|1/2\rangle\rangle+\frac{1}{\sqrt[4]{2}}|1/16\rangle\rangle \\
&\left|-\right\rangle=\frac{1}{\sqrt{2}}|0\rangle\rangle+\frac{1}{\sqrt{2}}|1/2\rangle\rangle-\frac{1}{\sqrt[4]{2}}|1/16\rangle\rangle \\
& \left|f\right\rangle =|0\rangle\rangle- |1/2\rangle\rangle.
  \end{align}
}
In Cardy's original paper \cite{cardyBoundaryConditionsFusion1989}, these three boundary states are argued to correspond to the three physical boundary conditions of the Ising model\footnote{this is an argument made by considering the transformation of the boundary states under $ \mathbb{Z}_2 $ symmetry of Ising CFT. It is not a rigorous theorem.}:
\begin{itemize}
  \item $|+\rangle$: fixed boundary condition with spins fixed to $+1$ at the boundary.
  \item $|-\rangle$: fixed boundary condition with spins fixed to $-1$ at the boundary.
  \item $|f\rangle$: free boundary condition with spins free to fluctuate at the boundary.
\end{itemize}

\subsubsection{Partition Functions of Ising BCFT}

With the Cardy states, we can compute the partition functions of Ising BCFT on a compatified strip (cylinder) with different boundary conditions:
\begin{table}[H]
\centering
\renewcommand{\arraystretch}{1.6}
\begin{tabular}{cccc}
\hline\hline
Partition Function & Close Channel & Expression in $\tilde q$ & Expression in $q$ \\
\hline\hline

$Z_{++} = Z_{--}$ 
& $\langle \pm | \tilde{q}^{\frac12(L_0+\bar L_0 - c/12)} | \pm \rangle$
& $\frac12 \chi_0(\tilde q) + \frac12 \chi_{1/2}(\tilde q) + \frac{1}{\sqrt{2}} \chi_{1/16}(\tilde q)$
& $\chi_0(q)$
\\

$Z_{+-} = Z_{-+}$ 
& $\langle \pm | \tilde{q}^{\frac12(L_0+\bar L_0 - c/12)} | \mp \rangle$
& $\frac12 \chi_0(\tilde q) + \frac12 \chi_{1/2}(\tilde q) - \frac{1}{\sqrt{2}} \chi_{1/16}(\tilde q)$
& $\chi_{1/2}(q)$
\\

$Z_{+f} = Z_{-f}$
& $\langle \pm | \tilde{q}^{\frac12(L_0+\bar L_0 - c/12)} | f \rangle$
& $\frac{1}{\sqrt{2}} \chi_0(\tilde q) - \frac{1}{\sqrt{2}} \chi_{1/2}(\tilde q)$
& $\chi_{1/16}(q)$
\\

$Z_{ff}$
& $\langle f | \tilde{q}^{\frac12(L_0+\bar L_0 - c/12)} | f \rangle$
& $\chi_0(\tilde q) + \chi_{1/2}(\tilde q)$
& $\chi_0(q) + \chi_{1/2}(q)$
\\

\hline\hline
\end{tabular}
\caption{Partition function and corresponding characters in both $\tilde q$ and $q$ expansions.}
\label{tab:ising_partition_functions}
\end{table}

We can see that the expression in $ q $ always have non-negative integer coefficients, which satisfy the Cardy condition.

