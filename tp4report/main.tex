% main.tex - 科研项目学习笔记模板
% !TeX root = main.tex
%%%%%%%%%%%%%%%%%%%%%%%%%%%%%% DOCUMENT 
\documentclass[12pt]{article}

%%%%%%%%%%%%%%%%%%%%%%%%%%%%%% PACKAGES

% 中文支持(XeLaTeX 编译)
% \usepackage[UTF8]{ctex}
% \usepackage{xeCJKfntef} 

% \setCJKmainfont{HanziPen SC}
% \setmainfont{HanziPen SC}


% 页面设置
\usepackage[a4paper, left=15mm, right=15mm, top=15mm, bottom=15mm]{geometry}

\PassOptionsToPackage{dvipsnames,svgnames,x11names}{xcolor}
\usepackage{xcolor}


% 数学环境及符号
\usepackage{amsmath, amssymb, amsfonts, amsthm,amsopn}
\usepackage{tensor}              % 张量指标管理
\usepackage{mathtools}           % amsmath增强
\usepackage{physics}             % 物理公式快捷命令
\usepackage{bbold}               % 数学黑体
\usepackage{dsfont}              % 另一种数字体
\usepackage[mathscr]{eucal}     % 花体字母
\usepackage{tensor}              % 张量指标管理
\usepackage{simpler-wick}       % Wick记号
\usepackage{mathrsfs}            % 另一种花体字母

% 颜色与图形相关
\usepackage{graphicx}           % 插图支持
\usepackage{float}              % 浮动体控制
\usepackage{tikz}               % 绘图库
\usetikzlibrary{math}           % tikz数学扩展
\usepackage{geometry}
% 表格与列表
\usepackage{makecell}           % 表格多行换行
\usepackage{multicol}           % 多栏排版
\usepackage{colortbl}           % 表格颜色
\usepackage{enumitem}           % 列表自定义

% 其他辅助
\usepackage{framed}             % 有边框环境
\usepackage{tcolorbox}          % 灵活盒子环境
\tcbuselibrary{breakable}       % 盒子内容分页
\usepackage{thmtools}           % 定理环境管理
\usepackage{thm-restate}        % 定理重述
\usepackage{showlabels}         % 显示标签,调试用(完成后可注释)
\usepackage[normalem]{ulem}     % 下划线、删除线
\usepackage{hyperref}           % 超链接(最后加载)
\usepackage{cleveref}           % 智能引用(紧跟hyperref)
\usepackage{soul}
\definecolor{lightred}{rgb}{1,.8,.8} 
\sethlcolor{lightred}   
\usepackage{hyperref}

\usepackage{natbib}
\bibliographystyle{plain}

% 自定义宏包
\usepackage{macros}

% 一个中文可以高亮的包
\usepackage{cjkhl}
\definecolor{lightblue}{rgb}{.8,.8,1}

%%%%%%%%%%%%%%%%%%%%%%%%%%%%%% 自定义命令
\newcommand{\tml}{Teichmüller space}
\newcommand{\hil}{Hilbert space}
\newcommand{\mtc}{Modular Tensor Category}

%%%%%%%%%%%%%%%%%%%%%%%%%%%%%% BEGINNING OF THE DOCUMENT

\begin{document}

\title{\boldmath Bosonization of Majorana CFT with Boundaries}
\author{Yu Liu}

\maketitle

\begin{abstract}
  It has been well known that the two-dimensional critical Ising model can be described by a free massless Majorana fermion CFT as well as the $c=\frac{1}{2}$ minimal model CFT. It is interesting to show the explicit correspondence between the two theories.

  In this note, we focus on how these 
\end{abstract}

\tableofcontents


\newpage
\section{Preliminaries 1: Ising CFT and BCFT}\label{sec:Boundary Ising CFT} % (fold)
\subsection{Lightning Review of BCFT} 

I will first give a very brief review of fundamental concepts and tools in Boundary Conformal Field Theory (BCFT). For a more detailed introduction, one can refer to my CFT and BCFT note \cite{Xiao-daihua_Phys_Learning_CFT_Primary_BCFT_2025, Xiao-daihua_Phys_Learning_CFT_Minimal_CFT_2025}, click \href{https://github.com/Xiao-daihua/Phys_Learning/blob/main/CFT/Primary_BCFT/main.pdf}{here}, and the following references \cite{cardyBoundaryConditionsFusion1989, northeYoungResearchersSchool2025, recknagelBoundaryConformalField}, these three references are found to be most helpful for my understanding of BCFT.

\subsubsection{Gluing Condition}

A Boundary CFT (BCFT) is a conformal field theory on a manifold with boundary induced from a parent CFT. It inherets the operator algebra data of the parent CFT, while having extra data due to the presence of boundaries. The simplest case is to consider a BCFT on the upper half plane (UHP) $\{z=x+iy|y\geq 0\}$, where the real axis $y=0$ serves as the boundary. 

To preserve conformal symmetry at the boundary, we need to impose the so-called gluing condition on the energy-momentum tensor $T(z)$ and $\bar{T}(\bar{z})$ at the boundary:
\defi{
  \textbf{Gluing Condition}

  At the boundary $y=0$, the energy-momentum tensor satisfies:
  \begin{align}
    T(z) = \bar{T}(\bar{z}) \quad \text{at } y=0.
  \end{align}
}
With this gluing condition, we can prove that the conformal symmetry is "partially" preserved, which leads to the existence of a single Virasoro algebra instead of two independent ones:
\thm{
  \textbf{Symmetry Algebra of BCFT}

  In a BCFT defined on the upper half plane with the gluing condition $T(z) = \bar{T}(\bar{z})$ at the boundary $y=0$, the symmetry algebra is generated by a single Virasoro algebra with generators:
  \begin{align}
    L_n^H = \frac{1}{2\pi i} \oint_{C} dz \, z^{n+1} T^H(z) 
  \end{align}
  where $ T^H(z) $ is the analytical continuation of $ T(z) $ and $ \bar{T}(\bar{z}) $. Satisfying:
  \begin{align}
    [L_n^H, L_m^H] = (n-m) L_{n+m}^H + \frac{c}{12} n(n^2-1) \delta_{n+m,0}
  \end{align}
}

\subsubsection{Boundary States Formalism}

There exist various ways of defining a BCFT from a parent CFT on a manifold by imposing different types of boundary conditions at the boundary. A tool of labeling and classifying these boundary conditions is the boundary state formalism. We define the boundary states as;
\defi{
  \textbf{Boundary State}

  Boundary states $\ket{\alpha}$ and $\ket{\beta}$ are defined as states living on
  the boundary of the parent CFT placed on a chopped cylinder, and they satisfy:
  \begin{itemize}
    \item The cylinder has the same geometry as a compactified BCFT on a strip,
          with length $L$ and circumference $\beta$.
    \item The partition function of the parent CFT on a cylinder with boundary
          states $\ket{\alpha}$ and $\ket{\beta}$ at its two ends is equivalent to
          the partition function of the BCFT on a strip with boundary conditions
          $\alpha$ and $\beta$ at the two ends. Mathematically:
  \end{itemize}
  \begin{align}
    Z_{\alpha\beta} = \bra{\alpha}\,
           \tilde{q}^{\frac{1}{2}(L_0 + \bar{L}_0 - c/12)}\,
         \ket{\beta} = \operatorname{Tr}_{\mathcal{H}_{\alpha\beta}}
      \left( q^{\,L_0^{H} - c/24} \right),
  \end{align}
  where $q = e^{-\pi \beta / L}$ and $\tilde{q} = e^{-4\pi L / \beta}$.Here $L_0$ and $\bar{L}_0$ are the Virasoro generators of the parent CFT, while $L_0^{H}$ is the Virasoro generator in the BCFT.
}
We can use the boundary states as a state in the Hilbert space of the parent CFT to encode the boundary conditions of the BCFT. To be consistent with the gluing condition, the boundary states must satisfy the following Ishibashi condition:
\thm{
  \textbf{Ishibashi Condition}

  Boundary states $\ket{\alpha}$ should satisfy:
  \begin{align}
    (L_n - \bar{L}_{-n}) \ket{\alpha} = 0, \quad \forall n \in \mathbb{Z}.
  \end{align}
}
A solution to the Ishibashi condition can be constructed from each primary state $\ket{h}$ of the parent CFT, leading to the definition of Ishibashi states:
\defi{
  \textbf{Ishibashi State}

  For each primary state $\ket{h}$ of the parent CFT, the corresponding Ishibashi state $\ket{h}\rangle$ is defined as:
  \begin{align}
    \ket{h}\rangle = \sum_{N} \ket{h;N} \otimes \overline{\ket{h;N}},
  \end{align}
  where $\{\ket{h;N}\}$ is an orthonormal basis of the Verma module built on the chiral primary state $\ket{h}$. 
}
We can prove that the Ishibashi States satisfy the Ishibashi Conditions.


\subsubsection{Cardy's Condition}

Consider the Parent CFT as a diagonal CFT. Indeed the main topic of the note Ising CFT is a diagonal CFT. There exist another consistency condition for the boundary state:
\thm{
  \textbf{Cardy's Condition}

  For two boundary states $ \ket{\alpha} $ and $ \beta $ they should satisfy that:
  \begin{align}
    \sum_h\langle\alpha|h\rangle\rangle\langle\langle h|\beta\rangle S_h^{h^{\prime}}=n_{\alpha\beta}^{h^{\prime}} \quad \text{or equivalently} \quad  \langle a|h^{\prime}\rangle\rangle\langle\langle h^{\prime}|b\rangle=\sum_hS_h^{h^{\prime}}n_{ab}^h.
  \end{align}
  where $ S^{h'}_h $ is the modular S matrix of the parent CFT and $ n_{\alpha\beta}^{h^{\prime}} $ are non-negative integers representing the multiplicities of the representation with highest weight $ h' $ in the open channel partition function $ Z_{\alpha\beta} $.
}
Proof: see Cardy's original paper \cite{cardyBoundaryConditionsFusion1989}. 

If we assume that the boundary states are linear combination of Ishibashi States, we can also write the Cardy's condition a relation of expansion coefficients:
\thm{\label{thm:cardycondition}
  \textbf{Cardy's Condition (coefficient constrain)}

  Consider the boundary states written as:
  \begin{align}
    \left|a\right\rangle=\sum_{i}B_{i}^{a}\left|i\right\rangle\rangle
  \end{align}
  The Cardy's condition says that:
  \begin{align}
    \sum_{ij} B_i^{a *}B_i^bS_{i}^j\chi_j(q)=\sum_in_{ab}^i\chi_i(q)\quad \Rightarrow \quad \sum_{i} B_i^{a *}B_i^bS_{i}^j=n_{ab}^j
  \end{align}
  where $ \chi_i $ is the character of the Virasoro representation. 
}

A solution to the Cardy's condition is given by comparing it with the Verlinder formula. These solutions are called the Cardy States:
\defi{
  \textbf{Cardy States}

  Cardy states are linear combination of Ishibashi states with coefficients given by:
  \begin{align}
    |a\rangle=\sum_{j}\frac{S_{a}^{j}}{(S_{0}^{j})^{1/2}}|j\rangle\rangle
  \end{align}
}
Proof: taking $ n_{ab}^i $ as the fusion matrix $ N^a_{bc} $ and Cardy states into Cardy's condition, we can see that the Cardy's condition \cref{thm:cardycondition} is exactly the Verlinder formula.


\subsubsection{Elementary Boundary States}

Looking at the Cardy's condition we might see that, if a state $ \ket{a} $ satisfy the condition then $ \mathbb{N}_+\ket{a} $ must also satisfy this condition, for we only insist $ n^i_{ab} $ to be non-negative integers. 

However, we usually only consider the so-called elementary boundary states that can not be decomposed into a sum of other boundary states. Thus we have the definition:
\defi{
  \textbf{Elementary Boundary States}

  For a parent CFT, a induced BCFT have following elementary boundary states $ \{\ket{a_i}\} $ that shall satisfy the following normalization and orthogonality condition: 
  \begin{align}
    n_{a_ia_j}^0 = \delta_{ij}
  \end{align}
}
The normalization is physical, this is because for a field theory, we assume to have a unique vaccum state, thus if $ n_{ab}^0 \geq 2 $ we will have multiple vacuum states in the open channel, which is unphysical, and shall be interpret as putting two identical theory togethor. Our goal is to find all elementary boundary states of a BCFT induced from a parent CFT.

Moreover, we can see that 
\thm{
 Cardy states are elementary boundary states. 
}
This is because, cardy's state is a solution when considering $ n_{ab}^i = N_{ab}^i $ which is the bulk fusion matrix. And we know that according to the 2 point function of bulk primary fields, the vacuum representation only appear once in the fusion of two primary fields when they are the same, written in operator algebra as:
\begin{align}
  \phi_a \times \phi_b = \mathbb{1} \delta_{ab} + \cdots
\end{align}

\subsection{Ising CFT}

Ising CFT as the simplist unitary minimal model $ M(3/4) $ with central charge $ c=1/2 $. Here I will briefly review some basic data of Ising CFT that will be used in the following sections. For a more detailed introduction to Ising CFT, one can refer to \cite{difrancescoConformalFieldTheory1997}.

\subsubsection{Ising CFT Data}
Ising CFT is the simplest A-series unitary minimal model $M(3,4)$ with central charge $c=\tfrac{1}{2}$. 
It has three primary fields:
\begin{itemize}
  \item Identity operator $\mathbb{1}$ with conformal weights 
    $h_{\mathbb{1}}=\bar{h}_{\mathbb{1}}=0$, 
  \item Spin operator $\sigma$ with conformal weights 
    $h_{\sigma}=\bar{h}_{\sigma}= \tfrac{1}{16}$
  \item Energy operator $\epsilon$ with conformal weights 
    $h_{\epsilon}=\bar{h}_{\epsilon}=\tfrac{1}{2}$.
\end{itemize}
They satisfy the folowing operator algebra (fusion rules):
\begin{align}
  &\sigma \times \sigma = \mathbb{1} + \epsilon, \\
  &\sigma \times \epsilon = \sigma, \\
  &\epsilon \times \epsilon = \mathbb{1}.
\end{align}

\subsubsection{Virasoro Characters of Ising CFT}

Charaters of a Virasoro Representation is defined as :
\defi{
  \textbf{Character}

The character $\chi_h(q)$ of a Virasoro representation with highest weight $h$ is defined as:
\begin{align}
  \chi_h(q) = \operatorname{Tr}_{\mathcal{H}_h} \left( q^{L_0 - c/24} \right),
\end{align}
}
For Ising CFT the characters of the three primary fields are given by:
\begin{align}
  &\chi_0(q)=\frac{1}{2}q^{-\frac{1}{48}}\left(\prod_{n=0}^\infty\left(1+q^{\frac{1}{2}+n}\right)+\prod_{n=0}^\infty\left(1-q^{\frac{1}{2}+n}\right)\right),\\
&\chi_{1/2}(q)=\frac{1}{2}q^{-\frac{1}{48}}\left(\prod_{n=0}^\infty\left(1+q^{\frac{1}{2}+n}\right)-\prod_{n=0}^\infty\left(1-q^{\frac{1}{2}+n}\right)\right),\\
&\chi_{1/16}(q)= q^{\frac{1}{24}}\prod_{n=1}^\infty\left(1+q^n\right)
\end{align}


\subsubsection{Modular S Matrix and Transformation of Characters}
The modular S Matrix of Ising CFT is given by:
\begin{align}
  S^i_j=\begin{pmatrix}
    \frac{1}{2} & \frac{1}{2} & \frac{1}{\sqrt{2}} \\
    \frac{1}{2} & \frac{1}{2} & -\frac{1}{\sqrt{2}} \\
    \frac{1}{\sqrt{2}} & -\frac{1}{\sqrt{2}} & 0
  \end{pmatrix}
\end{align}
with the column and row order $ \mathbb{1},\epsilon,\sigma $. The characters form a representation of the modular group with the transformation:
\begin{align}
  \chi_i\left(q\right)=\sum_j S_{i}^j\chi_j\left(\tilde{q}\right).
\end{align}
if we take $ q = e^{-\frac{\pi \beta}{L}} $ and $ \tilde{q} = e^{-\frac{4\pi L}{\beta}} $, we have:
\begin{align}
  &\chi_0(\tilde{q})=\frac{1}{2}\left(\chi_0(q)+\chi_{1/2}(q)\right)+\frac{1}{\sqrt{2}}\chi_{1/16}(q),\\
  &\chi_{1/2}(\tilde{q})=\frac{1}{2}\left(\chi_0(q)+\chi_{1/2}(q)\right)-\frac{1}{\sqrt{2}}\chi_{1/16}(q),\\
  &\chi_{1/16}(\tilde{q})=\frac{1}{\sqrt{2}}\left(\chi_0(q)-\chi_{1/2}(q)\right).
\end{align}
as well as the inverse transformation:
\begin{align}
  &\chi_0(q)=\frac{1}{2}\left(\chi_0(\tilde{q})+\chi_{1/2}(\tilde{q})\right)+\frac{1}{\sqrt{2}}\chi_{1/16}(\tilde{q}),\\
  &\chi_{1/2}(q)=\frac{1}{2}\left(\chi_0(\tilde{q})+\chi_{1/2}(\tilde{q})\right)-\frac{1}{\sqrt{2}}\chi_{1/16}(\tilde{q}),\\
  &\chi_{1/16}(q)=\frac{1}{\sqrt{2}}\left(\chi_0(\tilde{q})-\chi_{1/2}(\tilde{q})\right).
\end{align}
\rmk{
  For Ising CFT the modular S matrix is self-inverse, i.e. $ S = S^{-1} $.
}

\subsection{Ising BCFT}
For a BCFT induced from Ising CFT, we can construct its boundary states by solving the Ishibashi condition and the Cardy's condition.

\subsubsection{Ishibashi States of Ising BCFT}

The Ishibashi states of Ising CFT can be constructed from its three primary states:
\defi{
  \textbf{Ising Ishibashi States}

  Ising CFT has three Ishibashi states:
  \begin{align}
    &|0\rangle\rangle=\sum_{N}|0;N\rangle\otimes\overline{|0;N\rangle},\\
&|1/2\rangle\rangle=\sum_{N}|1/2;N\rangle\otimes\overline{|1/2;N\rangle},\\
&|1/16\rangle\rangle=\sum_{N}|1/16;N\rangle\otimes\overline{|1/16;N\rangle}.
  \end{align}
}

\subsubsection{Cardy States of Ising BCFT}

 Using the modular S matrix of Ising CFT, we can write down the Cardy states by definition:
\defi{
  \textbf{Ising Cardy States}

  Ising BCFT exist three Cardy states:
  \begin{align}
    &\left|+\right\rangle=\frac{1}{\sqrt{2}}|0\rangle\rangle+\frac{1}{\sqrt{2}}|1/2\rangle\rangle+\frac{1}{\sqrt[4]{2}}|1/16\rangle\rangle \\
&\left|-\right\rangle=\frac{1}{\sqrt{2}}|0\rangle\rangle+\frac{1}{\sqrt{2}}|1/2\rangle\rangle-\frac{1}{\sqrt[4]{2}}|1/16\rangle\rangle \\
& \left|f\right\rangle =|0\rangle\rangle- |1/2\rangle\rangle.
  \end{align}
}
In Cardy's original paper \cite{cardyBoundaryConditionsFusion1989}, these three boundary states are argued to correspond to the three physical boundary conditions of the Ising model\footnote{this is an argument made by considering the transformation of the boundary states under $ \mathbb{Z}_2 $ symmetry of Ising CFT. It is not a rigorous theorem.}:
\begin{itemize}
  \item $|+\rangle$: fixed boundary condition with spins fixed to $+1$ at the boundary.
  \item $|-\rangle$: fixed boundary condition with spins fixed to $-1$ at the boundary.
  \item $|f\rangle$: free boundary condition with spins free to fluctuate at the boundary.
\end{itemize}

\subsubsection{Partition Functions of Ising BCFT}

With the Cardy states, we can compute the partition functions of Ising BCFT on a compatified strip (cylinder) with different boundary conditions:
\begin{table}[H]
\centering
\renewcommand{\arraystretch}{1.6}
\begin{tabular}{cccc}
\hline\hline
Partition Function & Close Channel & Expression in $\tilde q$ & Expression in $q$ \\
\hline\hline

$Z_{++} = Z_{--}$ 
& $\langle \pm | \tilde{q}^{\frac12(L_0+\bar L_0 - c/12)} | \pm \rangle$
& $\frac12 \chi_0(\tilde q) + \frac12 \chi_{1/2}(\tilde q) + \frac{1}{\sqrt{2}} \chi_{1/16}(\tilde q)$
& $\chi_0(q)$
\\

$Z_{+-} = Z_{-+}$ 
& $\langle \pm | \tilde{q}^{\frac12(L_0+\bar L_0 - c/12)} | \mp \rangle$
& $\frac12 \chi_0(\tilde q) + \frac12 \chi_{1/2}(\tilde q) - \frac{1}{\sqrt{2}} \chi_{1/16}(\tilde q)$
& $\chi_{1/2}(q)$
\\

$Z_{+f} = Z_{-f}$
& $\langle \pm | \tilde{q}^{\frac12(L_0+\bar L_0 - c/12)} | f \rangle$
& $\frac{1}{\sqrt{2}} \chi_0(\tilde q) - \frac{1}{\sqrt{2}} \chi_{1/2}(\tilde q)$
& $\chi_{1/16}(q)$
\\

$Z_{ff}$
& $\langle f | \tilde{q}^{\frac12(L_0+\bar L_0 - c/12)} | f \rangle$
& $\chi_0(\tilde q) + \chi_{1/2}(\tilde q)$
& $\chi_0(q) + \chi_{1/2}(q)$
\\

\hline\hline
\end{tabular}
\caption{Partition function and corresponding characters in both $\tilde q$ and $q$ expansions.}
\label{tab:ising_partition_functions}
\end{table}

We can see that the expression in $ q $ always have non-negative integer coefficients, which satisfy the Cardy condition.


% section Boundary Ising CFT (end)

\section{Preliminaries 2: Majorana CFT}\label{sec:Majorana CFT} % (fold)
A Majorana CFT exibites a Lagrangian description. In this section, I will briefly review the Majorana CFT. A detailed introduction can be found in \cite{difrancescoConformalFieldTheory1997, ginspargAppliedConformalField1988a}. Note that I take the idea of the reference \cite{ginspargAppliedConformalField1988a} to quantize the theory on a complex plane first and then transform it to a cylinder. But I appreciate the convention of the reference \cite{difrancescoConformalFieldTheory1997}.

\subsection{Majorana CFT Path Integral Formalism}


\subsubsection{Action and Equations of Motion}

The action of a free Majorana fermion on a complex plane is given by:
\defi{
  \textbf{Majorana Fermion Action}

  The action of a free Majorana fermion $\psi(z)$ and $\bar{\psi}(\bar{z})$ on a complex plane is given by:
  \begin{align}
    S = \frac{1}{2\pi} \int d^2z \, \left( \psi \bar{\partial} \psi + \bar{\psi} \partial \bar{\psi} \right),
  \end{align}
  where $\partial = \frac{\partial}{\partial z}$ and $\bar{\partial} = \frac{\partial}{\partial \bar{z}}$ and $ d^2 z = d^2 x = dx dy $. 
}
In this convention, the equations of motion derived from the action are:
\begin{align}
  \bar{\partial} \psi(z) = 0, \quad \partial \bar{\psi}(\bar{z}) = 0,
\end{align}
which means that $\psi(z)$ is a holomorphic field and $\bar{\psi}(\bar{z})$ is an anti-holomorphic field. 

\subsubsection{Energy-Momentum Tensor}

With the standard convention, we calculate the E-M tensor from the action:
\begin{align}
  T(z) = - 2 \pi T_{zz}(z) = -\displaystyle\frac{1}{2} \psi(z) \partial \psi(z), \quad
  \bar{T}(\bar{z}) = - 2 \pi T_{\bar{z}\bar{z}}(\bar{z}) = -\displaystyle\frac{1}{2} \bar{\psi}(\bar{z}) \bar{\partial} \bar{\psi}(\bar{z}).
\end{align}

\rmk{
  Note that the definition of $ T(z) $ has different conventions. But it is due to different conventions of the action and the field. We here take the convention in the reference \cite{difrancescoConformalFieldTheory1997}.
}

\subsubsection{OPE Structure}

The OPE structure of the Majorana fermion fields can be derived from their two-point correlation functions:
\begin{align}
  &\langle\psi(z)\psi(w)\rangle=\frac{1}{z-w}\quad \langle\bar{\psi}(\bar{z})\bar{\psi}(\bar{w})\rangle=\frac{1}{\bar{z}-\bar{w}}\\
&\langle\psi(z)\bar{\psi}(\bar{w})\rangle=0
\end{align}
Thus we have the following OPEs:
\begin{align}
  \psi(z)\psi(w)\sim\frac{1}{z-w}, \quad \bar{\psi}(\bar{z})\bar{\psi}(\bar{w})\sim\frac{1}{\bar{z}-\bar{w}}, \quad \psi(z)\bar{\psi}(\bar{w})\sim 0.
\end{align}
\subsection{Majorana CFT Canonical Formalism}

\subsubsection{Quantization and Mode Expansion}

We then consider the radial quantization of the theory on a complex plane. The EoM gives us that the fields can be expanded as:
\begin{align}
  \psi(z)=\sum_kb_kz^{-k-1/2} \quad \bar{\psi}(\bar{z})=\sum_k\bar{b}_k\bar{z}^{-k-1/2}
\end{align}
it is a common convention to take the $ z^{1/2} $ in the mode expansion. The OPE of the fields gives us the anti-commutation relations of the modes:
\begin{align}
  \{ b_k, b_l \} = \delta_{k+l,0}, \quad \{ \bar{b}_k, \bar{b}_l \} = \delta_{k+l,0}, \quad \{ b_k, \bar{b}_l \} = 0.
\end{align}
Moreover the Majorana fermion condition $ \psi(z) = \psi^\dagger(z) $ gives us that the modes satisfy:
\begin{align}
  b_k^\dagger = b_{-k}, \quad \bar{b}_k^\dagger = \bar{b}_{-k}.
\end{align}
The E-M tensor can also be expanded in modes:
\begin{align}
  T(z) = - \frac{1}{2} :\psi(z) \partial \psi(z): =\sum_n L_n z^{-n-2}, \quad \bar{T}(\bar{z}) = - \frac{1}{2} :\bar{\psi}(\bar{z}) \bar{\partial} \bar{\psi}(\bar{z}): = \sum_n \bar{L}_n \bar{z}^{-n-2},
\end{align}
\rmk{
  We shall note that the quantum version of the E-M tensor we impose normal ordering as a convention. Here we take the convention that the modes with positive index are annihilation operators and the modes with negative index are creation operators.
}
The virasoro generator of the theory are thus explicitly given by:
\begin{align}
  L_n=\frac{1}{2}\sum_k(k+\frac{1}{2}):b_{n-k}b_k:
\end{align}


\subsubsection{Neveu-Schwarz and Ramond Sectors}

We then notice a subtlety in the theory. We can impose two different boundary conditions on the fermion fields when we consider the theory on a cylinder:
\begin{itemize}
  \item \textbf{Neveu-Schwarz (NS) Boundary Condition}:
        \begin{align}
          \psi(e^{2\pi i} z) = \psi(z), \quad \bar{\psi}(e^{-2\pi i} \bar{z}) = \bar{\psi}(\bar{z}).
        \end{align}
        This boundary condition leads to half-integer moding of the fermion fields:
        \begin{align}
          k \in \mathbb{Z} + \frac{1}{2}.
        \end{align}
      \item \textbf{Ramond (R) Boundary Condition}: 
        \begin{align}
          \psi(e^{2\pi i} z) = - \psi(z), \quad \bar{\psi}(e^{-2\pi i} \bar{z}) =- \bar{\psi}(\bar{z}).
        \end{align}
        This boundary condition leads to integer moding of the fermion fields:
        \begin{align}
          k \in \mathbb{Z}.
        \end{align}
\end{itemize}
Thus different boundary conditions lead to different mode expansions of the fermion fields as well as different Hilbert spaces of the theory. The Hilbert space of the theory is thus classified into two sectors: the NS sector and the R sector.

\subsubsection{Hamiltonian after Normalization}

The Hamiltonian of the theory on a complex plane is given by the generator of the time translation:
\thm{
  \textbf{Majorana CFT Hamiltonian on a Complex Plane}

  The Hamiltonian of the Majorana CFT on a complex plane is given by:
  \begin{align}
    & H \;=\; L_0 + \bar L_0 \\
    %
    & L_0 \;=\; \sum_{k>0} k\, b_{-k} b_k ,
    \qquad 
      \bar L_0 \;=\; \sum_{k>0} k\, \bar b_{-k} \bar b_k ,
    \qquad (\mathrm{NS}:~ k\in \mathbb{Z}+\tfrac12) 
    \\
    & L_0 \;=\; \sum_{k>0} k\, b_{-k} b_k + \frac{1}{16},
    \qquad
      \bar L_0 \;=\; \sum_{k>0} k\, \bar b_{-k} \bar b_k + \frac{1}{16},
    \qquad (\mathrm{R}:~ k\in\mathbb{Z})
\end{align}
Note that we have minus a constant in both sector to set the ground state energy of the NS sector to zero.
}
\rmk{
  Both calculations of the sectors gives a infinite vaccum enegery: $ 1/2\sum_{k = 1}^\infty k = -1/24$ for NS and $ 1/2 \sum_{k = 0}^\infty (k+1/2) =  1/48$. We regulate the vaccum energy by adding $ 1/24 $, thus we get $ 0 $ in the NS sector and $ 1/16 $ in the R sector.
}

\subsubsection{Zero Mode and Ground State Degeneracy}\label{sec:Zero Mode and Ground State Degeneracy in Majorana CFT}

The NS sector works straightforwardly with a zero vacuum energy after normalization and a unique ground state $ \ket{0}_{NS} $ defined by:
\begin{align}
  b_k \ket{0}_{NS} = 0, \quad \bar{b}_k \ket{0}_{NS} = 0, \quad k > 0.
\end{align}
However, the R sector is more subtle due to the presence of zero modes $ b_0 $ and $ \bar{b}_0 $. These zero modes satisfy the Clifford algebra:
\begin{align}
  \{ b_0, b_0 \} = 1, \quad \{ \bar{b}_0, \bar{b}_0 \} = 1, \quad \{ b_0, \bar{b}_0 \} = 0.
\end{align}
We can see that the zero modes $ b_0 $ and $ \bar{b}_0 $ generate a two-dimensional representation of the Clifford algebra. Thus the ground state of the R sector is doubly degenerate according to the following theorm:
\thm{
  Two dimensional Cliffoed Algebra Representation have non-trivial representation for at least two dimensions.
}
Thus, we can define two ground states $ \ket{+}_{R} $ and $ \ket{-}_{R} $ and the act of pauli matrix on it is:
\begin{align}
  \sigma_x \ket{\pm}_{\mathrm R} &= \ket{\mp}_{\mathrm R}, \quad \sigma_y \ket{\pm}_{\mathrm R} = \mp i\,\ket{\mp}_{\mathrm R}, \quad \sigma_z \ket{\pm}_{\mathrm R} = \pm \ket{\pm}_{\mathrm R}.
\end{align}
And then we can represent the zero modes and the fermion parity operator as:
\begin{align}
   & b_0 = \frac{1}{2}\left( \sigma_x + \sigma_y \right)
        \,(-1)^{\sum_{n>0} b_{-n} b_n + \bar b_{-n} \bar b_n}, \\
   & \bar b_0 = \frac{1}{2}\left( \sigma_x - \sigma_y \right)
        \,(-1)^{\sum_{n>0} b_{-n} b_n + \bar b_{-n} \bar b_n}, \\
\end{align}
In this convention, we have: 
\begin{align}\label{eq:Majorana zero mode action on R ground states}
  b_0|\pm\rangle_\mathrm{R}=\frac{1}{\sqrt{2}}e^{\pm i\pi/4}|\mp\rangle_\mathrm{R},\quad\bar{b}_0|\pm\rangle_\mathrm{R}=\frac{1}{\sqrt{2}}e^{\mp i\pi/4}|\mp\rangle_\mathrm{R}.
\end{align}

\subsubsection{Fermion Parity}\label{sec:Fermion Parity in Majorana CFT}

We hope to define (in many cases this operator might not be well defined) the fermion parity operator $ (-1)^F $ in the Majorana CFT as follows:
\defi{
  \textbf{Fermion Parity Operator}

  The fermion number operator $ F $ in a Majorana CFT is defined as:
\begin{align}
  \{(-1)^F, b_k\}=0, \quad \{(-1)^{F}, \bar{b}_k\}=0, \quad\text{ for all }k , 
\end{align}
}
\YL{[note]}
In the NS sector, the fermion parity operator is explicitly given by:
\begin{align}
  (-1)^F = (-1)^{\sum_{k>0} b_{-k} b_k + \sum_{k>0} \bar{b}_{-k} \bar{b}_k}.
\end{align}
However, in the R sector defining the fermion number operator is more subtle. There are two different ways of defining the fermion number operator in the R sector, write it in terms of $ b_0, \bar{b}_0$ :
\begin{align}
  (-1)^{F} =
\begin{cases}
  -2 i\, b_0 \bar b_0 \ (-1)^{F_{nz}}, \quad \text{type 0A} \\
  +2 i\, b_0 \bar b_0 \ (-1)^{F_{nz}}. \quad \text{type 0B}
\end{cases}
\end{align}
In string theory, they are called type 0A and type 0B. Commonly, we take the notation of $ (-1)^{F_{nz}} = (-1)^{\sum_{k>0} b_{-k} b_k + \sum_{k>0} \bar{b}_{-k} \bar{b}_k} $ non-zero mode fermion parity operator. 

\YL{[I'm not sure about do I understand this correctly.]}

\subsubsection{Hilbert Space Structure}

With the ground state structure and mode expansion, we can construct the Hilbert space of the Majorana CFT. The Hilbert space is classified into two sectors: the NS sector and the R sector. 
\begin{itemize}
  \item \textbf{NS Sector Hilbert Space}:
        The Hilbert space in the NS sector is constructed by acting the creation operators $ b_{-k}, \bar{b}_{-k} $ ($ k > 0 $ and $ k \in \mathbb{Z} + 1/2 $) on the unique ground state $ \ket{0}_{NS} $:
        \begin{align}
          \mathcal{H}_{NS} = \mathrm{span}\left\{ b_{-k_1} b_{-k_2} \cdots \bar{b}_{-l_1} \bar{b}_{-l_2} \cdots \ket{0}_{NS} \,|\, k_i, l_j > 0, k_i, l_j \in \mathbb{Z} + \frac{1}{2} \right\}.
        \end{align}
      \item \textbf{R Sector Hilbert Space}: 
        The Hilbert space in the R sector is constructed by acting the creation operators $ b_{-k}, \bar{b}_{-k} $ ($ k > 0 $ and $ k \in \mathbb{Z} $) on the doubly degenerate ground states $ \ket{\pm}_{R} $:
        \begin{align}
          \mathcal{H}_{R} = \mathrm{span}\left\{ b_{-k_1} b_{-k_2} \cdots \bar{b}_{-l_1} \bar{b}_{-l_2} \cdots \ket{\pm}_{R} \,|\, k_i, l_j > 0, k_i, l_j \in \mathbb{Z}  \right\}.
        \end{align}
\end{itemize}


\subsection{Majorana CFT on a Cylinder}\label{sec:Majorana CFT on a Cylinder}
Here we will consider the theory on a cylinder with circumference $ \beta $. As we do in CFT, we map the theory from a complex plane to a cylinder via a conformal map.


\subsubsection{Conformal Transformation to Cylinder}

We then transfrom the theory from a complex plane to a cylinder with circumference $\beta$ via the conformal map:
\begin{align}
  & w = \frac{\beta}{2\pi} \ln z, \quad \bar{w} = \frac{\beta}{2\pi} \ln \bar{z} \quad  z = e^{\frac{2\pi w}{\beta}}, \quad \bar{z} = e^{\frac{2\pi \bar{w}}{\beta}} \\ 
  & w = t - i \sigma, \quad \bar{w} = t + i \sigma, \quad \sigma \in [- \beta, 0]
\end{align}
Under this map the fermion fields transform as:
\begin{align}
  \psi^C(w) = \left( \frac{dz}{dw} \right)^{1/2} \psi(z) = \left( \frac{2\pi}{\beta} z \right)^{1/2} \psi\left( z \right), \\
  \bar{\psi}^C(\bar{w}) = \left( \frac{d\bar{z}}{d\bar{w}} \right)^{1/2} \bar{\psi}(\bar{z}) = \left( \frac{2\pi}{\beta} \bar{z} \right)^{1/2} \bar{\psi}\left( \bar{z} \right).
\end{align}
The mode expansion on a cylinder is thus given by:
\begin{align}
  \psi^C(w)=\sqrt{\frac{2\pi}{\beta}}\sum_kb_ke^{-2\pi kw/\beta} \quad \bar{\psi}^C(\bar{w})=\sqrt{\frac{2\pi}{\beta}}\sum_k\bar{b}_ke^{-2\pi k\bar{w}/\beta}
\end{align}
the mode $ b_k $ is exactly the same object as that on the complex plane.
\rmk{
  It is interesting to note that the NS boundary condition becomes antiperiodic boundary condition on the cylinder; while the R boundary condition becomes periodic boundary condition on the cylinder. Thus, it is ambiguous to call a boundary condition antiperiodic or periodic without specifying the geometry.
}
The E-M tensor on a cylinder is given by:
\begin{align}
  T^C(w) = \left( \frac{dz}{dw} \right)^2 T(z) + \frac{c}{12} \{ z, w \} = \left( \frac{2\pi}{\beta} \right)^2 \left( z^2 T(z) - \frac{1}{48} \right), \\
  \bar{T}^C(\bar{w}) = \left( \frac{d\bar{z}}{d\bar{w}} \right)^2 \bar{T}(\bar{z}) + \frac{c}{12} \{ \bar{z}, \bar{w} \} = \left( \frac{2\pi}{\beta} \right)^2 \left( \bar{z}^2 \bar{T}(\bar{z}) - \frac{1}{48} \right).
\end{align}
Due to the conformal symmetry, the action on a cylinder shall be in the exact same form as that on a complex plane:
\begin{align}
  S^C = \frac{1}{2\pi} \int d^2 w \, \left( \psi^C \bar{\partial} \psi^C + \bar{\psi}^C \partial \bar{\psi}^C \right),
\end{align}
here $ \partial = \frac{\partial}{\partial w} $ and $ \bar{\partial} = \frac{\partial}{\partial \bar{w}} $ and $ d^2 w = dt d\sigma $.

\subsubsection{Hamiltonian on a Cylinder}

Thus, explicitly, the Hamiltonian on a cylinder is given by the standard CFT result as:
\thm{
  \textbf{Hamiltonian on a Cylinder}

  The Hamiltonian of a general CFT on a cylinder with circumference $\beta$ is given by:
  \begin{align}
    H = \frac{2\pi}{\beta} \left( L_0 + \bar{L}_0 - \frac{c}{12} \right).
  \end{align}
}
Proof: See my BCFT note for details.


Thus for the Majorana CFT, the Hamiltonian on a cylinder is given by:
\thm{
  \textbf{Majorana CFT Hamiltonian on a Cylinder}

  The Hamiltonian of the Majorana CFT on a cylinder with circumference $\beta$ is given by:
  \begin{align}
    & H = \frac{2\pi}{\beta} \left( L_0 + \bar{L}_0 - \frac{1}{24} \right) \\
    %
    & L_0 = \sum_{k>0} k\, b_{-k} b_k ,
    \qquad 
      \bar L_0 = \sum_{k>0} k\, \bar b_{-k} \bar b_k ,
    \qquad (\mathrm{NS}:~ k\in \mathbb{Z}+\tfrac12) 
    \\
    & L_0 = \sum_{k>0} k\, b_{-k} b_k + \frac{1}{16},
    \qquad
      \bar L_0 = \sum_{k>0} k\, \bar b_{-k} \bar b_k + \frac{1}{16},
    \qquad (\mathrm{R}:~ k\in\mathbb{Z})
  \end{align}
}
We can prove this by calculating the generator of time translation on a cylinder directly from the energy-momentum tensor, for more details see my BCFT note.

\subsubsection{Conformal Transformation to Cylinder (another convention)}

Apart from the quite standard convention, we have another commonly used convention:
\begin{align}
  & w = \displaystyle\frac{i \beta}{2 \pi} \ln z, \bar{w} = -\displaystyle\frac{i \beta}{2 \pi} \ln \bar{z} \quad  z = e^{\frac{- 2 \pi i w}{\beta}}, \bar{z} = e^{\frac{2 \pi i \bar{w}}{\beta}} \\ 
  & w = \sigma + i t, \quad  \bar{w} = \sigma - i t
\end{align}
Under this map the fermion fields transform as:
\begin{align}
  &\psi^C(w) = \left( \frac{dz}{dw} \right)^{1/2} \psi(z) = \left( \frac{2\pi}{i \beta} z \right)^{1/2} \psi\left( z \right), \\
  &\bar{\psi}^C(\bar{w}) = \left( \frac{d\bar{z}}{d\bar{w}} \right)^{1/2} \bar{\psi}(\bar{z}) = \left( -\frac{2\pi}{i \beta} \bar{z} \right)^{1/2} \bar{\psi}\left( \bar{z} \right).
\end{align}
Thus the mode expansion on a cylinder is given by:
\begin{align}
  \psi^C(w)=\sqrt{\frac{2\pi}{i\beta}}\sum_kb_ke^{-2\pi i kw/\beta} \quad \bar{\psi}^C(\bar{w})=\sqrt{-\frac{2\pi}{i\beta}}\sum_k\bar{b}_ke^{2\pi k i\bar{w}/\beta}
\end{align}
The E-M tensor on a cylinder is given by:
\begin{align}
  T^C(w) = \left( \frac{dz}{dw} \right)^2 T(z) + \frac{c}{12} \{ z, w \} = - \left(\displaystyle\frac{2 \pi}{\beta}\right)^2 \left( z^2 T(z) - \frac{1}{48} \right), \\
  \bar{T}^C(\bar{w}) = \left( \frac{d\bar{z}}{d\bar{w}} \right)^2 \bar{T}(\bar{z}) + \frac{c}{12} \{ \bar{z}, \bar{w} \} = - \left( \displaystyle\frac{ 2 \pi}{\beta} \right)^2 \left( \bar{z}^2 \bar{T}(\bar{z}) - \frac{1}{48} \right).
\end{align}

\subsubsection{Hamiltonian on a Cylinder (another convention)}

With this convention, the Hamiltonian on a cylinder is given by:
\thm{
  \textbf{Hamiltonian on a Cylinder (another convention)}

  The Hamiltonian of a general CFT on a cylinder with circumference $\beta$ is given by:
  \begin{align}
    H^C = \frac{2\pi }{\beta} \left( L_0 - \bar{L}_0 - \frac{c}{12} \right).
  \end{align}
}
Proof: Similarly:
\begin{align}
  H^C = \int_0^{-\beta} d\sigma \, T_{tt}, \quad T_{tt} = -T_{ww} - T_{\bar{w}\bar{w}} =  \frac{1}{2\pi} \left( T^C(w) + \bar{T}^C(\bar{w}) \right).
\end{align}
\YL{[This is purely my own calculation, I have not find a reference doing this yet, so I have to double check it later.]}
Note that in this convention, the reltaion between $ T_{tt} $ and the E-M tensor components is different and the intergral contour is different. Thus, we have:
\begin{align}
  H^C &= \int_0^{-\beta} d\sigma \, \frac{1}{2\pi} \left( T^C(w) + \bar{T}^C(\bar{w}) \right) \\
      & = \displaystyle\frac{2 \pi}{\beta} \left( L_0 - \bar{L}_0 - \frac{c}{12} \right).
\end{align}

\rmk{
  Note that though we start from a different convention, the final form of the Hamiltonian is exactly the same with the previous one. 
}






% section Majorana CFT (end)

\newpage
\section{Majorana BCFT}\label{sec:Free Majorana Fermion BCFT} % (fold)
As a standard way of dealing with boundary CFT, we consider a Majorana CFT on a compatified strip with two boundaries or equivalently a cylinder. We study the boundary state of this theory. For detailed calculation, we refer to \cite{nepomechieConsistentSuperconformalBoundary2001a}. However, I might point out many discussions in these papers are naive and needs further clarification. Thus, I include some clarification here with the insight from \cite{ebisuFermionizationConformalBoundary2021, bachasWorldsheetExtensionOddZ2012, horiNotesBosonizationBoundary2009, chatterjeeExactPartitionFunction1995a}. 


\subsection{Boundary Conditions of Majorana CFT}

There are two conformally invariant boundary conditions for Majorana CFT. We then discuss how to write them down on a general manifold with boundaries. And then specify the manifold to a compatified strip with two boundaries. The idea is organized in \cite{chatterjeeExactPartitionFunction1995a}


\subsubsection{Boundary Conditions on UHP}

It is well know that the Majorana CFT has two conformal boundary conditions on the UHP:
\begin{itemize}
  \item \textbf{Free $ (+) $ Boundary Condition:} $\psi=\bar{\psi}$ at the boundary.
  \item \textbf{Fixed $ (-) $ Boundary Condition:} $\psi=-\bar{\psi}$ at the boundary.
\end{itemize}
It is belived that the free boundary condition corresponds to free spin boundary condition of Ising model while the fixed boundary condition corresponds to fixed spin boundary condition of Ising model.

\subsubsection{Boundary Conditions on General Manifold}

We then want to generalize these conditions to general 2-manifolds with boundaries. Consider a general manifold with boundary $ \mathcal{D} $ (not just the UHP) with boundaries $ \partial\mathcal{D} = \mathcal{B} = \cup_j \mathcal{B}_j  $ that are parametrized as $ \mathcal{B}_j = ( Z_j(t), \bar{Z}_j(t) ) $ paramatrized by $ t $. On the boundary $ \mathcal{B}_j $ the field is $ \psi(t) = \psi(Z_j(t)), \bar{\psi}(t) = \bar{\psi}(\bar{Z}_i(t)) $. We define the tangent vectors on the boundary as:
\begin{align}
  e_j(t)=\partial_{t}Z_j(t),\quad\bar{e}_j(t)=\partial_{t}\bar{Z}_j(t)
\end{align}
that satisfy:
\begin{itemize}
  \item normalization condition: $ e(t)\bar{e}(t)=1 $
  \item orientation condition: $ (ie,-i\bar{e}) $ vector shall be normal to the boundary and point inward the domain $ \mathcal{D} $
\end{itemize}
Thus, these two conformal boundary conditions on UHP can be generalized to:
\begin{itemize}
  \item \textbf{Free Boundary Condition:} $ [ e_j^{\frac12}\psi - \bar{e}_j^{\frac12}\bar{\psi} ]_{\mathcal{B}_j} = 0 $
  \item \textbf{Fixed Boundary Condition:} $ [ e_j^{\frac12}\psi + \bar{e}_j^{\frac12}\bar{\psi} ]_{\mathcal{B}_j} = 0 $
\end{itemize}

\subsubsection{Boundary Conditions on a Strip}

Then we focus on our main topic: Majorana CFT on a compatified strip with two boundaries. We consider the strip $ \mathcal{S} = \{ z = x + iy | x \in [0,L], y \in \mathbb{R} \} $ with two boundaries at $ x=0 $ and $ x=L $. Then strip is compatified by identifying $ y \sim y + \beta $. 
\begin{figure}[H]
  \centering
  \includegraphics[width=0.32\textwidth]{assets/cylindermanifold.png}
  \caption{Compatified strip with two boundaries at $ x=0 $ and $ x=L $, compatified by identifying $ y \sim y + \beta $.}
  \label{fig:cylindermanifold}
\end{figure}
On this manifold, we can write down the boundary at $ x=0 $ and $ x=L $ as:
\begin{align}
 & Z_1(t) = -it,\quad \bar{Z}_1(t) = it \quad \text{at } x=0\\
 & Z_2(t) = L + it,\quad \bar{Z}_2(t) = L - it \quad \text{at } x=L
\end{align}
with this orientation and normalization we have:
\begin{align}
  e_1(t) = -i,\quad \bar{e}_1(t) = i \quad &\text{at } x=0\\
  e_2(t) = i,\quad \bar{e}_2(t) = -i \quad &\text{at } x=L
\end{align}
Thus, the action of Majorana CFT on this compatified strip with two boundaries is:
\begin{itemize}
  \item \textbf{Free $ (+) $ Boundary Condition:}
    \begin{align}
      \psi(0,y) = i \bar{\psi}(0,y), \quad \psi(L,y) = -i \bar{\psi}(L,y)
    \end{align}
  \item \textbf{Fixed $ (-) $ Boundary Condition:}
    \begin{align}
      \psi(0,y) = -i \bar{\psi}(0,y), \quad \psi(L,y) = i \bar{\psi}(L,y)
    \end{align}
\end{itemize}



\subsection{Open Channel Partition Function}

With the boundary conditions in hand we can canonically quantize the Majorana CFT on this compatified strip and get the open channel partition function. Here we will point out some subtleties that are not discussed in some literatures. There ain't a well-defined fermion parity opertor in the open channel. Thus, we need to be careful when we do this. We do this with two steps:
\begin{itemize}
  \item Quantize the theory on a infinite strip with two boundaries, find out the Hilbert Space \textbf{This is Extremely Nontrivial}
    \item Compatify the strip and get the partition function
\end{itemize}

\subsubsection{Doubling Trick}

We aren't familiar with the mode expansion on a strip and quantization on a strip, however, we have already discussed the theory on a cylinder in \cref{sec:Majorana CFT on a Cylinder}. Thus we may use a trick to double a strip and get a cylinder. This is the so-called doubling trick. 

We can write the two boundary conditions in a compact form:
\begin{align}
  \psi(0,y)+ai\bar{\psi}(0,y)=0\quad \psi(L,y)-bi\bar{\psi}(L,y)=0,
\end{align}
\begin{itemize}
  \item when $ a = b =1 $ we have the fixed $ (-) $ boundary condition at both boundaries
  \item when $ a = b =-1 $ we have the free $ (+) $ boundary condition at both boundaries
  \item when $ a = -b =1 $ we have the mixed boundary condition: fixed at $ x=0 $ and free at $ x=L $
  \item when $ a = -b =-1 $ we have the mixed boundary condition: free at $ x=0 $ and fixed at $ x=L $
\end{itemize}
then we define a anxilary field $ \bar{\psi}_0(x,y) = - a \bar{\psi}(x,y) $. Thus the boundary condition can be written as:
\begin{align}
  \psi(0,y)= i\bar{\psi}_0(0,y)\quad \psi(L,y)= - i \displaystyle\frac{b}{a} \bar{\psi}_0(L,y)
\end{align}
Then we can extend the field $ \psi(x,y) $ to the region $ x \in [-L,L] $ by defining:
\defi{
  \textbf{Doubled Theory}

  We define a field on the strip $ [-L,L] \times \mathbb{R} $ with:
  \begin{align}
  \left.\Psi(x,y)=\left\{\begin{array}{lll}\psi(x,y)&\mathrm{if}&x\in[0,L]\\i\bar{\psi}_0(-x,y)&\mathrm{if}&x\in[-L,0]\end{array}\right.\right.
  \end{align}
The original action can be written as:
\begin{align}
  S^S = \displaystyle\frac{1}{2\pi} \int_0^L dx \int_{-\infty}^\infty dy\  [\psi\bar{\partial}\psi + \bar{\psi}\partial\bar{\psi}] =S^C =  \displaystyle\frac{1}{2\pi} \int_{-L}^L dx \int_{-\infty}^\infty dy \  \Psi\bar{\partial}\Psi = \displaystyle\frac{1}{2 \pi} \int_z \Psi \bar{\partial} \Psi
\end{align}
where $ z = x+iy $. 
}
We can see that at the $ x= 0 $ boundary, the field $ \Psi(x,y) $ is continuous. While at the $ x=L $ boundary, we have:
\begin{align}
  \Psi(L,y)=-\frac{b}{a}\Psi(-L,y)
\end{align}
Thus we can understand the theory of two decoupled field $ \psi(x,y) $ and $ \bar{\psi}(x,y) $ on a strip with boundary conditions as a theory of a single ifield $ \Psi(x,y) $ on a circle of circumference $ 2L $. 
\begin{itemize}
  \item same boundary conditions ($ a=b $) correspond to the Neveu-Schwarz (NS) sector with anti-periodic boundary condition: $ \Psi(L,y)=-\Psi(-L,y) $ 
  \item different boundary conditions ($ a=-b $) correspond to the Ramond (R) sector with periodic boundary condition: $ \Psi(L,y)=\Psi(-L,y) $
\end{itemize}

Eventually, we learn that the Majorana CFT on a strip is equivalent to a Chiral Majorana CFT on a cylinder with NS and R boundary conditions. We then can perform the quantization on a cylinder to get the quantized theory.

\subsubsection{Canonical Quantization}

A canonical method of quantizing on a cylinder is transform it onto a complex plane, do the radial quantization and then transform it back to the cylinder. We have already discussed this procedure in \cref{sec:Majorana CFT on a Cylinder}. Here we just list the results:
\begin{itemize}
  \item \textbf{Mode Expansion:} the mode expansion of the field $ \Psi(z) $ on the cylinder is:
    \begin{align}
      \Psi(z) = \sqrt{\frac{\pi}{iL}} \sum_k b_k e^{-\frac{\pi}{L}ik z}, \quad z = x + i y
    \end{align}
    Where the commutation relation of the modes is:
    \begin{align}
      \{ b_k, b_{k'} \} = \delta_{k+k',0}
    \end{align}
For the NS sector (Same BC), $ k \in \mathbb{Z} + \frac{1}{2} $ while for the R sector (Different BC), $ k \in \mathbb{Z} $.
\rmk{
  In fact there is a little convention problem here for x is defined on $ [-L,L] $we need an extra global phase factor. Yet it is a symmetry of the theory thus we ignore it here.
}
  \item \textbf{Hamiltonian:} The Hamiltonian of the theory is:
    \begin{align}\label{eq:Majorana BCFT Hamiltonian}
      H_{open} = \frac{\pi}{L} \left( \sum_{k>0} k b_{-k} b_k +E_0 \right) 
    \end{align}
    where the zero-point energy $ E_0 $ is:
    \begin{align}
      E_0 = \left\{\begin{array}{ll}-\frac{1}{48}&\mathrm{NS\ sector}\\\frac{1}{24}&\mathrm{R\ sector}\end{array}\right.
    \end{align}
    Proof: This is just the Hamiltonian on a cylinder with circumference $ \beta = 2L $ and central charge $ c=\frac{1}{2} $ see \cref{sec:Majorana CFT on a Cylinder}. Yet only one chiral part.
    \rmk{
      In fact, I have tried, in this standard convention, the Hamiltonian comes directly from legendre transformation of the chiral Lagrangian.
    }
\end{itemize}

\subsubsection{Zero Mode and Ground State Degeneracy(?)}

With a chiral Majorana fermion on a cylinder, when considering the Ramond sector, we know that the zero mode only has one-dimensional Clifford algebra:
\begin{align}
  \{ b_0, b_0 \} = 1 \quad  b_0^2 = \frac{1}{2}
\end{align}
Naively, we may think that we only have one ground state $ |0\rangle_R $ satisfying $ b_k |0\rangle = 0 $ for $ k>0 $. The zero mode $ b_0 $ then acts on the ground state as:
\begin{align}
  b_0 = \displaystyle\frac{1}{\sqrt{2}} \ket{0}_R \quad \text{or}\quad b_0 = -\displaystyle\frac{1}{\sqrt{2}} \ket{0}_R
\end{align}
However, things get worse when we try to define the fermion parity operator $ (-1)^F $. This is due to a mathematical theorem:
\thm{
  \textbf{Odd Dimensional Clifford Algebra Grading}

  Clifford algebra with odd number of generators doesn't admit a $ \mathbb{Z}_2 $ grading, ie there is no way to define a fermion parity operator $ (-1)^F $ that anti-commutes with all the generators.
}

\subsubsection{Boundary Fermion}

However, we \textbf{manually wish to define} a fermion parity operator $ (-1)^F $ and \textbf{we wish to} have a ground state degeneracy. The motivations are:
\begin{itemize}
  \item We wish to discuss Bosonization which needs a well-defined fermion parity operator to proceed. 
    \item We wish to have 4 different partition functions just as the cylinder case.
\end{itemize}
Thus, we play the following trick:
\thm{
  \textbf{Boundary Fermion Trick}

  We introduce an auxilary boundary free Majorana fermion $ \xi(t) $ with dynamical terms only on the boundary of one of the boundary conditions. we have two choices:
  \begin{itemize}
    \item Add $ \xi(t) $ on the free boundary condition boundary
    \item Add $ \xi(t) $ on the fixed boundary condition boundary
  \end{itemize}
  The action of the boundary fermion is:
  \begin{align}
    S_\xi = \frac{i}{4} \int dt\ \xi(t) \frac{d}{dt} \xi(t)
  \end{align}
  Under canonical quantization, the boundary fermion satisfies the anti-commutation relation:
\begin{align}
  \{ \xi(t), \xi(t') \} = 2\delta(t-t')
\end{align}
This defines a extra boundary Hilbert space $ \mathcal{H}_B = \text{Span}\{ |0\rangle_B, \xi |0\rangle_B \} $.
}
With this boundary fermion, we can define the fermion parity operator and have a ground state degeneracy.
\begin{itemize}
  \item \textbf{Case 1: Single Boundary Fermion} when we have different boundary conditions at the two boundaries, there exists a single boundary fermion $ \xi(t) $ on the free boundary only.
\end{itemize}
The boundary fermion serves as a quantum mechanical fermion degree of freedom on the boundary. Apparently, it has nothing to do with the bulk Majorana fermion $ \Psi(z) $. However, it provides an extra zero mode $ \xi $ that satisfies:
\begin{align}
  \{ \xi, \xi \} = 2 \quad \xi^2 = 2 , \quad \{ \xi, b_k \} = 0 \ \forall\ k
\end{align}
We certainly have a ground state degeneracy, to represent the Clifford algebra generated by $ b_0 $ and $ \xi $:
\begin{align}
  \{ b_0, \xi \} = 0, \quad b_0^2 = \frac{1}{2}, \quad \xi^2 = 2
\end{align}
Thus, we play exactly the same trick as in \cref{sec:Fermion Parity in Majorana CFT} to define the degenerate ground states. And the Hilbert space is given by acting the bulk negative modes $ b_{-k} $ ($ k= 1,2,3... $) on the degenerate ground states.

\begin{itemize}
  \item \textbf{Case 2: Double Boundary Fermions} when we have same boundary conditions at the two boundaries, there might exist two boundary fermions $ \xi_1(t) $ and $ \xi_2(t) $ on both boundaries.
\end{itemize}
In this case, the bulk Majorana fermion has no zero mode thus there is no ground state degeneracy from the bulk. However, the two boundary fermions $ \xi_1 $ and $ \xi_2 $ generate a two-dimensional Clifford algebra:
\begin{align}
  \{ \xi_i, \xi_j \} = 2\delta_{ij}, \quad \xi_i^2 = 2 , \quad \{ \xi_i, b_k \} = 0 \ \forall\ k
\end{align}
\YL{[I came up with this idea recently, I haven't find any reference discussing this case yet. I think it is true]}
We then can play the same trick as in \cref{sec:Zero Mode and Ground State Degeneracy in Majorana CFT} to define the degenerate ground states. And the Hilbert space is given by acting the bulk negative modes $ b_{-k} $ ($ k= 1/2,3/2... $) on the degenerate ground states.

\subsubsection{Fermion Parity Operator}

\begin{itemize}
  \item \textbf{Same boundary conditions with no boundary fermions}
\end{itemize}
For this case, there is no zero mode thus we can define the fermion parity operator as usual:
\begin{align}
  (-1)^F = (-1)^{F_\mathrm{nz}} = (-1)^{\sum_{k>0} b_{-k} b_k}
\end{align}
\begin{itemize}
  \item \textbf{Different boundary conditions with one boundary fermions}
\end{itemize}
In this case, we have two zero modes $ b_0 $ and $ \xi $ that generate a two-dimensional Clifford algebra.
We play the exact same trick to define the degenerate ground states as we do in \cref{sec:Zero Mode and Ground State Degeneracy in Majorana CFT}. We define the fermion parity operator as:
\begin{align}
  (-1)^{F} =
\begin{cases}
 - \sqrt{2}i\xi b_0(\boldsymbol{-}1)^{F_\mathrm{nz}}.\\ 
  \sqrt{2}i\xi b_0(\boldsymbol{-}1)^{F_\mathrm{nz}}.\\ 
\end{cases}
\end{align}
We have two choices here, yet they matter no diference in the final partition function. We may just pick any one of them. We commonly use $ (-1)^F = \sqrt{2}i\xi b_0(-1)^{F_\mathrm{nz}} $.
\begin{itemize}
  \item \textbf{Same boundary conditions with two boundary fermions}
\end{itemize}
In this case, we have two boundary fermions $ \xi_1 $ and $ \xi_2 $ that generate a two-dimensional Clifford algebra.
We play the exact same trick to define the degenerate ground states as we do in \cref{sec:Zero Mode and Ground State Degeneracy in Majorana CFT}. We define the fermion parity operator as:
\begin{align}
  (-1)^{F} =
\begin{cases}
 - i\xi_1 \xi_2(\boldsymbol{-}1)^{F_\mathrm{nz}}.\\ 
  i\xi_1 \xi_2(\bold{-}1)^{F_\mathrm{nz}}.\\ 
\end{cases}
\end{align}
We have two choices here, yet they matter no diference in the final partition function. We may just pick any one of them. We commonly use $ (-1)^F = i\xi_1 \xi_2(-1)^{F_\mathrm{nz}} $.

\subsubsection{0A Theory and 0B Theory}

We have claimed that there are two ways of adding the boundary fermion: on the free boundary or on the fixed boundary. We name these two theories as:
\begin{itemize}
  \item \textbf{0A Theory:} adding the boundary fermion on the free $(+)$ boundary condition boundary.
  \item \textbf{0B Theory:} adding the boundary fermion on the fixed $(-)$ boundary condition boundary.
\end{itemize}
As we will see eventually, these two theories correspond to the choice of type 0A fermion parity and type 0B fermion parity in the closed channel.
\textbf{From now on, we will focus on the 0A theory only. The 0B theory can be treated similarly.}


\subsubsection{Open Channel Hilbert Space}

With all the ingredients in hand, we can finally calculate the open channel partition function of Majorana CFT on a compatified strip with two boundaries. We first list out all sort of boundary conditions we can have and give them a name, we call \textbf{free boundary condition} as $ + $ and \textbf{fixed boundary condition} as $ - $. Thus we have 4 different types of boundary conditions:
\begin{itemize}
  \item $(+,+)$ boundary conditions: both boundaries have free boundary conditions. There are two boundary fermions, one on each boundary for the 0A theory.

  \item $(-,-)$ boundary conditions: both boundaries have fixed boundary conditions. There is no boundary fermion for the 0A theory.

  \item $(+,-)$ boundary conditions: the boundary at $x=0$ is free and the boundary at $x=L$ is fixed. There is a boundary fermion only on the free boundary.

  \item $(-,+)$ boundary conditions: the boundary at $x=0$ is fixed and the boundary at $x=L$ is free. There is a boundary fermion only on the free boundary.
\end{itemize}

The Hilbert spaces corresponding to the four types of boundary conditions are:
\begin{itemize}
  \item $(+,+)$ boundary conditions: the chiral NS sector Hilbert space tensored with two boundary-fermion Hilbert spaces,
  \[
     \mathcal{H}_{(++)}=\mathcal{H}^C_{\mathrm{NS}}
       \otimes \mathcal{H}_B^{(1)} \otimes \mathcal{H}_B^{(2)} .
  \]
  It is generated by acting negative modes $b_{-k}$ ($k\in\mathbb{Z}+\frac12>0$) on the degenerate ground states. The Hilbert space has twice the size of the $(-,-)$ case.

  \item $(-,-)$ boundary conditions: the chiral bulk NS sector Hilbert space,
  \[
     \mathcal{H}_{(--)}=\mathcal{H}^C_{\mathrm{NS}},
  \]
  generated by acting negative modes $b_{-k}$ ($k\in\mathbb{Z}+\frac12$) on the NS ground state $\ket{0}_{\mathrm{NS}}$.

  \item $(+,-)$ boundary conditions: the bulk R sector Hilbert space tensored with a single boundary-fermion Hilbert space,
  \[
     \mathcal{H}_{(+-)}=\mathcal{H}^C_{\mathrm{R}}\otimes\mathcal{H}_B .
  \]
  It is generated by acting negative modes $b_{-k}$ ($k\in\mathbb{Z}>0$) on the degenerate R ground states. This Hilbert space is twice as large as the $(+,+)$ case.

  \item $(-,+)$ boundary conditions: identical in structure to the $(+,-)$ case,
  \[
     \mathcal{H}_{(-+)}=\mathcal{H}^C_{\mathrm{R}}\otimes\mathcal{H}_B .
  \]
\end{itemize}


\subsubsection{Open Channel Partition Functions}

Then we can finally calculate the partition functions in the open channel. The partition function is defined as:
\defi{
  \textbf{Open Channel Partition Function}

  The open channel partition function is defined as:
\begin{align}
  Z = \mathrm{Tr}_{\mathcal{H}} \left[ e^{-\beta H} \right]
\end{align}
Yet for fermions we can compatify with periodic or anti-periodic boundary condition, thus we have 2 different partition functions:
\begin{align}
  Z_{-} = \mathrm{Tr}_{\mathcal{H}} \left[ (-1)^F e^{- \beta H} \right] \quad  Z_{+} = \mathrm{Tr}_{\mathcal{H}} \left[ e^{-\beta H} \right]
\end{align}
We commonly use $ q = e^{-\pi \beta /L} $ for simpler notation. 
}
The two kinds of partition functions are equivalent to compatifying the time direction with anti-periodic and periodic boundary conditions respectively. This can be shown easily using the operter formalism:
\begin{align}
  \mathrm{Tr}[\psi(\tau+\beta)e^{-\beta H}]=\mathrm{Tr}[e^{\beta H}\psi(\tau)e^{-\beta H}e^{-\beta H}]=\mathrm{Tr}[\psi(\tau)e^{-\beta H}].
\end{align}
If we then plug in the fermion parity operator $ (-1)^F $, we have:
\begin{align}
  \mathrm{Tr}[\psi(\tau+\beta)(-1)^Fe^{-\beta H}]=-\mathrm{Tr}[\psi(\tau)(-1)^Fe^{-\beta H}].
\end{align}
The $ -1 $ is taken when anti-commuting $ \psi $ passes through $ (-1)^F $. Thus, we see that the partition function with $ (-1)^F $ inserted corresponds to anti-periodic boundary condition in time direction while the one without $ (-1)^F $ corresponds to periodic boundary condition in time direction.

We then calculate the partition functions for the 4 types of boundary conditions, we review that the Hamiltonian is given by \cref{eq:Majorana BCFT Hamiltonian}. Thus, we have the following results, and they can be explicitly written in terms of Ising model characters as well:
% \begin{itemize}
%   \item $(-,-)$ boundary conditions partition functions:
%     \begin{align}
%       Z_{(--),-} 
%       &= \mathrm{Tr}_{\mathcal{H}_{(++)}} \!\left[ (-1)^{F}\,
%       e^{-\beta H} \right]
%       = q^{-\frac{1}{48}} \prod_{n=0}^\infty \left( 1 - q^{n+\frac{1}{2}} \right) = \chi_0(q) - \chi_{1/2}(q), \\
%       Z_{(--),+} 
%       &= \mathrm{Tr}_{\mathcal{H}_{(++)}} \!\left[
%       e^{-\frac{\pi \beta}{L}\left(L_0 - \frac{c}{24}\right)} \right]
%       = q^{-\frac{1}{48}} \prod_{n=0}^\infty \left( 1 + q^{n+\frac{1}{2}} \right) = \chi_0(q) + \chi_{1/2}(q).
%     \end{align}
%   \item $(+,+)$ boundary conditions partition functions:
%     \begin{align}
%       Z_{(++),-} 
%       &= \mathrm{Tr}_{\mathcal{H}_{(--)}} \!\left[ (-1)^{F}\,
%       e^{-\beta H} \right]
%       = 0 \\ 
%       Z_{(++),+} 
%       &= \mathrm{Tr}_{\mathcal{H}_{(--)}} \!\left[
%       e^{-\frac{\pi \beta}{L}\left(L_0 - \frac{c}{24}\right)} \right]
%       = 2 q^{-\frac{1}{48}} \prod_{n=0}^\infty \left( 1 + q^{n+\frac{1}{2}} \right)= 2\left(\chi_0(q) + \chi_{1/2}(q)\right)
%     \end{align}
%     We have zero for the degenerate ground states cancel each other out in the $ Z_{(--),-} $ case.
%   \item $(+,-)$ and $ (-,+) $ boundary conditions partition functions:
%     \begin{align}
%       Z_{(+-),-} = Z_{(-+),-}
%       &= \mathrm{Tr}_{\mathcal{H}_{(+-)}} \!\left[ (-1)^{F}\,
%       e^{-\beta H} \right]
%       = 0 \\ 
%       Z_{(+-),+} = Z_{(-+),+}
%       &= \mathrm{Tr}_{\mathcal{H}_{(+-)}} \!\left[
%       e^{-\frac{\pi \beta}{L}\left(L_0 - \frac{c}{24}\right)} \right]
%       = 2 q^{\frac{1}{24}} \prod_{n=1}^\infty \left( 1 + q^{n} \right) = 2\chi_{1/16}(q)
%     \end{align}
%     We have zero for the degenerate ground states cancel each other out in the $ Z_{(+-),-} $ case.
% \end{itemize}
% Under Modular transformation $ \tilde{q} = e^{-4\pi L/\beta} $, thus we can also write the partition functions in terms of $ \tilde{q} $:
% \begin{table}[H]
% \centering
% \renewcommand{\arraystretch}{1.6}
% \begin{tabular}{ccc}
% \hline
% Partition Function & $ q $ character & $ \tilde{q} $ character \\ 
% \hline\hline
% $Z_{(++),-}$ 
% & $\chi_0(q) - \chi_{1/2}(q)$
% & $\sqrt{2}\,\chi_{1/16}(\tilde q)$
% \\
% $Z_{(++),+}$ 
% & $\chi_0(q) + \chi_{1/2}(q)$
% & $\chi_0(\tilde q) + \chi_{1/2}(\tilde q)$
% \\
% $Z_{(--),-}$ 
% & $0$
% & $0$
% \\
% $Z_{(--),+}$ 
% & $2\big(\chi_0(q) + \chi_{1/2}(q)\big)$
% & $2\big(\chi_0(\tilde{q}) + \chi_{1/2}(\tilde{q})\big)$
% \\
% $Z_{(+-),-}$ 
% & $0$
% & $0$
% \\
% $Z_{(+-),+}$ 
% & $2\,\chi_{1/16}(q)$
% & $\sqrt{2}\big(\chi_0(\tilde q) - \chi_{1/2}(\tilde q)\big)$
% \\
% \hline\hline
% \end{tabular}
% \caption{Free fermion open channel partition functions with different boundary conditions in both $ q $ and $ \tilde{q} $ characters.}
% \end{table}

\begin{itemize}
  \item $(-,-)$ boundary conditions partition functions:
    \begin{align}
      Z_{(--),-} 
      &= \mathrm{Tr}_{\mathcal{H}_{(--)}} \!\left[ (-1)^{F}\,
      e^{-\beta H} \right]
      = q^{-\frac{1}{48}} \prod_{n=0}^\infty \left( 1 - q^{n+\frac{1}{2}} \right)
      = \chi_0(q) - \chi_{1/2}(q), \\
      Z_{(--),+} 
      &= \mathrm{Tr}_{\mathcal{H}_{(--)}} \!\left[
      e^{-\frac{\pi \beta}{L}\left(L_0 - \frac{c}{24}\right)} \right]
      = q^{-\frac{1}{48}} \prod_{n=0}^\infty \left( 1 + q^{n+\frac{1}{2}} \right)
      = \chi_0(q) + \chi_{1/2}(q).
    \end{align}

  \item $(+,+)$ boundary conditions partition functions:
    \begin{align}
      Z_{(++),-} 
      &= \mathrm{Tr}_{\mathcal{H}_{(++)}} \!\left[ (-1)^{F}\,
      e^{-\beta H} \right]
      = 0 \\ 
      Z_{(++),+} 
      &= \mathrm{Tr}_{\mathcal{H}_{(++)}} \!\left[
      e^{-\frac{\pi \beta}{L}\left(L_0 - \frac{c}{24}\right)} \right]
      = 2 q^{-\frac{1}{48}} \prod_{n=0}^\infty \left( 1 + q^{n+\frac{1}{2}} \right)
      = 2\left(\chi_0(q) + \chi_{1/2}(q)\right)
    \end{align}
    Again, the degenerate ground states cancel each other in $Z_{(++),-}$.

  \item $(-,+)$ and $(+,-)$ boundary conditions partition functions:
    \begin{align}
      Z_{(-+),-} = Z_{(+-),+}
      &= \mathrm{Tr}_{\mathcal{H}_{(-+)}} \!\left[ (-1)^{F}\,
      e^{-\beta H} \right]
      = 0 \\ 
      Z_{(-+),+} = Z_{(+-),-}
      &= \mathrm{Tr}_{\mathcal{H}_{(-+)}} \!\left[
      e^{-\frac{\pi \beta}{L}\left(L_0 - \frac{c}{24}\right)} \right]
      = 2 q^{\frac{1}{24}} \prod_{n=1}^\infty \left( 1 + q^{n} \right)
      = 2\chi_{1/16}(q)
    \end{align}
    As before, the degenerate ground states cancel in $Z_{(-+),-}$.
\end{itemize}

Under the modular transformation $\tilde q = e^{-4\pi L / \beta}$, the partition functions become:
\begin{table}[H]
\centering
\renewcommand{\arraystretch}{1.6}
\begin{tabular}{ccc}
\hline
Partition Function & $ q $ character & $ \tilde{q} $ character \\ 
\hline\hline

$Z_{(--),-}$ 
& $\chi_0(q) - \chi_{1/2}(q)$
& $\sqrt{2}\,\chi_{1/16}(\tilde q)$
\\

$Z_{(--),+}$ 
& $\chi_0(q) + \chi_{1/2}(q)$
& $\chi_0(\tilde q) + \chi_{1/2}(\tilde q)$
\\

$Z_{(++),-}$ 
& 0
& 0
\\

$Z_{(++),+}$ 
& $2\big(\chi_0(q) + \chi_{1/2}(q)\big)$
& $2\big(\chi_0(\tilde{q}) + \chi_{1/2}(\tilde{q})\big)$
\\

$Z_{(-+),-}$ 
& 0
& 0
\\

$Z_{(-+),+}$ 
& $2\,\chi_{1/16}(q)$
& $\sqrt{2}\big(\chi_0(\tilde q) - \chi_{1/2}(\tilde q)\big)$
\\

\hline\hline
\end{tabular}
\caption{Free fermion open-channel partition functions in both $ q $ and $ \tilde{q} $ characters.}
\label{tab:open channel partition functions}
\end{table}



\subsection{Closed Channel Partition Function}
After the calculation of the open channel partition function, we want to identify the boundary states that cooresponds to the two boundary conditions (in fact 4 boundary states due to that we consider 2 ways of compatifying the strip). Yet the discussion on BCFT are mostly on a Bosonic state, we need to define a proper boundary state formalism for Fermions. 

\subsubsection{Fermionic Boundary State Formalism}

We try to generalize the definition of boundary state in Bosonic CFT to Fermionic CFT. An obvious observation tells us different ways of compatifying the strip correspond to different boundary states. Thus it is natural to define:

\defi{
  \textbf{Fermionic Boundary State}

  We define the boundary state $ |\alpha, NS \rangle $ and  $ \ket{\alpha,R} $ in Fermionic CFT as the state that satisfies: test
  \begin{align}
  &\bra{\alpha,NS}  e^{-L_1 H_{\mathrm{closed}}} \ket{\beta,NS}
  \;=\; \mathrm{Tr}_{\mathcal{H}_{\alpha\beta}}\!\left( e^{-L_2 H_{\mathrm{open}}} \right) = Z_{(\alpha\beta),+},
    \\
  &\bra{\alpha,R } e^{-L_1 H_{\mathrm{closed}}}  \ket{\beta,\mathrm{R}}
  \;=\; \mathrm{Tr}_{\mathcal{H}_{\alpha\beta}}\!\left( (-1)^F e^{-L_2 H_{\mathrm{open}}} \right) = Z_{(\alpha\beta),-}.
  \end{align}
}
Thus for every boundary condition $ \alpha, \beta $, we have two boundary states $ |\alpha, NS \rangle $ and  $ \ket{\alpha,R} $. Here, we will manage to find the boudnary states that correspond to the 4 types of boundary conditions we have discussed in the open channel.


\subsubsection{Canonical Quantization in Closed Channel}

The closed channel is merely the Majorana CFT on a cylinder with circumference $ \beta $ and length $ L $. We have already discussed the canonical quantization of Majorana CFT on a cylinder in \cref{sec:Majorana CFT on a Cylinder}. To be more convenient, we just use the standard transformation from the complex plane to the cylinder, and here's the results:
\begin{itemize}
  \item \textbf{Mode Expansion:} the mode expansion of field $ \psi(z) $ and $ \bar{\psi}(\bar{z}) $ is given by:
    \begin{align}
        \psi^C(z)=\sqrt{\frac{2\pi}{\beta}}\sum_ka_ke^{-2\pi k z/\beta} \quad \bar{\psi}^C(\bar{z})=\sqrt{\frac{2\pi}{\beta}}\sum_k\bar{a}_ke^{-2\pi k\bar{z}/\beta}
    \end{align}
    where now $ z = x - iy $, now $ x $ direction is the time direction. The commutation relation of the modes is:
    \begin{align}
      \{ a_k, a_{k'} \} = \delta_{k+k',0}, \quad \{ \bar{a}_k, \bar{a}_{k'} \} = \delta_{k+k',0}
    \end{align}
  \item \textbf{Hamiltonian:} The Hamiltonian of the theory is:
    \begin{align}
      H_{closed}=\frac{2\pi}{\mathsf{\beta}}\left(2E_0+\sum_{k>0}k\left(a_{-k}a_k+\bar{a}_{-k}\bar{a}_k\right)\right)
    \end{align}
    Similarly:
    \begin{align}
      E_0 = \left\{\begin{array}{ll}-\frac{1}{48}&\mathrm{NS\ sector}\\ \frac{1}{24} &\mathrm{R\ sector}\end{array}\right.
    \end{align}
  \item \textbf{Fermion Parity:} for a free fermion on a cylinder we know that fermion parity operators are as follows:
    \begin{align}
      (-1)^F = \left\{\begin{array}{ll} (-1)^{F_\mathrm{nz}} & \mathrm{NS\ sector}\\ \pm 2 ia_0\bar{a}_0(-1)^{F_\mathrm{nz}} & \mathrm{R\ sector}\end{array}\right.
    \end{align}
    Here $ F_\mathrm{nz} = \sum_{k>0} a_{-k} a_k + \sum_{k>0} \bar{a}_{-k} \bar{a}_k $ is the non-zero mode fermion number operator. The $ \pm $ in the R sector corresponds to the two different choices of fermion parity operator. We then will see that when considering a 0A theory, we have to use the $ -2i a_0 \bar{a}_0 $ choice or so called type 0A fermion parity.
\end{itemize}
\rmk{for the modes in the closed channel, we use $ a_k, \bar{a}_k $ to show that it has nothing to do with the modes in the open channel $ b_k $.}

\subsubsection{Gluing Conditions and Ishibashi States}

Imposing the boundary conditions shall turm into gluing condition for the parent CFT. We now clarify the procedure and give out the results. First of all, the parent CFT is defined on a cylinder with coordinate $ z = x + iy $. It is quantized by radial quantization on a plane with coordinate $ \xi $. Till now this have nothing to do with a BCFT just a standard CFT on a cylinder. 

However, we have to relate it to a BCFT with boundary and project the boundary torwand a circle on $ \xi $ plane and which will lead to imposing the gluing condition. Thus we take the following standard procedure, we consider a Majorana CFT on a UHP with coordinate $ w = x+ iy $, where $ z = \bar{z} $ is the boundary. Then we have two types of boundary conditions:
\begin{itemize}
  \item \textbf{Free $ (+) $ Boundary Condition:} this boundary condition requires:
    \begin{align}
      \psi(z) = \bar{\psi}(\bar{z}) \quad \text{at } z = \bar{z}.
    \end{align}
  \item \textbf{Fixed $ (-) $ Boundary Condition:} this boundary condition requires:
    \begin{align}
      \psi(z) = - \bar{\psi}(\bar{z}) \quad \text{at } z = \bar{z}.
    \end{align}
\end{itemize}

\begin{itemize}
  \item \textbf{Inner Circle Condition}
\end{itemize}
Then we make a conformal transformation to transform the $ z = \bar{z} \in \mathbb{R}_+ $ part to the inner circle of the $ \xi $ plane. A commonly used conformal transformation is:
\begin{align}
  \xi = e^{\frac{2\pi i}{\beta}\ln z} \quad \bar{\xi} = e^{-\frac{2\pi i}{\beta}\ln \bar{z}}.
\end{align}
Under this transformation, the majorana fermion transforms as:
\begin{align}
  \psi^P(\xi) = \left( \frac{2\pi}{\beta} \displaystyle\frac{\xi}{w} \right)^{1/2} \psi^H(w), \quad \bar{\psi}^P(\bar{\xi}) = \left( -\frac{2\pi}{\beta} \displaystyle\frac{\bar{\xi}}{\bar{w}} \right)^{1/2} \bar{\psi}^H(\bar{w}).
\end{align}
\rmk{
  we use $ C $ to denote the field on the cylinder and $ P $ to denote the field on the plane. and we use $ H $ to denote the field on the UHP.
}
Thus for the inner circle $ |\xi| = 1 $, we have:
\begin{itemize}
  \item \textbf{Free $ (+) $ Boundary Condition:}
    \begin{align}
      \psi^P(\xi) = i \bar{\psi}^P(\bar{\xi}) \quad \text{at } |\xi| = 1.
    \end{align}
  \item \textbf{Fixed $ (-) $ Boundary Condition:}
    \begin{align}
      \psi^P(\xi) = - i \bar{\psi}^P(\bar{\xi}) \quad \text{at } |\xi| = 1.
    \end{align}
\end{itemize}
Then we mode expand the fields on the plane and plug in the mode expansion to get the gluing conditions on the modes. We finally have:
\begin{itemize}
  \item \textbf{Free $ (+) $ Boundary Condition:}
    \begin{align}
      \left(a_{k}-i\bar{a}_{-k}\right)|+, NS/R \rangle=0\mathrm{~,~}\quad\left\langle +, NS/R\right|\left(a_{-k}+i\bar{a}_{k}\right)=0\mathrm{~,}
    \end{align}
  \item \textbf{Fixed $ (-) $ Boundary Condition:}
    \begin{align}
      \left(a_{k}+i\bar{a}_{-k}\right)|-, NS/R \rangle=0\mathrm{~,~}\quad\left\langle -, NS/R\right|\left(a_{-k}-i\bar{a}_{k}\right)=0\mathrm{~,}
    \end{align}
\end{itemize}

For these gluing condition, it is able to construct a set of Ishibashi states that satisfy them:
\begin{itemize}
  \item \textbf{NS Sector Ishibashi States:}
    \begin{align}
    |\pm,NS\rangle\rangle=\prod_{k\in\mathbb{N}_+-1/2}e^{i\pm a_{-k}\bar{a}_{-k}}|0\rangle_{\mathrm{NS}}
    \end{align}
    The $ + $ state is solution to the gluing condition of free boundary condition while the $ - $ state is solution to the gluing condition of fixed boundary condition.
  \item \textbf{R Sector Ishibashi States:}
    \begin{align}
|\pm,R \rangle\rangle=\prod_{k\in\mathbb{N}_+}e^{i\pm a_{-k}\bar{a}_{-k}}|\pm\rangle_{\mathrm{R}}
    \end{align}
    The $ + $ state is solution to the gluing condition of free boundary condition while the $ - $ state is solution to the gluing condition of fixed boundary condition. And $ |\pm\rangle_R $ are the degenerate R ground states which we have discussed in \cref{sec:Zero Mode and Ground State Degeneracy in Majorana CFT}. solutions to different gluing conditions corresponds to different choices of R ground states, this is because the gluing conditions on zero modes are:
    \begin{align}
  & (a_0 - i \bar{a}_0) |+,R\rangle = 0, \\
  & (a_0 + i \bar{a}_0) |-,R\rangle = 0.
    \end{align}
    It can be seen by plugging in the definition \cref{eq:Majorana zero mode action on R ground states}. 
\end{itemize}
Proof: 

I'll sketch the proof for NS sector and the free $ (+) $ boundary condition here. R sector only differs from the discussion of the zero modes. We shall verify that the Ishibashi state indeed satisfies the gluing condition. Due to the anti-commutation relation of the modes, we have:
\begin{align}
  e^{i a_{-k}\bar{a}_{-k}} = 1 + i a_{-k}\bar{a}_{-k}.
\end{align}
This in total behaves as a bosonic operator so we can change the order of acting them on the vacuum state freely. Then we consider the action of $ (a_k - i \bar{a}_{-k}) $ on the Ishibashi state, we assume $ k > 0 $ the other case is rather similar. We have:
\begin{align}
  (a_k - i \bar{a}_{-k}) (1 + i a_{-k}\bar{a}_{-k})
  &= a_k + i a_k a_{-k} \bar{a}_{-k} - i \bar{a}_{-k} - \bar{a}_{-k} a_{-k} \bar{a}_{-k} \\
  &= a_k - i\bar{a}_{-k} + i a_k a_{-k} \bar{a}_{-k}\\ 
  & = a_k -i\bar{a}_{-k} (a_{-k} a_k ) \\
\end{align}
Acting on vaccum this will give zero.
\qed 

\subsubsection{Boundary States and Partition Functions}

We now that boundary states shall be linear combinations of the Ishibashi states we have constructed. However, we notice that for every boundary condition only exists one corresponding Ishibashi state. \textbf{Thus, the boundary states are merely the Ishibashi states themselves up to a normalization factor}
\begin{align}
  &\ket{+,NS} \;\propto\;  |\pm,NS\rangle\rangle, \qquad \ket{-,NS} \;\propto\;  |-,NS\rangle\rangle, \\
  &\ket{+,R} \;\propto\;  |+,R\rangle\rangle, \qquad \ket{-,R} \;\propto\;  |-,R\rangle\rangle.
\end{align}
We now have to find out the normalization factor using the definition of fermionic boundary state (which sometimes are called spin Cardy's condition). To do this we first calculate the partition function of Ishibashi states, this can be done explicitly:
\begin{align}
\langle\!\langle \pm,NS | e^{-L H_{\text{closed}}} | \pm,NS \rangle\!\rangle
&=
\tilde{q}^{-\frac{1}{48}}
\prod_{n=0}^{\infty}\left(1+\tilde{q}^{n+\frac{1}{2}}\right)
= \chi_{0}(\tilde{q}) + \chi_{1/2}(\tilde{q}),
\\
\langle\!\langle \mp,NS | e^{-L H_{\text{closed}}} | \pm,NS \rangle\!\rangle
&=
\tilde{q}^{-\frac{1}{48}}
\prod_{n=0}^{\infty}\left(1-\tilde{q}^{n+\frac{1}{2}}\right)
= \chi_{0}(\tilde{q}) - \chi_{1/2}(\tilde{q}),
\\
\langle\!\langle \pm,R | e^{-L H_{\text{closed}}} | \pm,R \rangle\!\rangle
&=
\tilde{q}^{\frac{1}{24}}
\prod_{n=1}^{\infty}\left(1+\tilde{q}^{n}\right)
= \chi_{1/16}(\tilde{q}),
\\
\langle\!\langle \mp,R | e^{-L H_{\text{closed}}} | \pm,R \rangle\!\rangle
&=
0 .
\end{align}
Proof: This is done by plugging in the definition and do straightforward calculation. For example, for the first one we can first calculate the commutation relation between the Hamiltonian and operators:
\begin{align}
  [H_{\mathrm{closed}},a_{-k}]=\frac{2\pi}{\beta}ka_{-k},\quad[H_{\mathrm{closed}},\bar{a}_{-k}]=\frac{2\pi}{\beta}k\bar{a}_{-k}
\end{align}
From this we can have:
\begin{align}
  e^{-LH_{\mathrm{closed}}}a_{-k}\bar{a}_{-k}e^{LH_{\mathrm{closed}}}=e^{-\frac{4\pi L}{\beta}k}a_{-k}\bar{a}_{-k}
\end{align}
Then we use the fact that:
\begin{align}
  (1- i\bar{a}_k a_{k}) (1 + i \omega_k a_{-k}\bar{a}_{-k}) = 1 + i \omega_k a_{-k}\bar{a}_{-k} - i \bar{a}_k a_{k} + \omega_k \bar{a}_k a_{k} a_{-k}\bar{a}_{-k}  \quad \text{where } \quad  \omega_k = e^{-\frac{4\pi L}{\beta}k}
\end{align}
When act on vaccum the only non-zero contribution comes from the first and the last term. Thus we have:
\begin{align}
  1+ \omega_k \bar{a}_k a_{k} a_{-k}\bar{a}_{-k} \sim 1 + \omega_k
\end{align}
contribution from each term. In total we have:
\begin{align}
  \langle\!\langle +,NS | e^{-L H_{\text{closed}}} | +,NS \rangle\!\rangle
  &=
    \tilde{q}^{-\frac{1}{48}}
  \prod_{k\in\mathbb{N}_+ - 1/2} \left(1 + e^{-\frac{4\pi L}{\beta}k}\right) \\
  &=
  \tilde{q}^{-\frac{1}{48}}
  \prod_{n=0}^{\infty}\left(1+\tilde{q}^{n+\frac{1}{2}}\right) 
\end{align}
The first term comes from the vaccum energy. The other cases are rather similar. 
\qed


where we have used the modular transformation of Ising characters and $ \tilde{q} = e^{-4\pi L/\beta} $. we compare these results with the open channel partition functions in \cref{tab:open channel partition functions} and find out the normalization factors. We have:
\begin{align}
 & \ket{-,NS} = | -,NS\rangle\rangle\\ 
 &\ket{-,R} = \sqrt[4]{2}|-,R\rangle\rangle, \\ 
 & \ket{+,NS} = \sqrt{2}|+,NS\rangle\rangle, \\ 
 & \ket{+,R} = 0|+,R\rangle\rangle = 0.
\end{align}
We calculate the partition functions and list them out in a table:
\begin{table}[H]
\centering
\renewcommand{\arraystretch}{1.6}
\begin{tabular}{cccc}
\hline\hline
Open Channel & $ q $ character & $ \tilde{q} $ character & Closed Channel \\ 
\hline\hline

$Z_{(--),-}$ 
& $\chi_0(q) - \chi_{1/2}(q)$
& $\sqrt{2}\,\chi_{1/16}(\tilde q)$
& $\langle -,R | e^{-L H_{\rm closed}} | -,R \rangle$
\\

$Z_{(--),+}$ 
& $\chi_0(q) + \chi_{1/2}(q)$
& $\chi_0(\tilde q) + \chi_{1/2}(\tilde q)$
& $\langle -,NS | e^{-L H_{\rm closed}} | -,NS \rangle$
\\

$Z_{(++),-}$ 
& 0
& 0
& $\langle +,R | e^{-L H_{\rm closed}} | +,R \rangle$
\\

$Z_{(++),+}$ 
& $2\big(\chi_0(q) + \chi_{1/2}(q)\big)$
& $2\big(\chi_0(\tilde{q}) + \chi_{1/2}(\tilde{q})\big)$
& $\langle +,NS | e^{-L H_{\rm closed}} | +,NS \rangle$
\\

$Z_{(-+),-}$ 
& 0
& 0
& $\langle -,R | e^{-L H_{\rm closed}} | +,R \rangle$
\\

$Z_{(-+),+}$ 
& $2\,\chi_{1/16}(q)$
& $\sqrt{2}\big(\chi_0(\tilde q) - \chi_{1/2}(\tilde q)\big)$
& $\langle -,NS | e^{-L H_{\rm closed}} | +,NS \rangle$
\\

\hline\hline
\end{tabular}
\caption{Free fermion open-channel partition functions in both $ q $ and $ \tilde{q} $ characters, with the corresponding closed-channel overlaps.}
\label{tab:open-closed-partition-single}
\end{table}
This table summarize the main results of this section. We have successfully identified the boundary states that correspond to the two boundary conditions in Majorana CFT with boundary fermions. 






% section Free Majorana Fermion BCFT (end)

\newpage 
\section{Bosonization of Majorana BCFT}\label{sec:Bosonization of Majorana BCFT} % (fold)
Now we finally get to the main topic of this note, Bosonization. We will discuss how to bosonize the Majorana CFT with boundaries in closed channel. This topic has been discussed in several references, \cite{horiNotesBosonizationBoundary2009, ebisuFermionizationConformalBoundary2021}. Yet non of them gives a complete and clear explanation of the process. The following will be my understanding of the bosonization of Majorana BCFT.

\subsection{GSO Projection and $ \mathbb{Z}_2 $ Orbifold}

Bosonization is transforming a fermionic theory into a bosonic theory. A naive idea of changing a fermionic theory into a bosonic theory is to simply bind all fermions in pairs. When two fermions are paired, they can behave as a boson. 

\subsubsection{GSO Projection}

This naive idea works well, and mathematically bosonization of a partition function and Hilbert space can be described as the GSO projection in fermionic theory or the $ \mathbb{Z}_2 $ orbifold. First let's define GSO projection.
\defi{
  \textbf{GSO Projection}

  Nsively, it can be understood as projecting into Hilbert Space with even fermion number. Mathematically, if we have a fermionic theory with a well defined $ \mathbb{Z}_2 $ grading of fermion parity operator $ (-1)^F $, we can define the GSO projected partition function as:
  \begin{align}
    Z_{GSO} = \operatorname{Tr}_{\mathcal{H}}\left[\frac{1+(-1)^F}{2}e^{-\beta H}\right]
  \end{align}
  We can see that we impose a projection operator $ P_{GSO} = \frac{1+(-1)^F}{2} $ on states satisdfy:
  \begin{align}
    \ket{\text{odd}}_{GSO} = \displaystyle\frac{1 +(-1)^F}{2} \ket{\text{odd}} = 0, \quad \ket{\text{even}}_{GSO} = \displaystyle\frac{1 +(-1)^F}{2} \ket{\text{even}} = \ket{\text{even}}
  \end{align}
  By definition.
}

\subsubsection{Bosonization with boundaries}

When it comes to a boudnary theory, we not only have to consider the GSO projection of the Hilbert space, but also the boundary conditions (in open channel) or the boundary states (in closed channel). For a abstract boundary condition, it might be hard to understand how to "project" it. However, for a boundary state, we certainly can just act the projector on it and use the projected boundary state to construct the bosonized theory.


\subsection{Closed channel Bosonization}
The closed channel is rather easy to understand (and I only understand this). I propose that the bosonization of boundary states shall satisfy the two consistent steps:
\begin{itemize}
  \item Boundary States are GSO projected, we only keep the boundary states with even fermion parity. 
    \item Choose the Linear combination of GSO projected boundary states that satisfies the Cardy's condition and are elementary boundary states. To make it a valid bosonic boundary state.
\end{itemize}
With these two conditions in have, we can uniquely determine the bosonized boundary states and see they are \textbf{exactly} the known boundary states in Ising CFT.

\subsubsection{Fermionic Parity and 0A/0B Theories}

In the closed channel, in order to impose the GSO projection, we first need to define the fermionic parity operator $ (-1)^F $. As we have discussed alot, we have two ways of defining the fermionic parity operator in R sector Majorana CFT, the type 0A and type 0B ones:
\begin{align}
  (-1)^{F} =
\begin{cases}
  -2 i\, b_0 \bar b_0 \ (-1)^{F_{nz}}, \quad \text{type 0A} \\
  +2 i\, b_0 \bar b_0 \ (-1)^{F_{nz}}. \quad \text{type 0B}
\end{cases}
\end{align}
Here, with the above discussion of 0A theory, \textbf{we can see that we must use type 0A fermionic parity} or the theory may be trivial in R sector. 

\subsubsection{GSO Projection of Boundary States}

We demand the first consistent condition: 
\begin{itemize}
  \item Boundary States are GSO projected, we only keep the boundary states with even fermion parity.
\end{itemize}

\textbf{For the NS sector} 
we may find the GSO projection is trivial, for we construct the Ishibashi states as:
\begin{align}
      |\pm,NS\rangle\rangle=\prod_{k\in\mathbb{N}_+-1/2}e^{i\pm a_{-k}\bar{a}_{-k}}|0\rangle_{\mathrm{NS}}
\end{align}
Both of then have even fermion parity for we generate it from the vaccum with $ a_{-k}\bar{a}_{-k} $ bounded together. Thus we have for the boundary states in NS sector:
\begin{align}
  | \pm, NS \rangle_{GSO} = | \pm, NS \rangle
\end{align}
  
\textbf{For the R sector}
in a 0A theory, we only have one non-zero boundary state $ \ket{-,R} $. If we define the fermion parity operator as type 0A one, we can easily check that:
\begin{align}
  (-1)^F \ket{-,R} = + \ket{-,R} \quad \Rightarrow \quad \ket{-,R}_{GSO} = \ket{-,R}
\end{align}
Witch seems to work well. However, if we define the fermion parity operator as type 0B one, we will get:
\begin{align}
  (-1)^F \ket{-,R} = - \ket{-,R} \quad \Rightarrow \quad \ket{-,R}_{GSO} = 0
\end{align}
which means the R sector is completely projected out. Thus we must use type 0A fermionic parity operator to have a non-trivial R sector. \textbf{That's also the reason why we call putting a fermion on $ (+) $ boundary a type 0A theory.} for we must use type 0A fermionic parity operator to have a non-trivial R sector.

\subsubsection{Final Bosonized Boundary States}

Thus after a GSO projection, we have three non-trivial boundary states:
\begin{itemize}
  \item \textbf{fixed boundary states: } we have two projected boundary states $ \ket{-,NS} $ and $ \ket{-,R} $.
  \item \textbf{free boundary state: } we have one projected boundary state $ \ket{+,NS} $.
\end{itemize}
Then we \textbf{propose} that the final bosonized boundary states shall be in the linear combination of these projected boundary states that:
\begin{itemize}
  \item Statisfy the Cardy's condition. 
  \item Are elementary boundary states.
\end{itemize}
To do this, we shall first list out the overlaps of these boundary states:
\begin{table}[H]
\centering
\renewcommand{\arraystretch}{1.6}
\begin{tabular}{ccc}
\hline\hline
$q$ character & $\tilde q$ character & Closed Channel \\
\hline\hline

$\chi_0(q) - \chi_{1/2}(q)$
& $\sqrt{2}\,\chi_{1/16}(\tilde q)$
& $\langle -,R | e^{-L H_{\rm closed}} | -,R \rangle$
\\

$\chi_0(q) + \chi_{1/2}(q)$
& $\chi_0(\tilde q) + \chi_{1/2}(\tilde q)$
& $\langle -,NS | e^{-L H_{\rm closed}} | -,NS \rangle$
\\

$2\big(\chi_0(q) + \chi_{1/2}(q)\big)$
& $2\big(\chi_0(\tilde{q}) + \chi_{1/2}(\tilde{q})\big)$
& $\langle +,NS | e^{-L H_{\rm closed}} | +,NS \rangle$
\\

$2\,\chi_{1/16}(q)$
& $\sqrt{2}\big(\chi_0(\tilde q) - \chi_{1/2}(\tilde q)\big)$
& $\langle -,NS | e^{-L H_{\rm closed}} | +,NS \rangle$
\\

\hline\hline
\end{tabular}
\caption{Free fermion partition functions keeping only even parity.}
\end{table}
If we only consider one sector, we can't construct such a solution. However, if we do a trick and "combine the R sector and the NS sector together" (this is quite subtle, I will later on exlain my understandings), we can find the solution:
\begin{align}
  \ket{f} &= \displaystyle\frac{1}{\sqrt{2}}\ket{+,NS} \\ 
  \ket{+} &= \frac{1}{\sqrt{2}}\left(\ket{-,NS} + \ket{-,R}\right) \\ 
  \ket{-} &= \frac{1}{\sqrt{2}}\left(\ket{-,NS} - \ket{-,R}\right)
\end{align}
In fact we can never "sum" a ramond sector state with a NS sector state. What we do is in fact \textbf{Direct Sum} the two Hilbert spaces, thus it should be written as:
\begin{align}
  \ket{f} &= \displaystyle\frac{1}{\sqrt{2}}\ket{+,NS} \oplus 0 \\ 
  \ket{+} &= \frac{1}{\sqrt{2}}\left(\ket{-,NS} \oplus \ket{-,R}\right) \\ 
  \ket{-} &= \frac{1}{\sqrt{2}}\left(\ket{-,NS} \oplus -\ket{-,R}\right)
\end{align}
And the evolution operator in the partition function should also be written as:
\begin{align}
  e^{-L H_{\rm closed}} = e^{-L H_{\rm closed NS}} \oplus e^{-L H_{\rm closed R}}
\end{align}
With these in hand we calculate the partition function of these direct sumed state in the direct sumed Hilbert space:

\begin{table}[H]
\centering
\renewcommand{\arraystretch}{1.6}
\begin{tabular}{ccc}
\hline\hline
$q$ character & $\tilde q$ character & Closed Channel \\
\hline\hline

$\chi_0(q) $
              & $ \frac{1}{2}\big(\chi_0(\tilde q) + \chi_{1/2}(\tilde q) \big) + \frac{1}{\sqrt{2}}\chi_{1/16}(\tilde q)$
& $\langle + | e^{-L H_{\rm closed}} | + \rangle$
\\
$\chi_0(q) $
              & $ \frac{1}{2}\big(\chi_0(\tilde q) + \chi_{1/2}(\tilde q) \big) + \frac{1}{\sqrt{2}}\chi_{1/16}(\tilde q)$
& $\langle - | e^{-L H_{\rm closed}} | - \rangle$
\\

$ \chi_{1/2}(q) $
&  $ \frac{1}{2}\big(\chi_0(\tilde q) + \chi_{1/2}(\tilde q) \big) + \frac{1}{\sqrt{2}}\chi_{1/16}(\tilde q)$
& $\langle - | e^{-L H_{\rm closed}} | + \rangle$
\\

$ \chi_0(q) + \chi_{1/2}(q) $
& $ \chi_0(\tilde{q}) + \chi_{1/2}(\tilde{q}) $
& $\langle f | e^{-L H_{\rm closed}} | f \rangle$
\\

$ \chi_{1/16}(q)$
& $\frac{1}{\sqrt{2}}\big(\chi_0(\tilde q) - \chi_{1/2}(\tilde q)\big)$
& $\langle + | e^{-L H_{\rm closed}} | f \rangle$
\\

$ \chi_{1/16}(q)$
& $\frac{1}{\sqrt{2}}\big(\chi_0(\tilde q) - \chi_{1/2}(\tilde q)\big)$
& $\langle - | e^{-L H_{\rm closed}} | f \rangle$
\\

\hline\hline
\end{tabular}
\caption{Partition function of the bosonized boundary states.}
\end{table}
We compare the above table with the known Ising CFT boundary states partition functions \cref{tab:ising_partition_functions} and we find they are exactly the same! Thus we identify the bosonized theory as the Ising CFT with boundary states:
\begin{align}
  \ket{f} &= \ket{f}_{\text{Ising}} \\ 
  \ket{+} &= \ket{+}_{\text{Ising}} \\ 
  \ket{-} &= \ket{-}_{\text{Ising}}
\end{align}

\subsection{Compare with Exact Partition Function Calculation}

We compare the boundary state result with the calculation from \cite{chatterjeeExactPartitionFunction1995a}. In this paper, it calculates the boundary states and partition function using another method. I will now compare the results in the paper to the bosonized results in this note, and show that they are consistent. 

\subsubsection{Boundary States}

If we explicitly write down the boundary states above is of free fermion vaccum states and modes. We can have:

\textbf{Fixed Boundary States:} 
\begin{align}
  \ket{+} =& \displaystyle\frac{1}{\sqrt{2}} (\ket{-,NS} + \ket{-,R}) \\
  =& \frac{1}{\sqrt{2}}\left(\prod_{k\in\mathbb{N}_+-1/2}e^{- i a_{-k}\bar{a}_{-k}}|0\rangle_{\mathrm{NS}} + \sqrt[4]{2} \prod_{n\in\mathbb{N}_+}e^{- i a_{-n}\bar{a}_{-n}}|-\rangle_{\mathrm{R}}\right) \\ 
  = & \frac{1}{\sqrt{2}}\prod_{k\in\mathbb{N}_+-1/2}e^{- ia_{-k}\bar{a}_{-k}}|0\rangle_{\mathrm{NS}} + \displaystyle\frac{1}{\sqrt[4]{2}} \prod_{n\in\mathbb{N}_+}e^{- i a_{-n}\bar{a}_{-n}}  |-\rangle_{\mathrm{R}}
\end{align}
\begin{align}
  \ket{-} =& \displaystyle\frac{1}{\sqrt{2}} (\ket{-,NS} + \ket{-,R}) \\
  =& \frac{1}{\sqrt{2}}\left(\prod_{k\in\mathbb{N}_+-1/2}e^{- i a_{-k}\bar{a}_{-k}}|0\rangle_{\mathrm{NS}} - \sqrt[4]{2} \prod_{n\in\mathbb{N}_+}e^{- i a_{-n}\bar{a}_{-n}}|-\rangle_{\mathrm{R}}\right) \\ 
  = & \frac{1}{\sqrt{2}}\prod_{k\in\mathbb{N}_+-1/2}e^{- ia_{-k}\bar{a}_{-k}}|0\rangle_{\mathrm{NS}} - \displaystyle\frac{1}{\sqrt[4]{2}} \prod_{n\in\mathbb{N}_+}e^{- i a_{-n}\bar{a}_{-n}} |-\rangle_{\mathrm{R}}
\end{align}

\textbf{Free Boundary State:}
\begin{align}
  \ket{f} = \displaystyle\frac{1}{\sqrt{2}} \ket{+,NS}  = \prod_{k\in\mathbb{N}_+-1/2}e^{ i a_{-k}\bar{a}_{-k}}|0\rangle_{\mathrm{NS}}
\end{align}

Looking at the results of \cite{chatterjeeExactPartitionFunction1995a}, we can find that the boundary states they obtain is:
\begin{align}
  |B_{\pm}\rangle&=e^{-R\varepsilon_{\lambda}}\Big(\frac{\alpha}{e}\Big)^{\alpha}\sqrt{\pi}\Big[\frac{1}{\Gamma(\alpha+\frac{1}{2})}\exp\{ i \sum_{n=0}^{\infty}\frac{n+\frac{1}{2}-\alpha}{n+\frac{1}{2}+\alpha} a_{n+1/2}^{\dagger} \bar{a}_{n+1/2}^{\dagger}\}|0\rangle\\
  &\pm\frac{2^{\frac14}\sqrt{\alpha}}{\Gamma(\alpha+1)} \exp\{ i \sum_{n=1}^{\infty} \frac{n-\alpha}{n+\alpha} a_{n}^{\dagger} \bar{a}_{n}^{\dagger}\}|\sigma\rangle\Big]
\end{align}
with $ \epsilon_\lambda \sim \lambda \log \lambda^2 $. Thus, we take a limit of $ \alpha = 0, \lambda = 0 $ then we have:
\begin{align}
  |B_{free}\rangle=\exp i\{\sum_{n=1}^\infty a_{n+\frac{1}{2}}^\dagger\bar{a}_{n+\frac{1}{2}}^\dagger\}|0\rangle+i\exp i\{\sum_{n=1}^\infty a_{n+\frac{1}{2}}^\dagger\bar{a}_{n+\frac{1}{2}}^\dagger\}a_{\frac{1}{2}}^\dagger\bar{a}_{\frac{1}{2}}^\dagger|0\rangle
\end{align}
which is exactly the free boundary state $ \ket{f} $ we obtained above. If we take the limit of $ \alpha \to \infty, \lambda \to \infty $, we have:
\begin{align}
  |B_{fixed\pm}\rangle&=\frac{1}{\sqrt{2}}\exp-i\{\sum_{n=1}^{\infty}a_{n+\frac{1}{2}}^{\dagger}\bar{a}_{n+\frac{1}{2}}^{\dagger}\}\left(|0\rangle+|\varepsilon\rangle\right)\\&\pm\frac{1}{2^{\frac{1}{4}}}\exp-i\{\sum_{n=1}^{\infty}a_{n}^{\dagger}\bar{a}_{n}^{\dagger}\}|\sigma\rangle
\end{align}
which is exactly the fixed boundary states $ \ket{\pm} $ we obtained above. Thus we see that the boundary states obtained in \cite{chatterjeeExactPartitionFunction1995a} are exactly the same as the bosonized boundary states we obtained above.




% section Bosonization of Majorana BCFT (end)

% \newpage 
% \section{Bosoniztaion from Exact Partition Function}\label{sec:Bosoniztaion from Exact Partition Function} % (fold)
% \subsection{Exact Action of Majorana BCFT}
The boundary Majorana CFT can also be described with an explicit action shown in \cite{chatterjeeExactPartitionFunction1995a}. We give a brief review here. 

\subsubsection{General QFT with a Boundary}

For a general QFT defined on a space with a boundary, the action can not only include the bulk term but also the boundary term. With the Lagrangian equation, the boundary term shall give out the boundary condition of the fields. For example, consider a general 2D field theory on the UHP, we can write down the action as:
\begin{align}
  S =\int_{-\infty}^\infty dx\ \int_0^\infty dy\mathcal{L}(\phi_i,\partial_\mu\phi_i)+\int_0^\infty dx\ \mathcal{L}_B(\phi_i^B,\frac{d}{dx}\phi_i^B)
\end{align}
The Ordinary Euler-Lagrange equation gives out the bulk equation of motion:
\begin{align}
  \partial_\mu\left(\frac{\partial\mathcal{L}}{\partial\partial_\mu\phi_i}\right)-\frac{\mathcal{L}}{\partial\phi_i}=0
\end{align}
while at the boundary the boundary condition we want to impose is given by:
\begin{align}
  \left.\frac{\partial\mathcal{L}_B}{\partial\phi_i^B}-\frac{d}{dx}\left(\frac{\partial\mathcal{L}_B}{\partial\left(\frac{d}{dx}\phi_i^B\right)}\right)-\frac{\partial\mathcal{L}}{\partial\frac{\partial\phi_i}{\partial y}}\right|_{y=0}=0,\quad y=0
\end{align}
Thus a delicate design of the boundary Lagrangian $\mathcal{L}_B$ can give out the desired boundary condition.

\subsubsection{Majorana CFT Boundary Conditions on UHP}


\subsubsection{Majorana CFT on UHP}

Generally, if we consider any boundary condition (not only conformal boundary condition) of a Majorana CFT on a UHP, we can write down the boundary Lagrangian as:
\begin{align}
  S&=\frac{1}{2\pi }\int_{-\infty}^{\infty}dx\int_{0}^{\infty}dy[\psi\bar{\partial}\psi+\bar{\psi}\partial \bar{\psi}]\\ 
  &+\int_{-\infty}^{\infty}dx\left[-\frac{i}{4\pi}\psi\bar{\psi}|_{y=0}+\frac{1}{2}\xi\frac{d}{dx}\xi\right]+ih\int_{-\infty}^{\infty}\xi(x)(\psi+\bar{\psi})(x,0)
\end{align}
We impose the boundary degree of freedom $\xi(x)$ to construct the boundary condition. $ h $ is a boundary coupling constant, interpreted as a boundary magnetic field in the Ising model. By varying the action, we can get the equation of motion in the boundary condition at $y=0$:
\begin{align}
  \left.\left(\partial+i\lambda\right)\psi-\left(\bar{\partial}-i\lambda\right)\bar{\psi}\right|_{z=\bar{z}}=0 \quad \text{with}\quad \lambda=4\pi h^2
\end{align}
we can easily check that when $\lambda=0$ we get the free boundary condition while when $\lambda\to\infty$ we get the fixed boundary condition. Thus by tuning the boundary coupling constant $h$, we can realize the RG flow from free boundary condition to fixed boundary condition in Majorana CFT.


\subsubsection{Majorana CFT on any Manifold}

We then consider a generalied manifold with boundary $ \mathcal{D} $ (not just the UHP) with boundaries $ \partial\mathcal{D} = \mathcal{B} = \cup_j \mathcal{B}_j  $ that are parametrized as $ \mathcal{B}_j = ( Z_i(t), \bar{Z}_j(t) ) $ paramatrized by $ t $, we have another form of the action for a general BC:
\begin{align}
  S&=\frac{1}{2\pi}\int_{\mathcal{D}}dxdy[\psi\bar{\partial}\psi+\bar{\psi}\partial\bar{\psi}\\ 
   &+\sum_{j=1}^{n}\left\{\int_{\mathcal{B}_{j}}dt[-\frac{i}{4\pi}\psi\bar{\psi}+\frac{1}{2}\xi\dot{\xi}]+ih\int_{\mathcal{B}_{j}}dt \ \xi(t)(e_{j}^{\frac{1}{2}}\psi+\bar{e}_{j}^{\frac{1}{2}}\bar{\psi})(t)\right\}
\end{align}
where on the boundary $ \mathcal{B}_j $ the field is $ \psi(t) = \psi(Z_j(t)), \bar{\psi}(t) = \bar{\psi}(\bar{Z}_i(t)) $. $ e_j(t) $ and $ \bar{e}_j(t) $ are defined as the tangent vectors on the boundary:
The boundary condition at $ \mathcal{B}_j $ is given by:
\begin{align}
  (\partial_{t}+i\lambda) e^{\frac{1}{2}}(t)\psi(Z(t)) =(\partial_{t}-i\lambda) \bar{e}^{\frac{1}{2}}(t)\bar{\psi}(\bar{Z}(t)) \quad \text{where} \quad \lambda=4\pi h^2
\end{align}


\subsubsection{Boundary Free Fermion}

% % section Bosoniztaion from Exact Partition Function (end)

\newpage 
\bibliography{references}


\end{document}
