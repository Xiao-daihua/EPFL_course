\subsection{Binar System的引力波辐射}

\subsubsection{垂直转动平面观察结果}

下面我们研究如下set up的观测着观察到的双星系统的引力波metrix。我们假设双星系统在$ x-y $平面上互相绕着转动,并且我们在$ z $轴很远的位置观测到引力波。
\begin{figure}[H]
  \centering
  \includegraphics[width=0.15\textwidth]{assets/doublestar.png}
  \caption{双星系统引力波辐射示意图}
  \label{fig:doublestar}
\end{figure}

\bigskip
\hlr{经典双星系统力学}

我们考虑两个质量分别为$ m_1, m_2 $的点粒子在相互引力作用下绕着质心做圆周运动。我们设定质心在原点,并且两个粒子在$ x-y $平面上做圆周运动。我们可以定义两个粒子的位置和相对位置$ x_0 $以及质心位置$ x_{cm} $为:
\begin{align}
  x_1, \quad x_2 \quad x_0 = x_1 - x_2, \quad x_{cm} = \displaystyle\frac{m_1 x_1 + m_2 x_2}{m_1 + m_2}
\end{align}
并且我们定义总质量$ m = m_1 + m_2 $以及约化质量$ \mu = \displaystyle\frac{m_1 m_2}{m_1 + m_2} $。我们会发现由于我们选择质心是原点,因此$ x_{cm} = 0 $。我们可以把两个粒子的位置表示为:
\begin{align}
  x_1 = \displaystyle\frac{m_2}{m} x_0, \quad x_2 = -\displaystyle\frac{m_1}{m} x_0
\end{align}

\bigskip
\hlr{质量分布计算}

下面我们计算两个粒子的质量分布。对于相对论点粒子系统,我们的能量密度可以近似的认为是$ m(x)c^2 $乘以一个delta函数分布:
\begin{align}
  T^{00}=\sum_Am_Ac^2\delta^{(3)}(\mathbf{x}-\mathbf{x}_A(t))\mathrm{~,}
\end{align}
因此,moment of inertia给出:
\begin{align}
  M_{ij}(t) = \int d^3x\, m(x) x_i x_j = m_1 x_{1i}(t) x_{1j}(t) + m_2 x_{2i}(t) x_{2j}(t) = \mu x_{0i}(t) x_{0j}(t)
\end{align}
我们会意识到,moment of inertia可以等效的认为是一个质量为$ \mu $的粒子在位置$ x_0 $处的moment of inertia。

\bigskip
\hlr{牛顿力学计算Moment of Inertia}

根据牛顿力学的计算我们知道两个粒子在相互引力作用下绕着质心做圆周运动。我们设定两个粒子在$ x-y $平面上绕着质心做圆周运动,并且角速度为$ \omega_s $。我们可以把相对位置表示为:
\begin{align}
  x_0(t) = R \cos(\omega_s t + \pi/2) \quad y_0(t) = R \sin(\omega_s t + \pi/2) \quad z_0(t) = 0
\end{align}
我们带入moment of inertia的关系可以得到:
\begin{align}
&\dot{M}_{11}=\frac{\mu R^{2}}{2}2w_{s}\sin(2w_{s}t)\quad \ddot{M}_{11}=\mu R^{2}2w_{s}^{2}\cos(2w_{s}t)\quad \dddot{M}_{11}=-\mu R^{2}4w_{s}^{3}\sin(2w_{s}t)\\
&\dot{M}_{12}=\frac{\mu R^{2}}{2}2w_{s}(-\cos (2w_{s}t))\quad \ddot{M}_{12}=\mu R^{2}2w_{s}^{2}\sin(2w_{s}t)\quad \dddot{M}_{12}=\mu R^{2}4w_{s}^{3}\cos(2w_{s}t) \\
\end{align}
并且我们知道$ M_{22} = -M_{11} $,以及$ M_{12} = M_{21} $。因此我们可以求出所有的分量的三阶导数。

\bigskip
\hlr{引力波极化分量计算}

根据\cref{eq:gwquadrupolepluscross}我们可以得到引力波的plus和cross极化分量:
\begin{align}\label{eq:binarygwpluscross}
  h_+(t,r) &= \displaystyle\frac{G}{r} \left(\ddot{M}_{11}(t-r) - \ddot{M}_{22}(t-r/c)\right) = \displaystyle\frac{4G\mu R^2 \omega_s^2}{r} \cos(2\omega_s(t-r)) \\
h_\times(t,r) &= \displaystyle\frac{2G}{r} \left(\ddot{M}_{12}(t-r)\right) = \displaystyle\frac{2G\mu R^2 \omega_s^2}{r} \sin(2\omega_s(t-r))
\end{align}
注意这里的讨论$ R $是两个粒子之间的距离,而$ r $是观察者距离双星系统质心的距离。

\subsubsection{一般观察方向的结果}

对于一般观测来说我们的观察者必然不是垂直于双星系统的转动平面的。因此我们需要考虑一般观察者的观测结果。由于我们知道计算$ z $方向的观察者比较简单,因此我们需要进行一个坐标变换,把沿着某个平面转动的$ M_{ij} $变换到观察者是$ z $轴方向的坐标系下。如下图所示:
\begin{figure}[H]
  \centering
  \includegraphics[width=0.5\textwidth]{assets/coor.png}
  \caption{一般观察方向的坐标变换示意图}
  \label{fig:coor}
\end{figure}

\bigskip
\hlr{Moment of Inertia的坐标变换}

考虑Moment of Inertia的定义$  M_{ij} = \int d^3x\, m(x) x_i x_j $,我们考虑一个坐标变换$ x_i' = R_{ij} x_j $,其中$ R_{ij} $是一个正交矩阵。另一个坐标系下面的Moment of Inertia为$ M_{ij}' = \int d^3x'\, m(x') x_i' x_j' $。我们可以把新的坐标系下的Moment of Inertia表示为旧的坐标系下的Moment of Inertia:
\begin{align}
  M_{ij}' &= \int d^3x'\, m(x') x_i' x_j' = \int d^3x\, m(x) (R_{ik} x_k)(R_{jl} x_l) = R_{ik} R_{jl} \int d^3x\, m(x) x_k x_l = R_{ik} R_{jl} M_{kl}
\end{align}
\rmk{我们注意一般$ T^{00} $也是要在坐标变换下联系的,但是这里由于我们考虑的是纯空间坐标的旋转,因此$ T^{00} $是不变的,因此我们直接使用了$ m(x) $不变的关系。}
对于坐标转动我们的选择并不唯一,我们仅仅要求观察这方向 $ n_i=\left(\sin\theta\sin\phi,\sin\theta\cos\phi,\cos\theta\right), $在新的坐标系下是$ z' $轴方向即可。也就是说:$ (0,0,1) = R ( \sin\theta\sin\phi,\sin\theta\cos\phi,\cos\theta) $或者说$ R^T (0,0,1) = ( \sin\theta\sin\phi,\sin\theta\cos\phi,\cos\theta)$。我们可以选择如下的旋转矩阵:
\begin{align}
  R^T=\begin{pmatrix}\cos\phi&\sin\phi&0\\-\sin\phi&\cos\phi&0\\0&0&1\end{pmatrix}\begin{pmatrix}1&0&0\\0&\cos\theta&\sin\theta\\0&-\sin\theta&\cos\theta\end{pmatrix}
\end{align}
因此我们有:
\begin{align}
  M'_{ij} = R_{ik} R_{jl} M_{kl} = (R M R^T)_{ij}
\end{align}
\rmk{
  这里我们的convention和GW书中3.3不一样,我们选择书中的$ \mathcal{R} $作为$ R^T $。
}

\bigskip
\hlr{引力波极化分量计算}

下面我们带入Binary System的$ M_{ij} $计算引力波的极化分量。我们有:
\begin{align}
  &h_{+}(t,\theta,\phi)=\frac{4G}{r}\mu \omega_{s}^{2}R^{2}\frac{1+\cos^{2}\theta}{2}\cos(2\omega_{s}(t-r)+2\phi)\\ 
  &h_{\times}(t,\theta,\phi)=\frac{4G}{r}\mu \omega_{s}^{2}R^{2}\cos\theta\sin(2\omega_{s}(t-r)+2\phi)
\end{align}
我们对比之前$ z $方向的结果\cref{eq:binarygwpluscross}可以发现多了一个角度相关的衰减。也就是说在z方向我们能够接收到最大的引力波信号,而在平行于转动平面的方向我们接收到的引力波信号是最小的。

\bigskip
\hlr{引力波信号的性质}

我们可以发现引力波信号的一些性质:
\begin{itemize}
  \item 引力波的波动频率是$ 2 \omega_s $,也就是说引力波的频率是双星系统转动频率的两倍。
\item 我们可以通过引力波的极化分量的大小来判断观测着相对于双星系统转动平面的角度。
\item 我们计算引力波的波长$ \bar{\lambda} = 1/2 \omega_s \sim R/2v >> R $,因此我们可以认为引力波的波长远大于双星系统的尺度。
\end{itemize}

\subsubsection{Binary System能量辐射}

我们下面计算这个系统的引力波的对于所有角度的能量辐射。根据辐射公式\cref{thm:gwpowerquadruple}我们知道计算Quadrupole Moment的三阶导数的平方的平均值即可。我们有:
\begin{align}
  P_{GW} = \displaystyle\frac{32}{5} G \mu^2 R^4 \omega_s^6
\end{align}

\subsection{Binary System的Inspiral}

下面我们讨论对于一个Binary System如果





\subsection{Gravitational Lensing}

\subsection{Questions and thoughts}

\question{为什么点粒子的E-M Tensor正文讨论的那样???}

注意哦!我们考虑的是相对论的情况,是经典力学的推广!我们对此需要知道能量动量什么的定义是:$ p^\mu = m u^\mu $。我们不可以naive的从经典的定义给过去,只能从相对论的定义取低速度极限变回去!

对于能动量张量,我们定义为\textbf{时空平移对称性的守恒流}只有这个定义是合理的。然后考虑一个单粒子我们有:
\begin{align}
  S=-m\int d\tau \quad \Rightarrow \quad T^{\mu\nu} = \int d\tau \displaystyle\frac{p^\mu p^\nu}{p^0}  \delta^{(3)}(\mathbf{x} - \mathbf{x}(\tau))
\end{align}
我们对于经典极限很喜欢使用$ dt $作为参数,所以我们有:
\begin{align}
  T^{\mu\nu} = \int dt m \displaystyle\frac{d x^\mu}{dt} \displaystyle\frac{d x^\nu}{dt} \delta^{(3)}(\mathbf{x} - \mathbf{x}(t)) 
\end{align}

