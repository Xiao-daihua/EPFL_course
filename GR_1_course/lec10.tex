\subsection{Binary System的引力波}




\subsection{Questions and thoughts}

\question{为什么点粒子的E-M Tensor正文讨论的那样???}

注意哦!我们考虑的是相对论的情况,是经典力学的推广!我们对此需要知道能量动量什么的定义是:$ p^\mu = m u^\mu $。我们不可以naive的从经典的定义给过去,只能从相对论的定义取低速度极限变回去!

对于能动量张量,我们定义为\textbf{时空平移对称性的守恒流}只有这个定义是合理的。然后考虑一个单粒子我们有:
\begin{align}
  S=-m\int d\tau \quad \Rightarrow \quad T^{\mu\nu} = \int d\tau \displaystyle\frac{p^\mu p^\nu}{p^0}  \delta^{(3)}(\mathbf{x} - \mathbf{x}(\tau))
\end{align}
我们对于经典极限很喜欢使用$ dt $作为参数,所以我们有:
\begin{align}
  T^{\mu\nu} = \int dt m \displaystyle\frac{d x^\mu}{dt} \displaystyle\frac{d x^\nu}{dt} \delta^{(3)}(\mathbf{x} - \mathbf{x}(t)) 
\end{align}

