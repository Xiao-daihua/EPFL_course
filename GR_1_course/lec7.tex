\subsection{引力波}
\subsubsection{TT Gauge下的Einstein Field Equation}

前面我们研究了有源的Einstein Field Equation在Hilbert Gauge下面的求解。下面我们希望求解无源的Einstein Field Equation在Hilbert Gauge下面的解,也就是引力波的解。

\bigskip
\hlr{homogenious EFE in Hilbert Gauge}

我们考虑之前讨论的EFE在Hilbert Gauge下面的形式:
\begin{align}
  \square\gamma_{\mu\nu}=0\mathrm{~.} \quad \partial^\nu \gamma_{\mu\nu}=0\mathrm{~.}
\end{align}
我们计算变量和方程的数目。
\begin{itemize}
  \item 变量$ \gamma_{\mu\nu} $,作为对称张量一共是10个变量
  \item \textbf{Gauge Condition}:Hilbert Gauge提供了4个约束条件
\end{itemize}
所以我们总共有10-4=6个自由度。但是这其实并没有完全固定我们的gauge condition。因为对于free field来说,我们还可以要求更多的条件。


\bigskip
\hlr{Tranceless-Transverse Gauge}

我们可以在Hilbert Grauge的基础上再要求更多的条件,我们给出下面这个gauge condition:
\defi{
  Transverse-Tranceless Gauge

对于无源的Einstein Field Equation,我们可以对于metric要求满足:
\begin{align}
  h_{\mu0}=0,\quad\sum_kh_{kk}=0,\quad \partial_j h_{kj}=0.
\end{align}
}
我们可以证明永远可以找到一个coordinate transformation对于无源的metric满足这个条件。

\bigskip
\hlr{TT Gauge选择证明}


首先复习一般坐标变换前后在线性近似成立的基础上满足\cref{eq:gammatransform}的关系,并且Hilbert Gauge左手边的项满足\cref{eq:hilbertgaugecondition}的要求。我们会发现,如果选择一个坐标变换满足:
\begin{align}
  \square\xi_\mu = 0\mathrm{~,}
\end{align}
那么Hilbert Gauge条件依然成立。说明有一些自由度并没有被固定,我们没有完整的选择一个坐标系,也就是一个gauge。我们发现对于一个$ \square \xi^\mu = 0 $可以根据求道和$ \square $算符对易知道:
\begin{align}
  \Box\, \delta\gamma_{\mu\nu} 
    &= \Box\, \xi_{\mu\nu} 
     = \Box\!\left( \partial_\mu \xi_\nu + \partial_\nu \xi_\mu 
       - \eta_{\mu\nu} \partial_\rho \xi^\rho \right)
     = 0 .
\end{align}
对于无源系统,我们总能够选择一个坐标保证变量部分是0。
\rmk{
  对于有源系统,我们不能够这样选择。因为比如选择$ \gamma_{0i} = 0 $但是$ T_{0i} \neq 0 $,那么我们就不能够满足EFE了。
}

我们可以进一步在Hilbert Gauge的基础上选择一个坐标变换满足上面的方程,然后选择合适的$ \xi_\mu $保证:
\begin{itemize}
  \item $ \xi_0 $的选择保证$ \gamma = 0 $
  \item $ \xi_i $的选择保证$ \gamma_{0i} = 0 $
\end{itemize}
根据hilbert gauge条件,我们有$ \partial_{0}h^{00} = 0 $我们不妨选择$ h_{00} = 0 $就得到了TT gauge的条件。这里我们比Hilbert Gauge多选择了4个约束条件,所以自由度为:
\begin{align}
  10 - 4 - 4 = 2\mathrm{~.}
\end{align}

\subsubsection{引力波的解}

\bigskip
\hlr{EFE在TT Gauge下的解}

\thm{
  EFE in TT Gauge 

  对于无源的EFE在TT Gauge下面,我们有:
  \begin{align}
    \square h_{ij}^{TT} = 0\mathrm{~,}
  \end{align}
}
对于这个方程我们有一个自然的ansatz:
\begin{align}
  h_{ij}^{\mathrm{TT}}(x)=e_{ij}(\mathbf{k})e^{ik^\mu x_\mu}
\end{align}
我们选择$ k^\mu = (\omega, k^i) $。并且$ e_{ij}(\mathbf{k}) $是极化张量,我们把这个ansatz带入EFE以及TT Gauge条件,我们有:
\begin{itemize}
  \item \textbf{EFE条件}:$ k^\mu k_\mu = 0 $也就是说引力波是以光速传播的。
  \item \textbf{TT Gauge条件}
    \begin{itemize}
      \item \textbf{Hilbert Gauge条件}:$ \partial^jh_{ij}=0 \Rightarrow n^ih_{ij}=0. $其中$ \hat{\mathbf{n}}=\mathbf{k}/|\mathbf{k}| $
      \item \textbf{Tranceless条件}:$ h^i_i = 0 $
    \end{itemize}
\end{itemize}
如果在一个坐标系之中我们的引力波沿着$ z $方向传播,那么我们可以写出极化张量的形式:
\begin{align}
  \left.h_{ij}^{\mathrm{TT}}(t,z)=\left(\begin{array}{ccc}h_+&h_\times&0\\h_\times&-h_+&0\\0&0&0\end{array}\right.\right)_{ij}e^{i(kz -\omega t)}\mathrm{~,}
\end{align}
其中$ h_+ $和$ h_\times $是两个独立的极化态,也就是之前讨论的两个独立自由度。
最后添加上平直时空的项给出完整的metric perturbation:
\thm{\label{thm:ttwave}
  TT Gauge下面z方向传播的引力波解

  对于一个沿着$ z $方向传播的引力波,在TT Gauge下面我们有metric:
\begin{align}
  ds^2&=-dt^2+dz^2 \\
      &+\{1+h_+\text{exp}[i(kz- \omega )t]\}dx^2+\{1-h_+\text{exp}[i(kz - \omega t)]\}dy^2+2h_\times\text{exp}[i(kz- \omega )t]\}dxdy.
\end{align}
}

\subsubsection{TT Gauge一些性质}\label{sec:ttgaugeprop}
\bigskip
\hlr{投影到TT Gauge之中}

对于一个Hilbert Gauge下面的解,我们可以使用一个系统方法投影到TT Gauge之中。我们定义一个投影算符:
\defi{
  pre TT Projection Operator

  考虑Hilbert Gauge下面的波动解沿着$ \mathbf{n} $方向进行传播,我们定义投影算符:
  \begin{align}
    P_{ij}(\hat{\mathbf{n}})=\delta_{ij}-n_in_j\mathrm{~.}
  \end{align}
}
我们发现这个投影算符满足: 对称的;Transverse的(也就是$ n^iP_{ij} = 0 $);Trance 2的。并且满足投影算符的定义:
\begin{align}
  P_{ik}P_{kj}=P_{ij}\mathrm{~.}
\end{align}
我们可以使用这个投影算符定义TT投影算符:
\defi{
  TT Projection Operator

  对于一个Hilbert Gauge下面的波动解$ h_{ij} $,我们定义TT投影算符:
  \begin{align}
    \Lambda_{ij,kl}(\hat{\mathbf{n}})=P_{ik}(\hat{\mathbf{n}})P_{jl}(\hat{\mathbf{n}})-\frac{1}{2}P_{ij}(\hat{\mathbf{n}})P_{kl}(\hat{\mathbf{n}})\mathrm{~.}
  \end{align}
}
对于这个投影算符我们很明显发现有下面的性质:
\begin{itemize}
  \item 这依旧是一个投影算符 : $ \Lambda_{ij,kl}\Lambda_{kl,mn}=\Lambda_{ij,mn}\mathrm{~.} $
  \item 对于所有index都是transverse的:$ n^i\Lambda_{ij,kl}=0\mathrm{~.} $
  \item Tranceless性质:$ \Lambda_{ii,kl}=0 = \Lambda_{ij,kk}\mathrm{~.} $
  \item 对于同时变换前后两个指标是对称的:$ (i,j)\leftrightarrow(k,l). $
    \item 具体用$ n $写出来形式是:
      \begin{align}
        \Lambda_{ij,kl}(\hat{\mathbf{n}})&=\delta_{ik}\delta_{jl}-\frac{1}{2}\delta_{ij}\delta_{kl}-n_jn_l\delta_{ik}-n_in_k\delta_{jl}\\&\left.+\frac{1}{2}n_kn_l\delta_{ij}+\frac{1}{2}n_in_j\delta_{kl}+\frac{1}{2}n_in_jn_kn_l\right..
      \end{align}
\end{itemize}

\thm{
  投影到TT Gauge

  对于一个Hilbert Gauge下面的波动解 $ h_{\mu\nu} $,我们选择其空间部分$ h_{ij} $,那么其在TT Gauge下面的形式为:
  \begin{align}
  h_{ij}^{\mathrm{TT}}=\Lambda_{ij,kl}(\hat{\mathbf{n}}) h_{kl}\mathrm{~.}
  \end{align}
  当然根据定义我们有$ h_{0\mu}^{\mathrm{TT}}=0. $其中$ \hat{\mathbf{n}} $是波动传播的方向。
}

\bigskip
\hlr{spin-2性质}

\YL{[并不是特别重要,有需要补充]}





\subsection{Questiona and thoughts}

\question{为什么有源系统我们不能选择TT gauge??有源系统为什么不需要进一步gauge fixing}

\YL{[目前没有完全理解捏]}

