\subsection{Coherent State与经典谐振子对应}

\hlr{Coherent State具体形式}

我们试图解满足上面说到的两个条件的量子态的具体形式,我们得到这个约束条件:
\begin{align}
  \hat{a}\left|\psi_{\alpha}\right\rangle=\alpha\left|\psi_{\alpha}\right\rangle. 
\end{align}
我们将态进行展开研究,带入上面的方程给出:
\begin{align}
  \left|\alpha\right\rangle=\sum_{n=0}^{\infty}C_{n}\left|n\right\rangle.\\ 
  C_n=\frac{\alpha^n}{\sqrt{n!}}C_0\mathrm{~.}
\end{align}
结合这两点我们可以写出coherent state up to 一个归一化常数:
\begin{align}
  \left|\alpha\right\rangle=\sum_{n=0}^{\infty}\frac{\alpha^{n}}{\sqrt{n!}}C_{0}\left|n\right\rangle=\sum_{n=0}^{\infty}\frac{\alpha^{n}}{\sqrt{n!}}C_{0}\left[\frac{\left(\hat{a}^{\dagger}\right)^{n}}{\sqrt{n!}}\left|0\right\rangle\right]=C_{0}e^{\alpha\hat{a}^{\dagger}}\left|0\right\rangle.
\end{align}
最后得到下面的形式:
\imp{Coherent State标准形式}{
  \begin{align}
    |\alpha\rangle=\exp\left(-\frac{1}{2}|\alpha|^{2}+\alpha\hat{a}^{\dagger}\right)|0\rangle.
  \end{align}
}
对于求归一化系数的时候我们可以使用一个技巧:
\tip{Glauber’s formula}{
  如果两个算符之间满足下面的关系:
  \begin{align}
    [A,[A,B]]=0\mathrm{~,~}\quad[B,[A,B]]=0\mathrm{~,}
  \end{align}
  那么这两个算符写在指数上面就会满足:
  \begin{align}
    e^Ae^B=e^{\frac{1}{2}[A,B]}e^{A+B}.
  \end{align}
}
对于产生湮灭算符这个关系就很好用,因为产生湮灭算符的对易子是1!!

\bigskip
\hlr{Coherent State的能量分布}

下面计算coherent state这个态在不同能量上面的概率也就是能量分布 $ \mathscr{P}_n = |\braket{n}{\alpha}|^2 $。很容易计算出来准确的结果是:
\begin{align}
  \mathscr{P}_n=|C_n|^2=e^{-|\alpha|^2}\frac{|\alpha|^{2n}}{n!}\mathrm{~.}
\end{align}
我们下面讨论这个分布在经典的行为,也就是当能量很大很大的时候,$ \alpha $很大很大时候的行为,我们使用下面的近似:
\begin{align}
  n!\sim n^ne^{-n}\sqrt{2\pi n}\mathrm{~,~}\quad n\to\infty\mathrm{~.}
\end{align}
最后试图把所有的项写在指数上面结果就是:
\begin{align}
  \mathscr{P}_n&\approx\frac{1}{\sqrt{2\pi n}}e^{S(n)},\\
S(n)&=-n\log n+n+2n\log|\alpha|-|\alpha|^2\mathrm{~.}
\end{align}
下面研究这个分布性质:
\begin{enumerate}
  \item 最大值:$ S'(n)=0 $,解出最大值在 $ n=|\alpha|^2 $ 处。
  \item 正好这个时候$ S(n) = 0 $
  \item 分布计算到二阶是:
\begin{align}
  \begin{aligned}\mathscr{P}_{n}&\approx\frac{1}{\sqrt{2\pi\bar{n}}}\exp\left[S(\bar{n})+\frac{1}{2}\left.\frac{\partial^2}{\partial n^2}S(n)\right|_{n=\bar{n}}(n-\bar{n})^2\right]\\&=\frac{1}{\sqrt{2\pi|\alpha|^2}}\mathrm{exp}\left[-\frac{\left(n-|\alpha|^2\right)^2}{2|\alpha|^2}\right].\end{aligned}
\end{align}
  % \item 这个分布很窄 $ \Delta n\sim|\alpha|\ll|\alpha|^2,\quad|\alpha|\gg1\mathrm{~.} $
\end{enumerate}

\hlr{计算动量,位置分布呜呜}

没啥意义,看看作业算了!

\bigskip
\hlr{Coherent State在位置表象下面}

没什么技术含量,唯一的提醒就是请引入$ \hat{D}(\alpha)=\exp\left(\alpha\hat{a}^\dagger-\alpha^*\hat{a}\right) $进行辅助求解!






\subsection{Questions and thoughts}

\tip{关于Coherent State重要计算细节}{
  我们需要知道这个态的定义给出了$ a \ket{\alpha} = \alpha \ket{\alpha} $但是从来没有说$ a^\dagger $作用上去会发生什么!!!!所以正确处理的方法就是把所有dagger的部分都放在左边作用在bra上面!!!
}

\question{空间平移算符是怎么推导出来的捏?}

我们一种推导方式「之后我们推导CFT的所有生成元的算符形式也是这么推的」,就是寻找这个变换前后的函数之间的关系。每一个变换必然是有一个参数进行描述的,我们考虑这个参数在无限接近于identity的时候的行为。

比如对于空间平移我们考虑 $ a = 0 + \epsilon $的行为我们有:
\begin{align}
  \psi(x-\delta a)\approx\psi(x)-\delta a\frac{d}{dx}\psi(x).\quad (\hat{T}(\delta a)\psi)(x)\approx\left(1-\delta a\frac{d}{dx}\right)\psi(x).\\ 
  \hat{T}(\delta a)\approx1-\frac{i}{\hbar}\delta a\hat{p},\hat{p}=-i\hbar\frac{d}{dx}.
\end{align}
所以我们就有这个算符的无限小变换的形式。为了得到完整的算符,我们不妨搞一个指数映射,从而给出这个!!给出的结论是:
\begin{align}
  \hat{T}(a)=\exp\left(-\frac{i}{\hbar}a\hat{p}\right).\\ 
  \psi(x-a) = (\hat{T}(a)\psi)(x).
\end{align}


\qed




\bigskip
\question{Heisenberg Picyure之中算符的演化计算?}

复习一下就是:
\begin{align}
  |\psi_H\rangle=|\psi_S(0)\rangle,\quad A_H(t)=U^\dagger(t)A_SU(t),U(t)=e^{-\frac{i}{\hbar}Ht}
\end{align}
\qed 


\bigskip
\question{时间演化作用在一个大算符上面,一定等于分别作用在小算符上面吗?}

注意!!!这并不一定是正确的!!
\begin{align}
  (AB)_H(t)\neq A_H(t)B_H(t)
\end{align}
\qed

\bigskip
\question{
  量子力学里面我们怎么计算一个巨大的指数上的算符演化一个算符呢?
}

也就是想问我们怎么算这种东西$ e^{-\hat{O}}\hat{b}e^{\hat{O}} $ 我们对此使用Baker-Campbell-Hausdorff (BCH) 展开进行处理!也就是:
\begin{align}
  e^{-\hat{O}}\hat{b}e^{\hat{O}}=\hat{b}+[\hat{b},\hat{O}]+\frac{1}{2!}[[\hat{b},\hat{O}],\hat{O}]+\frac{1}{3!}[[[\hat{b},\hat{O}],\hat{O}],\hat{O}]+\cdots
\end{align}
\qed
