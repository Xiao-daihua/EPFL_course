\subsection{Stationary Scattering State}


\bigskip
\hlr{一维散射理论analogue}

对于一维的散射理论我们一般使用了一种思路进行求解:
\begin{itemize}
  \item 首先假设入射和出射都是平面波形式,并且存在反射和透射系数:
    \begin{align}
      \begin{aligned}&\psi\to e^{ipx}+Re^{-ipx},\quad x\to-\infty,\\&\psi\to De^{ipx},\quad x\to+\infty.\end{aligned}
    \end{align}
    \item 将其作为边界条件,求解定态Schrodinger方程得到反射和透射系数。
      \item 得到一维的S矩阵以及散射振幅。注意,对于1维来说,由于入射和透射需要能量大小一样,所以方向其实只有两种可能$ p,-p $,因此S矩阵是一个$ 2\times2 $的矩阵:
\end{itemize}
\[
S = {\renewcommand{\arraystretch}{0.8}
\begin{pmatrix}
D & R\\
R & D
\end{pmatrix}}
\]



\bigskip
\hlr{一般散射理论推广}

我们对于一般的散射理论也可以使用类似方法求解,但是我们需要知道两个东西:
\begin{enumerate}
  \item 类比经典散射理论,有什么类似$ p $的标记能够展开所有的stationary scattering state吗?
    \item stationary scattering state的临界行为是什么?
\end{enumerate}

下面我们分别回答这两个问题。

\bigskip
\hlr{Sacttering State的展开复习}

这里我们再仔细回顾一下之前讨论的Hilbert Space的结构以及使用Scattering State和Bound State进行完备展开的内容。
对于之前散射理论的讨论,我们存在假设:
\begin{itemize}
  \item 整个Hilbert Space可以由自由粒子的波函数进行很好的展开。
\end{itemize}
因此对于一切in state我们都可以写作:
\begin{align}
  |\phi\rangle=\int d^3\mathbf{p}\ \phi(\mathbf{p})|\mathbf{p}\rangle\mathrm{~.}
\end{align}
然后我们知道在0时间的散射态可以通过in state作用上Moller Operator得到,所以我们有:
\begin{align}
  \boldsymbol{\Omega}_+|\boldsymbol{\phi}\rangle=\int d^3\mathbf{p}\boldsymbol{\phi}(\mathbf{p})\boldsymbol{\Omega}_+|\mathbf{p}\rangle=\int d^3\mathbf{p}\boldsymbol{\phi}(\mathbf{p})|\mathbf{p}_+\rangle\mathrm{~.}
\end{align}
其中我们定义$ \ket{\mathbf{p}_+} = \Omega_+ \ket{\mathbf{p}} $。这个态性质很好,恰好是Hamiltonian的本征值为$ p^2/2m $的本征态。并且一切的stationary scattering state都可以通过不同的$ \phi(\mathbf{p}) $进行线性组合得到。因此我们可以使用$ \mathbf{p} $来标记stationary scattering state。我们下面讨论其性质:
\begin{itemize}
  \item \textbf{Hamiltonian本征态}:
    由于Moller Operator的定义,我们有
    \begin{align}
      H|\mathbf{p}_+\rangle=H\boldsymbol{\Omega}_+|\mathbf{p}\rangle=\boldsymbol{\Omega}_+H_0|\mathbf{p}\rangle=\frac{p^2}{2m}|\mathbf{p}_+\rangle\mathrm{~.}
    \end{align}
  \item \textbf{正交性:} 根据Moller Operator的性质,我们有
    \begin{align}
      \langle \mathbf{p}'_+|\mathbf{p}_+\rangle=\langle \mathbf{p}'|\boldsymbol{\Omega}_+^\dagger\boldsymbol{\Omega}_+|\mathbf{p}\rangle=\langle \mathbf{p}'|\mathbf{p}\rangle=\delta(\mathbf{p}'-\mathbf{p})\mathrm{~.}
    \end{align}
  \item \textbf{完备性:} 我们之前假设Hilbert Space在散射时刻可以使用Moller Operator像空间「散射态」;以及Bounded State进行完备展开,数学上写出来就是:
    \begin{enumerate}
      \item 对于无穷远时间,我们有:
        \begin{equation}
          \mathbb{1} = \int d^3\mathbf{p}\,|\mathbf{p}\rangle\langle\mathbf{p}|
        \end{equation}

      \item 对于 $t=0$ 时间:
        \begin{align}
          \mathbb{1}
    &= \int d^3\mathbf{p}\,|\mathbf{p}_+\rangle\langle\mathbf{p}_+|
    + \sum_n |n\rangle\langle n|\\
    &= \int d^3\mathbf{p}\,|\mathbf{p}_-\rangle\langle\mathbf{p}_-|
    + \sum_n |n\rangle\langle n|
        \end{align}
    \end{enumerate}
\end{itemize}
\rmk{
  其实这个内容在section 7已经假设讨论了。这里只是总结强调一下。
}


\bigskip
\hlr{Stationary Scattering State的波函数渐进行为}

所以我们就要讨论$ \ket{\mathbf{p}_+} $的波函数形式的渐近行为。我们给出定理

\thm{
  Stationary Scattering State的波函数

  对于散射理论,stationary scattering state的波函数在远离散射中心时的渐近行为为:
  \begin{align}
   \langle \mathbf{x}|\mathbf{p}_+\rangle
= \frac{1}{(2\pi)^{3/2}}
\left[
e^{i\mathbf{p}\cdot\mathbf{x}}
+ f\!\left(p,\frac{\mathbf{x}}{r}\leftarrow\mathbf{p}\right)\,
\frac{e^{ipr}}{r}
\right] 
  \end{align}
}
Proof: 可以使用Moller Operator的Green函数表示进行证明,回顾\cref{thm:Moller_Green},我们带入:
\begin{align}
  |\mathbf{p}_+\rangle=|\mathbf{p}\rangle+G(E_p+i\boldsymbol{\varepsilon})V|\mathbf{p}\rangle.
\end{align}
然后我们使用T矩阵的特殊换位形式 \cref{eq:T_matrix_identities},我们有:
\begin{align}
  \langle\mathbf{x}|\mathbf{p}_+\rangle=\langle\mathbf{x}|\mathbf{p}\rangle+\langle\mathbf{x}|G_0(E_p+i\boldsymbol{\varepsilon})T(E_p+i\boldsymbol{\varepsilon})|\mathbf{p}\rangle
\end{align}
代入T矩阵和散射振幅的关系我们有:
\begin{align}
  \langle\mathbf{x}|\mathbf{p}_+\rangle=\frac{1}{(2\pi)^{3/2}}\left[e^{i\mathbf{p}\cdot\mathbf{x}}-\frac{1}{(2\pi)^2m}\int d^3\mathbf{x}^{\prime}d^3\mathbf{p}^{\prime}e^{i\mathbf{p}^{\prime}\cdot\mathbf{x}^{\prime}}f(\mathbf{p}^{\prime}\leftarrow\mathbf{p})\langle \mathbf{x}|G_0(E_p+i\varepsilon)|\mathbf{x}^{\prime}\rangle\right].
\end{align}
这里我们使用自由粒子Green函数的形式\cref{eq:free_particle_green_function},可以得到:
\begin{align}
  \left\langle\mathbf{x}|G_0(E+i\boldsymbol{\varepsilon})|\mathbf{x}^{\prime}\right\rangle=-\frac{m}{2\boldsymbol{\pi}|\mathbf{x}-\mathbf{x}^{\prime}|}e^{ip|\mathbf{x}-\mathbf{x}^{\prime}|-\boldsymbol{\varepsilon}|\mathbf{x}-\mathbf{x}^{\prime}|}.
\end{align}
下面考虑$ r = |\mathbf{x}| \to \infty $的渐近行为,进行一波计算我们有:
\begin{align}
   \langle \mathbf{x}|\mathbf{p}_+\rangle
  =\frac{1}{(2\pi)^{3/2}}\left[e^{i\mathbf{p}\cdot\mathbf{x}}-f\left(p\frac{\mathbf{x}}{r}\leftarrow\mathbf{p}\right)\frac{1}{r}e^{ipr}\right],
\end{align}

\subsection{S 矩阵的T Production表示}

对于S矩阵我们有一种特殊的表示方法,称为T Production表示,结论是:
\thm{
S 矩阵的T Production表示

S矩阵可以被势能表示为:
\begin{align}
  S=T\exp\left(-i\int_{-\infty}^\infty dtV(t)\right)\mathrm{~.}
\end{align}
其中T Product的定义是:
\begin{align}
  TV(t_1)V(t_2)=\theta(t_1-t_2)V(t_1)V(t_2)+\theta(t_2-t_1)V(t_2)V(t_1)
\end{align}
也就是永远保持算符时间从右到左时间递减排列。
}

\bigskip
\hlr{S 矩阵的Interacting Picture}

我们回顾一下S矩阵使用Interacting Picture的定义。Interaction Picture之中时间演化算符被定义为\cref{eq:interaction_picture_time_evolution},我们抄下来是:
\begin{align}
    S(t_1,t_2)=e^{iH_0t_1}e^{iH(t_2-t_1)}e^{-iH_0t_2}.
\end{align}
S 矩阵其实是这个算符在$ t_1\to+\infty,t_2\to-\infty $的极限。我们使用积分Trick进行计算这个算符,我们知道$ S(t_2,t_2) = 0 $,所以时间演化算符可以写作:
\begin{align}
  S(t_1,t_2) = 1 + \int_{t_2}^{t_1} dt \frac{d}{dt} S(t,t_2).
\end{align}
计算右边的导数我们有:
\begin{align}
  \frac{\partial}{\partial t_1}S(t_1,t_2)=-iU_0^\dagger(t_1)(H-H_0)U_0(t_1)S(t_1,t_2)=-iV(t_1)S(t_1,t_2),
\end{align}
带入之后我们会有:
\begin{align}
  S(t_1,t_2)=1-i\int_{t_2}^{t_1}dtV(t)S(t,t_2).
\end{align}
我们发现右边积分里面又有$ S(t,t_2) $,所以我们可以继续迭代这个公式:
\begin{align}
  S=\sum_{n=0}^\infty S_n^2, \quad S_n=(-i)^n\int_{t_2}^{t_1}dtV(t)\int_{t_2}^tdt^{\prime}V(t^{\prime})\int_{t_2}^{t^{\prime}}dt^{\prime\prime}V(t^{\prime\prime})\ldots
\end{align}
并且由于我们的迭代,这些V算符的时间顺序是从右到左递减的。因此我们可以使用T Product的形式以及对于一个cube积分进行改写:
\begin{align}
  S_n=(-1)^n\frac{1}{n!}\int_{t_1}^{t_2}dt\int_{t_1}^{t_2}dt^{\prime}\int_{t_1}^{t_2}dt^{\prime\prime}\ldots\left(TV(t)V(t^{\prime})V(t^{\prime\prime})\ldots\right).
\end{align}
\rmk{
  注意!最后这个式子我们是对于cube进行积分,积分上下限制完全都是一样的。
}
