这一章介绍的是古人对于相对论和量子力学结合的早期探索。内容从现在的观点是错误的,dirac equation是场方程而非波函数的Schrodinger方程。而真正本质的应该是相对论量子场论。

\subsection{Relativistic Schrodinger Equation}

\subsubsection{Failed Attempts}
古人的理想是使用协变的语言书写量子力学。因此进行了一些尝试。但是一些naive的尝试都最终失败了!

\bigskip
\hlr{直接使用相对论粒子能量}

我们知道对于相对论点粒子能量是$ E=\sqrt{p^2+m^2} $。因此我们可以尝试直接将Schrodinger方程中的Hamiltonian替换为这个相对论的形式:
\begin{align}
  i\frac{\partial\psi}{\partial t}=\sqrt{m^2c^4-\hbar^2c^2\nabla^2}\psi\mathrm{~.}
\end{align}
但遇到的问题是这个方程右手边可以改写为:
\begin{align}
  \sqrt{m^2c^4-\hbar^2c^2\nabla^2}\psi(x)=\int d^3\mathbf{x}^{\prime}F(\mathbf{x},\mathbf{x}^{\prime})\psi(\mathbf{x}^{\prime}) \quad \text{where} \quad F(\mathbf{x},\mathbf{x}^{\prime})\propto\int d^3\mathbf{p}e^{i\mathbf{p}\cdot(\mathbf{x}-\mathbf{x}^{\prime})}\sqrt{\mathbf{p}^2c^2+m^2c^4}.
\end{align}
这意味着波函数一点的时间演化会被整个空间的分布影响。这违反了相对论的locality原则!

\bigskip
\hlr{Klein-Gordon Equation}

由于上面的方程之中存在算符的平方根,很难处理。因此古人尝试将Schrodinger方程两边平方,得到:
\begin{align}
  -\square\psi=\frac{m^2c^4}{\hbar^2}\psi
\end{align}
也就是Klein-Gordon Equation。但是会发现这个方程存在两个问题:
\begin{itemize}
  \item 负能量解:这个方程显然存在平面波解但是其能量为: $ E=\pm\sqrt{\mathbf{p}^2c^2+m^2c^4}\mathrm{~.} $这表示能量没有下界。
    \item 概率密度无法定义:普通的Schrodinger Equation的概率密度是守恒的也就是说:
      \begin{align}
        \partial_{t} \int d^3\mathbf{x} |\psi|^2 = 0\mathrm{~.} \quad \text{with} \quad \psi^* \psi \geq 0\mathrm{~.}
      \end{align}
      这是因为Schrodinger方程是1阶时间导数的方程。但是Klein-Gordon Equation是二阶方程,如果我们希望定义一个概率的interpretation必然不是$ |\psi|^2 $这个形式。因此我们需要重新定义概率密度。古人定义了:
      \begin{align}
        \rho = \frac{i\hbar}{2m}(\psi^*\partial_t\psi - \psi\partial_t\psi^*)\mathrm{~.} \quad \text{satisfying} \quad  \partial_t \int d^3\mathbf{x} \rho = 0\mathrm{~.}
      \end{align}
      但是这个密度并不是正定的,也就是说这个流并不是永远大于0的,因此不能作为概率密度来解释。
\end{itemize}

\subsubsection{Dirac Equation}

为了强行构造一个协变的方程,另一个思路就是我们保留时间导数是一阶,但是强行要求Hamiltonian也是线性的,并且不要使用什么平方根的形式。也就是说我们要求:
\begin{align}
  H=c\sum_{i=1}^3\alpha_ip_i+\beta mc^2.
\end{align}
但为了能够和相对论量子力学匹配我们要求这个形式化的Hamiltonian的平方应该给出相对论的能量的平方。$ H^2 = p^2c^2 + m^2c^4 $。
\rmk{
  这个构造就相当于严格的定义了什么叫「开平方」!
}
对于这个构造我们会发现如果$ \alpha_i, \beta $是数字的话永远不能满足这个关系。因为我们计算会发现:
\begin{align}
  H^2 = c^2\sum_{i,j}\alpha_i\alpha_j p_ip_j + mc^3\sum_i(\alpha_i\beta+\beta\alpha_i)p_i + m^2c^4\beta^2.
\end{align}

