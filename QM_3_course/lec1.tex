\subsection{自由粒子与谐振子的量子经典对应}
\hlr{量子体系经典对应}
\begin{itemize}
  \item 自由粒子对应着一个波包!
    \begin{itemize}
      \item 这玩意实在不是人算的,也没啥讨论意义
    \end{itemize}
  \item 谐振子需要对应一个量子态,满足两个条件。
    \begin{itemize}
      \item 对于动量和位置算符的期望值,满足经典谐振子的运动方程。
      \item 对于能量值期望满足经典关系
    \end{itemize}
\end{itemize}
这样的两个约束条件下,我们可以形式化的写出来是:
\begin{align}
  \left\langle\psi_{\alpha_{0}}|\hat{a}|\psi_{\alpha_{0}}\right\rangle&=\alpha_{0}.\\ 
  \langle\psi_{\alpha_0}|\hat{H}|\psi_{\alpha_0}\rangle&=\hbar\omega\left(|\alpha_0|^2+\frac{1}{2}\right).
\end{align}
我们之后推导这样的量子态的具体形式。

\bigskip
\hlr{一维量子系统的归一化定义}

我们定义归一化是这样的,对于所有基态都是满足:
\begin{align}
  \int dx |x\rangle\langle x|&=1,\\
  \int dp |p\rangle\langle p|&=1. 
\end{align}
并且在此基础上我们根据动量算符的定义,解本征方程会给出:
\begin{align}
  -i\hbar\frac{d}{dx}\langle x|p\rangle=p\langle x|p\rangle. 
\end{align}
解出:
\begin{align}
  \langle x|p\rangle&=\frac{1}{\sqrt{2\pi\hbar}}e^{ipx/\hbar}. 
\end{align}
所以我们有傅立叶变换正反变换的关系是:
\begin{align}
  \tilde{\boldsymbol{\psi}}(p)=\frac{1}{\sqrt{2\pi\hbar}}\int_{-\infty}^{\infty}dx\boldsymbol{\psi}(x)e^{-\frac{i}{\hbar}px}.\\ 
  \tilde{\boldsymbol{\psi}}(x)=\frac{1}{\sqrt{2\pi\hbar}}\int_{-\infty}^{\infty}dp\boldsymbol{\psi}(p)e^{\frac{i}{\hbar}px}
\end{align}
接下来我们也可以定义合理的delta函数:
\rmk{注意!delta函数的形式和我们选择的归一化没有任何关系!}
\begin{align}
  \int_{-\infty}^\infty dpe^{\frac i\hbar pz}&=2\pi\hbar\delta(z)\mathrm{~,}\\
  \int_{-\infty}^\infty dz\operatorname{\delta}(z)&=1\mathrm{~.}
\end{align}
对于三维的情况我们也有:
\begin{align}
  \int d^3\mathbf{x}e^{i(\mathbf{k-p})\cdot\mathbf{x}}=(2\pi)^3\delta^3(\mathbf{k-p})
\end{align}

\subsection{量子态的完备性条件和归一化的两种convention}

对于QFT和QM我们一般使用两个布艺样的动量本征态的定义,虽然两者仅仅差了一个系数,同时使用了不一样的归一化条件,导致是等价的,但我们下面列举其区别:


\bigskip
\hlr{QM使用}

\begin{table}[H]
\centering
\begin{tabular}{ccc}
\toprule
\textbf{项目} & \textbf{位置表象 $|\mathbf{x}\rangle$} & \textbf{动量表象 $|\mathbf{p}\rangle$} \\
\midrule
归一化条件 & 
$\langle \mathbf{x}|\mathbf{x}'\rangle = \delta^{(3)}(\mathbf{x}-\mathbf{x}')$ & 
$\langle \mathbf{p}|\mathbf{p}'\rangle = \delta^{(3)}(\mathbf{p}-\mathbf{p}')$ \\
完备性条件 & 
$\displaystyle \int d^3x\,|\mathbf{x}\rangle\langle \mathbf{x}| = \mathbf{1}$ & 
$\displaystyle \int d^3p\,|\mathbf{p}\rangle\langle \mathbf{p}| = \mathbf{1}$ \\
本征态在另一个表象下的表示 & 
$\langle \mathbf{x}|\mathbf{p}\rangle = \frac{1}{(2\pi)^{3/2}} e^{i\mathbf{p}\cdot\mathbf{x}}$ & 
$\langle \mathbf{p}|\mathbf{x}\rangle = \frac{1}{(2\pi)^{3/2}} e^{-i\mathbf{p}\cdot\mathbf{x}}$ \\
空间表象下波函数 $\phi(\mathbf{x})$ & 
$\phi(\mathbf{x}) = \langle \mathbf{x}|\phi\rangle$ & 
$\phi(\mathbf{x}) = \frac{1}{(2\pi)^{3/2}}\int d^3p\, e^{i\mathbf{p}\cdot\mathbf{x}} \tilde{\phi}(\mathbf{p})$ \\
动量表象波函数 $\tilde{\phi}(\mathbf{p})$ & 
$\tilde{\phi}(\mathbf{p}) = \frac{1}{(2\pi)^{3/2}} \int d^3x\, e^{-i\mathbf{p}\cdot \mathbf{x}} \phi(\mathbf{x})$ & 
$\tilde{\phi}(\mathbf{p}) = \langle \mathbf{p}|\phi\rangle$ \\
\bottomrule
\end{tabular}
\caption{量子力学(QM)常用规范下位置与动量表象对照}
\end{table}


\bigskip
\hlr{QFT使用}

\begin{table}[h!]
\centering
\begin{tabular}{ccc}
\toprule
\textbf{项目} & \textbf{位置/场算符表象 $|\mathbf{x}\rangle$ / $\phi(\mathbf{x})$} & \textbf{动量表象 $|\mathbf{p}\rangle$ / $\tilde{\phi}(\mathbf{p})$} \\
\midrule
归一化条件 & 
$\langle \mathbf{x}|\mathbf{x}'\rangle = \delta^{(3)}(\mathbf{x}-\mathbf{x}')$ & 
$\langle \mathbf{p}|\mathbf{p}'\rangle = (2\pi)^3 \delta^{(3)}(\mathbf{p}-\mathbf{p}')$ \\
完备性条件 & 
$\displaystyle \int d^3x\,|\mathbf{x}\rangle\langle \mathbf{x}| = \mathbf{1}$ & 
$\displaystyle \int \frac{d^3p}{(2\pi)^3}\,|\mathbf{p}\rangle\langle \mathbf{p}| = \mathbf{1}$ \\
本征态在另一个表象下的表示 & 
$\langle \mathbf{x}|\mathbf{p}\rangle = e^{i\mathbf{p}\cdot\mathbf{x}}$ & 
$\langle \mathbf{p  }|\mathbf{x}\rangle = e^{-i\mathbf{p}\cdot\mathbf{x}}$ \\ 
空间表象下波函数or场算符 & 
$\phi(\mathbf{x}) = \langle \mathbf{x}|\phi\rangle$ & 
$\phi(\mathbf{x}) = \int \frac{d^3p}{(2\pi)^3} e^{i\mathbf{p}\cdot\mathbf{x}} \tilde{\phi}(\mathbf{p})$ \\
动量表象下波函数or场算符 & 
$\tilde{\phi}(\mathbf{p}) = \int d^3x\, e^{-i\mathbf{p}\cdot \mathbf{x}} \phi(\mathbf{x})$ & 
$\tilde{\phi}(\mathbf{p}) = \langle \mathbf{p}|\phi\rangle$ \\
\bottomrule
\end{tabular}
\caption{QFT 常用 \((2\pi)^3\) 归一化规范下的位置/动量表象对照}
\end{table}

对于这两种convention,我们的delta函数的定义是完全一样的:
\begin{align}
  \int d^3\mathbf{x}e^{i(\mathbf{k-p})\cdot\mathbf{x}}&=(2\pi)^3\delta^3(\mathbf{k-p})\mathrm{~,}\\
  \int d^3\mathbf{p}e^{-i(\mathbf{k-p})\cdot\mathbf{x}}&=(2\pi)^3\delta^3(\mathbf{k-p})\mathrm{~.}
\end{align}


\subsection{Questions and thoughts}

 \question{实正定矩阵有什么对角化性质,有什么相似变换性质?}

 所有实正定矩阵都可以进行对角化,对角化之后的矩阵是实对角矩阵,并且所有的特征值都是正实数。并且将其对角化的矩阵是正交矩阵,也就是满足:
 \begin{align}
  \det Q=\pm1 \quad Q^{-1}=Q^T
 \end{align}
对角化不会改变矩阵的特征值捏!!

 但是,我们可以通过一个变换让所有对角元素都是1.就是两边乘上对角元素是$ \displaystyle\frac{1}{\sqrt{\lambda_i}} $的矩阵,但是这个已经不是对角化了捏!!但是这是一个合理的坐标变换呀呀呀,更具体的说,其实就是一个rescaling!!

\qed
