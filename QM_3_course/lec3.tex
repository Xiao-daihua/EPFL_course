\subsection{Take home messages}

\hlr{WKB 近似基本定义}

我们希望解决一个一维的stationary schrodinger Equation,保证势能是任意的,并且得到一个合理的近似解应该怎么办捏?我们对于波函数进行下面改写:
\begin{align}
  \psi(x,t)=e^{\frac{i}{\hbar}S(x,t)}
\end{align}
带入定态方程之后得到结果是:
\begin{align}
  -i\frac{\hbar}{2m}S^{\prime\prime}+\frac{1}{2m}S^{\prime2}=E-V\mathrm{~.}
\end{align}
这就意味着我们的S必然是包含$ \hbar $的,我们将$ S $进行一个$ \hbar $的级数展开:
\begin{align}
  S=\sum_{n=0}^\infty\left(\frac{\hbar}{i}\right)^nS_n\mathrm{~.}
\end{align}
展开之后对比阶数给出一系列方程:
\begin{enumerate}
  \item \textbf{0 阶方程} 
    \begin{align}
      \frac{1}{2m}\left(S_0^{\prime}\right)^2=E-V.
    \end{align}
    这个方程解出来的结果是:
    \begin{align}
      S_0=\pm\int dxp(x)\mathrm{~,~}\quad p(x)=\sqrt{2m(E-V(x))}\mathrm{~.}
    \end{align}
  \item \textbf{1 阶方程}
    \begin{align}
      -i\frac{\hbar}{2m}S_0^{\prime\prime}+\frac{\hbar}{i}\frac{1}{2m}2S_0^{\prime}S_1^{\prime}=0\quad\Rightarrow\quad S_1^{\prime}=-\frac{S_0^{\prime\prime}}{2S_0^{\prime}}.
    \end{align}
\end{enumerate}
最后结合一下前两阶近似给出的结果是:
\begin{align}
  \psi(x)=\frac{C_1}{\sqrt{p(x)}}\mathrm{exp}\left(\frac{i}{\hbar}\int p(x)dx\right)+\frac{C_2}{\sqrt{p(x)}}\mathrm{exp}\left(-\frac{i}{\hbar}\int p(x)dx\right). 
\end{align}
\rmk{我们一定要记住对于两个$ \int p dx $务必选择同样的积分上下限!!}

\bigskip
\hlr{WKB 近似的适用条件}

最重要的有两个方面的适用条件;
\begin{enumerate}
  \item 前后项之间的差距很大 $ \hbar^nS_{n+1}(x)$ 远大于$\hbar^{n-1}S_n(x)^2, $ 
  \item 每一项都足够趋于0,因此$ \hbar^iS_{i+1}(x) << 1 ,\quad\text{for all }i>k $
\end{enumerate}
对于LO Approximation上面的条件可以写作【虽然我们一般仅仅验证后面两个,这个是最重要的!毕竟前两项都是要留下来的】:
\begin{align}
  |S_{1}|<<|S_{0}/\hbar|\mathrm{~,~}\quad|\hbar S_{2}|<<|S_{1}|\mathrm{~,~}\quad|\hbar S_{2}|<<1. 
\end{align}

\line
还有一个有直接物理意义的适用条件也就是,我们对比1阶和0阶的方程。如果能忽略一阶的方程那么就需要一阶方程远小于0阶的。我们找到leading term分别是$ S_0'' $和$ \hbar S_0'^2 $。


所以如果近似合理,我们需要满足:$ \left|\hbar S_{0}^{\prime\prime}\right| >>\left|S_{0}^{\prime2}\right|. $这就意味着德布罗伊波长要变化不大!这给出了另一个重要的近似条件:
\begin{align}
  |\lambda(x)'| << 1
\end{align}


\bigskip
\hlr{WKB近似在不同区域的求解}

这里比较复杂的推导不进行说明,但是我们可以知道类似简单的定态方程的求解。WKB近似给出;
\begin{enumerate}
  \item 对于classically allowed region $ E>V(x) $,可以有类似平面波的解,并且有两个待定的系数
  \item 对于classically forbidden region $ E<V(x) $,可以有类似指数衰减的解,并且有两个待定的系数;但是\textbf{有一个系数必然是0因为可以通过物理不能爆炸来决定} 
  \item 最后对于turning point的连接。我们这里不能适用WKB近似,但是可以适用线性势能近似。给出的结果就是Airy Function。我们用这个方程连接两边,给出系数之间的关系。最后只剩下一个通过归一化待定的系数!!
\end{enumerate}

\subsection{Questions and thoughts}

\question{WKB近似我们选择$ S_0 $的积分区间是什么?我们似乎很随意??}

一个事实就是我们考虑一个波函数是:$ \psi(x,t)  = e^{i/\hbar S(x,t)}$所以我们积分上下限不同其实就是一个相位因子而已,而对于一个波函数来说差一个相位对于物理完全没有影响。

值得注意的是,有影响的是相位差。所以我们务必要保证所有出现的$ S_0 $选择同样的一个常数,也就是我们把积分上或者下限选择成了$ x $,另一个上或者下限务必选择成【同一个位置!】

一个很好的选择就是turning point的位置捏!!
\qed



