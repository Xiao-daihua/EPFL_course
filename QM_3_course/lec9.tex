

求解S矩阵十分困难,但是我们可以通过一个变形把求解的困难转化为对于一个成为Green函数的求解。进一步我们会发现,可以使用Born近似以微扰展开的方法处理整个散射的问题!

\subsection{Moller Operator的Green Function方法}


\hlr{有限时间Moller算子}

对于有闲时间的Moller算符我们可以使用求导再进行定积分的方式进行改写:
\begin{align}
  U^\dagger(t)U_0(t)=1+\int_0^tdt^{\prime}\frac{d}{dt^{\prime}}\left[U^\dagger(t^{\prime})U_0(t^{\prime})\right]
\end{align}
我们带入时间演化算符的定义,然后进行形式化的求导计算可以得到:
\begin{align}
 U^\dagger(t)U_0(t) =1+i\int_0^tdt^{\prime}U^{\dagger}(t^{\prime})VU_0(t^{\prime}).
\end{align}

\bigskip
\hlr{Moller 算子作用在out state}

我们考虑Moller算子作用在out state上面的形式:
\begin{align}
  |\psi_0\rangle= &\lim_{t \to \infty} U^\dagger(t)U_0(t)|\psi_{\mathrm{out}}\rangle \\ 
 = &|\psi_{\mathrm{out}}\rangle+i\int_0^\infty d\tau U^\dagger(\tau)VU_0(\tau)|\psi_{\mathrm{out}}\rangle
\end{align}
上面的积分发散,所以我们需要使用一个数学Trick:
\begin{itemize}
  \item 在积分之中引入一个衰减因子,并认为只有这样处理才是物理的:$ e^{-\epsilon \tau} $
\end{itemize}
于是我们得到:
\begin{align}
  \ket{\psi_0} = |\boldsymbol{\psi}_{\mathrm{out}}\rangle\boldsymbol{+}\lim_{\boldsymbol{\varepsilon}\to0}\int d^{3}\mathbf{p}\frac{1}{E_{p}\boldsymbol{-}H\boldsymbol{-}i\boldsymbol{\varepsilon}}V\left|\mathbf{p}\right\rangle\left\langle\mathbf{p}\left|\boldsymbol{\psi}_{\mathrm{out}}\right\rangle\right..
\end{align}
其中我们虽然H写在了分母上面,我们的意思其实是这个算符的线性组合的逆算符,对于这样的算符我们称之为\hlr{Green函数}。
\defi{
  Green Function 

  给定一个Hamiltonian $ H $,我们定义:
  \begin{align}
    G(z)=\frac{1}{z - \hat{H}}\mathrm{~.}
  \end{align}
  $ z $是一个复数变量。我们称之为Hamiltonian对应的\hlr{Green函数}。

  特别的对于自由粒子Hamiltonian $ H_0 $,我们定义:
  \begin{align}
    G_0(z)=\frac{1}{z - \hat{H}_0}\mathrm{~.}
  \end{align}
}
因此我们可以使用Green Function改写Moller算子。对于out state我们有:
\begin{align}
  |\psi_0\rangle=|\psi_\mathrm{out}\rangle+\lim_{\boldsymbol{\varepsilon}\to0}\int d^3\mathbf{p}\ G(E_p-i\boldsymbol{\varepsilon})V|\mathbf{p}\rangle\langle\mathbf{p}|\psi_\mathrm{out}\rangle.
\end{align}
类似的我们也可以得到in state的结果:
\begin{align}
  |\psi_0\rangle=|\psi_{\mathrm{in}}\rangle+\lim_{\varepsilon\to0}\int d^3\mathbf{p}\ G(E_p+i\varepsilon)V|\mathbf{p}\rangle\langle\mathbf{p}|\psi_{\mathrm{in}}\rangle.
\end{align}
因此我们有结论:
\thm{\label{thm:Moller_Green}
  Moller Operator的Green函数表示

  两个Moller Operator可以使用Green函数表示为:
  \begin{align}
   \Omega_{\pm}=1+\lim_{\boldsymbol{\varepsilon}\to0}\int d^{3}\mathbf{p}\ G(E_{p}\pm i\boldsymbol{\varepsilon})V|\mathbf{p}\rangle\langle\mathbf{p}|\mathrm{~.}
  \end{align}
}

\subsection{Green 函数到T矩阵再到散射振幅}

\subsubsection{Green函数的结构}

\hlr{Green函数的解析性质}

根据之前Hilbert Space结构的讨论,我们可以使用第二种完备性关系对Green函数进行展开。我们知道Hilbert Space包含了两部分:
\begin{itemize}
  \item \textbf{离散的束缚态谱}:$ E_n < 0 $对应$ \ket{n} $
  \item \textbf{连续的散射态谱}:$ E_p \geq 0 $对应$ \ket{\mathbf{p}_+} = \Omega_+\ket{\mathbf{p}}  $
\end{itemize}
所以,我们可以把Green函数进行本征展开:
\begin{align}
  G(z)=\sum_{n=1}^N\frac{|n\rangle\langle n|}{z-E_n}+\int d^3\mathbf{q}\frac{|\mathbf{q}_+\rangle\langle\mathbf{q}_+|}{z-E_q}.
\end{align}
观察函数形式,会发现这是一个复变函数,在实轴上面有一系列的极点以及割线:
\begin{itemize}
  \item \textbf{束缚态极点}:$ z=E_n<0 $,并对应留数$ \mathrm{res}_{z=E_n}G(z)=|n\rangle\langle n|\mathrm{~.} $
  \item \textbf{散射态割线}:$ z\in[0,+\infty) $
\end{itemize}
同样的对于自由粒子Hamiltonian $ H_0 $,我们也可以得到类似的结果。只不过没有束缚态极点,只有散射态割线。通过这个分析我们有一个很重要的结论:
\thm{
Green函数与Spectrum 

对于一个散射理论的set up,如果我们清楚了Green函数的形式也就是知道了解析形式,那么就等价于我们知道了这个理论的Hamiltonian谱的信息。
}
所以我们研究散射理论可以把Green函数作为整个理论的核心对象。

\bigskip
\hlr{自由粒子Green函数}

求解Green函数十分困难,但是自由粒子的Green Function十分容易求解,我们不妨先分析自由粒子Green函数的性质:
\begin{itemize}
  \item \textbf{自由粒子Green函数的本征展开}:
    \begin{align}
      G_0(z)=\int d^3\mathbf{q}\frac{|\mathbf{q}\rangle\langle\mathbf{q}|}{z-E_q}\mathrm{~.}
    \end{align}
  \item \textbf{空间表象展开:}
    自由粒子Green函数在空间表象下的形式为:
    \begin{align}\label{eq:free_particle_green_function}
      \langle\mathbf{x}|\hat{G}_{0}(z)|\mathbf{x}^{\prime}\rangle=-\frac{m}{2\pi}\frac{e^{i\sqrt{2mz}|\mathbf{x}-\mathbf{x}^{\prime}|}}{|\mathbf{x}-\mathbf{x}^{\prime}|}.
    \end{align}
\end{itemize}
所以我们发现自由粒子Green函数的形式是非常简单,可以解析的计算的。


\bigskip
\hlr{Lippmann-Schwinger方程}

Green函数本身十分复杂。但是我们会发现自由粒子Green函数和势能形式都很简单。所以一个求解思路就是使用自由粒子Green函数和势能线性的构造出完整的Green函数。这个可以通过Lippmann-Schwinger方程进行实现:
\thm{
  Lippmann-Schwinger方程

  给定一个Hamiltonian $ H=H_0+V $,其对应的Green函数满足下面的方程:
  \begin{align}
    G(z)=G_0(z)+G_0(z)VG(z)=G_0(z)+G(z)VG_0(z)\mathrm{~.}
  \end{align}
}
证明使用了下面的数学公式并选择了特殊的算符$ A,B $:
\begin{align}
  A^{-1}=B^{-1}+B^{-1}(B-A)A^{-1}.
\end{align}


\subsubsection{T矩阵与散射振幅}

\bigskip
\hlr{T矩阵改写Green函数}

我们知道真正的可观测量是散射振幅,所以我们需要一个方法把Green函数和散射振幅联系起来。我们定义T矩阵:
\defi{
  T-Matrix 

  给定一个Hamiltonian $ H=H_0+V $,我们定义T矩阵为:
  \begin{align}
    T(z)=V+VG(z)V\mathrm{~.}
  \end{align}
  同时有逆变换:
  \begin{align}
    G(z)=G_0(z)+G_0(z)T(z)G_0(z)\mathrm{~.}
  \end{align}
}
我们会发现这个函数和Green函数有一样的解析结构,因此其也包含了Hamiltonian的谱信息。同时我们类似的可以使用Lippmann-Schwinger方程对T矩阵进行改写。先引入两个恒等式,可以直接根据T矩阵的定义以及Lippmann-Schwinger方程得到:
\begin{align}\label{eq:T_matrix_identities}
  G_0(z)T(z)=G(z)V,\\ 
  T(z)G_0(z)=VG(z),
\end{align}
第一个恒等式加上Lippmann-Schwinger方程的形式我们可以得到拟变换。
\begin{itemize}
  \item 我们知道T矩阵如同Green函数的一个变形改写,几乎包含了类似的信息。
\end{itemize}
所以我们同样可以构造一个类似的Lippmann-Schwinger方程:
\thm{
  T矩阵的Lippmann-Schwinger方程

  给定一个Hamiltonian $ H=H_0+V $,其对应的T矩阵满足下面的方程:
  \begin{align}
    T(z)=V+VG_0(z)T(z).
  \end{align}
}

\bigskip
\hlr{S矩阵与T矩阵的关系}

我们使用T矩阵等价的改写Green函数是因为其可以很好的和S矩阵联系起来。从而我们可以得到散射振幅的信息。回顾S矩阵的定义是:
\begin{align}
  S=\lim_{t\to\infty}e^{iH_0t}e^{-2iHt}e^{iH_0t}.
\end{align}
同样的我们可以使用求导再\textbf{引入衰减因子}再积分的技巧进行改写:
\begin{align}
  S=1-i\lim_{\varepsilon\to+0}\int_0^\infty dte^{-\varepsilon t}\left[e^{iH_0t}Ve^{-2iHt}e^{iH_0t}+e^{iH_0t}e^{-2iHt}Ve^{iH_0t}\right],
\end{align}
我们考虑如果一个散射in and out态是动量本征态$ |\mathbf{p}\rangle $,那么我们有:
\begin{align}
  \bra{\mathbf{p'}}S \ket{\mathbf{p}}=\delta^{(3)}(\mathbf{p}-\mathbf{p}^{\prime})- \frac{1}{2}\lim_{\varepsilon\to+0}\langle\mathbf{p}^{\prime}|VG\left(\frac{E_{\mathbf{p}^{\prime}}+E_{\mathbf{p}}}{2}+i\boldsymbol{\varepsilon}\right)+G\left(\frac{E_{\mathbf{p}^{\prime}}+E_{\mathbf{p}}}{2}+i\boldsymbol{\varepsilon}\right)V|\mathbf{p}\rangle\mathrm{~.}
\end{align}
我们使用前面的恒等式将Green函数改写为T矩阵形式:
\begin{align}
  \bra{\mathbf{p'}}S \ket{\mathbf{p}}=\delta^{(3)}\left(\mathbf{p}^{\prime}-\mathbf{p}\right)-\frac{1}{2}\lim_{\boldsymbol{\varepsilon}\to+0}\left\langle\mathbf{p}^{\prime}\right|T\left(\frac{E_{\mathbf{p}^{\prime}}+E_{\mathbf{p}}}{2}+i\boldsymbol{\varepsilon}\right)|\mathbf{p}\rangle\left[\frac{2}{E_{\mathbf{p}^{\prime}}-E_{\mathbf{p}}+i\boldsymbol{\varepsilon}}+\frac{2}{E_{\mathbf{p}}-E_{\mathbf{p}^{\prime}}+i\boldsymbol{\varepsilon}}\right]
\end{align}
这里我们使用一个极其重要的数学技巧处理分母上面的复变函数;
\begin{align}
  \frac{1}{x+i\varepsilon}=\mathscr{P}\left(\frac{1}{x}\right)-i\pi\delta(x)\mathrm{~,}
\end{align}
\thm{\label{thm:S_T_relation}
  S矩阵,T矩阵和散射振幅的关系

  给定一个Hamiltonian $ H=H_0+V $,其对应的S矩阵和平面波本征态之间的关系为:
  \begin{align}
  \left\langle\mathbf{p}^{\prime}\right|S\left|\mathbf{p}\right\rangle= \delta^{(3)}\left(\mathbf{p}^{\prime}-\mathbf{p}\right)-2\pi i \delta\left(E_{\mathbf{p}}-E_{\mathbf{p}^{\prime}}\right)\langle\mathbf{p}|T\left(E_{\mathbf{p}}+i\varepsilon\right)\left|\mathbf{p}^{\prime}\right\rangle.
  \end{align}
  对于散射振幅以及T矩阵的关系为:
  \begin{align}
    t(p' \leftarrow p)
= \langle p' | T(E_p + i\varepsilon) | p \rangle
= -\frac{1}{(2\pi)^2 m} f(p' \leftarrow p), \quad \text{if } E_{p'} = E_p.
  \end{align}
}

\bigskip
\hlr{On shell讨论}

我们对于散射振幅,t振幅的定义都是在on shell的基础上的,也就是说$ E_{p'}=E_p $。但是T矩阵的定义本身并没有这个限制,所以我们可以考虑T矩阵的off shell推广。然后使得对于一些能量的入射是on shell的。


\subsection{Born近似}

T矩阵以及散射振幅我们很难进行解析求解。所以我们会使用一个近似手段也就是Born近似进行处理。我们假设相互作用相对于自由粒子Hamiltonian来说是一个微扰,那么我们可以使用微扰论的方法处理。

\bigskip
\hlr{T矩阵的微扰展开}

考虑系统的Hamiltonian呈现:
\begin{align}
  H=H_0+\lambda V\mathrm{~.}
\end{align}
我们改写T矩阵的Lippmann-Schwinger方程:
\begin{align}
  T(z)=\lambda V+\lambda VG_0(z)T(z)\quad \Rightarrow \quad T(z)=(1-\lambda V G_0(z))^{-1}\lambda V\mathrm{~.}
\end{align}
因此我们对于$ (1- \lambda V G_0(z))^{-1} $进行微扰展开:
\begin{align}
  T=\lambda V+\lambda^2VG_0V+\lambda^3VG_0VG_0V+...
\end{align}

\bigskip
\hlr{一阶Born近似}

对于一阶的Born近似,我们只保留T矩阵展开的第一项:
\begin{align}
  T^{(1)}=\lambda V\mathrm{~.}
\end{align}
我们根据T矩阵和散射振幅的关系\cref{thm:S_T_relation}可以得到:
\begin{align}
  f^{(1)}(\mathbf{p^{\prime}\leftarrow p})&=-(2\pi)^2m\left\langle\mathbf{p}^{\prime}|V|\mathbf{p}\right\rangle\\
                                          &=-(2\pi)^2m\int d^3\mathbf{x}\left\langle\mathbf{p}^{\prime}|V|\mathbf{x}\right\rangle\left\langle\mathbf{x}|\mathbf{p}\right\rangle\\
                                          &=-\frac{m}{2\pi}\int d^3\mathbf{x}e^{-i\mathbf{q}\cdot\mathbf{x}}V(\mathbf{x})\mathrm{~,}
\end{align}
这里我们使用第一章的量子力学的convention进行计算:
\begin{align}\label{eq:plane_wave_convention}
  \bra{p'} V \ket{p} = \int d^3x \bra{p'} x \rangle \langle x | V | p \rangle = \int d^3x \frac{e^{-i p' \cdot x}}{(2\pi)^{3/2}} V(x) \frac{e^{i p \cdot x}}{(2\pi)^{3/2}} = \frac{1}{(2\pi)^3} \int d^3x e^{-i (p' - p) \cdot x} V(x).
\end{align}
其中我们使用了$ \mathbf{q}=\mathbf{p}^{\prime}-\mathbf{p} $表示动量转移。注意,动量转移如果我们使用入射和散射角度表示的话,有:
\begin{align}
  |\mathbf{q}|=2p\sin\left(\frac{\theta}{2}\right)\mathrm{~.}
\end{align}
如果是求解径向对称的势能$ V(\mathbf{x})=V(r) $,我们可以很方便的得到散射振幅以及微分散射截面。 

\bigskip
\hlr{二阶Born近似}

对于二阶的Born近似,我们保留T矩阵展开的前两项:
\begin{align}
  T^{(1)}+T^{(2)}=\lambda V+\lambda^2VG_0V\mathrm{~,}
\end{align}
第一项上面已经完成讨论,对于第二项我们有:
\begin{align}
  f^{(2)}(\mathbf{p^{\prime}\leftarrow p})&=-(2\pi)^2m\left\langle\mathbf{p}^{\prime}|VG_0V|\mathbf{p}\right\rangle\\
                                          &=-(2\pi)^2m\int d^3\mathbf{q}\left\langle\mathbf{p}^{\prime}|V|\mathbf{q}\right\rangle\left\langle\mathbf{q}|G_0(E_p+i\varepsilon)|\mathbf{q}\right\rangle\left\langle\mathbf{q}|V|\mathbf{p}\right\rangle\\
                                          &=-(2\pi)^2m\int d^3\mathbf{q}\left\langle\mathbf{p}^{\prime}|V|\mathbf{q}\right\rangle\left\langle\mathbf{q}|V|\mathbf{p}\right\rangle\frac{1}{E_p-E_q+i\varepsilon}\mathrm{~,}
\end{align}
\rmk{
  注意,我们的t-振幅和散射振幅都需要$ E_p = E_p' $成立才有意义,所以我们在这里默认这两个是一样的!
}
这里我们可以使用之前一阶结果\cref{eq:plane_wave_convention}进行进一步计算!
