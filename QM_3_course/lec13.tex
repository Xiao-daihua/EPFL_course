\subsection{Pauli Equation II}
我们会发现Pauli Equation很好的半经典的描述了电磁场和费米子的相互作用。

\bigskip
\hlr{磁场中的Pauli Equation}

我们假设使用Coulomb Gauge进行研究,$ \mathbf{A} = 1/2 \mathbf{B} \times \mathbf{r} $,我们考虑系统仅仅有磁场的存在,并且仅仅展开到$ c^{-1} $的一级。我们有:
\begin{align}
  -\frac{\hbar}{i}\frac{\partial\phi}{\partial t}=\left[\frac{\mathbf{p}^{2}}{2m}-\mu \left(\mathbf{L}+2\mathbf{s}\right)\cdot\mathbf{B}\right]\boldsymbol{\phi}.
\end{align}
其中$ \mathbf{p} = \{p^i\} $而且$ \mu = e \hbar / 2 m c $是玻尔磁子,$ \mathbf{L} = \mathbf{r} \times \mathbf{p} $是轨道角动量算符,$ \mathbf{s} = \hbar / 2 \{ \sigma^i\} $是自旋算符。这很好的描述了波尔磁子的结果!!


\bigskip
\hlr{原子中的电子}

我们的Pauli Equation也可以很好的描述原子中的电子行为的相对论修正。对于Schrodinger 方程给出的结果为:
\begin{align}
  -\frac{\hbar}{i}\frac{\partial\phi}{\partial t}=H{\phi}\quad H=\frac{\mathbf{p}^2}{2m}-\frac{Ze^2}{r}\mathrm{~.}
\end{align}
如果我们考虑使用Pauli Equation,会需要使用Foldy-Wouthuysen transformation的方法进行相对论修正。考虑对于$ (mc^2)^{-1} $的一阶展开,我们得到:
\begin{align}
  H^{1} = H +V_1 +V_2 +V_3
\end{align}
我们给出每一个修正的表达式,对于$ V(r)=-\frac{Ze^{2}}{r} $我们有:
\begin{itemize}
  \item $ V_1 $是动能修正:$ V_1=-\frac{1}{2mc^2}\frac{\left(|\mathbf{p}|^2\right)^2}{4m^2} $
    \item $ V_2 $是自旋轨道耦合修正:$ V_{2}=\frac{1}{2mc^{2}}\frac{1}{r}\frac{dV(r)}{dr}\mathbf{s}\cdot\mathbf{L} $
    \item $ V_3 $是Darwin项:$ V_3=\frac{\hbar^2\pi}{2m^2c^2}Ze^2\delta(x) $
\end{itemize}
这些修正项很好地解释了氢原子光谱中的精细结构。

\bigskip
\hlr{一阶近似的使用条件}

对于上面的一阶近似,我们需要满足很接近非相对论的条件。 也就是说我们需要满足:
\begin{align}
  \frac{p^2}{2m}<< m\mathrm{~.}
\end{align}
由于我们粒子的动能和轨道有关系我们有$ p^2/2m \sim Ze^2/r $,根据量子力学不确定关系我们有$ r \sim \hbar / p $,因此我们得到:
\begin{align}
  p \sim Ze^2 m / \hbar \mathrm{~.}
\end{align}
因此我们的一阶近似关系要求:
\begin{align}
  Z << \frac{1}{\alpha} \approx 137\mathrm{~.} \alpha = \frac{e^2}{\hbar c} \mathrm{~.}
\end{align}
也就是说这个近似仅仅对于轻元素有效。






\subsection{Relativistic Boson \& Scalar Field}

\subsection{Relativistic Fermion \& Dirac Field}
对于这两部分我表示,会在QFT课程中进行详细的介绍,这里不再赘述。
