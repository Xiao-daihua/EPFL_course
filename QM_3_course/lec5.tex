\subsection{WKB近似的应用}

\subsubsection{应用1:Planck Formula}

\hlr{Bohr-Sommerfield的理解}

当我们写出Bohr-Sommerfield能级条件的时候,我们其实存在一个interpretation,我们考虑一个周期的运动的积分:
\begin{align}
  \oint pdx=2\pi\hbar\left(n+\frac{1}{2}\right),
\end{align}
这个其实是在说,我们的相空间的轨道并不是连续存在的,而是这个轨道包住的相空间的大小是量子化的。并且最小的一份是:$ 2 \pi \hbar$

\bigskip
\hlr{WKB近似给出能级差:Plank Formula}

我们首先知道轨道包围相空间的大小正比于一些单位元。并且对于谐振子的计算,我们发现这个会告诉我们能量的条件。我们现在希望具体计算单位包住的相空间意味着多少能量。我们计算有;
\begin{align}
  \frac{\partial}{\partial E}\int_{x_1}^{x_2}pdx=p(x_2)\frac{\partial x_2}{\partial E}-p(x_1)\frac{\partial x_1}{\partial E}+\int_{x_1}^{x_2}\frac{\partial p}{\partial E}dx.
\end{align}
\rmk{很不常用所以几乎都忘了的公式:对于积分求导!我们需要对于上下限求导然后再加上对被积函数求导}
然后我们知道turning point处动量为0,所以第一个两项为0。然后我们有:
\begin{align}
  \begin{aligned}\frac{\partial}{\partial E}\int_{x_1}^{x_2}pdx&=\int_{x_1}^{x_2}\frac{\partial}{\partial E}\sqrt{2m(E-V)}dx\\&=\int_{x_1}^{x_2}\sqrt{\frac{m}{2(E-V)}}dx.\end{aligned}
\end{align}
下面我们量子的不知道怎么做了,就使用经典的动量的定义进行一个换元:
\begin{align}
  m\frac{dx}{dt}=\sqrt{2m(E-V)}\quad\Rightarrow\quad dx=dt\sqrt{\frac{2(E-V)}{m}}.
\end{align}

于是我们给出了一个时间和能量相关的关系:
\begin{align}
  \frac{\partial}{\partial E}\oint pdx=T.
\end{align}
带入Bohr-Sommerfield能级条件我们有:
\begin{align}
  2\pi\hbar\frac{\partial n}{\partial E}=T\quad\Rightarrow\quad\Delta E=\hbar\omega\cdot\Delta n\mathrm{~,}
\end{align}

\subsubsection{应用2:一般势垒隧穿}

\bigskip
\hlr{一般势垒隧穿的概念}

我们研究过方形的势垒的隧穿,现在希望通过WKB近似研究更一般的隧穿问题。下面在WKB近似语境下面考虑一些基本概念,我们研究下面这个图之中的一般势垒。


\begin{figure}[H]
  \centering
  \includegraphics[width=0.5\textwidth]{assets/barrier.png}
  \caption{一般势垒隧穿}
  \label{fig:barrier}
\end{figure}

对于一般的势垒在远离势垒的情况我们可以认为是平面波的形式:
\begin{align}
  \psi_{x\to-\infty}=e^{\frac{i}{\hbar}px}+Re^{-\frac{i}{\hbar}px},\quad\psi_{x\to\infty}=De^{\frac{i}{\hbar}px}.
\end{align}
\begin{itemize}
  \item 注意,这个解的形式是我们考虑物理意义之后选择的,排除了没有physical inerpretation的解。
  \item 对于系数的选择我们定义入射波的系数为1,反射波的系数为$ R $,透射波的系数为$ D $。然后透射出来的概率就是$ |D|^2 $,反射的概率是$ |R|^2 $。并且我们有概率守恒条件:
    \begin{align}
      |R|^2+|D|^2=1.
    \end{align}
\end{itemize}



\bigskip
\hlr{WKB解析延拓方法求解系数关系}

下面我们在一般讨论的基础上使用WKB近似条件研究turning point 附近的行为从而对上这些系数之间的关系,求出反射和透射的概率。先对于$ x > b $的区域我们设波函数是:
\begin{align}
  \psi_{III} &=\frac{C}{\sqrt{p}}\text{exp}{(\frac{i}{\hbar}\int_{b}^{x}dx^{\prime}p)}.\\ 
  \int_b^xdx^{\prime}p &=\frac{2}{3}(-2mV^{\prime}(b))^{1/2}(x-b)^{3/2}.
\end{align}
其中使用了线性近似的条件,并且注意我们的积分是要求对于实函数进行积分的。然后我们延拓到复平面并绕过b点研究$ x \in (a,b) $区间之内的情况。得到【上半平面】延拓后的函数是:
\begin{align}
  \psi_{II} =  \left.\tilde{\psi}_{III}(x)\right|_{x<b}=\text{exp}{(-\frac{i\pi}{4}}\frac{C}{\sqrt{|p|}}e^{\frac{1}{\hbar}\int_{x}^{b}dx^{\prime}|p|)}
\end{align}
\rmk{
  注意,我们不考虑下半平面转过去,只考虑上半平面转过去;是因为只有这个是physical的!!是物理的!!所以我们认为这个就是II区间的波函数。
}

然后我们希望把这个波函数延拓到I区间,这里我们玩了一个trick:
\begin{align}
  \psi_{II}(x) =\text{exp}{(-\frac{i\pi}{4}}\frac{C}{\sqrt{|p|}})\text{exp}{(\frac{1}{\hbar}\int_{a}^{b}dx|p|}e^{-\frac{1}{\hbar}\int_{a}^{x}dx^{\prime}|p|}).
\end{align}
也就是积分分成两段,第一段给出了一个常数。第二段正是我们一般延拓的样子。然后我们继续绕过a点到达I区间:
  
\begin{align}
  \begin{aligned}\psi_{I}(x)&=\frac{2C}{\sqrt{p}}e^{-\frac{i\pi}{4}}e^{\frac{1}{\hbar}\int_{a}^{b}dx|p|}\cos\left(\frac{1}{\hbar}\int_{x}^{a}dx\mathrm{~p}-\frac{\pi}{4}\right)\\&=\frac{C}{\sqrt{p}}e^{\frac{1}{\hbar}\int_a^bdx{|p|}}\left(e^{\frac{i}{\hbar}\int_x^adx{p-\frac{i\pi}{4}}}+e^{-\frac{i}{\hbar}\int_x^adx{p+\frac{i\pi}{4}}}\right).\end{aligned}
\end{align}

\bigskip
\hlr{Match 系数}

我们对一下I,III区间的波函数和远离势垒的波函数,我们给出的结论是,注意是我们使用统一的归一化导致的系数:
\begin{align}
  R=e^{\frac{i\pi}{2}},\quad D=e^{\frac{i\pi}{2}}e^{-\frac{1}{\hbar}\int_{a}^{b}dx|p|}.
\end{align}
所以我们知道反射率就是1也就是几乎不可能存在透射的情况。但是透射的概率存在一个特别小的数值:
\begin{align}
  P=|D|^2=e^{-\frac{2}{\hbar}\int_a^bdx|p|}.
\end{align}

\bigskip
\hlr{近似适用条件}

首先我们使用了LOWKB的近似,这也就是意味着我们动量变化要很小。同样的这也等价于意味着我们的透射概率特别特别小,因为下面两个条件基本一致:
\begin{align}
  \frac{1}{\hbar}\int_a^bdx\left|p\right|\gg 1\mathrm{~,} , \quad \frac{b-a}{\lambda}\gg 1\mathrm{~,}
\end{align}
也就是说potential barrier需要足够宽远大于波长。

同时也注意到,我们的近似完全没有使用势能在无穷的aymptotic condition,所以这个结果是非常general的。适用于在无穷各种趋近的方式的势垒(不包含趋近正无穷大的)


\subsubsection{应用3:metastable state的lifetime}

考虑一个束缚态的势阱,我们知道这个势阱有一个势垒,这个势垒允许粒子隧穿出去。我们希望计算这个束缚态的lifetime。我们可以定义单位时间穿梭出去的概率:
\begin{align}
  \frac{dP}{dt}=\frac{|D|^2}{T},
\end{align}
根据这个公式我们可以计算出来一个时间:
\begin{align}
  \tau=\frac{T}{|D|^2}\mathrm{~,}
\end{align}

Naively使用能量数量级进行估算是特别不准确的 $ \tau_{nai\nu e}\sim1\mathrm{~Ме}V^{-1}\thicksim10^{-21}\mathrm{~s~}, $于是我们可以假设下面的模型用WKB进行估算,一波骚操作之后更准确了一点点但不多:
\begin{align}
  |D|^2\approx\exp\left(-\frac{\pi\beta}{\hbar}\sqrt{\frac{2m}{E}}\right)=\exp\left(-\frac{2\pi\beta}{\hbar\nu}\right)\mathrm{~,}
\end{align}

\hlr{细节可以看讲义,但我不觉得很好玩}

\subsubsection{应用4: Double Well potential的本征态能级差}

我们考虑下面的一个势能,我们研究其对称和反对称解之间的能级差。
\begin{figure}[H]
  \centering
  \includegraphics[width=0.5\textwidth]{assets/doublewell.png}
  \caption{Double Well potential}
  \label{fig:doublewell}
\end{figure}

\bigskip
\hlr{波函数形式分析}

显然一个自然的想法是分别考虑两个well的波函数。但是问题是,在一维情况下,不存在简并「可以数学证明」,我们不可以简并的考虑两个well之中叠加,务必整体的考虑。

并且根据对称性分析能量算符和parity算符对易,我们选择共同对角化的基,所以本征态是parity算符的本征态。所以由下面两种情况:
\begin{align}
  \psi(-x)=\psi(x),\text{ or odd }\psi(-x)=-\psi(x)
\end{align}

\bigskip
\hlr{基态附近的波函数形式}
\rmk{这里我们才引入,我们考虑的是基态,前面的讨论都是一般的}

下面我们希望研究【最低能量态的能极差】,所以我们假设一个波函数是$ \psi_0(x) $是一个类似于单个well基态波函数的函数localized在右边的well,我们物理上假设整体波函数是:
\begin{align}
  \psi_{1}(x)=\frac{1}{\sqrt{2}}(\psi_{0}(x)+\psi_{0}(-x))\mathrm{~,~}\quad\psi_{2}(x)=\frac{1}{\sqrt{2}}(\psi_{0}(x)-\psi_{0}(-x))\mathrm{~.}
\end{align}
我们物理上,根据函数形式的观察以及解schrodinger方程的经验\YL{[可以仔细讨论一下为什么是基态,但是我懒]}认为$ \psi_1 $是整体的基态波函数,$ \psi_2 $是第一激发态波函数。

\bigskip
\hlr{定态薛定谔方程以及能级差}

我们试图计算能极差,给出schrodinger equation:
\begin{align}
  \psi_1^{\prime\prime}+\frac{2m}{\hbar^2}(E_1-V)\psi_1=0\mathrm{~,~}\quad\psi_2^{\prime\prime}+\frac{2m}{\hbar^2}(E_2-V)\psi_2=0\mathrm{~.}
\end{align}
并且对其进行一个组合:
\begin{align}
  \psi_{1}^{\prime\prime}\psi_{2}- \psi_{2}^{\prime\prime}\psi_{1}+\frac{2m}{\hbar^{2}}(E_{1}-E_{2})\psi_{1}\psi_{2}=0\mathrm{~.}
\end{align}
然后进行一个积分。并且利用了$ \psi_0 $的归一化。一通计算我们使用$ \psi_0 $表示的结果是:
\begin{align}
  2{\psi}_{0}(0){\psi}_{0}^{\prime}(0)+\frac{{m}}{{\hbar}^{2}}(E_{1}{-}E_{2})=0\mathrm{~.}
\end{align}

\bigskip
\hlr{WKB近似计算能级差}

我们下面利用WKB近似计算$ \psi_0(0),\psi_0'(0) $。我们知道在classical forbidden region里面【考虑$ x \in (x, x_0) $的区域】波函数是:
\begin{align}
  \psi_{0}(x)=\sqrt{\frac{m{\omega}}{2\pi p}}e^{-\frac{1}{\hbar}\int_{x}^{a}dx|p|},
\end{align}
\rmk{这里我们使用了之前计算的半经典的归一化条件,对于一个well内波函数进行归一化。同时我们选择$ a \in (0, x_0) $就是$ \phi_0 $对应态的turning point}
最后得出结论:
\begin{align}
  E_{2}{-}E_{1}=\frac{{\omega\hbar}}{{\pi}}e^{-\frac{1}{\hbar}\int_{-a}^{a}dx|p|}.
\end{align}













\subsection{Questions and Thoughts}






