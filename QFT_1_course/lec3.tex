\subsection{对称性与群}

\hlr{群和物理的坐标变换}

给出了群的基本概念,值得注意的是我们会发现,所有某一个集合到自己的双射都构成了一个群结构。

我们不妨考虑一个群,也就是一个流形上面不同的坐标系之间的映射:
\begin{align}
  q_i\mapsto q_i^{\prime}=f_i\left(\{q_i\}\right)
\end{align}
所以我们知道\textbf{物理上的坐标系变换其实就是一个群的作用}

下文之中我们只考虑那些变换前后的空间是一样的,比如用$ \mathbb{R}^2 $描述和用$ [0,2 \pi) \times \mathbb{R}_+  $就是不一样的,我们不考虑这样的捏!

\bigskip
\hlr{对称性的定义}

但是我们并不在乎全部的坐标系变换。我们只在乎少量特殊的坐标变换。
\defi{Symmetry\label{def:Symmetry}

  我们把Symmetry定义为一个【坐标变换】保证了EoM的基本形式不变的变换。数学上就是EoM的形式替换成新的坐标的coordinate之后依然成立!
}
\rmk{
  这里坐标变换需要特别提醒,有的时候Symmetry仅仅是对于场的形式作用。所以其实我们更严格的应该是使用jet bundle的语言来描述这个Symmetry变换。
}
\rmk{几个重要的说明:
  \begin{enumerate}
    \item 存在一个Symmetry也就意味着我们不可能通过实验区分Symmetry变换前后的物理系统。
    \item Symmetry是一个坐标变换不是变分,这就意味着对于Symmetry Transformation。但是这个坐标变换在无限小的情况下对于场来说可以induce出一个场的构型空间的变分。
  \end{enumerate}
}

\bigskip
\hlr{Lie Group}

\defi{Lie Group

我们这里定义Lie Group需要满足下面的性质:
\begin{itemize}
  \item N dim analytical manifold,以及其所有元素都可以通过一个N维的坐标系进行描述
  \item 群的乘法和逆元在坐标系下是analytical function也就是;
    \begin{align}
      \begin{aligned}g(\alpha)g(\beta)&=g(p(\alpha,\beta)),\\g^{-1}(\alpha)&=g(r(\alpha)),\end{aligned}
    \end{align}
\end{itemize}

}
我们会发现所有的连续群都是Lie Group。我们对于这个manifold显然可以选择一个坐标系进行描述,但是坐标系我们有个convention是:
\begin{align}
  g(\vec{0})=e
\end{align}

\bigskip
\hlr{Lie Group Realization}

我们希望考虑一个群元素作用在某一个空间上面的结果【这其实就是群在这个空间上面的表示】我们严格化的写出就是:
\defi{Lie Group Representation

  首先定义GL(V)是一个【线性空间V】上,所有【可逆线性映射】构成的空间。我们定义一个Lie Group Representation是一个映射:
  \begin{align}
    \begin{aligned}D:G \to & GL(V)\\g \mapsto & D(g)\end{aligned}
  \end{align}
  并且这个映射需要满足:
  \begin{enumerate}
    \item $ D(g_1)D(g_2)=D(g_1g_2),\quad\forall\mathrm{~}g_1,g_2\in G $
    \item $ D(e)=I $
  \end{enumerate}
}

\bigskip
\hlr{表示的分类}

我们可以把representation分成主要两类:
\begin{itemize}
  \item Reducible Representation:存在一个非平凡子空间$ W \subset V $使得$ D(g)W \subset W $,也就是这个子空间在群作用下不变。
  \item Irreducible Representation:不存在这样的子空间。
\end{itemize}
并且对于reducible reprensentation我们或许可以进行complete reduction也就是把V分解成一系列invariant subspace的直和
\rmk{注意!并不是所有的representation都可以complete reduction!!有的时候我们的矩阵仅仅是上三角矩阵,这样子可以有一个不变子空间但是不能complete reduction!!}

考虑reprensentation的其他性质也可以有分类:
\begin{itemize}
  \item Equivalent Representation:存在一个可逆线性映射$ S:V \to V' $使得$ D'(g)=S D(g) S^{-1} $【对于$ V = V' $的情况下也就是线性映射可以通过相似变换进行变换,这个在矩阵的意义上就是换了一个基】
  \item Unitary Representation:存在一个内积$ \langle\cdot|\cdot\rangle $使得对于任意的$ g \in G $我们有$ D(g)^\dagger D(g)=I $【注意,对于Unitary的定义我们务必先定义一个内积结构才能讨论,因为dagger的定义就是$ \langle Ux,y\rangle=\langle x,U^\dagger y\rangle. $】
  \item Faithful Representation:$ D $是单射,也就是$ g_1\neq g_2\Longrightarrow D(g_1)\neq D(g_2)\mathrm{~.} $
\end{itemize}

\bigskip
\hlr{Shur's Lemma}

对于irreducible representation我们有下面三个很强的结论。考虑下面的set up:
\begin{itemize}
  \item $ D_1, D_2 $是某一个群$ G $的两个irreducible representation,分别作用在$ V_1,V_2 $上面。
  \item 如果存在一个线性映射$ A:V_1 \to V_2 $使得对于任意的$ g \in G $我们有:
    \begin{align}
      A D_1(g)=D_2(g) A
    \end{align}
    也就是$ A $把$ D_1 $和$ D_2 $联系起来。
\end{itemize}
\begin{enumerate}
  \item \textbf{Shur's Lemma 1}: $ A $只有两个选择 是零映射或者是一个invertible mapping
    \begin{itemize}
      \item 对于第二种情况,$ V_1 $和$ V_2 $空间有一个相同的维度,并且$ D_1 $和$ D_2 $是equivalent representation
    \end{itemize}
  \item \textbf{Shur's lemma 2}: 如果$ V_1 = V_2 = V $并且$ D_1 = D_2 = D $那么$ A $需要满足:$  A = \lambda Id$。
  \item \textbf{Shur's lemma 3}:如果上面的$ D $并非一个irrrep 而是一堆irrrep的直和,那么$ A $可以是一个block diagonal matrix,每一个block对应一个irrrep,并且每一个block是$ \lambda_i Id $。【注意,每一个block拥有一个$ \lambda_i $】
    
\end{enumerate}


\subsection{Questions and Thoughts}

\question{对称性定义到底是什么?}

注意!我们把对称性比较简单的定义为:\textbf{所有保证EoM不变的reparametration变换!}。
\bigskip

\rmk{对称性是坐标变换不是变分。但是对于无穷小对称性变换,我们可以把它写作一个场构型的无穷小变分!!}

但是在我研究diffeo的时候有一些困扰。因为diffeo似乎是local的变换。但是不完全是local的,似乎Killing Vector给出的是global的?这个又是为什么呢?
\qed 

\bigskip
\question{存在diffeomorphism的时候,我们怎么理解和研究对称性?}

我终于终于理解了【25.9.25】。首先,我们说对称性是坐标变换而不是变分。但是,坐标变换不意味着是local的:我们考虑各种不一样的diffeo坐标变换:
\begin{itemize}
  \item 如果这个坐标变换是Killing Vector Generate的,那么我们会发现这些坐标变换是global的symmetry。
  \item 如果这个坐标变换不是Killing Vector Generate的,那么我们会发现这些坐标变换是local的symmetry。
\end{itemize}
我们只是挑出来了一些坐标变换,并且说这些是global的!!

\qed 


\bigskip
\question{场论之中我们怎么确定一个连续变换是一个global symmetry还是gauge symmetry?}

请看我们的附录!!部分!!!
\qed


\bigskip
\question{怎么理解Killing Vector生成的变换是Global Symmetry Transformation的?或者说怎么理解Killing Vector给出了一个Global Symmetry?}

\bigskip
差一嘴:Killing Vector在主动变换和被动变换下面是怎么理解的:
\begin{enumerate}
  \item 主动变换下也就是metric不变的变换$ \mathscr{L}_V g = 0 $
  \item 被动变换我们理解为不同坐标分量,也就是$ \bar{g}_{ab}(y) = g_{\mu\nu}(x) $,对于Killing Vector生成的坐标变换$ x \to y $。也就是函数形式完全没有变换捏,只是直接把自变量替换一下!
\end{enumerate}

已经在上上面的问题之中给出了回答捏!!\qed


\bigskip
\question{同样的,我们能不能从场论的角度理解Diffeo变换是一个Gauge Symmetry?直接从引力的action出发研究对称性?}

首先我们知道Diffeo是引力理论的gauge symmetry。我们可以说明,引力方程EoM给出的$ G_{\mu\nu} = 0 $这个对于所有的diffeo都是不变的。所以我们知道这是一个类似于gauge symmetry的东西。同时,很好理解其实Diffeo必然是local的,因为我们可以随便变换无限小量!!

\qed

\bigskip
\question{能不能证明量子场论的Gauge变换都类似于diffeo这种换一个坐标描述的意思??而不是一个正经的物理变换,只不过是长得很像动力学场的变换而已?}

似乎某种意义下面应该是可以的,但是可能和Diffeo完全不是一个概念。可以去看看Gauge Field的书对于那些已经研究明白的gauge theory有什么性质。看看这些Gauge Field有没有几何的描述【似乎是有的!!】。但是这个讨论有一点点大了!

\YL{感觉是一个特别有意思的问题捏!!甚至可以问是不是所有的Gauge Symmetry都有一个几何interpretation!不仅仅是我们一般讨论的Yang-Mills}

\bigskip
\question{
 Diffeo会改变时空的结构「曲率」吗?
}

不会!!Diffeo只是一个坐标变换而已!它不会改变任何物理量!所以曲率不变!!Diffeo对应一个李导数的作用,曲率在这个操作下是不会变换的。

如果我们想变分这个$ S = \int \sqrt{g} d^4 x R $我们做的是对于$ g_{\mu\nu} $进行一个perturbation,而不是做一个diffeo!而是强行进行一个加!!【variation】不被任何diffeo induce的variation!
\qed

\bigskip
\question{为什么对于$ t \to t' = \lambda t $的变换我们有自由度的协变是$ q \to q'(t') = \lambda^p q(t)$}

这是来自于,我们强行希望这是一个对称性!也就是EoM不变,为此只有特殊的$ q(t) $协变才能做到这个。我们请从这个思路理解!\qed


\bigskip
\question{技术上,$ \epsilon^{\mu\nu \rho \sigma} $是怎么退化到三阶的??}

存在这么一个好用的关系捏:
\begin{align}
  \epsilon_{0ijk}=\epsilon_{ijk},\quad i,j,k=1,2,3
\end{align}













