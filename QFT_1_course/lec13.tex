\subsection{Quantization的思路与准备}

\subsubsection{Quantization的思路}

\hlr{Quantization是什么}

对于量子化我们的操作是,把所有经典的量的Poisson Bracket替换为量子力学之中的对易关系。但是,对于Classical Spinor Field,我们完全没有研究过Poisson Bracket以及分析力学的性质再研究起来可能会十分麻烦\textbf{因为它是一个约束系统}。因此我们或许需要给出更一般的量子化的方式:
\axm{
  正则量子化公理

  对于经典的场我们存在动力学场$ \phi_i(x) $以及其共轭动量场$ \pi_i(x) $,量子化即为下方的操作:
  \begin{itemize}
    \item \textbf{场替换为算符}:我们把经典场替换为算符$ \phi_i(x) $但并不一定Hermite,只有其中实数部分变成了Hermite算符。
    \item \textbf{共轭对易关系}:算符需要满足共轭对易关系:
      \begin{align}
        [\phi_i(\mathbf{x},t),\pi_j(\mathbf{y},t)]_\pm = i\delta_{ij}\delta^3(\mathbf{x}-\mathbf{y}),
      \end{align}
      其中$ [A,B]_\pm = AB \pm BA $,上标$ + $表示反对易关系,$ - $表示对易关系。对于不同的场我们有自由选择不一样的量子化方式。
  \end{itemize}
}

\bigskip
\hlr{Hilbert Space的构造}

在量子化之后我们需要构造Hilbert Space,我们期望其Hilbert Space是Fock Space,这样我们可以构建粒子的interpretation。为此,我们需要研究Hamiltonian量子化后的性质。我们的思路是:
\begin{itemize}
  \item 经典上进行求解Hamiltonian
  \item 量子化:给出对易关系或者反对易关系(同时可以验证一下Hamiltonian确确实实是生成元满足$ \dot{\psi}(x) = i\left[H,\psi(x)\right] $;毕竟我们没有讨论经典力学的情况所以验证一下还是令人放心的)
  \item 通过Mode Expansion进行对角化Hamiltonian,给出on shell的产生湮灭算符。
  \item 构造真空态,产生算符作用在真空态上构造Fock Space。
\end{itemize}

\subsubsection{Quantization 准备工作}

\rmk{注意:我们这一小节依旧是\textbf{纯经典的内容}。但是其实里面讨论基本上是不严谨的,因为Dirac Field是一个约束系统,我们并没有讨论Dirac Bracket的内容。所以这里的内容只能作为一种直观的理解,严格的处理需要使用Dirac Bracket的内容。}
量子化之前我们需要进行一些准备工作:求解Hamiltonian形式;给出理论的Canonical Variable;给出Mode Expansion的反向变换方便我们通过场对易关系给出产生湮灭算符的对易关系。

\bigskip
\hlr{Hamiltonian的求解}

对于Lagrangian,我们希望认为$ \psi $和$ \bar{\psi} $是独立变量,因此我们可以通过分部积分的方式重新改写Lagrangian保证两者都存在动力学项:
\begin{align}
  \mathcal{L}=\frac{i}{2}\left(\bar{\psi}\partial\psi-(\partial_\mu\bar{\psi})\gamma^\mu\psi\right)-m\bar{\psi}\psi
\end{align}
对此我们求解EM Tensor可以以得到Hamiltonian密度,然后对于等时面进行积分得到Hamiltonian以及动量算符
\thm{
  Dirac Field的Hamiltonian和动量算符
  \begin{align}
    H=\int d^3x\,\bar{\psi}(-i\gamma^i\partial_i+m)\psi ,\quad P^i =- P_i =  -i\int d^3\mathbf{x}\psi^\dagger\partial_i\psi
  \end{align}
}
\rmk{
  我们知道对于标量场来说,我们可以通过Legendre变换;时间平移对称性守恒量;时间平移对称性生成元 三种方式完全等价的求解Hamiltonian。但是对于Dirac Field这个约束系统来说这些并不完全等价。严格的解释需要使用Dirac Bracket。但是我们这里选择了合适的路径至少保证了后两者是等价的!
}

\bigskip
\hlr{Canonical Variable的求解}

对于Dirac Field,我们可以通过Lagrangian求解其共轭动量,这个时候不能够使用分布积分之后的Lagrangian,严格的解释需要\textbf{使用Dirac Bracket}:
\thm{
  Dirac Field的Canonical Variable
  \begin{align}
   \psi(x), \quad \pi(x)=\frac{\partial \mathcal{L}}{\partial \dot{\psi}(x)}=i\psi^\dagger(x)
  \end{align}
}

\bigskip
\hlr{反向Mode Expansion}

我们经常需要使用场的Canonical对易关系给出产生湮灭算符的对易关系,因此我们需要给出反向的Mode Expansion。我们的on shell mode expansion是包含时间的,现在我们不妨考虑$ t = 0 $的等时面上面的情况。我们回顾一下on shell mode expansion:
\begin{align}
  \psi(x)=\int d\Omega_\mathbf{p}e^{- ip_\mu x^\mu}\sum_s\left(a_s(p)u_s(p)+b_s(-p)v_s(-p)\right)
\end{align}
对于$ t = 0 $的等时面上我们不使用协变的指标进行书写,使用$ \mathbf{p}\cdot \mathbf{x} = \sum_i x^ip^i $进行书写$ exp(-ip_\mu x^\mu) = exp(i\mathbf{px}) $。因此我们有:
\begin{align}
  \psi(\mathbf{x})=\int d\Omega_\mathbf{p}e^{i\mathbf{p}\cdot \mathbf{x}}\sum_s\left(a_s(p)u_s(p)+b_s(-p)v_s(-p)\right)
\end{align}
我们对其进行三维的反向Forier变换:
\begin{align}
  \int d^3\mathbf{x}e^{-i\mathbf{p}\cdot \mathbf{x}}\psi(\mathbf{x})=\int d\Omega_\mathbf{p'}\sum_s\left(a_s(p')u_s(p')+b_s(-p')v_s(-p')\right)\int d^3\mathbf{x}e^{i(\mathbf{p'}-\mathbf{p})\cdot \mathbf{x}}
\end{align}
我们使用\cref{sec:spinor_basis_properties}之中的性质可以知道:
\begin{align}
  u^\dagger_s(p) \int d^3\mathbf{x}e^{-i\mathbf{p}\cdot \mathbf{x}}\psi(\mathbf{x}) = a_s(p) ,\quad v^\dagger_s(-p) \int d^3\mathbf{x}e^{-i\mathbf{p}\cdot \mathbf{x}}\psi(\mathbf{x}) = b_s^\dagger(-p)
\end{align}


\subsection{Quantization of Dirac Field}
由于正则量子化的方法是在一个等时面上面进行的,因此我们需要把之前的on shell mode expansion限制在等时面上面。我们不妨选择$ t  = 0 $的等时面上面进行量子化。其上的on shell mode expansion为:
\begin{align}
  \psi(\mathbf{x})=\int d\Omega_\mathbf{p}e^{i\mathbf{p}\cdot \mathbf{x}}\sum_s\left(a_s(p)u_s(p)+b_s(-p)v_s(-p)\right)
\end{align}
我们之后每一步操作都默认是在\textbf{0时间的等时面上面}进行的。

\subsubsection{Fermionic Quantization}

\hlr{正则对易关系}

在各种尝试之后人们发现,量子化Dirac Field的最好方式是使用反对易关系,这样可以避免出现各种的问题(负能量,non-Unitary)。我们现在Promote所有的场为算符,并且给出反对易关系:
\thm{
  Dirac Field的量子化

  我们现在promote所有的Dirac Field作为算符,并且要求满足反对易关系:
  \begin{align}
   \{\psi_{\alpha}(\mathbf{x}),\psi_{\beta}(\mathbf{y})\}\boldsymbol{=}0\quad\{\psi_{\alpha}(\mathbf{x}),\pi_{\beta}(\mathbf{y})\}\boldsymbol{=}i\delta_{\alpha\beta}\delta^3(\mathbf{x}-\mathbf{y}) 
  \end{align}
}
我们带入canonical momentum的表达式$ \pi(x) = i\psi^\dagger(x) $,我们可以得到:
\begin{align}
  \{\psi_\alpha(\mathbf{x}),\psi_\beta^\dagger(\mathbf{y})\}=\delta_{\alpha\beta}\delta^3(\mathbf{x}-\mathbf{y})
\end{align}

\bigskip
\hlr{产生湮灭算符的反对易关系}

下面我们计算量子化之后的mode的系数$ a_s(p),b_s(p) $的反对易关系。我们使用反向mode expansion的结果:
\begin{align}
  \left\{a_r(p),a_s(k)^{\dagger}\right\}=(2\pi)^32\omega_\mathbf{p}\delta_{rs}\delta^3(\mathbf{k}-\mathbf{p})\quad\left\{b_r(p),b_s^{\dagger}(k)\right\}=(2\pi)^32\omega_\mathbf{p}\delta_{rs}\delta^3(\mathbf{k}-\mathbf{p})
\end{align}
我们之前的展开convention很好的给出了covariant的delta函数,这样两边其实都是Lorentz协变的力!


\subsubsection{Hamiltonian的对角化}

\hlr{Hamiltonian的对角化}

下面我们研究Hamiltonian量子化之后的性质,我们显然知道Hamiltonian是一个非对角的形式,我们希望将其对角化进行研究。对于自由标量场我们这个时候选择了三维的mode expansion。对于Dirac Field我们在标量场的启发下已经先见之明的使用了on shell mode expansion,我们现在发现其\textbf{正好能对角化Hamiltonian}。我们把量子化之后的mode expansion带入Hamiltonian:
\thm{
  Dirac Field的Hamiltonian的对角形式 
  \begin{align}
    H=\int d\Omega_\mathbf{p}\sum_s\omega_\mathbf{p}\left(a_s^{\dagger}(p)a_s(p)-b_s^{\dagger}(p)b_s(p)\right)
  \end{align}
  我们这个时候为了消除负能量,一般选择另外一组产生湮灭算符$ \tilde{b}_s = b_s^{\dagger} $,$ \tilde{b}_s^{\dagger} = b_s $,这样Hamiltonian可以写成:
  \begin{align}
    H=\int d\Omega_\mathbf{p}\sum_s\omega_\mathbf{p}\left(a_s^{\dagger}(p)a_s(p)+\tilde{b}_s^{\dagger}(p)\tilde{b}_s(p)\right) + \text{常数项}
  \end{align}
}

\bigskip
\hlr{Hamiltonian对易关系}

我们根据Fermionic反对易关系给出Hamiltonian和Mode系数的对易关系。注意,我们对于Hamiltonian的生成的定义依旧是\textbf{反对易}关系,只是反对易关系其实可以使用对易关系进行给出:
\begin{align}
  [A,BC]=\{A,B\}C - B\{A,C\}, \quad [AB,C]=A\{B,C\} - \{A,C\}B
\end{align}
因此我们可以计算出:
\begin{align}
  [H,a_r^\dagger(p)]=\omega_\mathbf{p}a_r^\dagger(p);\quad [H,\tilde{b}_r^\dagger(p)]=\omega_\mathbf{p}\tilde{b}_r^\dagger(p);
\end{align}
这也确实说明我们选择的新的$ \tilde{b} $才是标准的正能量的产生湮灭算符。


\subsection{Dirac Field的Hilbert Space}

\hlr{单一Fermionic谐振子的Hilbert Space构造}

为了更好理解Fermionic的量子化,我们先考虑一个简单的单一Fermionic谐振子。其Hamiltonian为:
\begin{align}
  H=\omega b^\dagger b = -\omega bb^\dagger + \omega
\end{align}
其中$ b $和$ b^\dagger $满足反对易关系:$ \{b,b^\dagger\}=1 $。我们现在构造其Hilbert Space。我们定义真空态$ \ket{0} $满足$ b\ket{0} = 0 $,我们可以构造出唯一的激发态$ \ket{1} = b^\dagger\ket{0} $。我们理解如果$ \omega >0 $的话,$ \ket{0} $是最低能量态;如果$ \omega <0 $的话,$ \ket{1} $是最低能量态。

\hlr{Hamiltonian和动量算符的Mode形式}

我们使用场的on shell mode expansion再改写动量算符得到:
\begin{align}
 P^i =  \int d\Omega_{\mathbf{p}}p^i\left[\sum_{r}\left(a_{r}^{\dagger}(p)a_{r}(p)+\tilde{b}_{r}^{\dagger}(p)\tilde{b}_{r}(p)\right)\right]
\end{align}
和Hamiltonian进行结合我们就有四动量算符的形式:
\begin{align}
  P^\mu =  \int d\Omega_{\mathbf{p}}p^\mu\left[\sum_{r}\left(a_{r}^{\dagger}(p)a_{r}(p)+\tilde{b}_{r}^{\dagger}(p)\tilde{b}_{r}(p)\right)\right]
\end{align}
以及这个算符和产生湮灭算符的对易关系:
\begin{align}
  [P^\mu,a_r^\dagger(p)]=p^\mu a_r^\dagger(p),\quad [P^\mu,\tilde{b}_r^\dagger(p)]=p^\mu \tilde{b}_r^\dagger(p)
\end{align}

\bigskip
\hlr{Dirac Field的Hilbert Space构造}

因此我们知道这个理论其实有四种激发态$ a_r^\dagger(p)\ket{0},\tilde{b}_r^\dagger(p)\ket{0} ,r = 1,2 $。每一种产生算符会激发一个on shell动量为$ p^\mu $的粒子。作用在真空态上面给出Hilbert Space:
\thm{
  Dirac Field的Hilbert Space

  我们定义真空态$ \ket{0} $满足$ a_r(p)\ket{0} = 0 ,\tilde{b}_r(p)\ket{0} = 0 $,我们可以通过作用产生算符在真空态上面构造Hilbert Space:
  \begin{align}
    \mathcal{H} = \text{span}\left\{\prod_{i=1}^N a_{s_i}^\dagger(p_i)\prod_{j=1}^M \tilde{b}_{r_j}^\dagger(k_j)\ket{0}\mid N,M = 0,1,2,\cdots; s_i,r_j = 1,2\right\}
  \end{align}
}
对于其中的任意一个量子态都是$ P^\mu $的本征态,其本征值为所有粒子的动量之和。我们意识到这个Hilbert Space中粒子满足Fermi-Dirac统计:
\begin{itemize}
  \item 所有的产生算符之间都是anti-commute的;因此不可能存在两个完全一样的粒子在一个量子态上面,这就是\textbf{泡利不相容原理}。
    \item 由于反对易,所以波函数必然是全反对称的。
\end{itemize}

\bigskip
\hlr{Spin and Statistics Theorem}

我们量子化Dirac Field得到FD统计的粒子,这其实是一个非常深刻的结论。我们可以把这个结论推广为更一般的\textbf{Spin and Statistics Theorem}:
\thm{
  Spin and Statistics Theorem

  对于一个Lorentz群的表示协变场,按照$ (j_-,j_+) $表示进行协变。如果量子化的理论需要满足Unitary, Causality以及Positivity of Energy,那么该场量子化之后的统计性质由$ j_- + j_+ $决定:
  \begin{itemize}
    \item 如果$ j_- + j_+ $为整数,那么该场量子化之后满足Bose-Einstein统计。
    \item 如果$ j_- + j_+ $为半整数,那么该场量子化之后满足Fermi-Dirac统计。
  \end{itemize}
}
关于为什么仅仅和$ j_-+j_+ $有关,是因为我们的场是按照$ j_- \otimes j_+ $的表示进行协变的。这个表示等价于很多表示的直和,而这些表示的spin的奇偶行完全由$ j_- + j_+ $决定。


\subsection{Heisenberg Picture构造全时空的场算符}

前面的量子化过程是在$ t = 0 $这个等时面上面完成的。但是我们希望构建全时空的场算符。为此我们使用Heisenberg Picture的方式进行构造。就可以得到一个on shell的全时空的场算符,我们给出:
\begin{align}
  \psi(x)\equiv e^{iHt}\psi(\mathbf{x})e^{-iHt}
\end{align}
为了计算这个,我们依旧先证明一个数学上的关系:
\begin{align}
  e^{iHt}a_r(p)e^{-iHt}=a_r(p)e^{-i\omega_\mathbf{p}t},\quad e^{iHt}\tilde{b}_r(p)e^{-iHt}=\tilde{b}_r(p)e^{-i\omega_\mathbf{p}t}.
\end{align}
我们把这个关系带入上面的式子,我们就有:
\thm{
  Dirac Field的全时空场算符的协变形式

  Dirac Field的全时空场算符为:
  \begin{align}
    \psi(x) = \int d\Omega_{\mathbf{p}}\left[\sum_re^{-ip_\mu x^\mu}a_r(p)u_r(p)+e^{ip_\mu x^\mu }\tilde{b}_r^\dagger(p)v_r(p)\right]
  \end{align}
  注意我们使用的仍然是$ (+,-,-,-) $的metric。
}

\subsection{U(1)对称性与电荷守恒}

在一般时空isometry对称性之外,Dirac Field还存在一个非常重要的内部对称性:U(1)对称性。对于下面的Global U(1)变换,Lagrangian是满足对称性的条件的:
\begin{align}
  \psi\to e^{-i\alpha}\psi\quad\bar{\psi}\to e^{i\alpha}\bar{\psi}
\end{align}
自然我们可以计算其守恒流是:
\begin{align}
  j^\mu = \bar{\psi}\gamma^\mu\psi \Rightarrow \partial_\mu j^\mu = 0
\end{align}
并且我们也可以给出守恒荷:
\begin{align}
  J^0 = \psi^\dagger\psi \Rightarrow Q = \int d^3\mathbf{x}\,\psi^\dagger\psi = \int d\Omega_\mathbf{p}\sum_r\left(a_r^\dagger(p)a_r(p)-\tilde{b}_r^\dagger(p)\tilde{b}_r(p)\right)
\end{align}
对于这个守恒荷我们给出下面的Interpretation:
\begin{itemize}
  \item $ a_r^\dagger(p) $产生的粒子携带守恒荷$ +1 $;$ \tilde{b}_r^\dagger(p) $产生的粒子携带守恒荷$ -1 $。
    \item 我们可以解释其为电荷,Dirac Field描述的粒子携带电荷$ +1 $,反粒子携带电荷$ -1 $。所以我们解释其描述了两种带+1和-1电荷的粒子。
\end{itemize}


\subsection{角动量和Boost守恒荷}

\hlr{经典Lorentz变换守恒流与守恒荷}

我们回顾一下经典的情况。在Lorentz变化下,根据Dirac Field的定义其满足下面的协变关系:
\begin{align}
  \psi(x)\to[\Lambda_D(\omega)]\psi(\Lambda^{-1}(\omega)x). \quad \Lambda_D=e^{-i\frac{1}{2}\omega_{\mu\nu}S^{\mu\nu}} \quad \Lambda = e^{-i\frac{1}{2}\omega_{\mu\nu}\mathrm{J}^{\mu\nu}}
\end{align}
我们可以计算其守恒流以及守恒荷为:
\begin{align}
  J_{\mu\nu}^\rho=x_\mu T_\nu^\rho-x_\nu T_\mu^\rho+\frac{1}{2}\bar{\psi}\{\gamma^\rho,S_{\mu\nu}\}\psi, \quad J_{\mu\nu}\equiv\int d^3x J_{\mu\nu}^0
\end{align}
我们一般会把Lorentz Boost 守恒荷分为角动量和Boost两部分,角动量守恒荷为:
\begin{align}
  J_k = \displaystyle\frac{1}{2} \epsilon_{ijk} J_{ij}, \quad J_k =\int d^3\vec{x}\psi^\dagger\left[\vec{x}\wedge(-i\vec{\nabla})+\frac{1}{2} \Sigma \right]\psi., \quad \Sigma = \begin{pmatrix}
    \vec{\sigma} & 0 \\ 0 & \vec{\sigma}
  \end{pmatrix}
\end{align}
其中$ \Sigma $就是对角的自选算符,而前面是轨道角动量!

\bigskip
\hlr{角动量作用于单粒子态}

这里我们可以看出结果:
\begin{itemize}
  \item Dirac Field量子化的$ r $指标实际上是自旋指标,$ r = 1,2 $分别对应$ s_z = +\frac{1}{2},-\frac{1}{2} $的粒子态。
\end{itemize}
验证方法就是作用角动量守恒量在单粒子态上面,我们interpret这个守恒量是角动量量子化之后的算符,我们发现不同的$ r $指标确实对应不同的z方向自旋取值。


\subsection{Questions and Thoughts}

\question{
  为什么对称性的守恒荷和角动量,动量可以认为是一个东西?为什么量子化之后的算符作用在量子态上面的本征值就是角动量,动量的取值?
}
思路是这样的,我们使用旋转对称性和角动量给出一个说明:
\begin{itemize}
  \item 首先经典力学告诉我们,旋转对称性给出的Nother定理的守恒量正好是经典对应的角动量
  \item 下面我们\textbf{定义}对于场论or一切经典体系:角动量定义为旋转对称性的守恒量。
    \item 接下来量子力学公理告诉我们,守恒量量子化之后的算符的本征值是量子态在该守恒量下的取值。
\end{itemize}
这个思路是十分清晰的!\qed
