\subsection{区分量子态}
\hlr{量子态区分概率最大化}

给定两个量子态$ \rho_0 $以及$ \rho_1 $我们希望通过POVM对其进行区分。假设我们使用的POVM是有两个元素$ \{ M_0,M_1\} $。其中$ M_i $回答的问题是是不是$ \rho_i $态。我们给出下面定理:
\thm{
  量子态区分概率
  
  给定两个量子态$ \rho_0 $以及$ \rho_1 $,其先验概率分别假设为$ 1/2 $。我们使用POVM$ \{ M_0,M_1\} $对其进行区分,我们\textbf{【主观上认为】测量到$ M_0 $结果量子态就是$ \rho_0 $,测量到$ M_1 $结果量子态就是$ \rho_1 $}。那么我们成功区分的概率为:
  \begin{align}
    p=\frac{1}{2}+\frac{1}{2}\tr(M_0 (\rho_0 - \rho_1))\mathrm{~.}
  \end{align}
}

Proof: 我们计算成功概率:
\begin{align}
  \mathbb{P}(\mathrm{succes})&=\mathbb{P}(\rho=\rho_0)\mathbb{P}(\mathrm{measure~}0\mid\rho=\rho_0)+\mathbb{P}(\rho=\rho_1)\mathbb{P}(\mathrm{measure~}1\mid\rho=\rho_1)\\
                             &=\frac{1}{2}\operatorname{Tr}(M_0\rho_0)+\frac{1}{2}\operatorname{Tr}(M_1\rho_1).
\end{align}
根据POVM的完备性我们有$ M_1=I-M_0 $,可以直接给出上面的结论。
\rmk{
  注意,我们是主观上认为测量到$ M_0 $结果量子态就是$ \rho_0 $,测量到$ M_1 $结果量子态就是$ \rho_1 $。实际上如果不小心设计了一个$ M_0 $导致$ \rho_0 $永远测不到这个结果,我们只能说这个测量设定很不合理,但是也不是不行。
}

\thm{
量子态区分最优POVM 

给定两个量子态$ \rho_0 $以及$ \rho_1 $,其先验概率分别假设为$ 1/2 $。我们使用POVM$ \{ M_0,M_1\} $对其进行区分,我们主观上认为测量到$ M_0 $结果量子态就是$ \rho_0 $,测量到$ M_1 $结果量子态就是$ \rho_1 $。那么使得成功区分概率最大的POVM为:
\begin{align}
   p_{opt}=\max_{0\leq M_0\leq I}\frac{1+\mathrm{Tr}(M_0(\rho_0-\rho_1))}{2}
\end{align}
其中$ M_0 $是任意的正半定矩阵且$ M_0\leq I $。
}

\bigskip
\hlr{例:纯态区分}

我们考虑区分两个纯态$ |\psi_0\rangle $以及$ |\psi_1\rangle $。我们不妨构建一个POVM保证$ M_0 = \ketbra{0}{0} $ 以及$ M_1 = \ketbra{1}{1} $,其中$ |0\rangle $以及$ |1\rangle $是两个态的正交归一化基底。我们计算成功概率:
\begin{align}
  p=\frac{1+\mathrm{Tr}(M_0(\rho_0-\rho_1))}{2}=\frac{1+1}{2}=1.
\end{align}
因此我们可以完美区分两个纯态。

\bigskip
\hlr{例:混态区分}

对于single qubit的混态我们考虑区分下面两个态:
\begin{align}
  \rho_0 = \frac{1}{2}\left( I + \vec{r}_0 \cdot \vec{\sigma} \right), \quad
\rho_1 = \frac{1}{2}\left( I + \vec{r}_1 \cdot \vec{\sigma} \right).
\end{align}
我们不妨更一般的构建下面一个POVM:
\begin{align}
  M_0 = a\!\left(I + \vec{x}\!\cdot\!\vec{\sigma}\right),
\quad
M_1 = I - M_0 .
\end{align}
但是generally这两个矩阵需要满足一些约束条件才能是一个合理的POVM,我们下面给出定理:
\thm{
  Single qubit POVM的约束条件

  给定一个single qubit的POVM$ \{ M_0,M_1\} $,其中$ M_0 = a\!\left(I + \vec{x}\!\cdot\!\vec{\sigma}\right) $以及$ M_1 = I - M_0 $。那么其需要满足下面的约束条件:
  \begin{align}
    \left\|\vec{x}\right\|\leq1 \quad a\in\left[0,1/\left(1+\left\|\boldsymbol{\vec{x}}\right\|^2\right)\right]
  \end{align}
}
Proof: 我们干的事情就是计算两个的本征值,保证本征值非复即可。

\bigskip
在这个基础上我们计算这两个混态的区分概率:
\thm{
  Single qubit混态区分概率

  给定两个single qubit混态$ \rho_0 = \frac{1}{2}\left( I + \vec{r}_0 \cdot \vec{\sigma} \right) $以及$ \rho_1 = \frac{1}{2}\left( I + \vec{r}_1 \cdot \vec{\sigma} \right) $,其先验概率分别假设为$ 1/2 $。我们使用POVM$ \{ M_0,M_1\} $对其进行区分,我们主观上认为测量到$ M_0 $结果量子态就是$ \rho_0 $,测量到$ M_1 $结果量子态就是$ \rho_1 $。那么成功区分的概率为:
  \begin{align}
    p \;=\; \frac12 \;+\; \frac{a}{4}\bigl( 2\,\vec{x}\!\cdot\!(\vec{r}_0 - \vec{r}_1) \bigr).
  \end{align}
}
这个计算之中我们使用了Pauli矩阵的trace性质:
\begin{align}
  \mathrm{Tr}(\sigma_i\sigma_j)=2\delta_{ij}, \quad \tr (\sigma_i)=0.
\end{align}
我们使用Pauli Basis展开single qubit的量子态的原因正是因为这个性质。
\thm{
  Single qubit混态区分最优概率

  对于上方的set up,使得成功区分概率最大的POVM为:
  \begin{align}
    \vec{x}_{\mathrm{opt}}
= \frac{\vec{r}_0 - \vec{r}_1}{\|\vec{r}_0 - \vec{r}_1\|},
\qquad
M_{0,\mathrm{opt}}
= \frac12\!\left( I + \vec{x}_{\mathrm{opt}}\!\cdot\!\vec{\sigma} \right).
  \end{align}
  此时成功区分的最优概率为:
  \begin{align}
    p_{\mathrm{opt}}
= \frac12 + \frac14 
\left(
\frac{(\vec{r}_0 - \vec{r}_1)\!\cdot\!(\vec{r}_0 - \vec{r}_1)}
{\|\vec{r}_0 - \vec{r}_1\|}
\right)
= \frac12 + \frac14 \|\vec{r}_0 - \vec{r}_1\|.
  \end{align}
}
Proof:在上面约束条件下面进行放缩就好了!!这里我们最后一步放缩的时候会使用:
\begin{align}
  p=\frac{1}{2}+\frac{1}{2}\cdot\frac{1}{1+\left\|\boldsymbol{\vec{x}}\right\|^{2}}\|\boldsymbol{\vec{x}}\|\|\boldsymbol{\vec{r}}_{0}-\boldsymbol{\vec{r}}_{1}\|\cos(\boldsymbol{\vec{x}};\boldsymbol{\vec{r}}_{0}-\boldsymbol{\vec{r}}_{1}).
\end{align}
然后我们对此考虑到函数:
\begin{align}
  f(t)=\frac{t}{1+t^2},\quad t\geq0
\end{align}
在$ t=1 $处取得最大值$ 1/2 $。然后我们带回去就得到了最终结果!!

\bigskip
\hlr{数学工具:1-Norm}

上面讨论的都是单个bit的情况,我们考虑更高维度的量子态区分。为此我们先要定义1-norm的数学工具:
\defi{
  Matrix 1-Norm

  对于一个矩阵$ X $我们定义其1-Norm为:
  \begin{align}
    \left\|X\right\|_1=\mathrm{Tr}\left(\sqrt{X^\dagger X}\right)=\sum_i\sqrt{\lambda_i(X^\dagger X)},
  \end{align}
  其中$ \lambda_i(X^\dagger X) $是矩阵$ X^\dagger X $的本征值。
}
显然对于Unitary的矩阵$ U $我们有$ \left\|U\right\|_1=\mathrm{Tr}(I)=d $,其中$ d $是矩阵的维度。对于Hermitean矩阵$ H $我们有$ \left\|H\right\|_1=\sum_i|\lambda_i(H)| $,其中$ \lambda_i(H) $是矩阵$ H $的本征值。


\bigskip
\hlr{highier dimension state区分}

我们考虑区分两个高维度量子态$ \rho_0 $以及$ \rho_1 $。我们有下面定理:
\thm{
  Highier dimension量子态区分概率

  给定两个量子态$ \rho_0 $以及$ \rho_1 $,其先验概率分别假设为$ 1/2 $。我们使用POVM$ \{ M_0,M_1\} $对其进行区分,我们主观上认为测量到$ M_0 $结果量子态就是$ \rho_0 $,测量到$ M_1 $结果量子态就是$ \rho_1 $。那么成功区分的概率为:
  \begin{align}
    p_{opt}=\frac{1}{2}\left(1+\max_{0\leq M\leq I}\tr(M(\rho_0-\rho_1))\right)= \displaystyle\frac{1}{2}+\frac{1}{4}\|\rho_0-\rho_1\|_1.
  \end{align}
}
Proof:其中我们显然使用了一个恒成立的数学关系:
\begin{align}
  \frac{1}{2}\|\rho_0-\rho_1\|_1=\max_{0\leq M\leq I}\operatorname{Tr}(M(\rho_0-\rho_1)).
\end{align}
证明的核心是构造一个只有$ \rho_0 - \rho_1 $本征值为正的本征子空间的投影算符作为POVM的一个元素。

\subsection{矩阵模长}

\bigskip
\hlr{一般定义}

上面我们给出了1-Norm的定义,下面我们逐渐给出一些常用的定义与性质。
\defi{
  一般矩阵Norm

  对于一个矩阵如果我们希望赋予一个Norm的概念$ \|.\|:K^{m\times n}\mapsto\mathbb{R} $,那么需要满足一些必要的性质:
  \begin{itemize}
    \item 非负性:$ \|A\|\geq0 $
      \item 零点对应:$ \|A\|=0\Leftrightarrow A=0 $
      \item 三角不等式:$ \|A+B\|\leq\|A\|+\|B\| $
      \item 齐次性:$ \|\alpha A\|=|\alpha|\|A\| $
  \end{itemize}
}
辅助定义我们还需要定义一些概念:
\defi{
  Normal Matrix

  对于一个矩阵$ A\in K^{n\times n} $,如果其满足$ AA^\dagger=A^\dagger A $,那么我们称其为Normal Matrix。
}
显然Hermitean矩阵、Unitary矩阵都是Normal Matrix。我们考虑之前对于这两种normal矩阵的1-Norm的讨论可以有下面的定理:
\thm{
  Normal Matrix的1-Norm

  对于一个Normal Matrix $ A\in K^{n\times n} $,其1-Norm可以表示为:
  \begin{align}
    \left\|A\right\|_1=\sum_i|\lambda_i(A)|,
  \end{align}
  其中$ \lambda_i(A) $是矩阵$ A $的本征值。
}
我们一般都只考虑Normal Matrix的情况,因为演化算子是Unitary的量子态和测量算子是Hermitean的。所以基本上都是Normal Matrix。

\rmk{
  我们需要注意,对于Unitary矩阵来说$ \left\|U\right\|_1=d $,其中$ d $是矩阵的维度。这意味着:
  \begin{itemize}
    \item Unitary矩阵的\textbf{本征值模长}之和等于矩阵的维度。
      \item Unitary矩阵的本征值求和不一定等于矩阵的维度。
  \end{itemize}
  请区分模长求和与直接求和!!
}

\bigskip
\hlr{Schatten p-Norm}

\defi{
  Schatten p-Norm

  对于一个矩阵$ A\in K^{m\times n} $,我们定义其Schatten p-Norm为:
  \begin{align}
    \left\|A\right\|_p=\left(\sum_{i=1}^d\left|\sqrt{\lambda_i(A^\dagger A)}\right|^p\right)^{1/p}.
  \end{align}
  其中$ \lambda_i(A^\dagger A) $是矩阵$ A^\dagger A $的本征值。
}
对于一般的矩阵我们可以使用奇异值分解来分析其Schatten p-Norm的性质。我们有定理:
\thm{
  矩阵的奇异值分解与Schatten p-Norm 

  对于一个矩阵$ A\in K^{m\times n} $,其奇异值分解为$ A=U\Sigma V^\dagger $,其中$ U\in K^{m\times m} $以及$ V\in K^{n\times n} $是Unitary矩阵,$ \Sigma\in K^{m\times n} $是一个对角矩阵,其对角线元素为矩阵$ A $的奇异值。我们有:
  \begin{align}
    \left\|A\right\|_p=\left(\sum_{i=1}^d\sigma_i(A)^p\right)^{1/p},
  \end{align}
}
Proof: 这是因为矩阵的奇异值和$ A^\dagger A $的本征值有关系:
\begin{align}
  \sigma_i=\sqrt{\lambda_i(A^\dagger A)}
\end{align}
同样的考虑对于Normal Matrix的情况,我们存在一个特例:
\thm{
  Normal Matrix的Schatten p-Norm

  对于一个Normal Matrix $ A\in K^{n\times n} $,其Schatten p-Norm可以表示为:
  \begin{align}
    \left\|A\right\|_p=\left(\sum_i|\lambda_i(A)|^p\right)^{1/p},
  \end{align}
  其中$ \lambda_i(A) $是矩阵$ A $的本征值。
}
\begin{itemize}
  \item 我们从此只考虑Normal Matrix的情况。
\end{itemize}

\bigskip
\hlr{Examples of Schatten p-Norm}

如果$ p = 1 $那么Schatten 1-Norm就是我们之前定义的1-Norm。对于Normal Matrix来说:
\thm{
  Normal Matrix的Schatten 1-Norm

  对于Normal Matrix $ A\in K^{n\times n} $,其Schatten 1-Norm可以表示为:
\begin{align}
  \lVert A\rVert_1=\sum_i\lvert\lambda_i\rvert.
\end{align}
}
对于$ p = 2 $我们有Schatten 2-Norm:
\begin{align}
  \lVert A\rVert_2=\left(\sum_i\lvert\lambda_i\rvert^2\right)^{1/2}.
\end{align}
但是我们可以证明对于Normal Matrix其存在另一种更漂亮的形式
\thm{
  Normal Matrix的Schatten 2-Norm

  对于Normal Matrix $ A\in K^{n\times n} $,其Schatten 2-Norm可以表示为:
  \begin{align}
    \lVert A\rVert_2=\sqrt{\mathrm{Tr}(A^\dagger A)}.
  \end{align}
  等价的也可以表示为:
  \begin{align}
    \left\| A\right\|_2=\sqrt{\sum_{i,j}\lvert A_{ij}\rvert^2}.
  \end{align}
}
我们可以使用量子计算机来计算两个量子态区别的Schatten 2-Norm,因为数学上我们有:
\begin{align}
  \begin{aligned}\left\|\rho_0-\rho_1\right\|_2&=\sqrt{\mathrm{Tr}((\rho_0-\rho_1)(\rho_0-\rho_1))}\\&=\sqrt{\mathrm{Tr}(\rho_0^2)+\mathrm{Tr}(\rho_1^2)-2\mathrm{~Tr}(\rho_0\rho_1)}.\end{aligned}
\end{align}
其中的信息可以通过SWAP测试来获得。
\begin{align}
  \mathrm{Tr}\left(\rho_i^2\right)=\mathrm{Tr}((\rho_i\otimes\rho_i)SWAP),\quad\mathrm{Tr}(\rho_1\rho_2)=\mathrm{Tr}((\rho_1\otimes\rho_2)SWAP).
\end{align}

\bigskip
\hlr{Infinity Norm}

我们考虑Schatten p-Norm在$ p\to\infty $的极限情况,我们有下面定理:
\thm{
  Normal Matrix的Infinity Norm

  对于Normal Matrix $ A\in K^{n\times n} $,其Schatten Infinity-Norm可以表示为:
  \begin{align}
    \lVert A\rVert_\infty=\max_i|\lambda_i(A)|,
  \end{align}
  其中$ \lambda_i(A) $是矩阵$ A $的本征值。
}
也就是说这个给出了矩阵的最大本征值模长。一个很现实的意义就是他给出了一个算子的最大期望值:
\begin{align}
  \left|\mathrm{Tr}(M\rho)\right|\leq\left|\lambda_{max}\right|=\left\|M\right\|_\infty.
\end{align}
