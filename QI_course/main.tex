% !TeX root = main.tex
%%%%%%%%%%%%%%%%%%%%%%%%%%%%%% DOCUMENT 
\documentclass[11pt]{article}

%%%%%%%%%%%%%%%%%%%%%%%%%%%%%% PACKAGES

% 中文支持(XeLaTeX 编译)
\usepackage[UTF8]{ctex}
\usepackage{xeCJKfntef} 

% \setCJKmainfont{HanziPen SC}
% \setmainfont{HanziPen SC}


% 页面设置
% \usepackage[a4paper, left=15mm, right=15mm, top=15mm, bottom=15mm]{geometry}
\usepackage[
  a4paper,
  left=20mm,
  right=20mm,
  top=25mm,
  bottom=25mm
]{geometry}


\PassOptionsToPackage{dvipsnames,svgnames,x11names}{xcolor}
\usepackage{xcolor}


% 数学环境及符号
\usepackage{amsmath, amssymb, amsfonts, amsthm,amsopn}
\usepackage{tensor}              % 张量指标管理
\usepackage{mathtools}           % amsmath增强
% \usepackage{physics}             % 物理公式快捷命令
             % Dirac符号
\usepackage{bbold}               % 数学黑体
\usepackage{dsfont}              % 另一种数字体
\usepackage[mathscr]{eucal}     % 花体字母
\usepackage{tensor}              % 张量指标管理
\usepackage{simpler-wick}       % Wick记号
\usepackage{mathrsfs}            % 另一种花体字母

% 颜色与图形相关
\usepackage{graphicx}           % 插图支持
\usepackage{float}              % 浮动体控制
\usepackage{tikz}               % 绘图库
\usetikzlibrary{math}           % tikz数学扩展
\usepackage{geometry}
% 表格与列表
\usepackage{makecell}           % 表格多行换行
\usepackage{multicol}           % 多栏排版
\usepackage{colortbl}           % 表格颜色
\usepackage{enumitem}           % 列表自定义

% 其他辅助
\usepackage{framed}             % 有边框环境
\usepackage{tcolorbox}          % 灵活盒子环境
\tcbuselibrary{breakable}       % 盒子内容分页
\usepackage{thmtools}           % 定理环境管理
\usepackage{thm-restate}        % 定理重述
\usepackage{showlabels}         % 显示标签,调试用(完成后可注释)
\usepackage[normalem]{ulem}     % 下划线、删除线
\usepackage{hyperref}           % 超链接(最后加载)
\usepackage{cleveref}           % 智能引用(紧跟hyperref)
\usepackage{soul}

% 自定义宏包
\usepackage{macros}

% 一个中文可以高亮的包
\usepackage{cjkhl}
\definecolor{lightblue}{rgb}{.8,.8,1}

%%%%%%%%%%%%%%%%%%%%%%%%%%%%%% 自定义命令
\newcommand{\tml}{Teichmüller space}
\newcommand{\hil}{Hilbert space}
\newcommand{\mtc}{Modular Tensor Category}
\newcommand{\bra}[1]{\left\langle #1 \right|}
\newcommand{\ket}[1]{\left| #1 \right\rangle}
\newcommand{\braket}[2]{\left\langle #1 \middle| #2 \right\rangle}
\newcommand{\ketbra}[2]{\left| #1 \middle\rangle\!\middle\langle #2 \right|}
\newcommand{\tr}{\operatorname{Tr}}

%%%%%%%%%%%%%%%%%%%%%%%%%%%%%% BEGINNING OF THE DOCUMENT

\begin{document}

\title{\boldmath Quantum Information Theory 1 \\ \large EPFL 2025-2026 学年第一学期课程笔记}

\author{X. D. H.}

\maketitle

\begin{abstract}
  这是EPFL 2025秋季学期的QI的课程笔记。
\end{abstract}

\tableofcontents
\newpage
\section{Lecture 2: Quantum 2 bit System}\label{sec:Lecture 2} % (fold)
\subsection{经典力学基础}

\hlr{Action Principle 动力学描述}

对于有限自由度的经典力学我们使用Least action principle来描述系统的动力学。
\thm{Least Action Principle

假设已经知道$ q(t_i) = q_i, \ q(t_f) = t_f $【Derichlet Boundary Condition】粒子运动轨迹满足:
\begin{align}
  S\left[q\right]\equiv\int_{t_{i}}^{t_{f}}L\left(q,\dot{q}\right)dt, \quad \delta S[q] = 0
\end{align}
}

\rmk{注意几个需要用的变分学技巧

\begin{enumerate}
  \item 变分的基本定义—— functional 的线性主要部分:
\begin{align}
  \delta S[q] = \delta^{(1)} S[q] = S[q+\delta q] - S[q] |_{\text{linear part in } \delta q}
\end{align}
\item 变分计算的性质:
  \begin{itemize}
    \item 「我们认为自由度和自由度的导数独立并分别变分」$ \frac{\partial L}{\partial q_a}\delta q_a+\frac{\partial L}{\partial\dot{q}_a}\delta\dot{q}_a $
    \item 「变分和时间导数交换」$ \delta \dot{q} = \displaystyle\frac{d}{d t} \delta q $这样才能进行分部积分
  \end{itemize}
\item 我们最后给出lagrangian是因为bulk的$ \delta q $是可以取任意无限小的函数形式变换。所以我们才能够需要Euler-Lagrangian Equation = 0; boundary的$ \delta q $是0所以不需要考虑。「这两点都是因为合理的边界条件导致的」
\end{enumerate}
}

对于变分原理的一些特性有两个特别重要的讨论:
\begin{enumerate}
  \item 为什么仅仅对一个$ dt $进行积分为什么Lagrangian不会依赖于很多的时间点的函数?【需要保证locality,不同时间点不会互相作用】
  \item 为什么Lagrangian不会依赖于二阶导数?【需要保证EoM是二阶微分方程】同时,如果含有高阶导数某些情况可以写成更多变量的一阶导数形式。
\end{enumerate}

\textbf{Action 从更基础的物理「quantum mechanics」的视角出发其实是更fundamental 的物理量;而非Lagrangian}

\bigskip
\hlr{Hamiltonian 动力学描述}

对于Hamiltonian我们需要定义一个时间方向「这是一个非协变的理论框架」。定义共轭动量和Hamiltonian并且动力学可以通过下面方程描述:
\begin{align}
  \dot{q}_a&=\frac{\partial H}{\partial p_a}, \quad\dot{p}_a=-\frac{\partial H}{\partial q_a}
\end{align}

\bigskip 
\hlr{Hailtonian的代数结构}

Hamiltonian动力学除了通常的使用动力学方程的视角进行描述也可以直接使用一套代数公理进行描述:
\begin{itemize}
  \item 所有客观测量是相空间的函数 $ A(p,q) $
  \item 客观测量空间「相空间的函数空间$ f: \text{phase space} \to \mathbb{R} $」存在李代数结构:
    \begin{align}
      \{A,B\}\equiv\frac{\partial A}{\partial p_a}\frac{\partial B}{\partial q_a}-\frac{\partial A}{\partial q_a}\frac{\partial B}{\partial p_a}\mathrm{~.}
    \end{align}
    \item 李代数保证了所有客观测量都可以generate一个无穷小变换"canonical transformation":
      \begin{align}
        q_{a}^{\prime}\equiv q_{a}+\epsilon\left\{A,q_{a}\right\}_{a}^{\prime}\equiv p_{a}+\epsilon\left\{A,p_{a}\right\}
      \end{align}
      其中一些canonical transformation被称为是Symmetry如果$ A(p,q) $是一个守恒量。
      \item 所有客观测量的时间演化由Hamiltonian这个客观测量的canonical transformation生成:
        \begin{align}
          \dot{q}_a=\{H,q_a\}, \quad \dot{p}_a=\{H,p_a\}, \quad \dot{A}=\{H,A\}
        \end{align}
\end{itemize}
这样的力学代数结构给出了一种代数量子化的手段"canonical quantization"


\bigskip
\hlr{Action Principle of Field Theory}

对于场论我们可以理解为有流形结构的无限自由度系统。我们认为场的lagrangian是一个等时面上面的函数:
\begin{align}
  L[\phi,\dot{\phi}]=\int d^3\mathbf{x}\left.\mathcal{L}\left(\phi(\mathbf{x},t),\dot{\phi}(\mathbf{x},t),\vec{\nabla}\phi(\mathbf{x},t)\right)\right..
\end{align}
并且Action可以理解为下面的形式:
\begin{align}
  S\left[\phi\right]=\int dtL[\phi,\dot{\phi}]=\int d^4x\mathcal{L}\left(\phi(x),\partial_\mu\phi(x)\right)
\end{align}
\rmk{
  对于上下指标我们有这样的规定$ x^\mu = (t,x^i), \ \partial_\mu = (\partial_{t}, \partial_{i}) $ 
}
下面给出least action principle:
\thm{Least Action Principle of Field Theory

  对于一个四维的时空流形之中的一个区域$ \Omega $,如果边界上面存在Dirichlet Boundary Condition:$ \delta \phi_a |_{\partial \Omega} $那么场在这个时空区域内的动力学构型满足:
  \begin{align}
    \delta S_\Omega\left[\phi\right] = 0
  \end{align}
  同样的利用【边界上$ \delta \phi_a= 0 $】以及【Bulk里面的$ \delta \phi_a $随便】我们得到等价的Euler-Lagrange Equation of Field Theory:
  \begin{align}
    \frac{\partial\mathcal{L}}{\partial\phi_\alpha}-\partial_\mu\frac{\partial\mathcal{L}}{\partial(\partial_\mu\phi_\alpha)}=0
  \end{align}
}

\rmk{
  注意,我们这里的边界条件不仅仅要求的是时间边界条件还需要每一个时间的空间边界条件!边界条件需要包裹整个讨论动力学的时空区域。
}


\bigskip
\hlr{常用数学概念:functional derivative}

对于场的变分以及之后和action相关的计算,我们常常使用functional derivative的数学操作进行计算,这个概念给出了对于functional合理的导数的意义,也就是当我们的场构型变一点点的时候我们的functional的变化行为是什么样子的。我们定义:
\begin{align}
  \frac{\delta S_\Omega}{\delta\phi(x)}\equiv\left(\frac{\partial\mathcal{L}}{\partial\phi}-\partial_\mu\frac{\partial\mathcal{L}}{\partial(\partial_\mu\phi)}\right)
\end{align}
更一般的数学上我们定义一个积分functional是:
\begin{align}
  F[\phi]=\int d^nxf(\phi,\partial_i\phi,x)
\end{align}
其Functional derivative定义为:
\begin{align}
  \frac{\delta F}{\delta\phi(x)}\equiv\frac{\partial f}{\partial\phi}-\partial_i\frac{\partial f}{\partial(\partial_i\phi)}
\end{align}
一个特殊的functional derivative的例子是:
\begin{align}
  \frac{\delta\phi(y)}{\delta\phi(x)}=\delta^{(4)}(x-y)
\end{align}
所以我们之后写这个不是强行使用一个看似合理的符号,而是使用functional derivative严格的数学概念进行书写的!!


\imp{区分functional derivative和一般函数导数}{
  我们通常会使用两个记号需要区分:
  \begin{itemize}
    \item functional derivative: $ \displaystyle\frac{\delta F}{\delta\phi(x)} $这说的就是一个积分泛函对于其变量函数的functional derivative定义为:
      \begin{align}
  \frac{\delta F}{\delta\phi(x)}\equiv\frac{\partial f}{\partial\phi}-\partial_i\frac{\partial f}{\partial(\partial_i\phi)}
      \end{align}
    \item   对于某一个函数的导数比如:$ \displaystyle\frac{\partial G(f(x))}{\partial f(x)} $是把一个函数$ f(x) $当成自变量进行的导数
  \end{itemize}
  这两者的意义完全不一样需要进行区分捏!!!
}

\bigskip
\hlr{Hamiltonian Formalism基本概念}

\textbf{需要着重强调对于Hamiltonian Formalism的定义!!}我们需要下满几个步骤进行理论构建:
\begin{enumerate}
  \item 确定一个"时间方向"划定等时面!【注意,我们lagrangian的讨论是几乎把时间和空间等价讨论的,但是Hamiltonian我们第一步是打破这种等价性】
  \item 定义共轭动量场:$ \pi(x)\equiv\displaystyle\frac{\delta L}{\delta\dot{\phi}(x)}=\displaystyle\frac{\partial\mathcal{L}}{\partial\dot{\phi}} $
\begin{itemize}
  \item 注意,我们定义canonical momentum field为lagrangian对于动力学场的functional derivative 
  \item 但是由于lagrangian并不包含$ \nabla_i \dot{\phi_a} $这样的项我们的共轭动量场才正好是Lagrangian density对于$ \dot{\phi} $的偏导数
\end{itemize}

  \item 定义Hamiltonian 是一个生活在等时面上面的函数「并不体现协变」:
    \begin{align}
      H(\pi,\phi)=\int d^3\mathbf{x}\left(\pi(x)\dot{\phi}(x)-\mathcal{L}(\phi,\pi)\right)
    \end{align}并且辅助性定义Hamiltonian density:
    \begin{align}
      \mathcal{H}\equiv\pi(x)\dot{\phi}(x)-\mathcal{L}(\phi,\pi)
    \end{align}
\end{enumerate}


\bigskip
\hlr{场的Hamiltonian力学的代数形式}

我们在上面的概念基础上使用代数形式描述场的动力学。
\begin{enumerate}
  \item 客观测量推广到functional:
    \begin{align}
      A[\pi,\phi]=\int d^3\mathbf{x}\left.a(\pi,\phi)\right.\quad B[\pi,\phi]=\int d^3\mathbf{x}\left.b(\pi,\phi)\right.
    \end{align}
    \item 通过functional derivative 定义李代数结构:
  \begin{align}
    \{A,B\}\equiv\int d^3\mathbf{x}\left(\frac{\delta A}{\delta\pi}\frac{\delta B}{\delta\phi}-\frac{\delta A}{\delta\phi}\frac{\delta B}{\delta\pi}\right)
  \end{align}
  \begin{itemize}
    \item 李代数结构对应着canonical对易关系:
      \begin{align}
        \begin{aligned}&\{\pi(\mathbf{x},t),\phi(\mathbf{y},t)\}=\delta^{(3)}(\mathbf{x}-\mathbf{y})\\&\{\pi(\mathbf{x},t),\pi(\mathbf{y},t)\}=0=\{\phi(\mathbf{x},t),\phi(\mathbf{y},t)\}\end{aligned}
      \end{align}
  \end{itemize}
  
  \item 动力学量的时间演化由Hamiltonian生成:
    \begin{align}
      \dot{\phi}(x)=\{H,\phi(x)\}, \quad \dot{\pi}(x)=\{H,\pi(x)\}, \quad \dot{A}=\{H,A\}
    \end{align}
    \begin{itemize}
      \item 这个正好对于场和共轭动量场给出了functional derivative版本的Hamilton's Equation:
        \begin{align}
          \begin{aligned}&\dot{\pi}=\{H,\pi\}=-\frac{\delta H}{\delta\phi}=\vec{\nabla}\cdot\frac{\partial\mathcal{H}}{\partial(\vec{\nabla}\phi)}-\frac{\partial\mathcal{H}}{\partial\phi}\\&\dot{\phi}=\{H,\phi\}=\frac{\delta H}{\delta\pi}=\frac{\partial\mathcal{H}}{\partial\pi}\end{aligned}
        \end{align}
    \end{itemize}
\end{enumerate}

\rmk{注意所有对于积分functional的推广都是把导数推广到了functional derivative然后再积分「还需要积分是因为functional derivative是定义在local量的导数的,需要积分才能得到一个等时面上的量」。可见这是一个合理的推广方案!!}


\subsection{Questions and Thoughts}

\question{为什么对于多粒子的Lagrangian没有二阶导数,如果有二阶导数是什么,functional derivative是什么?}

显然Lagrangian作为任意的函数我们是十分随意的,我们可以Generally写作:
\begin{equation}
  L(q(t),q(t'),...,\dot{q},\ddot{q},\cdots,q^{(n)},t)
  \label{eq:generallage}
\end{equation}
但是我们会发现我们因为各种物理的原因收敛到了一般的样子$ L(q,\dot(q) )$因为下面的原因:
\begin{enumerate}
  \item 如果我们Action存在$ S = \int dt dt' L(q(t),g(t')) $我们意味着两个不同时间的自由度会存在相互作用。这显然是不符合物理的时间维度的locality的。
  \item 如果我们存在高阶导数:一个自然的结果是Lagrange Equation是高阶微分方程。而牛顿定律告诉我们力学规律是二阶微分方程。
    \begin{itemize}
      \item 另一个解释是,如果我们存在高阶导数。那么其实可以把这个变分变成更多的变量的只有一阶导数的变分问题。
    \end{itemize}
  \item 对于不显含时间,这意味着能量不守恒,存在其他变量影响捏。
\end{enumerate}

\qed 


\question{momentum 一定可以反解出来自由度的导数吗?}

共轭动量能够反解出来速度是有条件的,只有当下面的Hessian矩阵非奇异【也就是可逆】的时候才可以:
\begin{align}
  \partial^2L/\partial\dot{q}_i\partial\dot{q}_j
\end{align}
如果是奇异的话意味着下面的东西:
\begin{enumerate}
  \item 我们使用了过多的变量来描述系统。使得我们的共轭动量需要满足一些约束。只有在这个约束面上面才是physical的。
  \item 这样的Hamiltonian力学系统需要使用Dirac的约束力学来处理。
  \item 这种系统一般是有gauge symmetry的。
\end{enumerate}
\qed 

\question{Poisson bracket构成的线性映射的李代数,为什么是无穷维度的?怎么研究这个性质?}

我们注意Poisson Bracket是赋予在所有$ A: \mathbb{R}^n \times \mathbb{R}^n \to \mathbb{R} $的线性映射构成的线性空间的。我们研究这个线性空间的性质,会发现独立的线性映射的个数是无穷多个。

一个直观的基是所有的delta函数的组合。显然所有的delta函数是有无穷多的。

\qed 

\question{经典力学我们认为相空间是自由度?但是,量子化之后p和q其实都是算符?量子的自由度到底是什么?}

这个问题其实是说:量子力学系统的自由度到底是什么?

量子的自由度最基础的理解其实是一组canonical operator!!就像是经典力学里我们的自由度是$ (p,q) $这一组canonical变量一样。量子力学我们认为自由度是这样的一组【注意一定是一组】$ [p_i,x_j] = i \hbar \delta_{ij}  $的算符。

\hlr{注意:我们虽然一般只用某一个算符的本征矢量描述自由度,但是实际上决定的是一组canonical算符的代数关系;经典的【自由度】的概念量子化后变成了【客观测量的代数关系】}

这样的算符代数关系连同Identity operator构成了一个李代数,我们称之为Heisenberg Algebra。然后我们的Hilbert Space其实是这个李代数的不等价不可约表示空间构成的。

\imp{量子化的步骤}{
我们上面的思路分成下面几个步骤:
\begin{enumerate}
  \item 找到一组canonical的自由度$(p_i,q_j)$
  \item 量子化成为canonical的算符$ [\hat{p}_i,\hat{q}_j] = i \hbar \delta_{ij} $
  \item 研究这个李代数的表示论,构建Hilbert Space
\end{enumerate}
我们永远不是先有Hilbert Space,而是先确定我们的自由度,再通过表示论给出Hilbert Space。
}

对于量子场论来说,我们也是一样的。我们先找到一组canonical的自由度$ [a_p^\dagger,a_p] $然后我们作用在一个假设的Vaccum State上面,构建出Hilbert Space。

但是一般的量子场论体系下,我们会发现这个代数空间不仅仅是Heisenberg代数的表示空间,也可能构成其他代数(Virasoro Algebra, Kac-Moody Algebra)。

\tip{代数关系和Hilbert空间}{
  我们意识到,我们永远是现有代数关系再有Hilbert空间的捏!!从最基础的量子力学来都是这样的。
}
\qed


\question{为什么场论的可以写作Lagrangian积分的形式?而不是混沌的一个等时流形上面的函数?}

注意空间的locality!!!和时间一样。这也是为什么我们要把x variable进行区分出来。因为我们需要特殊强调这个变量需要时local的!!

\qed

\question{我们场论使用的量子力学的动量和位置3+1维度的归一化是什么??Weinberg之中讨论的协变归一化是什么意思??}

\hlr{非相对论一般归一化}

对于三维的量子力学来说,我们需要保证【完备性条件】和【归一化条件】形式不变也就是:
\begin{align}
  \int dx|x\rangle\langle x|=1,\quad \int dp|p\rangle\langle p|=1.\\ 
  \langle x|x'\rangle=\delta(x-x'),\quad \langle p|p'\rangle=\delta(p-p').
\end{align}
当然动量我们只能delta函数归一化,显然需要从一维变成三维!这样的一个结论就是动量本征态在空间表象写作;
\begin{align}
  \langle\mathbf{x}|\mathbf{p}\rangle=\frac{1}{(2\pi)^{3/2}}e^{i\mathbf{p}\cdot\mathbf{x}}.
\end{align}

\hlr{Weinberg讨论的协变归一化}

我们对于量子态的归一化是有自由度的,因为我们只需要保证在一定的测度下完备;并且应该正交的态互相正交就好!所以我们不妨定义下面的归一化条件:



下面解释为什么要选择这个归一化条件:





\question{Method of Stationary Phase是什么捏?}

讲一下这种近似方法



% section Lecture 2 (end)

\newpage
\section{Lecture 3: Purification and POVM measurement }\label{sec:Lecture 3} % (fold)
\subsection{Covariant Derivative}

\subsubsection{Covariant Derivative的定义和性质}

\hlr{Tangent Bundle上导数算符的视角}

我们现在有一个问题就是:怎么计算tangent bundle上面的导数。回顾平直时空我们naively的tangent bundle上面的导数(也就是方向导数),我们会发现这个算符满足下面两个视角:
\begin{enumerate}
  \item \textbf{方向导数视角: }确定一个向量场作为输入,然后把一个标量场映射成另一个标量场:
    \begin{align}
    v^i \partial_{i} f = V(f) \in \mathcal{F}
    \end{align}
  \item \textbf{张量视角: }把一个(1,0)张量映射成一个(1,1)张量。也就是:
    \begin{align}
      \partial_{i} v^j  \in \mathcal{T}(1,1)
    \end{align}
    相当于一个$(1,1)$张量。
\end{enumerate}
我们希望延续这两个视角,并且推广到一般的manifold上面。

我们定义一个算符,叫做\textbf{协变导数}(Covariant Derivative),记作$ \nabla $,这个算符满足下面两个视角:
\begin{enumerate}
  \item \textbf{方向导数视角: }确定一个向量场作为输入,然后把一个(k,l)张量映射成另一个(k,l)张量:
    \begin{align}
    \nabla_V T \in \mathcal{T}(k,l)
    \end{align}
    其中$ V \in \mathcal{T}(1,0) $是一个向量场,$ T \in \mathcal{T}(k,l) $是一个$(k,l)$张量场。
  \item \textbf{张量视角: }把一个(k,l)张量场映射成一个(k,l+1)张量场。也就是:
    \begin{align}
      \nabla T \in \mathcal{T}(k,l+1)
    \end{align}
    相当于一个$(k,l+1)$张量场。
\end{enumerate}
\rmk{
  很自然可以发现两个视角是等价的:
  \begin{enumerate}
    \item 方向导数视角如果选择是一个基矢量方向,那么可以把这个基矢量指标当作一个张量指标,给出张量视角
    \item 张量视角下,如果把第一个指标和一个向量场进行contraction,那么可以给出方向导数视角
  \end{enumerate}
  一般严格的书之中我们是使用第二个视角的。但是第一个视角方便我们进行推广,也方便我们进行理解(更好类比平直时空)。}

\bigskip
\hlr{Covariant Derivative的定义}


为了推广的方便我们使用第一种视角进行理解。我们认为协变导数是这样的一个算符:
\begin{align}
  \nabla: \mathcal{T}(1,0) \times \mathcal{T}(k,l) \to \mathcal{T}(k,l)
\end{align}
也就是说我们输入一个向量场和一个$(k,l)$张量场,输出一个$(k,l)$张量场。

我们不会直接给出定义而是一步步进行构造,我们思考如果推广导数算符需要满足哪些性质。下面两个是最最最基本的导数的性质:
\begin{itemize}
  \item \textbf{Linearity: }对于任意的$ T,S \in \mathcal{T}(k,l) $,以及任意的$ a,b \in \mathbb{R} $,我们有:
    \begin{align}
      \nabla_V (aT + bS) = a\nabla_V T + b\nabla_V S
    \end{align}
  \item \textbf{Leibniz Rule: }对于任意的$ T \in \mathcal{T}(k,l) $,$ S \in \mathcal{T}(m,n) $,我们有:
    \begin{align}
      \nabla_V (T \otimes S) = (\nabla_V T) \otimes S + T \otimes (\nabla_V S)
    \end{align}
\end{itemize}
接下来我们考虑如果要和一般的导数算符很像我们还需要满足什么性质。我们会发现和平直时空的方向导数类似还需要满足:
\begin{itemize}
  \item \textbf{Action on functions: }对于任意的标量场$ f \in \mathcal{F} $,我们有:
    \begin{align}
      \nabla_V f = V(f)
    \end{align} 
  \item \textbf{Linearity in the vector field: }对于任意的$ V,W \in \mathcal{T}(1,0) $,以及任意的$ a,b \in \mathbb{R} $,我们有:
    \begin{align}
      \nabla_{aV + bW} T = a\nabla_V T + b\nabla_W T
    \end{align}
\end{itemize}

\rmk{这里我们已经可以意识到,Linearity in Vector Field已经意味着如果我们使用基矢量$ \partial_{\mu} $作为这个vector field的话。

那么协变导数算符在不同坐标系下形式其实就是矢量变换的形式,也就是说协变导数算符作用之后相当于多了一个covariant的指标。这就很自然的有第二种视角了。

并且Action on functions可以很自然的写作:
\begin{align}
  \nabla_{V} f = V^\mu \nabla_{\mu} f.
\end{align}
}
最终还有一个最最最最重要的性质:
\begin{itemize}
  \item \textbf{Commutation with contraction: }对于任意的$ T \in \mathcal{T}(k,l) $,以及任意的contraction操作$ C $,我们有:
    \begin{align}
      \nabla_V (C(T)) = C(\nabla_V T)
    \end{align}
\end{itemize}
\rmk{
  最后这个性质特别重要。因为一般张量我们都可以写作(比如vector):
  \begin{align}
    v = v^\mu \partial_\mu 
  \end{align}
  那么作用上一个协变导数算符(比如 $ \nabla_\mu $)的结果其实可以先把$ v^\mu $当作一个scalar field和$ \partial_{\mu} $进行tensor product,分别用lebniz rule 作用,然后再进行contraction。也就是这样:
  \begin{align}
    \nabla_{\boldsymbol{u}}(f\boldsymbol{v})=\boldsymbol{u}(f)\boldsymbol{v}+f\nabla_{\boldsymbol{u}}\boldsymbol{v}.
  \end{align}

同样的对于一般的张量我们也可以进行这样的操作。
}

\bigskip
\hlr{协变导数作用在contravariant vector上的形式}

上面的五条性质已经define了一个导数算符(虽然并不是唯一的)。我们现在不妨就计算一下这样的性质下的导数算符有什么更多的性质。
\begin{enumerate}
  \item \textbf{作用在基矢量上面的形式: }我们考虑协变导数作用在基矢量上面的形式。我们有:
    \begin{align}
      \nabla_{\partial_\mu} \partial_\nu = \Gamma^\lambda_{\mu\nu} \partial_\lambda
    \end{align}
    其中$ \Gamma^\lambda_{\mu\nu} $是一个新的张量,叫做\textbf{Connection}。这个张量不是一个真正的张量,因为它的变换形式并不满足张量的变换形式。 

\rmk{
  为了记号方便我们会写$ \nabla_\mu $意思是$ \nabla_{\partial_\mu} $。更细致的讨论我们放在question部分。
}
\rmk{
  其实从定义就很容易看出来Connection并不是一个张量,因为给出其指标的并不是张量指标,而是【选定一个坐标系】之后的基矢量的指标。这是坐标系dependent的。我们可以计算出其变换法则是:
  \begin{align}
    \Gamma_{\beta\gamma}^{\prime\alpha}=\frac{\partial x^\mu}{\partial x^{\prime\beta}}\frac{\partial x^\nu}{\partial x^{\prime\gamma}}\frac{\partial x^{\prime\alpha}}{\partial x^\rho}\Gamma_{\mu\nu}^\rho+\frac{\partial^2x^\rho}{\partial x^{\prime\beta}\partial x^{\prime\gamma}}\frac{\partial x^{\prime\alpha}}{\partial x^\rho}\mathrm{~.}
  \end{align}
}
    \item \textbf{作用在一般向量场:}

      我们前面已经计算了作用在基矢量上面的形式,那么可以使用第二个remark的结果计算作用在一般向量场的形式:
      \begin{align}
        \nabla_{\partial_\mu} V = \nabla_{\partial_\mu} (V^\nu \partial_\nu) = (\partial_\mu V^\nu) \partial_\nu + V^\nu (\nabla_{\partial_\mu} \partial_\nu) = (\partial_\mu V^\nu + \Gamma^\nu_{\mu\lambda} V^\lambda) \partial_\nu
      \end{align}
\rmk{
  为了记号方便我们会写$ \nabla_\mu V^\nu \equiv (\nabla_{\partial_\mu} V)^\nu $。也就是$ \nabla_\mu V $在$ \partial_{\mu} $所在的坐标系下面的分量。
}
\end{enumerate}

\bigskip
\hlr{协变导数作用在covariant co-vector上的形式}

类似的我们研究协变导数是怎么用作用在covariant co-vector上的。
\begin{enumerate}
  \item \textbf{作用在基协变矢量上面的形式: }我们考虑协变导数作用在基协变矢量上面的形式。我们有:
    \begin{align}
      \nabla_{\partial_\mu} dx^\nu = -\Gamma^\nu_{\mu\lambda} dx^\lambda
    \end{align}
    其中$ \Gamma^\nu_{\mu\lambda} $和前面定义的connection是一样的\textbf{这个证明见讲义}
    \item \textbf{作用在一般协变向量场:}

      我们前面已经计算了作用在基协变矢量上面的形式,那么可以使用第二个remark的结果计算作用在一般协变向量场的形式:
      \begin{align}
        \nabla_{\partial_\mu} \omega = \nabla_{\partial_\mu} (\omega_\nu dx^\nu) = (\partial_\mu \omega_\nu) dx^\nu + \omega_\nu (\nabla_{\partial_\mu} dx^\nu) = (\partial_\mu \omega_\nu - \Gamma^\lambda_{\mu\nu} \omega_\lambda) dx^\nu
      \end{align}
      \rmk{
        为了记号方便我们会写$ \nabla_\mu \omega_\nu \equiv (\nabla_{\partial_\mu} \omega)_\nu $。也就是$ \nabla_\mu \omega $在$ \partial_{\mu} $所在的坐标系下面的分量。
      }
\end{enumerate}

\bigskip
\hlr{协变导数作用在一般张量上的形式}

类似上面的情况我不想多说了,直接写出$ \nabla_\mu $作用之后在对应坐标系下的分量形式
\begin{align}
 \begin{aligned}\nabla_\lambda T_{\nu_1...\nu_l}^{\mu_1...\mu_k}&=\partial_\lambda T_{\nu_1...\nu_l}^{\mu_1...\mu_k}\\&+\Gamma_{\lambda\rho}^{\mu_1}T_{\nu_1...\nu_k}^{\rho\mu_2...\mu_k}+\Gamma_{\lambda\rho}^{\mu_2}T_{\nu_1...\nu_k}^{\mu_1\rho...\mu_k}+\cdots\\&-\Gamma_{\lambda\nu_1}^\rho T_{\rho\nu_2...\nu_l}^{\mu_1...\mu_k}-\Gamma_{\lambda\nu_2}^\rho T_{\nu_1\rho...\nu_l}^{\mu_1...\mu_k}-\cdots.\end{aligned} 
\end{align}


\subsubsection{parallel transport}

有了协变导数的定义之后,我们可以定义怎么把一个张量沿着流形上面的一个曲线进行平行移动(parallel transport)。我们定义:

\defi{
  Parallel Transport

  给定一个流形上面的曲线$ C: \mathbb{R} \to \mathcal{M}, C(t) \in \mathcal{M} $选择一个coordinate system之后我们可以给出一个切向量:$ t^\mu = \displaystyle\frac{dx^\mu}{d t}  =  \displaystyle\frac{d (\psi \circ C(t))^\mu}{dt}$。于是我们可以定义,如果一个张量场$ T $满足:
  \begin{align}
    \nabla_{t} T = 0 
  \end{align}
  那么我们说这个张量场$ T $沿着曲线$ C $是平行移动的。
}
我们使用分量的语言写下来平行移动也就是在某个坐标系下面满足:
\begin{align}
  t^\mu\partial_\mu v^\nu+t^\mu\Gamma_{\mu\alpha}^\nu v^\alpha=\frac{dv^\nu}{dt}+t^\mu\Gamma_{\mu\alpha}^\nu v^\alpha=0.
\end{align}



\subsubsection{Physical Covariant Derivative}

满足上面的性质的协变导数并不是唯一的。但是物理世界我们用来描述世界的导数必然是唯一的。我们会发现上面的要求缺失了对于【距离】的要求。

真是的物理世界为了描述时空我们需要引入metric tensor作为距离的概念。所以我们需要metric对于协变导数有一个兼容性要求。

下面我们给出两个物理上的要求,确定协变导数的具体形式

\bigskip
\hlr{Torsion Free协变导数}

物理上我们一般要求协变导数满足下面的限制条件:
\begin{itemize}
  \item \textbf{Torsion Free: }对于任意的标量场$ f \in \mathcal{F} $,我们有:
    \begin{align}
      \nabla_a \nabla_b f = \nabla_b \nabla_a f
    \end{align}
\end{itemize}
\rmk{
  注意,torsion free的要求仅仅针对于光滑的标量场。对于一般的张量,torsion是不会为0的,我们后面知道这个会包含时空本身的curvature信息。
}
这个要求给出了一个很重要的结论。
\begin{enumerate}
  \item connection是对称的:
\begin{align}
  \Gamma^\lambda_{\mu\nu} = \Gamma^\lambda_{\nu\mu}
\end{align}
\item 对易子的另一种形式:
  \begin{align}
    [\mathbf{v},\mathbf{w}]^\nu=\left(v^\mu\nabla_\mu w^\nu-w^\mu\nabla_\mu v^\nu\right). 
  \end{align}
  这是因为我们的connection是对称的,所以对于一个反对称的对易子加入connection这种东西只会产出0。所以我们可以把对易子定义的导数直接写作协变导数,这样对易子的形式就更加协变了。
\end{enumerate}

\bigskip
\hlr{Metric Compatible协变导数}

物理上我们不仅仅需要平行移动的时候向量的方向不变,我们还需要距离是不变的。所以如果一个流形上我们赋予一个metric tensor $ g_{ab} $,我们还需要协变导数满足下面的限制条件:
\begin{align}
  t^\alpha\nabla_\alpha\left(g_{\mu\nu}v^\mu w^\nu\right)=0.
\end{align}
也就是如果两个向量场沿着一个曲线平行移动,那么它们的内积也是不变的。这个条件等价于:
\begin{align}
  t^\alpha v^\mu w^\nu\nabla_\alpha g_{\mu\nu}=0.
\end{align}
所以我们为了保证这个需求还需要协变导数满足:
\begin{itemize}
  \item \textbf{Metric Compatible: }对于任意的metric tensor $ g_{ab} $,我们有:
    \begin{align}
      \nabla_a g_{bc} = 0
    \end{align}
\end{itemize}
当然这个条件还可以推出一些重要的结论:
\begin{enumerate}
  \item Metic不论是上指标还是下指标的协变导数都是0(见question部分的证明)
  \item 协变导数可以和升降指标操作交换顺序!!我们可以随意的【升降协变导数内部的指标 $ \nabla_\mu V^\nu \to \nabla_\mu V_\nu $】也可以【升降协变导数本身的指标$ \nabla^\mu $】
\end{enumerate}


\bigskip
\hlr{Levi-Civita Connection}

给出了上面的两个物理要求之后,我们可以唯一的确定协变导数的connection的形式,这个connection叫做Levi-Civita Connection,形式如下:
\begin{align}
  \Gamma_{\mu\nu}^\rho=\frac{1}{2}g^{\rho\sigma}\left(\frac{\partial g_{\nu\sigma}}{\partial x^\mu}+\frac{\partial g_{\mu\sigma}}{\partial x^\nu}-\frac{\partial g_{\mu\nu}}{\partial x^\sigma}\right)
\end{align}

\subsection{Curvature of Manifolds}

\subsubsection{Riemann Curvature Tensor}

\bigskip
\hlr{无限小平行移动的差距}

一个自然的观察就是,如果我们在一个有曲率的流形上面沿着一个闭合曲线进行平行移动一个向量的话,最终得到的向量和原来的向量并不一样。这并非我们的「平行移动」非良定,而是流形本身的曲率导致的。

因此我们考虑一个无限小的loop进行平行移动,那么我们只用考虑两个方向的导数算符的对易子:
\begin{align}
  [\nabla_\mu,\nabla_\nu]v^\rho 
\end{align}
一通计算之后我们给出结论:
\begin{align}
  [\nabla_\mu,\nabla_\nu]v^\rho =\quad\left(\partial_\mu\Gamma_{\nu\sigma}^\rho-\partial_\nu\Gamma_{\mu\sigma}^\rho+\Gamma_{\mu\lambda}^\rho\Gamma_{\nu\sigma}^\lambda-\Gamma_{\nu\lambda}^\rho\Gamma_{\mu\sigma}^\lambda\right)v^\sigma.
\end{align}
\rmk{
  其中计算我们需要选择一个坐标系(不妨选择$ \partial_{\mu}, \partial_{\nu} $对应的那个坐标系)。然后从外到内的计算分量。【特别值得注意的是,我们外面的导数算符的Connection作用里面的$ \nabla_\nu v^\rho $是一个(1,1)张量需要作用两下】就像这样:
  \begin{align}
    \partial_\mu\left(\nabla_\nu v^\rho\right)-\Gamma_{\mu\nu}^\lambda\nabla_\lambda v^\rho+\Gamma_{\mu\sigma}^\rho\nabla_\nu v^\sigma
  \end{align}
}


\bigskip
\hlr{Riemann Curvature Tensor的定义}

在上面计算讨论之后我们定义Riemann Curvature Tensor如下:
\defi{
  Riemann Curvature Tensor坐标定义

  给定一个流形上面的一个coordinate system。并且给出这个coordinate system上面的Levi-Civita Connection $ \Gamma_{\mu\nu}^\rho $,我们定义Riemann Curvature Tensor为:
  \begin{align}
    \large[\nabla_\mu,\nabla_\nu]v^\rho=R^\rho{} _{\sigma\mu\nu}v^\sigma.
  \end{align}
  其中我们可以直接计算出来:
  \begin{align}
    R^\rho{}_{\sigma\mu\nu}=\partial_\mu\Gamma_{\nu\sigma}^\rho-\partial_\nu\Gamma_{\mu\sigma}^\rho+\Gamma_{\mu\lambda}^\rho\Gamma_{\nu\sigma}^\lambda-\Gamma_{\nu\lambda}^\rho\Gamma_{\mu\sigma}^\lambda.
  \end{align}

}

如果我们希望协变的定义Riemann Curvature Tensor的话,我们可以使用下面的视角:

\defi{Riemann Curvature Tensor协变定义

  给定一个流形上面的协变导数$ \nabla $,我们定义Riemann Curvature Tensor为下面的映射:
\begin{align}
  R(*,*) * :& TM \times TM \times TM \to TM\\ 
  R(X,Y)Z =& \nabla_X \nabla_Y Z - \nabla_Y \nabla_X Z - \nabla_{[X,Y]} Z
\end{align}
可以看到Riemann Curvature Tensor是一个$(1,3)$张量。如果上面的$ X, Y $正好是基矢量$ \partial_\mu, \partial_\nu $。我们知道基矢量是commute的(毕竟一眼看上去导数算符是commute的)所以可以得到之前坐标依赖的定义。
}
\rmk{
  为什么会有最后一项$ \nabla_{[X,Y]} Z $呢?这是因为计算$ \nabla_X \nabla_Y Z $的时候会出现$ \partial_{a}\nabla_Y Z^\nu $这样的项。这样$ \partial_{a} $需要对于$ Y^\mu $有一个导数,这个我们需要通过对易子进行消除。
}

如果把这个定义用坐标的形式写出来我们可以得到:
\begin{align}
  R_{\sigma\mu\nu}^\rho X^\mu Y^\nu Z^\sigma=X^\mu\nabla_\mu\left(Y^\nu\nabla_\nu Z^\rho\right)-Y^\mu\nabla_\mu\left(X^\nu\nabla_\nu Z^\rho\right)-(X^\mu\nabla_\mu Y^\nu-Y^\mu\nabla_\mu X^\nu)\nabla_\nu Z^\rho,
\end{align}

\rmk{
  对于Riemann Curvature Tensor的计算我们一定要注意,如果出现$ \partial_{a}\nabla_\mu X^\nu$这样的东西我们务必先把$ \nabla $作用完再进行导数算符的作用。
}

\bigskip
\hlr{协变导数的对易子作用在一般张量}

我们使用协变导数的对易子作用在矢量上面给出了Riemann Curvature Tensor的定义。那么我们可以推广到一般的张量上面。我们给出下面的结论:
\begin{itemize}
  \item \textbf{作用在vector上:}
    \begin{align}
      [\nabla_\mu,\nabla_\nu]v^\rho=R^\rho{}_{\sigma\mu\nu} v^\sigma.
    \end{align}
  \item \textbf{作用在covector上:}
    \begin{align}
      [\nabla_\mu,\nabla_\nu]\omega_\rho=-R^\sigma{}_{\rho\mu\nu} \omega_\sigma.
    \end{align}
  \item \textbf{作用在一般张量上:}
    \begin{align}
      [\nabla_\mu,\nabla_\nu]T^{\sigma_1...\sigma_k}{}_{\rho_1...\rho_l}=\sum_{i=1}^k R^{\sigma_i}{}_{\lambda\mu\nu} T^{\sigma_1...\lambda...\sigma_k}{}_{\rho_1...\rho_l}-\sum_{j=1}^l R^\lambda{}_{\rho_j\mu\nu} T^{\sigma_1...\sigma_k}{}_{\rho_1...\lambda...\rho_l}.
    \end{align} 
\end{itemize}
因此我们会发现Riemann Curvature Tensor本身就仿佛完全决定了所有这样的流形上的曲率的信息。而不是一个仅仅对于vector有用的量。

\subsubsection{Properties of Riemann Curvature Tensor}

Riemann Curvature Tensor标main上有着很多指标也有海量的分量。但是其实有着很多很好的对称性保证其独立的分量其实没有多少。我们下面结果这些对称性结果:
\begin{enumerate}
  \item \textbf{反对称性1: }Riemann Curvature Tensor在最后两个指标上是反对称的:
    \begin{align}
      R^\rho{}_{\sigma\mu\nu} = - R^\rho{}_{\sigma\nu\mu}
    \end{align}
    如果是使用协变的定义我们也可以写作:
    \begin{align}
      R(X,Y)=-R(Y,X)
    \end{align}
  \item \textbf{反对称性2: }Riemann Curvature Tensor在前两个指标是反对称的(如果把指标降下来)【本性质仅仅适用于「Levi-Civita Connection」】
    \begin{align}
      R_{\rho\sigma\mu\nu} = - R_{\sigma\rho\mu\nu}
    \end{align}
  \item \textbf{三指标全反对称:}对于后三个指标其实是全反对称的:【本性质仅仅适用于「Levi-Civita Connection」】
    \begin{align}
      R_{\rho[\sigma\mu\nu]} = 0
    \end{align}
    也就是:
    \begin{align}
      R_{\rho\sigma\mu\nu} + R_{\rho\mu\nu\sigma} + R_{\rho\nu\sigma\mu} = 0
    \end{align}
  \item \textbf{交换对称性: }Riemann Curvature Tensor交换前两个指标和后两个指标是对称的:【本性质仅仅适用于「Levi-Civita Connection」】
    \begin{align}
      R_{\rho\sigma\mu\nu} = R_{\mu\nu\rho\sigma}
    \end{align}
    \item \textbf{Bianchi Identity: }Riemann Curvature Tensor满足Bianchi Identity:
      \begin{align}
        \nabla_{[\lambda} R_{\rho\sigma]\mu\nu} = 0
      \end{align}
\end{enumerate}

对于Bianchi Identity我们展开的写出来意味着下面的等式成立:
\begin{align}
  \begin{aligned}[[\nabla_\lambda,\nabla_\rho],\nabla_\sigma]+[[\nabla_\rho,\nabla_\sigma],\nabla_\lambda]+[[\nabla_\sigma,\nabla_\lambda],\nabla_\rho]=0,\end{aligned}
\end{align}
这很像是一个导数算子的Jacobi Identity。


\subsubsection{Riemann Curvature Tensor相关的张量}

\hlr{Ricci Tensor and Ricci Scalar}

我们考虑从Riemann Curvature Tensor中contract出一些新的张量。我们发现很多的contraction会给出0 因为对称性。所以最终发阿羡合法的contraction只有下面一个:
\defi{Ricci Tensor and Ricci Scalar

我们定义Ricci Tensor为Riemann Curvature Tensor的contraction:
\begin{align}
  R_{\mu\nu} = R^\lambda{}_{\mu\lambda\nu}.
\end{align}
我们定义Ricci Scalar为Ricci Tensor的contraction:
\begin{align}
  R = g^{\mu\nu} R_{\mu\nu}.
\end{align}
}
我们可以显然的知道Ricci Tensor是一个【对称的】$(0,2)$张量,Ricci Scalar是一个标量场。



\bigskip
\hlr{
Einstein Tensor
}

根据Bianchi Identity我们其实还可以知道另一个很好玩的张量。我们把bianchi Identity进行一个contraction可以得到:
\begin{align}
  \nabla_\alpha R_\mu^\alpha+\nabla_\beta R_\mu^\beta-\nabla_\mu R=0
\end{align}
为此我们可以自然定义一个协变导数为0的张量:
\defi{
  Einstein Tensor

  我们定义Einstein Tensor为:
  \begin{align}
    G_{\mu\nu} = R_{\mu\nu} - \frac{1}{2} R g_{\mu\nu}.
  \end{align}
  自然满足:
  \begin{align}
    \nabla^\alpha G_{\alpha\beta}=0,\quad G_{\alpha\beta}=R_{\alpha\beta}-\frac{1}{2}g_{\alpha\beta}R.
  \end{align}
}


\bigskip
\hlr{Riemann Tensor的Decomposition}

Riemann tensor作为一个巨大的张量,显然可以decompose成为一些其他张量的组合。一个组合就是:
\begin{align}
  R_{\mu\nu\sigma\rho}=C_{\mu\nu\sigma\rho}+\frac{2}{n-2}\left(g_{\mu[\sigma}R_{\rho]\nu}-g_{\nu[\sigma}R_{\rho]\mu}\right)-\frac{2}{(n-1)(n-2)}Rg_{\mu[\sigma}g_{\rho]\nu}.
\end{align}
具体的讨论请参考各种教材。这里并不作详细展开了。


\subsection{Geodesic I}

\subsubsection{Geodesic Equation definition}

我们试图推广「直线」的概念。我们会发现,一个很自然的定义就是,如果一个曲线的切向量场沿着这个曲线是平行移动的,那么我们就说这个曲线是一个「测地线(geodesic)」。

\defi{
  Geodesic

  给定一个流形上面的曲线$ C: \mathbb{R} \to \mathcal{M}, C(t) \in \mathcal{M} $选择一个coordinate system之后我们可以给出一个切向量:$ t^\mu = \displaystyle\frac{dx^\mu}{d t}  =  \displaystyle\frac{d (\psi \circ C(t))^\mu}{dt}$。于是我们可以定义,如果这个切向量场满足:
  \begin{align}
    \nabla_{t} t = 0 
  \end{align}
  那么我们说这个曲线$ C $是一个测地线。
}
对于这个定义我们使用分量形式写出来就是:
\begin{align}
  T^\alpha\nabla_\alpha T^\beta=0. 
\end{align}
然后我们再展开协变导数写出来就是:
\begin{align}
  \frac{dT^\alpha}{dt}+\Gamma_{\beta\gamma}^\alpha T^\beta T^\gamma=0,
\end{align}
在带入切向量的定义之后我们可以得到测地线方程的最终形式:
\begin{align}
  \frac{d^2x^\alpha}{dt^2}+\Gamma_{\beta\gamma}^\alpha\frac{dx^\beta}{dt}\frac{dx^\gamma}{dt}=0. 
\end{align}
\rmk{
  注意,虽然一般我们对于任意的参数化都可以定义这样的方程。但是!!【只有用affine Parameter定义的Geodesic Equation是有物理意义的】!!!
}

\subsubsection{Geodesic as Extremal of Length}

下面我们就可以体现【Levi-Civita Connection的物理意义】的部分。因为我们发现这样的协变导数给出的geodesic其实就是极值距离曲线。

\bigskip
\hlr{流形上曲线长度}

我们可以通过metric和切向量的定义赋予流形上的一个曲线【长度】的概念。我们定义:
\defi{
  Curve Length on Manifold

  给定一个流形上面的曲线$ C: \mathbb{R} \to \mathcal{M}, C(t) \in \mathcal{M} $我们把可以定义下面的长度:
  \begin{align}
    l=\int\sqrt{g_{\mu\nu}T^\mu T^\nu}dt.
  \end{align}
}
我们会发现这个曲线长度的定义很合理并且满足下面的要求:
\begin{enumerate}
  \item \textbf{Parameterization Invariance: }如果我们对曲线进行一个reparameterization $ t \to t'(t) $,那么曲线的长度是不变的。
  \item \textbf{Coordinate Invariance: }如果我们对流形进行一个coordinate transformation $ x^\mu \to x^{\prime\mu}(x) $,那么曲线的长度是不变的。
  \item \textbf{Norm Preserve along curve:} 【如果研究的curve是一个geodesic】那么根据定义模长:$ \nabla_{\mathbf{T}}(g_{\mu\nu}T^\mu T^\nu)=0. $在曲线方向守恒的。
\end{enumerate}

\rmk{
  这个模长守恒其实根据Blau的说法,体现了参数化平移不变性的对称性!!
}

\bigskip
\hlr{Lorentz Signiture下的曲线}

在Lorentz Signiture下我们需要区分三种不同的曲线:
\begin{itemize}
  \item \textbf{Timelike Curve: }如果一个曲线的切向量满足$ g_{\mu\nu}T^\mu T^\nu < 0 $,那么我们说这个曲线是一个timelike curve。
  \item \textbf{Spacelike Curve: }如果一个曲线的切向量满足$ g_{\mu\nu}T^\mu T^\nu > 0 $,那么我们说这个曲线是一个spacelike curve。
  \item \textbf{Null Curve: }如果一个曲线的切向量满足$ g_{\mu\nu}T^\mu T^\nu = 0 $,那么我们说这个曲线是一个null curve。
\end{itemize}
狭义相对论已经告诉我们,timelike curve对应的物体是有质量的粒子的世界线,所以如果要描述time like curve的长度我们一般定义另一个也就是proper time:
\begin{align}
  \tau=\int\sqrt{-g_{\mu\nu}T^\mu T^\nu}dt.
\end{align}


\subsection{Questions and Thoughts}
\question{我们要求协变导数是metric compatible的,能不能同时说明对于metric两个上和两个下指标都是0?}

我们一般只证明了$ \nabla_a g_{bc} = 0 $我们应该怎么证明$ \nabla_a g^{bc} = 0 $。这个涉及一点点计算技巧:

首先我们证明delta函数的协变导数是0。我们有:
\begin{align}
  \nabla_a(\delta^b{}_cV^c)=\nabla_aV^b. 
\end{align}
所以可以根据lebniz rule推导出来:
\begin{align}
  (\nabla_a\delta_c^b)V^c+\delta_c^b(\nabla_aV^c)=\nabla_aV^b.
\end{align}
对比两边的式子,delta张量的协变导数就是0。

下一步我们知道张量的逆矩阵$ g_{ab}g^{bc} = \delta_{a}{}^{c} $所以我们可以给出:
\begin{align}
  \left(\nabla_ag^{bd}\right)g_{dc}+g^{bd}\left(\nabla_ag_{dc}\right)=0.
\end{align}
由于我们已经知道$ \nabla_ag_{bc} = 0 $所以我们有:
\begin{align}
  \left(\nabla_ag^{bd}\right)g_{dc}=0.
\end{align}
由于$ g_{ab} $是非退化的,所以我们可以乘上$ g^{ce} $得到:
\begin{align}
  \nabla_ag^{be}=0.
\end{align}
所以我们证明了metric的两个上指标的协变导数也是0。

\rmk{
  在这样的证明之后,我们知道协变导数的规则之中。我们可以【随意在协变导数内进行升降指标】也可以【升降协变导数的指标】!随意的写各种$ \nabla^\mu $之类的东西。
}

\bigskip
\question{为什么我们的Covariant derivative形式上确实从(k,l)张量变成了(k,l+1)张量,但是我们怎么形式化的写出基变多了一个呢?}

\bigskip
\textbf{协变导数观点1: }确定一个向量场作为输入,然后把一个(k,l)张量映射成一个(k,l)张量。
\bigskip

这种观点下面我们已经确定了一个向量场$ V $,然后对于这个向量场的协变导数被定义为:$ \nabla: V \times F_M(k,l) \to F_M(k,l) $,其中$ V $是确定的。

在这种观点下,表面上我们可以多一个下标记(当然更一般的这个下标应该写作一个张量,$ \nabla_\mu $的意思其实是$ \nabla_{\partial_{\mu}} $也就是$ x^\mu $坐标系的一个基矢量)。但是其实我们并没有真正多出一个基。因为我们写$ \nabla_\mu $的意思其实是在说$ \nabla(\partial_{\mu}, * ) $所以,其实相当于放了一个协变的向量进去。

但是根据我们的规则:$ \nabla(V , *) = v^\mu \nabla(\partial_\mu,*) $所以其实这个协变导数的形式在不同基下面是不一样的,可以形式化的书写这个观点下的协变导数在不同基下面的变换:
\begin{align}
  \nabla_a = \nabla(\partial_a, *) = \nabla(\displaystyle\frac{\partial x^\mu}{\partial y^a}\partial_\mu, *) = \displaystyle\frac{\partial x^\mu}{\partial y^a}\nabla_\mu
\end{align}
这样的观点下面我们有的时候仿佛多了一个指标但实际上没有。给出的张量还是一个(k,l)张量。

\rmk{这个观点下我们也经常写作基矢量分量形式展开,这是因为我们基矢量的covariant derivative是可以计算的。我们可以给出$ \partial + \Gamma $的形式。}

\bigskip
\textbf{协变导数观点2: }把一个(k,l)张量映射成一个(k,l+1)张量。
\bigskip


这种观点下面我们协变导数被定义为:$ \nabla : F_M(k,l) \to F_M(k,l+1) $。在这样的观点下面。我们可以形式化的写出被一个对应于$ \partial_{\mu} $基矢量,的协变导数的分量:
\begin{align}
  (\nabla_\mu T)^{a_1,...,a_k}_{b_1,...,b_l}
\end{align}
有的时候我们会有下面的这个记号,并且这个分量数值可以很直接的被计算:
\begin{align}
  \nabla_{\mu} V^\nu \equiv (\nabla_\mu V)^\nu = \partial_\mu V^\nu + \Gamma^\nu_{\mu\lambda}V^\lambda
\end{align}
我们根据之前的讨论,协变导数的指标会在不同基下面展开满足张量的变换。所以这个分量其实完全可以理解为一个(k,l+1)张量的分量。所以我们说协变导数形式上确实是一个(k,l+1)张量。

\rmk{

由于协变导数是【会作用在基上的】,所以我们单独写协变导数的时候数学形式上很不好像张量这样的,写出来这个东西$ \nabla_\mu dx^\mu $这个太让人误解了。虽然我们协变导数本身很难写出这样的东西,但是【作用在张量之后的协变导数】就是一个合法的(k,l+1)张量,所以我们可以这样写:
\begin{align}
  (\nabla_\mu v^\nu) dx^\mu \otimes \partial_\nu
\end{align}
这样再来进行运算就很舒服了!并且这个分量的数值是根据第一种观点讨论我们已经知道的,可以计算的!!
}

\bigskip
\question{协变导数的计算怎么用纯粹分量形式进行,而且不出错??}

首先我们讨论一个记号混淆的问题。特别是我们简化写作分量形式之后。

\imp{协变导数记号混淆}{
  我们常常会碰上这样的记号 $ \nabla_\nu v^\mu $这个记号似乎有两种理解:
  \begin{enumerate}
    \item 把$ v^\mu $作为一个标量场来看待,然后协变导数作用在这个标量场上面,得到一个标量场。其实就是$ \partial_\nu v^\mu $
    \item 把$ v^\mu $作为一个向量场来看待,然后协变导数作用在这个向量场上面,得到一个(1,1)张量。其实就是$ \partial_\nu v^\mu + \Gamma^\mu_{\nu\lambda} v^\lambda $
  \end{enumerate}
  注意【第一种理解仅仅在把基和分量张量积起来的视角下考虑】。如果我们没有explicitly写出来基矢量。也就是【分量计算】我们永远使用第二种理解
}
这个讨论也就涉及一个很重要的问题,我们在计算协变导数的时候,协变导数是需要作用在基矢量上面的!!但是我们写作分量形式进行计算的话其实我们没有explicitly写出基矢量。我们应该怎么办:

\imp{分量形式的协变导数计算}{

使用我们的$ \nabla_\mu v^\nu $的记号,然后形式化书写
\begin{itemize}
  \item 使用lebniz rule形式化的每一个部分(务必加入Chris Symbol!!)的协变导数分量$ \nabla_\mu v^\nu $
  \item 最后进行contraction
\end{itemize}
下面就是一个使用分量形式化书写的例子:
\begin{align}
  \begin{aligned}\nabla_\mu\left(\omega_\lambda v^\lambda\right)&\begin{array}{rcl}=&\left(\nabla_\mu\omega_\lambda\right)v^\lambda+\omega_\lambda\left(\nabla_\mu v^\lambda\right)\end{array}\\&=\quad(\partial_\mu\omega_\lambda)v^\lambda+\tilde{\Gamma}_{\mu\lambda}^\sigma\omega_\sigma v^\lambda+\omega_\lambda\left(\partial_\mu v^\lambda\right)+\omega_\lambda\Gamma_{\mu\rho}^\lambda v^\rho\end{aligned}
\end{align}
这样分量计算的结果一定是对的。但是是形式化的,我们真正在干的事情是补回来基矢量,然后先作用张量积的协变导数lebniz rule再进行contraction!!
}

\bigskip
\question{对于metric的计算物理上怎么通过$ ds^2 $形式化的进行计算?}

这个源自于习题之中的计算。我们已经知道一个metric写作:
\begin{align}
  ds^2 = g_{\mu\nu} dx^\mu dx^\nu 
\end{align}
然后我们希望形式化的做一个坐标变换:$ x^\mu(y^a) $。我们怎么快速的得到这样的变换的metric呢?所以我们需要形式化的写:$ dx^{\mu 2} $和$ dy^{a 2} $的关系我们使用下面的式子:
\begin{align}
  dy^2=(adx+dz)^2=a^2dx^2+2adxdz+dz^2
\end{align}
这个其实就是把两个基矢量变换放在了一起【务必这么数学上理解】但是形式上真的很超然。我把这个结果整理写到上一章的question and Thoughts里面了,请看\cref{dis:metrictrans}里面的讨论
\qed


 
% section Lecture 3 (end)

\newpage
\section{Lecture 4: Kraus Representation of Quantum Channel}\label{sec:Lecture 4} % (fold)
\subsection{Take home messages}

\hlr{Turning Point的复分析处理方法}

我们之前使用Airy Function处理了turning point附近的波函数连接问题。但是,这个方法比较复杂。下面介绍如何使用解析延拓的方法处理。

\textbf{Step 1:} 先考虑经典不允许区域的波函数,我们设置为:
\begin{align}
  \psi_{CF}=\frac{C}{2\sqrt{|p|}}\text{exp}{-\frac{1}{\hbar}\int_{x_0}^xdx^{\prime}|p|},
\end{align}
\bigskip

\textbf{Step 2:} 线性势能近似。一般的势能很难进行研究,但是如果我们认为turning point 附近是线性的,那么会很大程度简化。我们选择:
\begin{align}
  V(x)=E+\frac{dV}{dx}|_{x_0}(x-x_0)+\mathscr{O}((x-x_0)^2)\approx E+F(x-x_0).
\end{align}
所以对于动量我们有:
\begin{enumerate}
  \item 对于经典允许区域$ x < x_0 $,我们有$ p(x) = \sqrt{2m(E-V(x))} = \sqrt{2mF(x_0-x)} $;
  \item 对于经典不允许区域$ x > x_0 $,我们有$ p(x) = \sqrt{2m(V(x)-E)} = \sqrt{2mF(x-x_0)} $。
\end{enumerate}

\bigskip
\textbf{Step 3:} 我们研究这个函数的解析情况,发现除了$ x = x_0 $的时候$ \sqrt{p}  = 0$存在一个不能消除的奇点之外,似乎解析延拓到复平面是没有问题的。我们进行一个替换$ x-x_0 \to z = \rho e^{i \phi} $。于是在线性近似前提下解析延拓后的波函数是:
 
\begin{align}
  \tilde{\boldsymbol{\psi}}_{CF}(z)=\frac{C}{2}(2mF)^{-1/4}z^{-1/4}e^{-\frac{2}{3\hbar}(2mF)^{1/2}z^{3/2}}.
\end{align}

\bigskip
\textbf{Step 4:} 奇点的存在意味着我们需要进行割线选取。我们发现有两个选择【自然对应着经典允许区域解空间两个正交基】,一个是$ \phi = 0 $到$ \phi = \pi $,另一个是$ \phi = 0 $到$ \phi = -\pi $。

我们选择前者(正虚轴上的割线)作为割线。于是我们可以得到:
\begin{align}
  \tilde{\psi}_{CF}(\rho e^{-i\pi})=\frac{C}{2(2 m F)^{1/4}}\frac{1}{\rho^{1/4}e^{-i\pi/4}}\exp\left[-\frac{2}{3\hbar}(2 m F)^{1/2}\rho^{3/2}e^{-3i\pi/2}\right]
\end{align}
对于后者(负虚轴上的割线)
\begin{align}
  \tilde{\psi}_{CF}(\rho e^{i\pi})=\frac{C}{2(2mF)^{1/4}}\frac{1}{\rho^{1/4}e^{i\pi/4}}\exp\left[-\frac{2}{3\hbar}(2mF)^{1/2}\rho^{3/2}e^{3i\pi/2}\right]
\end{align}

\bigskip
\textbf{Step 5:} 计算经典允许区域的下面两个数值,并使用$ \rho $进行代换:
\begin{align}
  \int^{x_0}_x p(x')dx'=\int^{x_0}_x\sqrt{2m(E-V(x'))}dx' \quad \sqrt{p(x)}
\end{align}
\hlb{我们注意!!经典允许区域的动量积分什么的是没有绝对值的!!!}
一个更需要额外注意的事情是,在经典不允许区域 $ \rho = x-x_0 $但是在经典允许区域$ \rho = -(x-x_0) $!!!这之间转了一个180度的角度!!!由于注意的地方比较subtle我们单独写开:
\begin{align}
  \int_{x}^{x_0} (2mF)^{1/2} (x_0 - x)^{1/2} = - (2mF)^{1/2} 2/3 (x_0 - x)^{3/2}\bigg|_{x}^{x_0} = \frac{2}{3}(2mF)^{1/2}\rho^{3/2} 
\end{align}
这里最关键的一点是我们使用的是对于$ (2mF)^{1/2}(x_0 - x)^{1/2} $这个实数进行积分!!我们千万不要强行使用$ i (2mF)^{1/2}(x- x_0)^{1/2} $进行积分,这样子积分很可能就掉进某一个多值分支里面了!!!!【面对多值函数,不要随便使用复数,会死的很惨!】

\bigskip
\textbf{Step 6:} 用$ \int p, \sqrt{p} $代换$ \rho $给出经典允许区域波函数:
\begin{align}
 \psi_1(x) = \frac{C}{2\sqrt{p}}\exp\left[+\frac{i}{\hbar}\int_x^{x_0}pdx-\frac{i\pi}{4}\right],\\ 
  \psi_2(x) =\frac{C}{2\sqrt{p}}\exp\left[-\frac{i}{\hbar}\int_x^{x_0}pdx+\frac{i\pi}{4}\right].
\end{align}
我们对比与经典区域的标准波函数的系数我们会发现系数满足关系:
\begin{align}
  C_1=\frac{C}{2}e^{i\pi/4},\quad C_2=\frac{C}{2}e^{-i\pi/4}.
\end{align}
同时波函数也可以写作:
\begin{align}
  \psi_{CA}=\frac{C}{\sqrt{p}}\cos\left(\frac{1}{\hbar}\int_x^{x_0}pdx-\frac{\pi}{4}\right).
\end{align}

\imp{Comment on 非经典区域约束经典区域}{
  我们普通的量子力学解波函数的系数很多的是依靠边界条件的。但是我们这里是研究非经典区域对于经典区域的联通。其实这个也是一种边界条件的体现,我们选择的边界条件是在非经典区域波函数衰减到0.然后我们通过解析延拓把这个边界条件传递到经典区域。
}


\bigskip
\hlr{复分析处理法的适用情况}

显然复分析方法用了很多的近似,所以我们要考虑每一点的近似条件是什么:
\begin{enumerate}
  \item 线性近似条件:也就是二阶展开要远小于一阶:
    \begin{align}
      \left|\frac{1}{2}V^{\prime\prime}(x_0)(x-x_0)^2\right|<< \left|V^{\prime}(x_0)(x-x_0)\right|
    \end{align}
    
    \item WKB近似条件:也就是动量变化要远小于动量本身:
      \begin{align}
        |\lambda'| <<1 \quad \to \quad \left|\frac{\hbar}{\sqrt{2m|V^{\prime}(x_0)|}}\frac{1}{2}\frac{1}{(x-x_0)^{3/2}}\right|<<1\mathrm{~.} 
      \end{align}
\end{enumerate}
综合上面两个条件我们有:
\begin{align}
  |V^{\prime}(x_0)|^2\gg\frac{\hbar}{\sqrt{m}}|V^{\prime\prime}(x_0)|^{3/2}.
\end{align}

\bigskip
\hlr{半经典归一化条件}

我们使用半经典的视角看看归一化条件,我们下面有很多个近似。首先,我们认为波函数只在经典允许区域有贡献,我们忽略不允许区域的波函数:
\begin{align}
  1\approx\int|\psi_{\mathrm{CA}}|^2dx=\int_{x_1}^{x_2}dx\frac{C^2}{p}\cos^2\left(\frac{1}{\hbar}\int_{x_1}^x|p|dx-\phi\right)
\end{align}
然后我们考虑半经典的情况,也就是考虑$ \hbar << 1 $的时候,震荡特别快,所以我们可以把$ \cos^2 $的平均值取为$ 1/2 $,于是我们有:
\begin{align}
  \approx\frac{1}{2}C^2\int_{x_1}^{x_2}\frac{dx}{|p|}=\frac{1}{2}C^2\int_{x_1}^{x_2}\frac{dx}{mdx/dt}=\frac{1}{2}C^2\frac{1}{m}\frac{T}{2}\mathrm{~.}
\end{align}
其中也使用了经典的关系$ p = dx / dt $。并且其中$ T $来自对于时间的积分,也就是周期:
\begin{align}
  C=2\sqrt{\frac{m}{T}}\mathrm{~.}
\end{align}
最后给出上方的归一化系数,其中$ T $是经典周期。


\bigskip
\hlr{Bohr-Sommerfield能级近似}

对于WKB近似,如果我们考虑粒子处于一个束缚态,我们有两个turning point $ x_1,x_2 $。我们考虑经典允许区域的波函数,那么我们match系数有通过左边边界条件和右边边界条件两种。

但是由于是描述同一束缚态区域,所以我们存在一个约束条件:
\begin{align}
  \cos\left[\frac{1}{\hbar}\int_x^{x_2}pdx^{\prime}-\frac{\pi}{4}\right]=\pm\cos\left[\frac{1}{\hbar}\int_{x_1}^xpdx^{\prime}-\frac{\pi}{4}\right]
\end{align}
这里存在$ \pm $是因为我们不知道是选哪个割线的情况互相match。我们简化上面的式子给出条件:
\begin{align}
  \frac{1}{\hbar}\int_{x_1}^xpdx^{\prime}-\frac{\pi}{4}=\frac{1}{\hbar}\int_x^{x_2}pdx^{\prime}-\frac{\pi}{4}+\pi n\mathrm{~,~or}\quad \frac{1}{\hbar}\int_{x_1}^xpdx^{\prime}-\frac{\pi}{4}=-\left(\frac{1}{\hbar}\int_x^{x_2}pdx^{\prime}-\frac{\pi}{4}\right)+\pi n.
\end{align}
对于这两个条件其实只有第二个条件是可以成立的!【第一个条件不对所有x成立】。所以最终可以推出:
\begin{align}
  \frac{1}{\hbar}\int_{x_1}^{x_2}pdx=\pi\left(n+\frac{1}{2}\right).
\end{align}
这个条件就是Bohr-Sommerfield能级量化条件。


\subsection{Questions and thoughts}

\question{为什么解析延拓结果是不唯一的,并且我需要叠加才能给出物理结果?}

割线选取不同解析延拓结果必然不同。但是这里我们发现正好correspond tos经典允许区域的两个正交基。所以我们需要把两个结果叠加才能给出物理结果。

这是一个纯属巧合的物理过程。但是我们不妨这么按照直觉这么算。
\qed




\bigskip
\question{
  我们解析延拓的时候是不是转了一圈之后我们需要令$ \rho = -(x-x_0) $这样子才成了?但是因为我们取了模长,所以所有的符号都没了?
}

不对!我们计算的是classically alowed区域的波函数,所以对于动量应该必然是大于0的,所以我们并没有取模长!
\qed

\bigskip
\question{习题在计算的时候会遇见一些特别奇怪的问题!因为复数方法涉及多值函数,所以可能会计算出特别诡异的结果,我们下面来讨论一下!!!}

我们如果无脑进行计算的话!!!可能会掉进一些多值函数的坑里面。比如我们计算:
\begin{align}
  (1)^{1/2} = ((-1)^{2k})^{1/2} = (-1)^k  
\end{align}
这样子会!!凭空产生相位!!!我们该怎么避免这种不小心掉进各种奇奇怪怪的多值分支之中的情况呢?我们还要保证【永远用实数进行计算!!!】

\hlr{这个comment极其的重要!!!务必参考作业一同食用!!!效果更佳!!!}




% section Lecture 4 (end)

\newpage 
\section{Lecture 5: Stinespring representation and Choi representation}\label{sec:Lecture 5} % (fold)
\subsection{Lie Group Representations}

\subsubsection{一般李代数,李群表示的构造}

\hlr{从代数表示构建群表示}

为了之后的代数和群的讨论做铺垫,我们发现,如果给定一个代数的矩阵表示之后,我们可以通过一个操作得到整个群的表示。

\defi{
  Exponential Map 

  给定一个Lie代数的矩阵表示$ T^i $,我们可以通过如下的操作得到Lie群的矩阵表示:
  \begin{align}
    D[g(\alpha)]=\exp\left(i\alpha^iT^i\right).
  \end{align}
  其中$ \alpha_i $是一些实参数,Lie Parameters。
}

\bigskip
\hlr{群表示找到代数}


同样的一波观察,根据第一个Lie Theorem的操作,我们发现我们对于任何李群表示我们在$ e $附近进行展开有:
\begin{align}
  D(\alpha)=1+i\alpha^iX_i
\end{align}
其中$ X_i $是某些矩阵构成了这个李群的一个李代数的表示。【这个结论可以在Lie Theorem 1-2之中看出来】我们也可以通过群的乘法法则得到这些矩阵的对易关系,从而得到这个李代数的结构常数。这个显然是一个李代数的表示,并且正好就是群对应的


\bigskip
\hlr{任意群表示都可以写成Exponential Map的形式}

所以,对于任意的表示,我们可以把它写成某个李代数的表示Exponential Map的形式。上面的代数表示和群表示的关系告诉我们了一个操作,分为两个步骤:
\begin{enumerate}
  \item 先把某个群表示在单位元附近展开,得到代数表示$ \tilde{X}_i $。
  \item 通过代数表示exp map重构群表示。
\end{enumerate}

\thm{
  Convert to Exp Map

  对于任意Lie群的矩阵表示$ D[g(\alpha)] $,都可以通过某种方式写成Exponential Map的形式。下面具体构造就是:
  \begin{align}
    D(\alpha)\equiv\lim_{n\to\infty}\tilde{D}(\alpha/n)^n=\lim_{n\to\infty}\left(1+i\frac{\alpha_i}{n}\tilde{X}_i+\mathcal{O}(1/n^2)\right)^n\equiv e^{i\alpha_i\tilde{X}_i}
  \end{align}
}

\bigskip
\hlr{一般表示空间李群的表示}

\imp{
一般表示空间李群的表示
}{

我们在研究物理的时候一般希望能够构建一般表示空间上李群的表示。下面给出一个general的方法:
\begin{enumerate}
  \item 从李群的defining representation出发,通过在$ e $附近的展开,给出defining representation对应的李代数表示$ X_i $。
    \begin{align}
      D(\alpha)=1+i\alpha^iX_i
    \end{align}
  \item 通过李代数的表示$ X_i $,给出李代数的结构。并且使用李代数的表示理论给出李代数不同维度,在各种表示空间上的表示。
  \item 通过exp map的方法,给出这些表示空间上李群的表示:
    \begin{align}
      D[g(\alpha)]=e^{i\alpha^i\tilde{X}_i}
    \end{align}
\end{enumerate}
}


\subsubsection{Adjoint Representation}


\bigskip
\hlr{Adjoint Representation}

李代数的空间也是一个线性空间。我们发现这个空间可以很自然的作为一个表示空间构造一个群或者代数的表示,称为Adjoint Representation。

给定一个Lie群$ G $,它的李代数是$ \mathfrak{g} $。我们考虑李代数空间之中的一组基础$ X_i $,任意李代数的元素都可以写成$ X=\alpha^iX_i $的形式。现在我们考虑群作用在李代数空间之中的效果:
\begin{align}
  &D(\alpha)=e^{i\alpha^iX_i}\\
  &v\mapsto e^{i\alpha^jX_j}ve^{-i\alpha^jX_j}
\end{align}
发现这个操作给出了一个新的李代数元素。这个映射相当于在李代数的线性空间内给出了一个李群的表示。

\defi{
  Adjoint Representation

  给定一个Lie群$ G $,它的李代数是$ \mathfrak{g} $。我们考虑李代数空间之中的一组基础$ X_i $,任意李代数的元素都可以写成$ v= v^iX_i $的形式。定义Adjoint Representation为:
  \begin{align}
    D_{Adj}(g(\alpha)):\left.v\mapsto D(g(\alpha))\right.v\left.D(g(\alpha))^{-1}\right.
  \end{align}
  其中$ D(g(\alpha))=e^{i\alpha^iX_i} $是群的表示。
}

\bigskip
\hlr{李代数的Adjoint Representation}

我们考虑一个$ \alpha_i \to 0 $的情况下的群元素作用在李代数空间的元素$ v $上面的效果。我们有:
\begin{align}
  \begin{aligned}(1+i\alpha^jX_j)v(1-i\alpha^jX_j)&=v+i\alpha^j[X_j,v]\\&=v+i\alpha^jv^k[X_j,X_k]\\&=v^iX_i+i\alpha^jv^kif_{jk}^lX_l\\&=(v^i-\alpha^jv^kf_{jk}^i)X_i\\&\equiv v^{\prime i}X_i\end{aligned}
\end{align}
于是我们发现:
\begin{align}
  v^i\mapsto v^i-\alpha^jv^kf_{jk}^i=(\delta_k^i-\alpha^jf_{jk}^i)v^k
\end{align}
使用之前李群的表示给出李代数的表示的方法,我们知道李代数的adjoint表示是:
\begin{align}
  (\tilde{X}_j)_k^i=if_{jk}^i,
\end{align}
其中$ f_{jk}^i $是李代数的结构常数。回顾定义是:
\begin{align}
[X_i,X_j]=if_{ij}^kX_k,  
\end{align}

\bigskip
\hlr{Adjoint Representation的exp形式}

我们之前知道任意表示我们可以写成exp map的形式。那么Adjoint Representation自然也可以写成exp map的形式,我们可以自然的exp:
\begin{align}
  v^i\mapsto D_{Adj}(g(\alpha))_j^iv^j\equiv(e^{i\alpha^k\tilde{X}_k})_j^iv^j.
\end{align}

\bigskip
\hlr{Adjoint Representation作用在李代数的基}

下面考虑Adjoint Representation作用在李代数的基$ X_i $上面的效果:
\begin{align}
  g(\alpha):X_i\mapsto e^{i\alpha^jX_j}X_ie^{-i\alpha^jX_j}=X_iD_{Adj}(g(\alpha))_j^i
\end{align}
对于无限小变换,左边的基本定义下是:
\begin{align}
  X_i\mapsto X_i+i\alpha^j[X_j,X_i]+\ldots. 
\end{align}
然后根据之前讨论的exp map形式的Adjoint Representation,我们发现:
\begin{align}
  [X_j,X_i]=if_{ji}^kX_k
\end{align}
也就是李代数的结构常数的定义,我们发现这些都是consistent的。


\bigskip
\hlr{李群作用在李代数上}

我们在物理上经常使用到李群作用在自己的李代数上面的情况。Adjoint Representation就给了一个自然的数学语言来描述这个过程。我们发现李群作用在李代数上面可以描述为:
\begin{align}
  v\mapsto D(g(\alpha))v\left.D(g(\alpha))^{-1}\right.
\end{align}
然后考虑无限小的情况下,对于李代数的基$ X_i $,我们有:
\begin{align}
  X_i\mapsto X_i+i\alpha^j[X_j,X_i]+...
\end{align}
我们认为commutator可以描述李代数作用在其自己上面:
\begin{align}
  [X_j,X_i]=if_{ji}^kX_k
\end{align}

\subsection{Lorentz and Poincare Group}

\hlr{Poincare Group的定义}

Poincare Group是描述Lorentz转动+平移的群。定义为
\defi{Poincare Group(ISO(3,1))

  Poincare Group定义为满足下面的关系的流形上的坐标变换构成的群:
  \begin{align}
    x^\mu\mapsto x^{\prime\mu}=\Lambda^\mu{}_\nu x^\nu+a^\mu,\quad\eta^{\mu\nu}\Lambda^\rho{}_\mu\Lambda^\sigma{}_\nu=\eta^{\rho\sigma}
  \end{align}
}
这个群的参数一共有10个,自由度。4个来自平移部分$ a^\mu $,6个来自Lorentz转动部分$ \Lambda^\mu{}_\nu $。

\bigskip
\hlr{Lorentz Group的分类}

Lorentz Group是Poincare Group的一个子群,我们记作O(3,1),包含了所有不包含平移的坐标变换。我们下面发现Lorantz Group存在分区。
\begin{itemize}
  \item \textbf{行列式分类}:$ \det(\Lambda)=\pm1 $
\end{itemize}
根据Lorentz Group的定义,我们发现行列式只能是1或者-1。于是我们可以把Lorentz Group分成两个部分,分别是$ \det(\Lambda)=1 $的部分和$ \det(\Lambda)=-1 $的部分。
\begin{enumerate}
  \item $ \det(\Lambda)=1 $的部分称为\textbf{Proper Lorentz Group},记为$ SO(3,1) $。
  \item $ \det(\Lambda)=-1 $的部分称为\textbf{Improper element}。这部分不含identity,所以并不构成一个子群。
\end{enumerate}

\begin{itemize}
  \item \textbf{时间方向分类}:$ \Lambda^0{}_0\geq1 $或者$ \Lambda^0{}_0\leq-1 $
\end{itemize}

我们根据$ \eta_{00} = 1 $以及定义可以推导:
\begin{align}
  1=\left(\Lambda^0{}_0\right)^2-\sum_i\left(\Lambda^i{}_0\right)^2\Rightarrow\left(\Lambda^0{}_0\right)^2\geq1
\end{align}
于是我们可以根据$ \Lambda^0{}_0 $的符号把Lorentz Group再分成两个部分:
\begin{enumerate}
  \item $ \Lambda^0{}_0\geq1 $的部分称为\textbf{orthochronous Lorentz Group},记为$ SO(3,1)^\uparrow $。
  \item $ \Lambda^0{}_0\leq-1 $的部分称为\textbf{non-orthochronous 部分}。
\end{enumerate}

概括上面的分类我们有:

\begin{figure}[H]
  \centering
  \includegraphics[width=0.6\textwidth]{assets/lorentzgroup.png}
  \caption{Lorentz Group的分类}
  \label{fig:lorentzgroup}
\end{figure}

我们发现洛伦兹群有三个子群,其中最小的是:
\begin{align}
  \mathscr{L}_+^\uparrow\equiv SO^\uparrow(1,3)
\end{align}
也就是proper orthochronous Lorentz group。同时包含这个群的还有两个子群:
\begin{align}
  \mathscr{L}_+^\uparrow\oplus\mathscr{L}_+^\downarrow=SO(1,3)\mathrm{~and~}\mathscr{L}_+^\uparrow\oplus\mathscr{L}_-^\uparrow=O^\uparrow(1,3)
\end{align}


\bigskip
\hlr{Parity and Time Reversal}

Lorentz群存在这么多个分支的一个原因是因为其中存在离散的变换。而分成四个部分就是因为两个离散变换以及其组合。
\begin{align}
  P:(t,\vec{x})\mapsto(t,-\vec{x})\in\mathscr{L}_-^\uparrow\\ 
T:(t,\vec{x})\mapsto(-t,\vec{x})\in\mathscr{L}_-^\downarrow\\
 PT:(t,\vec{x})\mapsto(-t,-\vec{x})\in\mathscr{L}_+^\downarrow
\end{align}
我们最后会发现,其实Lorantz Group就是Proper orthochronous Lorentz Group和这两个离散变换生成的群。
\begin{align}
  O(1,3)=\mathscr{L}_+^\uparrow\circ\{1,P,T,PT\}
\end{align}



\subsection{Questions and thoughts}

\question{作业之中O(N)和SO(N)的维度是一样的,是怎么导致群不一样的?}

给出这两个群的定义:
\begin{align}
  &O(N)\equiv\left\{R\in GL(N,\mathbb{R}){\large|}RR^T=R^TR=1_N\right\}.\\ 
  &SO(N)\equiv\left\{R\in GL(N,\mathbb{R})|RR^T=R^TR=1_N,\det(R)=1\right\}.
\end{align}
我们使用物理学家常用的exp map的形式:
\begin{align}
  D[g(\alpha)]=1+i\alpha^iT^i+O(\alpha^2). 
\end{align}
然后会发现这两个群对应的代数都是so(N),也就是$ N\times N $维度的全反对称纯虚数矩阵空间。但是两者的群结构是不一样的。这是因为O(N)包含了SO(N)的一个离散部分,也就是行列式为-1的那些矩阵。这个离散部分无法通过连续变化从单位元到达,因此O(N)和SO(N)是不同的群,尽管它们的李代数是相同的。

离散部分的生成元是这样的三个很像是Parity变换的矩阵:
\begin{align}
  \left.P_x=\left[\begin{array}{ccc}-1&0&0\\0&1&0\\0&0&1\end{array}\right.\right],\quad P_y=\begin{bmatrix}1&0&0\\0&-1&0\\0&0&1\end{bmatrix},\quad P_z=\begin{bmatrix}1&0&0\\0&1&0\\0&0&-1\end{bmatrix}.
\end{align}
由于他们是离散的,不可以被切空间和exp map这样的构造包含进去,所以O(N)和SO(N)的群结构是不一样的,但是李代数是可以一样的。
\qed

% section Lecture 5 (end)

\newpage 
\section{Lecture 6: Choi representation and Complete Positivity}\label{sec:Lecture 5} % (fold)
\subsection{Choi Representation继续}


我们之前研究知道通过Vectorization我们可以将一组Kraus Operators映射为一个Choi State。下面我们希望反过来通过Choi State来得到Channel的Kraus Operators。

\bigskip
\hlr{例子:一个特殊channel的choi state}

首先我们计算一个特别的channel的Choi State。我们考虑下面的channel:
\begin{align}
  \mathcal{E}(\rho)=\frac{1}{3}\left(\mathrm{Tr}(\rho)I+\rho^T\right).
\end{align}
我们计算这个channel的Choi State,也就是对于$ \ket{Vec(I)} $进行$ \mathcal{E}(\rho) $的作用我们计算的时候会把前一个和后一个qubit进行区分开来计算:
\begin{align}
  J(\mathcal{E})=\sum_{i,j}\mathcal{E}(|i\rangle\langle j|)\otimes|i\rangle\langle j|
\end{align}
于是我们给出结果是:
\begin{align}
  J(\mathcal{E}) = =\frac{1}{3}\left(\sum_iI\otimes|i\rangle\langle i|+\sum_{i,j}\lvert j\rangle\langle i|\otimes|i\rangle\langle j|\right)\\
=\frac{1}{3}\left(\sum_iI\otimes|i\rangle\langle i|+\sum_{i,j}\lvert j\rangle\langle i|\otimes|i\rangle\langle j|\right)
\end{align}
我们会发现这个Choi State的第二个部分是一个SWAP Operator。我们下面定义:
\begin{itemize}
  \item SWAP Operator:
    \begin{align}
      S=\sum_{i,j}|i\rangle\langle j|\otimes|j\rangle\langle i|
    \end{align}
    \item SWAP Operator的本征态,就是所有的Bell State并且除了最后一个本征值都是1,我们可以decompose成为下面的形式:
      \begin{align}
        SWAP = | \phi^+ \rangle \langle \phi^+ | + | \phi^- \rangle \langle \phi^- | + | \psi^+ \rangle \langle \psi^+ | - | \psi^- \rangle \langle \psi^- |
      \end{align}
    \item SWAP Operator相当于对于一个2qubit系统进行「转置」也就是:
\begin{align}
  SWAP|ij\rangle=|ji\rangle
\end{align}
\end{itemize}
根据这个信息我们发现Choi State可以写作:
\begin{align}
  J(\mathcal{E})=&\frac{1}{3}\left(I\otimes I + SWAP\right)\\ 
   = &\frac{2}{3}(|\phi^+\rangle\langle\phi^+|+ |\phi^-\rangle\langle\phi^-| + |\psi^+\rangle\langle\psi^+|)
\end{align}
于是很显然我们发现这个Choi State可以写作三个vectorization的矩阵的形式:
\begin{align}
  \begin{aligned}A_0&=\frac{|0\rangle\langle0|+|1\rangle\langle1|}{\sqrt{3}}=\frac{1}{\sqrt{3}}I,\\
    A_1&=\frac{|0\rangle\langle0|-|1\rangle\langle1|}{\sqrt{3}}=\frac{1}{\sqrt{3}}Z,\\
  A_2&=\frac{|0\rangle\langle1|+|1\rangle\langle0|}{\sqrt{3}}=\frac{1}{\sqrt{3}}X.\end{aligned}
\end{align}
并且我们可以验证这些Kraus Operators满足完备关系:
\begin{align}
  \sum_iA_i^\dagger A_i=\frac{1}{3}{\left(I^2+Z^2+X^2\right)}=\frac{1}{3}(3I)=I.
\end{align}
所以我们成功地通过Choi State得到了Kraus Operators。

\bigskip
\hlr{不唯一的Kraus Operators表示}

但是另一个观察会发现这个Kraus Operators的表示并不是唯一的。我们可以进行一个unitary变换,可以得到:
\begin{align}
J(\mathcal{E})=\frac{2}{3}|00\rangle\langle00|+\frac{2}{3}|11\rangle\langle11|+\frac{2}{3}|\psi^+\rangle\langle\psi^+|.  
\end{align}
我们分别验证这个Choi State对应的Kraus Operators:
\begin{align}
  A_0^{\prime}=\sqrt{\frac{2}{3}}|0\rangle\langle0|,\quad A_1^{\prime}=\sqrt{\frac{2}{3}}|1\rangle\langle1|,\quad A_2^{\prime}=\frac{1}{\sqrt{3}}X. 
\end{align}
发现他们也满足完备关系。

\bigskip
\hlr{通过Choi State得到Kraus Operators的一般方法}

我们,嗯分析上面的方法发现可以一套完备的方法通过Quantum Channel得到Choi State再得到Kraus Operators:
\begin{enumerate}
  \item \textbf{计算量子态的Choi State}: 通过$ J(\mathcal{E}) = (\mathcal{E} \otimes I)(|Vec(I)\rangle\langle Vec(I)|) $计算Choi State。
  \item \textbf{Choi State的谱分解}: 计算Choi State的谱分解$ J(\mathcal{E}) = \sum_i \lambda_i |\psi_i\rangle\langle \psi_i| $。
  \item \textbf{得到Kraus Operators}: 通过$ A_i = \sqrt{\lambda_i} \ket{\psi_i}\ket{\psi_i} $得到Kraus Operators。
  \item \textbf{检查完备关系}: 检查$ \sum_i A_i^\dagger A_i = I $是否成立。
\end{enumerate}

\bigskip
\hlr{Unitary Mixing Freedom}

上面已经发现Kraus Operators的表示并不是唯一的。我们可以通过一个unitary变换得到另外一组Kraus Operators。我们总结一下这个结论:
\thm{Unitary Mixing

对于任意的一个量子态(显然Choi State是一个量子态)的两种不同的谱分解:
\begin{align}
  \rho=&\sum_k\lambda_k|\lambda_k\rangle\langle\lambda_k|.\\ 
  \rho=&\sum_ip_i|\varphi_i\rangle\langle\varphi_i|.
\end{align}
那么这两组本征态之间存在一个unitary变换$ U $使得:
\begin{align}
  \sqrt{p}_i|\varphi_i\rangle=\sum_kU_{ik}\sqrt{\lambda_k}|\lambda_k\rangle.
\end{align}
}
下面我们证明这个结论。首先根据两种谱分解我们给出两种将$ \rho $ purification的方式:
\begin{align}
  |\psi\rangle=\sum_k\sqrt{\lambda_k}|\lambda_k\rangle\otimes|k\rangle,\quad|\varphi\rangle=\sum_i\sqrt{p_i}|\varphi_i\rangle\otimes|i\rangle.
\end{align}
Schmidt Decomposition定理告诉我们这两种purification方式之间存在一个unitary变换$ U $使得:
\begin{align}
  |\varphi\rangle=(I\otimes U)|\psi\rangle.
\end{align}
我们写开会发现:
\begin{align}
  \begin{aligned}\sqrt{p_i}|\varphi_i\rangle&\large=(I\otimes\langle i|)(I\otimes U)|\psi\rangle\\&=\sum_k\sqrt{\lambda_k}(I\otimes\langle i|U)(|\lambda_k\rangle\otimes|k\rangle)\\&=\sum_k\sqrt{\lambda_k}|\lambda_k\rangle\cdot(\langle i|U|k\rangle)\\&=\sum_k\sqrt{\lambda_k}U_{ik}|\lambda_k\rangle.\end{aligned}
\end{align}
\qed 
然后我们会发现上面的general量子态的定理如果使用在Choi State上面就得到了我们下面的结论:
\thm{
  Unitary Mixing Freedom for Kraus Operators

  对于任意的一个Quantum Channel $ \mathcal{E} $的两组不同的Kraus Operators表示$ \{A_i\} $和$ \{B_j\} $,那么这两组Kraus Operators之间存在一个unitary变换$ U $使得:
  \begin{align}
    B_j = \sum_i U_{ji} A_i.
  \end{align}
}
其中就是使用了Vectorization线性的性质,可以把外面的unitary拉进去!


\subsection{Representation of Quantum Channel}

\hlr{Completely Positive}

我们最后回来讨论一下什么样子的映射才是一个Quantum Channel。这里特别强调一下completely positive的概念。
\begin{itemize}
  \item \textbf{Positive Map}: 一个映射$ \mathcal{E} $是positive map如果对于任意的正定矩阵$ \rho\geq0 $都有$ (I \otimes \mathcal{E})(\rho)\geq0 $。
\end{itemize}


\bigskip
\hlr{例子:Positive Map but not Completely Positive Map}
  
一个最基本的例子就是转置映射$ T(\rho)=\rho^T $。我们验证这个映射是positive map但是不是completely positive map。

我们考虑转置映射 $ \otimes $ Identity 映射作用在一个bell state上面:
\begin{align}
  (T\otimes I)(|\phi^+\rangle\langle\phi^+|)
\end{align}
发现会给出SWAP Operator。我们知道SWAP Operator不是正定的所以这个映射不是completely positive map。

\bigskip 
\hlr{Representation Thm}

\YL{[我懒得写了真的(((]}



\subsection{Questions and Thoughts}







% section Lecture 5 (end)


\newpage 
\section{Lecture 7: Vectorization的数学技巧}\label{sec:Lecture 5} % (fold)
\subsection{Symmetry变换的定义}

\subsubsection{Symmetry Transformation for Lagrangian Formalism}

\bigskip
\hlr{对称性的定义「对于场的lagrangian formalism」}

我们之前有定义对称性为一个coordinate transformation,使得EoM不变。下面我们从Lagrangian formalism出发重新定义:

\defi{Global Symmetry Transformation

  首先考虑一个Lie Group给出的Coordinate Transformation。【正如之前claim的,并不一定是坐标变换,这里只是粗浅的理解为坐标变换和场的协变】对于一个Lie Group $ G $,我们选择其coordinate system是$ \{\alpha_i\} $。我们考虑这个群在一个Field上面的表示。我们有:
  \begin{align}
    \left\{\begin{array}{lll}x^{\prime\mu}=f^\mu(x,\alpha)\\\phi_a^{\prime}(x^{\prime})=F_a(\phi(x),\alpha)\end{array}\right.
  \end{align}
 如果这个变化满足下面的方程:
 \begin{align}
  d^4x\left[\mathcal{L}(\phi(x),\partial\phi(x))\right]=d^4x^{\prime}\left[\mathcal{L}(\phi^{\prime}(x^{\prime}),\partial^{\prime}\phi^{\prime}(x^{\prime}))+\partial_\mu^{\prime}K^\mu(\phi^{\prime})\right]
 \end{align}
那么我们称之为Symmetry Transformation。
}
\rmk{
注意我们这个定义的细节是,对于同样的一个Lagrangian的形式,我们输入两种场函数$ \phi(x), \partial \phi(x) $以及 $ \phi'(x), \partial' \phi'(x') $。【注意,不是同点比较!!】但是$ \mathrm{L} $的函数形式是一样的!!
}
我们对比其与最一般的定义的关系,\cref{def:Symmetry}。我们发现:
\begin{itemize}
  \item 这显然就是一个coordinate transformation。我们之前讨论了坐标变换可以通过一个群表示 
  \item 这个变换保证了EoM不变。我们下面证明:
    \begin{enumerate}
      \item 首先注意到作用量需要满足:
        \begin{align}
          S=\int_\Omega d^4x\mathcal{L}(\phi(x),\partial\phi(x))=\int_{f(\Omega,\alpha)}d^4x^{\prime}\mathcal{L}(\phi^{\prime}(x^{\prime}),\partial^{\prime}\phi^{\prime}(x^{\prime}))+\int_{\partial f(\Omega,\alpha)}d\sigma^{\prime\mu}K^\mu(\phi^{\prime}).
        \end{align}
        \item 下面我们对这个作用量分别对于$ \phi_a(x) $以及$ \phi_z^\prime (x^\prime) $进行变分得到结果为:
          \begin{align}
            \delta S=\int_\Omega d^4x \  \Gamma(\phi,\partial)^a\delta\phi_a=\int_{f(\Omega,\alpha)}d^4x^{\prime}\  \Gamma(\phi^{\prime},\partial^{\prime})^a\delta\phi_a^{\prime}
          \end{align}
    \end{enumerate}
    我们会发现两个坐标系下,EoM的函数形式是完全一样的!「虽然输入的场是不一样的」都是$ \Gamma(*) $,其中$ * $代表的就是对应坐标系下输入的场。
\end{itemize}

因此我们证明了这个定义其实就是Symmetry最一般定义 \cref{def:Symmetry} 的一个特例。


\bigskip
\hlr{对称性变换的interpretation}

此外我们还会发现:
\begin{itemize}
  \item 如果$ \phi_a(x) $是一个解的话,那么做一个对称性变换之后$ \phi_a'(x') = F_a(\phi(f^{-1}(x^{\prime},\alpha)),\alpha) $也是一个解。这两个函数形式都是运动方程的解。
\end{itemize}
这个也就意味着
\begin{itemize}
  \item Symmetry意味着存在一组observer,他们描述的系统是完全一样的。
\end{itemize}

\bigskip
\hlr{例子:标量场的平移对称性}

我们考虑一个标量场的lagrangian:
\begin{align}
  \mathcal{L}=\frac{1}{2}(\partial\phi)^2-\frac{1}{2}m^2\phi^2
\end{align}
可以证明下面的变换是一个symmetry transformation:
\begin{align}
  \left\{\begin{array}{lll}x^{\prime\mu}=x^\mu-a^\mu\\\phi^{\prime}(x^{\prime})=\phi(x)\end{array}\right.
\end{align}
这个是一个只有时空坐标变换的对称性变换。

\bigskip
\hlr{例子:复标量场的全局U(1)对称性}

考虑下面这个Lagrangian:
\begin{align}
  \mathcal{L}=\partial_\mu\phi\partial^\mu\phi^*-m^2\phi\phi^*
\end{align}
我们可以证明下面的变换是一个symmetry transformation:
\begin{align}
  \left\{\begin{array}{lll}x^{\prime\mu}=x^\mu\\\phi^{\prime}(x^{\prime})=e^{i\alpha}\phi(x)\end{array}\right.
\end{align}

\bigskip
\hlr{某个函数在Transformation下变换}

我们物理上会很随意的使用一个名词【求解一个量在xxx Transformation之下的变换】,但是从来没说过这个词到底是什么意思。我现在给一个严格说法:
\defi{某个函数在Transformation下变换

  考虑一个coordinate transformation:
  \begin{align}
    \left\{\begin{array}{lll}x^{\prime\mu}=f^\mu(x,\alpha)\\\phi_a^{\prime}(x^{\prime})=F_a(\phi(x),\alpha)\end{array}\right.
  \end{align}
  我们考虑一个函数$ G(\phi(x),\partial\phi(x)) $,那么我们定义这个函数在这个coordinate transformation下的变换为:
  \begin{align}
    G'= G(\phi^{\prime}(x^{\prime}),\partial^{\prime}\phi^{\prime}(x^{\prime}))
  \end{align}
}
也就是说,我们在保证函数形式完全不变的情况下,把场完全替换成变换之后的场$ \phi'(x') $把坐标导数也完全直接替换乘变换之后的坐标导数$ \partial'  $。


\subsubsection{无限小坐标变换以及场构型空间变分}

显然使用坐标变换和场的协变的描述Coordinate Transformation很不方便。我们会发现,如果考虑无穷小坐标变换还有一个更简单的描述的方式。

当我们考虑无限小Coordinate Transformation的时候,我们可以把其当作场构型空间的一个特殊的变分来研究。这样子我们就可以使用熟悉的变分的方法来研究啦!

\bigskip
\hlr{Infinitesimal Coordinate Transformation}

我们考虑一个任意的群generate的一个coordinate transformation,可以写作:
\begin{align}
    \left\{\begin{array}{lll}x^{\prime\mu}=f^\mu(x,\alpha)\\\phi_a^{\prime}(x^{\prime})=F_a(\phi(x),\alpha)\end{array}\right.
\end{align}
我们考虑Lie Parameter无限趋于0的情况,也就是变换无限小的情况,并考虑一阶近似。我们有:
\begin{align}
  x^{\prime\mu}&=x^\mu-\epsilon_i^\mu(x)\alpha^i \equiv x^\mu-\epsilon^\mu(x)\\
\phi_a^{\prime}(x^{\prime})&=\phi_a(x)+\mathcal{E}_{ai}(\phi(x))\alpha^i\equiv\phi_a(x)+\mathcal{E}_a(x)
\end{align}
在这个情况下我们下面进行一个trick,首先对于$ \phi_a'(x') $这个函数在$ x $点进行Taylor展开:
\begin{align}
  \phi_a^{\prime}(x^{\prime})=\phi_a^{\prime}(x)-\epsilon^\mu(x)\partial_\mu\phi_a^{\prime}(x)
\end{align}
然后我们再把场的协变的无限小变换带入进去,消去$ \phi_a'(x') $,我们得到:
\begin{align}
  \phi_a^{\prime}(x)=\phi_a(x)+\left(\mathcal{E}_{ai}(\phi(x))+\epsilon_i^\mu(x)\partial_\mu\phi_a(x)\right)\alpha^i\equiv\phi_a(x)+\Delta_{ai}(\phi(x))\alpha^i\equiv\phi_a(x)+\Delta_a(x)
\end{align}
所以我们的结论是:

\thm{
  无限小坐标变换作为变分

  对于无限小对称性变换,我们可以等价的写作一个场构型的变分:
  \begin{align}
    \phi_a'(x) = \phi_a(x)+\Delta_a(x)
  \end{align}
  并且其中:
  \begin{align}
    \Delta_a(x)= \Delta_{ai}(\phi(x))\alpha^i = \left(\mathcal{E}_{ai}(\phi(x))+\epsilon_i^\mu(x)\partial_\mu\phi_a(x)\right)\alpha^i
  \end{align}
}


\rmk{
  其实等价的,对称性变换从这个视角下,就是:
  \begin{itemize}
    \item 某一个特殊的无限小变分,保证作用量的变分在off-shell情况下是不变的!!
  \end{itemize}
  Weinberg的书之中就是使用这个视角来定义对称性的!!
}

\bigskip
\hlr{例子:标量场的平移变换以及复标量场的全局U(1)变换}

对于标量场的平移变换,我们有:
\begin{align}
  \epsilon^\mu=a^\mu, \quad\mathcal{E}_a=0, \quad \Delta_a=a^\mu\partial_\mu\phi_a 
\end{align}
对于复标量场的全局U(1)变换,我们有:
\begin{align}
  \begin{aligned}
x^{\prime\mu}&=x^\mu&&\Longrightarrow\epsilon^\mu=0\\
\phi^{\prime}&=e^{i\alpha}\phi\approx(1+i\alpha)\phi&&\Longrightarrow\mathcal{E}=i\alpha\phi\\
\phi^{\prime*}&=e^{-i\alpha}\phi^*\approx(1-i\alpha)\phi^{\prime}&&\Longrightarrow\mathcal{E}^*=-i\alpha\phi^*\\
\Delta&=\mathcal{E}
\end{aligned}
\end{align}




\bigskip
\hlr{无穷小对称性变换下Boundary Term}

之前我们一直考虑的是【一般坐标变换】,现在我们开始考虑【对称性变换】。我们考虑对称性变换的定义我们会发现,很显然我们有:
\begin{align}
  \lim_{\alpha\to0}\mathcal{L}(\phi^{\prime}(x^{\prime}),\partial^{\prime}\phi^{\prime}(x^{\prime}))d^4x^{\prime}=\mathcal{L}(\phi(x),\partial\phi(x))d^4x
\end{align}
因此对于多出来的Boundary Term我们有:
\begin{align}
  \lim_{\alpha\to0}\partial_\mu^{\prime}K^\mu=0 
\end{align}
所以我们可以把$ K^\mu(x) $如下进行展开
\begin{align}
  K^\mu=\alpha^iK_i^\mu+\mathcal{O}(\alpha^2)\equiv\tilde{K}^\mu+\mathcal{O}(\alpha^2)
\end{align}




\subsection{Noether's Theorem}

\subsubsection{Noether's Theorem的陈述与证明}

\bigskip
\hlr{Noether's Theorem的陈述}

在上面的讨论基础上我们现在可以证明Noether's Theorem了!!我们给出定理:

\defi{
  Noether's Theorem

  对于一个给定的Lie Group $ G $,我们选择一个coordinate system之后存在N个Lie Parameters $ \{\alpha_i\}_{i=1}^N $。如果这个群对于我们研究的系统是一个Symmetry Transformation,那么对于每一个Lie Parameter $ \alpha_i $,我们都可以定义一个守恒流$ J_i^\mu $,【在onshell的情况下】满足:
  \begin{align}
    \partial_\mu J_i^\mu=0
  \end{align}
  其中守恒量$ J_i^\mu $定义为:
  \begin{align}
    J_i^\mu\equiv\frac{\partial\mathcal{L}}{\partial(\partial_\mu\phi_a)}\Delta_{ai}-\epsilon_i^\mu\mathcal{L}+K_i^\mu 
  \end{align}
  其中$ \Delta_{ai} $是无限小对称性变换下的场的变分提出$ \alpha_i $的系数,$ \epsilon_i^\mu $是无限小对称性变换下的坐标变换提出$ \alpha_i $的系数,$ K_i^\mu $是无穷小对称性变换下的boundary term提出$ \alpha_i $的系数。
}
\rmk{虽然这个定理explicitly给出了守恒流的形式,但是我们一般不会直接使用这个形式来计算守恒流。我们一般会使用后面讨论的一个更简单的方法计算守恒流;所以不必纠结这些奇奇怪怪的项是什么东西。}


\bigskip
\hlr{Noether's Theorem的证明}

下面我们给出证明。首先我们考虑对称性的定义式在无穷小对称性变换下面的形式:
\begin{align}
  d^4x\left[\mathcal{L}(\phi(x),\partial\phi(x))\right]=d^4x^{\prime}\left[\mathcal{L}(\phi^{\prime}(x^{\prime}),\partial^{\prime}\phi^{\prime}(x^{\prime}))+\partial_\mu^{\prime}K^\mu(\phi^{\prime})\right]
\end{align}
我们对上面这个式子右边的部分进行一个无穷小的展开我们。
\begin{itemize}
  \item Volume Element的展开:我们知道体元是按照Jacobi Matrix的行列式进行变换的。所以我们有:
    \begin{align}
      \left|\frac{\partial x^{\prime\mu}}{\partial x^{\nu}}\right|=\det\left(\delta_{\nu}^{\mu}-\partial_{\nu}\epsilon^{\mu}\right)=1-\mathrm{Tr}\left(\partial_{\nu}\epsilon^{\mu}\right)=1-\partial\epsilon
    \end{align} 
    \item Lagrangian的展开,我们直接在$ x $处进行Taylor展开:
      \begin{align}
        \begin{aligned}\mathcal{L}(\phi^{\prime}(x^{\prime}),\partial^{\prime}\phi^{\prime}(x^{\prime}))&=\mathcal{L}(\phi^{\prime}(x),\partial\phi^{\prime}(x))-\epsilon^\mu\partial_\mu\mathcal{L}(\phi^{\prime},\partial\phi^{\prime})\\&=\mathcal{L}(\phi^{\prime}(x),\partial\phi^{\prime}(x))-\epsilon^\mu\partial_\mu\mathcal{L}(\phi,\partial\phi)\end{aligned}
      \end{align}
      其中第二行,因为我们只考虑线性项;由于已经有了$ \epsilon $是对于$ \alpha $一阶的!所以直接把$ \phi' $替换成$ \phi $。

      然后我们进一步展开$  \mathcal{L}(\phi^{\prime}(x),\partial\phi^{\prime}(x))$这一部分,我们发现可以使用变分的视角来进行展开,得到:
      \begin{align}
        \mathcal{L}(\phi^{\prime}(x^{\prime}),\partial^{\prime}\phi^{\prime}(x^{\prime}))=\mathcal{L}(\phi(x)+\Delta(x),\partial(\phi(x)+\Delta(x))-\epsilon^\mu\partial_\mu\mathcal{L}(\phi,\partial\phi)
      \end{align}
    然后RHS的第一项可以展开得到:
    \begin{align}
      \mathcal{L}(\phi(x),\partial\phi(x))+\Delta_a\frac{\partial\mathcal{L}}{\partial\phi_a}+\partial_\mu\Delta_a\frac{\partial\mathcal{L}}{\partial(\partial_\mu\phi_a)}
    \end{align}
\end{itemize}

在上面的两个展开的结果带入原式,我们有:
\begin{align}
  \mathcal{L}(\phi,\partial\phi)d^4x=\left[\mathcal{L}(\phi,\partial\phi)+\Delta_a\frac{\partial\mathcal{L}}{\partial\phi_a}+\partial_\mu\Delta_a\frac{\partial\mathcal{L}}{\partial(\partial_\mu\phi_a)}-\epsilon^\mu\partial_\mu\mathcal{L}+\partial_\mu K^\mu\right](1-\partial_\nu\epsilon^\nu)d^4x
\end{align}
进过化简我们发现:
\begin{align}
  0 =\partial_\mu\left(-\epsilon^\mu\mathcal{L}+\Delta_a\frac{\partial\mathcal{L}}{\partial(\partial_\mu\phi_a)}+\tilde{K}^\mu\right)+\Delta_a\left(\frac{\partial\mathcal{L}}{\partial\phi_a}-\partial_\mu\frac{\partial\mathcal{L}}{\partial(\partial_\mu\phi_a)}\right)
\end{align}
注意到上面的第二项正好是EoM的形式,所以在onshell的情况下,这一项为0。我们就得到了Noether定理的结论。
\begin{align}
  0=\alpha_i\partial_\mu\left[-\epsilon_i^\mu\mathcal{L}+\Delta_{ai}\frac{\partial\mathcal{L}}{\partial(\partial_\mu\phi_a)}+K_\mu^i\right]
\end{align}

\bigskip
\hlr{Noether Charge}

我们发现有了conserve current之后我可以给出一个等时面上的物理量,这个物理量在时间是守恒的。
\defi{Noether Charge

  对于一个守恒流$ J_i^\mu $,我们可以定义一个等时面上的守恒量:
  \begin{align}
    Q_i(t)=\int d^3x J_i^0(\mathbf{x},t)
  \end{align}
  这个量在时间上是守恒的:
  \begin{align}
    \frac{dQ_i}{dt}=0
  \end{align}
}
证明很简单,也就是计算:
\begin{align}
  \frac{dQ_i}{dt}=\int d^3x \partial_0 J_i^0 = -\int d^3x \nabla\cdot J_i =-1
\end{align}

\subsubsection{Noether Current的计算方法}

\bigskip
\hlr{使用Local Transformation计算Noether Current}

虽然Noether定理之中给出了守恒流的形式,但是这个形式其实并不好用。我们一般不会直接使用这个形式来计算守恒流。我们一般会使用下面的一个trick!!我们可以使用一个symmetry lift出来的local transformation来计算Noether Current!!具体步骤如下:

\begin{itemize}
  \item \textbf{Step 1: 计算无限小变分} 

    首先我们需要根据对称性变换计算出无限小变分:
    \begin{align}
      \delta\phi_a=\phi_a'(x)-\phi_a(x)=\Delta_{ai}(\phi(x))\alpha^i
    \end{align}
    \item \textbf{Step 2: Lift to Local Transformation} 

      然后我们把这个无限小变分提升为一个local transformation,也就是如果我们现在认为Lie Parameter是和时空有关的,我们可以把这个transformation lift成为对应的local transformation:
      \begin{align}
        \delta\phi_a(x)&\equiv\phi_a^{\prime}(x)-\phi_a(x)=\Delta_{ai}\alpha^i(x)\\
        \delta\partial_\mu\phi_a(x)&\equiv\partial_\mu(\Delta_{ai}\alpha^i(x))=(\partial_\mu\Delta_{ai})\alpha^i(x)+\Delta_{ai}\partial_\mu\alpha^i(x)
      \end{align}
      其中$ \alpha^i(x) $是任意的时空函数。
    \item \textbf{Step 3: 计算作用量的变分} 

      接下来我们计算这个local transformation下作用量的变分,然后使用on shell的条件会发现我们的变分结果会变得非常简单:
      \begin{align}
        \Delta S\equiv\int d^4x\left[\mathcal{L}(\phi^{\prime},\partial\phi^{\prime})-\mathcal{L}(\phi,\partial\phi)\right]=\int_\Omega J_i^\mu\partial_\mu\alpha^id^4x\mathrm{~+~boundary~term}
      \end{align}
      \item \textbf{Step 4: 读出Noether Current} 

        最后我们就可以直接从上面的式子读出Noether Current了!!
\end{itemize}

\rmk{
  上面的计算之中,我们不一定使用运动方程进行Noether Current的计算。但是有的时候也是需要带入运动方程进行简化计算的。

  但是无论如何,我们可以知道$ J^\mu $是守恒的,是因为运动方程满足的时候$ \delta S = 0 $恒成立。因此我们分部积分得到$ \partial_{\mu} J^\mu = 0 $。

  但是有的时候会发现,没有使用运动方程条件,推导一堆之后 $ \delta S $自动就是0。这个时候就说明没有守恒流存在,也很可能就是一个Gauge Symmetry!!
}

\subsection{Noether Current变分计算技巧}

使用变分计算Noether Current是一个路径依赖的过程。因为,很可能推导着就变成了变分原理的恒等式,也就是带入运动方程之后$ \delta S = 0 $恒成立了,看不出守恒流的情况。所以我们会有一些奇技淫巧来帮助我们计算Noether Current。

\begin{itemize}
  \item \textbf{拼凑Lie Parameter乘以全导数}

    我们一般带入变分后的定义会产生 $ a f[\phi] $这样的项,一个思路就是尽可能的将$ f[\phi] $拼凑成为一个全导数的形式。这样分部积分后就会给出一个对于Noether Current的贡献。

    例子:标量场的时空平移

  \item \textbf{无质量波色子Lagrangian全微分变形}
\end{itemize}





\subsection{Symmetry and Conservation of Poincare}

下面我们讨论一个Poincare协变场的理论。如果Poincare群是这个理论的一个对称性变换构成的群的话有什么结果。

\subsubsection{时空平移对称性}

首先我们考虑时空平移对称性对应的对称荷和守恒量。由于我们知道,对于Poincare协变场,场的协变都是随着Lorentz变换协变的,而在时空平移变换下按照标量场进行变换。

所以我们列出一般Poincare协变场的时空平移变换下的无限小变换:
\begin{align}
  \begin{array}{lcl}x^{\prime\mu}=x^\mu-a^\mu\equiv x^\mu-\epsilon_i^\mu(x)\alpha^i\\
\phi_a^{\prime}(x^{\prime})=\phi_a(x)\equiv\phi_a(x)+\mathcal{E}_{ai}(\phi(x))\alpha^i\end{array}
\end{align}
对于这样的结果我们可以计算出对应的场构型空间的无限小变分:
\begin{align}
    &\epsilon^\mu=a^\mu=a^\nu\delta_\nu^\mu\Rightarrow\epsilon_\nu^\mu\equiv\delta_\nu^\mu,\\
    &\mathcal{E}_a=0\\
    &\Delta_a=a^\nu\partial_\nu\phi_a\Rightarrow\Delta_{a\nu}\equiv\partial_\nu\phi_a
\end{align}
如果我们假定平移变换对于这个Lagrangian系统并不产生boundary term的话,那么我们有$ K_i^\mu=0 $。「平移变换的Jacobi自然是0」。也就是说对称性条件更强了是:
\begin{align}
  \mathcal{L}(\phi(x),\partial\phi(x))=\mathcal{L}(\phi^\prime(x^\prime),\partial^\prime\phi^\prime(x^\prime))
\end{align}
我们可以推导出守恒流,定义为能动量张量!
\defi{能动量张量

对于一个Poincare协变场的Lagrangian系统,如果时空平移变换是其对称性变换,并且没有产生Boundary term的情况下,其对应的守恒流是:
\begin{align}
  T^\mu{}_\nu\equiv\frac{\partial\mathcal{L}}{\partial(\partial_\mu\phi_a)}\partial_\nu\phi_a-\delta_\nu^\mu\mathcal{L}
\end{align}
将其定义为能动量张量。
}
自然的我们也可以给出对应的守恒荷,也就是能动量4-vector:
\begin{align}
  P_\mu\equiv\int T^0{}_\mu d^3\mathbf{x}
\end{align}
\rmk{
  我们上面只是对于一个数学上的守恒流取了一个名字。但是上面的推导之中我们完全不能知道其物理意义。我们是通过研究很多已知的系通发现这个量就是对应的物理上的能动量,才以后interprate它为能动量张量的!!
}

\subsubsection{Lorentz对称性}

\bigskip
\hlr{一般Poincare协变场的Lorentz变换}

对于lorentz变换,我们知道,Poincare协变场在Lorentz变换下不仅仅有时空坐标系的变换还有场的非平凡协变。我们考虑一个Poincare协变场的Lorentz变换:
\begin{align}
  &x^{\prime\mu}={\Lambda^\mu}_\nu x^\nu\\
&\phi_a^{\prime}(x^{\prime})={D(\Lambda)}_a{}^b\phi_b(x)
\end{align}
其中矩阵:
\begin{align}
  D{\left(\Lambda\right)_{a}}^{b}\equiv\left(\exp\left.-\frac{i}{2}\omega_{\mu\nu}\Sigma^{\mu\nu}\right)_{a}^{b}\right.
\end{align}
这里$ \Sigma^{\mu\nu} $是这个Poincare协变场对应的一个lorentz代数的表示。


\bigskip
\hlr{无限小Lorentz变换下的场构型空间变分}

我们继续考虑无限小变换的情况。我们有:  
\begin{align}
  &x^{\prime\mu}=x^\mu+\omega^\mu{}_\nu x^\nu\\
&\phi_a^{\prime}(x^{\prime})=\phi_a(x)-\frac{i}{2}(\omega_{\mu\nu}\Sigma^{\mu\nu})_a{}^b\phi_b(x)
\end{align}
我们可以计算出在无限小Lorentz变换下的场构型空间变分:
\begin{align}
  \begin{aligned}\delta \phi_a(x) = \Delta_{a}&=-\frac{i}{2}(\omega_{\mu\nu}\Sigma^{\mu\nu})_a^b\phi_b(x)-\omega^{\mu\nu}x_\nu\partial_\mu\phi_a(x)\\&=-\frac{i}{2}\omega^{\mu\nu}\left(\Sigma_{\mu\nu}+i(x_\mu\partial_\nu-x_\nu\partial_\mu)\delta\right)_a^b\phi_b(x)\\&\equiv-\frac{i}{2}\omega^{\mu\nu}\left(\Sigma_{\mu\nu}+\mathrm{J}_{\mu\nu}\right)_a^b\phi_b(x).\end{aligned}
\end{align}
这里我们使用了$ J_{\mu\nu}{}_a{}^b $的定义是:
\begin{align}
  [J_{\mu\nu}{}]_a{}^b\equiv i(x_\mu\partial_\nu-x_\nu\partial_\mu)\delta_a^b
\end{align}

\bigskip
\hlr{标量场的Lorentz对称性守恒流}

为了简单,我们现在考虑标量场的情况。对于标量场,$ \Sigma^{\mu\nu} = 0$,也就是Trivial表示。所以我们知道守恒流是:
\begin{align}
  J^\rho{}_{\mu\nu} = (x_\mu T^\rho{}_\nu-x_\nu T^\rho{}_\mu)
\end{align}


\bigskip
\hlr{能动量张量对称性}


我们发现这个守恒流方程外加上能动量张量守恒流方程可以推导出来一个特别的结论:
\begin{align}
  T_{\mu\nu}\equiv\eta_{\mu\rho}T^\rho{}_\nu=\eta_{\nu\rho}T^\rho{}_\mu\equiv T_{\nu\mu}
\end{align}
但很可惜,只有标量场的情况才自然满足这个条件。但是对于一般的Poincare协变场来说,能动量张量并不一定是对称的!但我们可以通过一个叫做Belinfante-Rosenfeld procedure的方法把能动量张量改造成对称的形式!!

因为我们发现,如果对于能动量张量进行一个Modify,其依旧是守恒流:
\begin{align}
  \large\Theta^{\mu\nu}\equiv T^{\mu\nu}+\partial_\sigma A^{\sigma\mu\nu}
\end{align}
其中$ A^{\sigma\mu\nu} $满足$ A^{\sigma\mu\nu}=-A^{\mu\sigma\nu} $,那么我们发现$ \Theta^{\mu\nu} $依旧是一个守恒流。

\bigskip
\hlr{从表示空间分析能动量张量对称化}

我们从表示空间的角度分析,我们把能动量张量对称化到底意味着什么?能动量张量存在两个Lorentz指标,所以其可以看作是$ \mathbb{R}^4 $到一个$ (1/2,1/2) \otimes (1/2,1/2) $表示空间的映射。我们知道这个表示空间可以分解为:
\begin{align}
  \begin{aligned}(1/2,1/2)\otimes(1/2,1/2)&=(1/2\otimes1/2,1/2\otimes1/2)\\&=(0\oplus1,0\oplus1)\\&=(0,0)\oplus(0,1)\oplus(1,0)\oplus(1,1)\end{aligned}
\end{align}
所以我们发现能动量张量的分量可以分为三个部分:
\begin{itemize}
  \item $ (0,0) $部分:这个部分对应的就是能动量张量的trace部分。
  \item $ (0,1)\oplus(1,0) $部分:这个部分对应的就是能动量张量的antisymmetric部分。
  \item $ (1,1) $部分:这个部分对应的就是能动量张量的symmetric部分。
\end{itemize}
\rmk{对于所有rank 2的张量我们都可以进行这样的分解!!}
我们意识到,如果一个系统只有时空平移对称性的话,我们可以有任意形式的能动量张量;但是如果一个标量场系统同时具有Lorentz对称性的话,我们发现antisymmetric部分必须为0!如果不是标量场,我们也可以通过Belinfante-Rosenfeld procedure把antisymmetric部分去掉!
\rmk{
  所以我们发现其实能动量张量包含的信息比我们想象的要多。Lorentz Symmetry的信息实际上包含在其中。
}

\bigskip
\hlr{Lorentz对称性的守恒荷}


考虑一个已经被对称化之后的能动量张量。我们的守恒荷是:
\begin{align}
  J_{\mu\nu}\equiv\int d^3\mathbf{x}J_{\mu\nu}^0=\int d^3\mathbf{x}\left(x_\mu T^0{}_\nu-x_\nu T^0{}_\mu\right)
\end{align}
我们分两个进行讨论:
\begin{enumerate}
  \item \textbf{空间旋转部分$ J_{ij} $}

    这个部分对应的守恒荷是:
    \begin{align}
      J_{ij}=\int d^3\mathbf{x}\left(x_i T^0{}_j - x_j T^0{}_i\right)
    \end{align}
    这个量对应的物理意义是角动量算符!!
  \item \textbf{Boost部分$ J_{0i} $}

    这部分对应守恒荷是:
    \begin{align}
      K_i\equiv J_{i0}=\int d^3\mathbf{x}(x_i\rho-x_0p_i)
    \end{align}
    其中$ \rho = T^0{}_0 $我们理解为能量密度。如果我们定义一个坐标是质心位置,我们将其改写Boost守恒荷:
    \begin{align}
      X_i^{CM}&\equiv\frac{\int d^3\mathbf{x}x_i\rho}{\int d^3\mathbf{x}\rho},\quad K_i\equiv P_0X_i^{CM}-tP_i
    \end{align}
    我们发现Boost守恒意味着质心沿直线进行运动!
\end{enumerate}
\rmk{
  一个很重要的事实,就是守恒荷不一定和Hamiltonian对易。就像Boost的守恒荷和Hamiltonian并不是对易的。但是这个守恒荷显含时间,所以多出了一项,保证是守恒的!
}
\subsection{补充:对称性和守恒流的另一种定义}

对于对称性以及Moether Theorem我们存在另一种等价的定义方式,下面进行讨论。下方内容复制自我的TP4的笔记里面的讨论,所以是英语的。



\subsubsection{Symmetry and Conserved Current}

\hl{A more standard definition of Symmetry}

In Prof. Rattazzi's lecture the symmetry of a field theory is defined as:
\defi{
  \textbf{Symmetry(QFT EPFL)}

  Consider a field theory with a series of dynamical fields $ \phi_a(x) $ and a dynamical Lagrangian $ \mathcal{L}[\phi] $. A symmetry of the action is a transformation with parameters $ \alpha $:
  \begin{align}
    &x^{\prime\mu}=f^{\mu}(x,\alpha)\\ 
    &\phi_a^{\prime}(x^{\prime})=F_a(\phi(x),\alpha)
  \end{align}
  so that satisfy that:
  \begin{align}
  d^4x\left[\mathcal{L}(\phi_a(x),\partial\phi_a(x))\right]=d^4x^{\prime}\left[\mathcal{L}(\phi_a^{\prime}(x^{\prime}),\partial^{\prime}\phi_a^{\prime}(x^{\prime}))+\partial_\mu^{\prime}K^\mu(\phi^{\prime})\right]
  \end{align}
}
\rmk{
  We haven't say anything about what \textbf{transformation} means. Because it has 2 understandings with exactly the same mathematical expression:
  \begin{itemize}
    \item \textbf{Active Transformation}: The coordinates are fixed, but the fields are transformed.
    \item \textbf{Passive Transformation}: The coordinates are transformed, and the field transform covariantly.
  \end{itemize}
}
This definition can be rewrite a bit more elegantly when we define the change of Action under this coordinate transformtaion:
\defi{
  \textbf{Induced Change of Field and Acton under Transformation}

  Consider a field theory with dynamical fields $ \phi_a $ and a action $ S $ that takes the form:
\begin{align}
    S=\int d^dx\ \mathcal{L}(\phi(x),\partial_\mu\phi(x))
\end{align}
Form a transformation:
  \begin{align}
    &x^{\prime\mu}=f^{\mu}(x,\alpha)\\ 
    &\phi^{\prime}(x^{\prime})=F(\phi(x),\alpha)
  \end{align}
We can induce a change of the field configuration:
\begin{align}
  \phi(x)\to \phi^{\prime}(x)=F(\phi(f^{-1}(x)),\alpha)
\end{align}
We define the action after the as:
  \begin{align}
    S^{\prime}=\int d^dx\ \mathcal{L}(\phi^{\prime}(x),\partial_\mu\phi^{\prime}(x))
  \end{align}
}
 We define Symmetry as:
\defi{
  \textbf{Symmetry(QFT standard)}

  A transformation is a symmetry of the action if the action is invariant up to a total derivative:
  \begin{align}\label{eq:Symmetry definition standard}
    S^{\prime}=S+\int d^dx\ \partial_\mu K^\mu
  \end{align}
  Pluging in the definition of action we have:
  \begin{align}
    \delta S[\phi] = \text{Boundary Term}
  \end{align}
  For we can see that $ S^\prime - S $ is the standard definition of variation of action functional induced by the transformation.Thus we can rephrase as:
  \textbf{A transformation is a symmetry if the induced variation of the action functional is a boundary term.}
}
We can Prove that these two definitions are equivalent.We consider \cref{eq:Symmetry definition standard} and do a variable substitution $ x\to x^{\prime} $, we have:
\begin{align}
  S^{\prime}=\int d^dx^{\prime} \mathcal{L}(\phi^{\prime}(x^{\prime}),\partial_\mu^{\prime}\phi^{\prime}(x^{\prime}))+\int d^dx^{\prime}\ \partial_\mu^{\prime} K^\mu = S = \int d^dx\ \mathcal{L}(\phi(x),\partial_\mu\phi(x))
\end{align}
Thus we compare the integrands and get the first definition.

\rmk{
  Why do we use the Second definition?
  Because in the path integral formalism this is more natural and make the derivation more clean. However, the first definition is more intuitive easier to understand.
}

\bigskip
\hl{Noether's Theorem and Conserved Current}

Consider a transformation that is a \textbf{Symmetry of Rigid Parameters}, we can have a Coserved Current and the form of the current is related to the variation of action functional. 

\thm{
  \textbf{Noether's Theorem(Field Theory Standard)}

  Consider a field theory with dynamical fields $ \phi_a $ and a action $ S $ that takes the standard form. Assume that we have a transformation that is a symmetry:
  \begin{align}
    x^{\prime\mu}&=x^\mu+\omega_a\frac{\delta x^\mu}{\delta\omega_a}\\ 
    \phi^{\prime}(x^{\prime})&=\phi(x)+\omega_a\frac{\delta\mathcal{F}}{\delta\omega_a}(x)
  \end{align}
  where $ \omega_a $ are \textbf{rigid parameters}. Then there exists a conserved current $ J^\mu{}_{a} $ that satisfies, if we \textbf{lift the rigid parameters to local parameters} $ \omega_a(x) $:
  \begin{align}
    \delta S=-\int dxJ^\mu{}_a\partial_\mu\omega_a + \text{Boundary Terms}
  \end{align}
  We only have the derivative term $ \partial_{\mu} \omega_a $ this is because the transformation is a symmetry for rigid parameters. One explicit form of $ J^\mu{}_a $ is (\textbf{Canonical Form}):
  \begin{align}
    J^\mu{}_a=\left\{\frac{\partial\mathcal{L}}{\partial(\partial_\mu\phi)}\partial_\nu\phi-\delta_\nu^\mu\mathcal{L}\right\}\frac{\delta x^\nu}{\delta\omega_a}-\frac{\partial\mathcal{L}}{\partial(\partial_\mu\phi)}\frac{\delta\mathcal{F}}{\delta\omega_a}
  \end{align}
If the EoM is satisfied, we have the conservation equation:
  \begin{align}
    \partial_\mu J^\mu{}_a =0
  \end{align}
}
\textbf{Proof:}

\YL{[leave it for later.]}


\hl{Series of Conserved Currents}

  The Conserved Current is not Unique. We can always add a term of the form:
  \begin{align}
    J^\mu{}_a\to J^\mu{}_a + \partial_\nu B^{\mu\nu}{}_a \quad \text{where}\quad B^{\mu\nu}{}_a = -B^{\nu\mu}{}_a
  \end{align}
  Then it is also a conserved current and Satisfy the relation with the variation of action functional. This is because:
  \begin{align}
    \int d^dx\ \partial_\nu B^{\mu\nu}{}_a \partial_\mu \omega_a = \int d^dx\ \partial_\mu\left(B^{\mu\nu}{}_a \partial_\nu \omega_a\right) - \int d^dx\ B^{\mu\nu}{}_a \partial_\mu\partial_\nu \omega_a
  \end{align}
  The first term is a boundary term and the second term is symmetric in $ \mu\nu $ while $ B^{\mu\nu}{}_a $ is antisymmetric in $ \mu\nu $ thus it vanishes.  

\subsubsection{Energy-Momentum Tensor}

\hl{Canonical Energy-Momentum Tensor}

Consider the spacetime translation symmetry:
\begin{align}
  &x^{\prime\mu}=x^\mu + \epsilon^\mu\\ 
  &\phi^{\prime}(x^{\prime})=\phi(x)
\end{align}
We can construct its Canonical Conserved Current which is defined as the canonical
\defi{
  \textbf{Canonical Energy-Momentum Tensor}

  The Canonical Energy-Momentum Tensor is defined as the conserved current of spacetime translation symmetry:
  \begin{align}
    T_c^\mu{}_\nu = \frac{\partial\mathcal{L}}{\partial(\partial_\mu\phi)}\partial_\nu\phi - \delta^\mu_\nu \mathcal{L}
  \end{align}
}
According to the Noether's Theorem, we can see that if a theory has translation symmetry, we lift the rigid translation parameters $ \epsilon^\mu $ to local parameters $ \epsilon^\mu(x) $, we have:
\begin{align}
  \delta S = -\int d^dx\ T_c^\mu{}_\nu \partial_\mu \epsilon^\nu + \text{Boundary Terms}
\end{align}
and if the EoM is satisfied, we have the conservation equation:
\begin{align}
  \partial_\mu T_c^\mu{}_\nu =0
\end{align}
This is generally, the canonical E-M tensor is \textbf{not symmetric} in its indices and has a \textbf{non-vanishing trace}.

\bigskip
\hl{Symmetric Traceless Energy-Momentum Tensor}

If the theory has Lorentz Symmetry and Scale Symmetry, we can always construct a symmetric traceless energy-momentum tensor from the canonical one by adding an improvement term. 

\thm{
  \textbf{Symmetric Traceless Energy-Momentum Tensor}

  Consider a field theory with Lorentz and Scale Symmetry. We can construct a \textbf{Symmetric and Traceless} Energy-Monentum Tensor:
  \begin{align}
    T^{\mu\nu}=T_c^{\mu\nu}+\partial_\rho B^{\rho\mu\nu}+\frac{1}{2}\partial_\lambda\partial_\rho X^{\lambda\rho\mu\nu}
  \end{align}
  where $ B $ and $ X $ are constructed tensors. Moreover, the Lorentz and Scale Conserved Currents can be constructed from this symmetric traceless energy-momentum tensor as:
  \begin{align}
    J_{Lorentz}^{\mu\rho\sigma} &= T^{\mu\rho}x^\sigma - T^{\mu\sigma}x^\rho\\ 
    J_{Scale}^\mu &= T^{\mu\nu}x_\nu
  \end{align}
}
Proof: See yellow book chap 4.2.2 and chap 2.5.1. 


\subsection{补充:Noether Charge 在Hamiltonian力学的意义}\label{sec:Noetherinham}

我们发现前面的讨论都是基于Lagrangian力学的视角的。但是对于量子力学我们希望研究这些对称性和守恒荷在Hamiltonian力学中的意义。这样我们可以直接使用代数力学的方法量子化变成量子力学。

\bigskip
\hlr{Hamiltonian与时间平移对称Noether Charge}

我们可以从两者的定义出发推导发现这两者【在没有边界term $ K $的情况下】必然是一个东西。有的时候一个视角方便计算我们就使用一个视角进行计算!!

我们观察场论的能动量张量的定义:
\begin{align}
  T^\mu{}_\nu\equiv\frac{\partial\mathcal{L}}{\partial(\partial_\mu\phi_a)}\partial_\nu\phi_a-\delta_\nu^\mu\mathcal{L}
\end{align}
我们会意识到,$ T^0{}_0 $的形式和Hamiltonian Density:
\begin{align}
  \mathcal{H}\equiv\pi(x)\dot{\phi}(x)-\mathcal{L}(\phi,\pi)
\end{align}
的定义的形式是完全一样的。所以其实这就证明了,时间平移对称性的守恒荷也就是守恒流在等时面上的积分就是Hamiltonian。


\bigskip
\hlr{守恒荷与Hamiltonian对易关系}

守恒荷是守恒的,所以我们根据Hamiltonian力学之中代数时间演化方程,知道:
    \begin{align}
      \frac{dQ}{dt}=\{Q,H\}=0.
    \end{align}
所以我们知道守恒荷和Hamiltonian在代数上很可能是对易的。
\begin{itemize}
  \item 注意!!如果守恒荷显含时间的话,那么我们有:
    \begin{align}
      \frac{dQ}{dt}=\frac{\partial Q}{\partial t}+\{Q,H\}=0
    \end{align}
    这个时候守恒荷和Hamiltonian不一定对易!!
\end{itemize}
一个最典型的例子就是Boost守恒荷和Hamiltonian不对易!!


\bigskip
\hlr{守恒荷代数与Lagrangian Formalism生成元算符代数}

这两个代数是一样的。所以我们如果想找到守恒荷的代数结构,我们可以直接从Lagrangian Formalism的视角下找到生成元算符然后计算其代数结构。我们给出Lagrangian Formalism里面生成元的定义:
\defi{生成元定义:

对于一个对称性变换$ \omega $,我们可以induce一个场构型空间的无限小变分。这个变分可以写作;
\begin{align}
  \delta_\omega\phi(x)\equiv\phi^{\prime}(x)-\phi(x)\equiv-i\omega_aG_a\phi(x)
\end{align}
其中$ G_a $是一个微分算符(一般包含一些函数和导数算符的组合),我们把这个算符叫做这个对称性变换的生成元算符。
}
对称性变换的生成元可以通过导数算符互相作用的方式给出一个李代数。我们会发现这个李代数和守恒荷在Poisson Bracket下面构成的代数是一致的。我们给出证明:

首先对于两个对称性变换,我们可以计算不同作用顺序的差值:
\begin{align}
  \{\{F,Q_a\},Q_b\}-\{\{F,Q_b\},Q_a\}=\delta_b(\delta_aF)-\delta_a(\delta_bF)=[\delta_b,\delta_a]F.
\end{align}
然后根据Poisson Bracket的Jacobi Identity,我们有:
\begin{align}
  \{F,\{Q_a,Q_b\}\}=\{\{F,Q_a\},Q_b\}-\{\{F,Q_b\},Q_a\}.
\end{align}
因此我们得到:
\begin{align}
  [\delta_b,\delta_a]F=\delta_{[b,a]}F=f_{ab}^c\delta_cF=f_{ab}^c\{F,Q_c\}.
\\\{F,\{Q_a,Q_b\}\}={f_{ab}}^c\{F,Q_c\}.
\end{align}
这就意味着生成元在算符作用下构成的李代数的结构常数和守恒荷在Poisson Bracket下面构成的李代数的结构常数是一样的!!因此两个李代数在数学结构上是一样的!!

\bigskip
\hlr{守恒荷作为对称性变换下动力学变量构型变换的生成元}

我们知道任意的相空间上的函数都可以Generate一个场构型的变换。那么我们想问守恒荷也是一个相空间上的函数。其Generate了什么样子的变换呢?答案是:守恒荷Generate了对应的对称性变换!!

我们会发现如果我们的场构型的无限小变分是$ \delta \phi_a $的话,那么这个无限小变换可以写作下面的形式:
\begin{align}
  \delta_\alpha\phi_a(x)\equiv\phi_a^{\prime}(x)-\phi_a(x)=\alpha^i\left\{Q_i,\phi_a(x)\right\}=\alpha^i\Delta_{ai}(x).
\end{align}
这个证明过程可以看作业,其实就是把守恒荷的定义带入Poisson Bracket的定义之中。然后使用functional derivative的定义进行计算就好。

所以我们知道:
\begin{itemize}
  \item 对称性变换对应的守恒荷和对称性变换generator这个相空间的函数,是一个东西的两种定义方法!
\end{itemize}

特殊的一个经典的说法就是:\textbf{Hamiltonian = 时间平移变换生成元 = 时间平移对称性守恒荷}

\subsection{Questions and thoughts}


\question{对于对称性,Weinberg的定义和这里的定义是怎么对应的?}

我们对称性有两种定义方式:
\begin{itemize}
  \item 课程的定义:定义对称性作为一个coordinate transformation。这个变换保证EoM不变。也就是【Lagrangian的【形式不变】】
  \item Weinberg的定义:定义对称性作为一个field variation。这个变换如果在任何情况下【不需要on-shell】都保证【作用量取值】不变。
\end{itemize}
这两种定义方式是完全等价的。因为coordinate transformation可以完全induce出来一个field variation。反过来field variation也可以induce出来一个coordinate transformation。并且我们可以证明:
\begin{itemize}
  \item 两种定义下对应给出的守恒荷是完全一样的!!
\end{itemize}
第一种定义更加physical给予了丰富的物理诠释;第二种定义更加计算concrete,方便判断symmetry以及求解计算守恒流。
\begin{table}[H]
\centering
\renewcommand{\arraystretch}{1.4}
\begin{tabular}{p{0.45\textwidth} | p{0.45\textwidth}}
\hline
\textbf{课程中的定义} & \textbf{Weinberg 的定义} \\
\hline\hline
对称性\textbf{坐标变换以及场的协变}(coordinate transformation) &
对称性被定义为\textbf{场的变分}(field variation)\\
\hline
要求:作用量的形式不变的情况下plug in 变换前后的场和导数算符,取值相同;等价于运动方程形式不变 &
要求:作用量在变分前后的数值不变 \\
\hline
Local symmetry是群参数依赖于时空点的对称性变换 &
Local symmetry是无限小变分系数$ \epsilon(x) $依赖于时空点的场变分 \\
\hline\hline
\end{tabular}
\caption{两种对称性定义的比较与对应关系}
\end{table}



\qed 


\bigskip
\question{为什么对于经典力学(并非场论)的时间平移对称性的推导之中不需要使用运动方程呢?}

我们注意!!!
\begin{itemize}
  \item 我们需要【满足运动方程】来证明守恒流是守恒的!!
  \item 但是推导守恒流的形式的时候没有必要使用运动方程!!
\end{itemize}
比如,经典力学的时间平移对称性的推导之中,我们并没有使用运动方程就可以读出守恒流:
\begin{align}
  \delta S = \int dt \dot{a} (\dot{q} \displaystyle\frac{L}{\dot{q}} - L) +\partial_{t}(aL)
\end{align}
我们自然可以直接读出守恒流是:
\begin{align}
  H = \dot{q} \displaystyle\frac{L}{\dot{q}} - L 
\end{align}
如果需要证明这个流是守恒的,我们需要运动方程满足,这个时候:
\begin{align}
  \delta S = 0 \Rightarrow \frac{dH}{dt} = 0
\end{align}
如果存在一个对称性,不论运动方程是否满足我们发现$ \delta S = 0$,那么才是Gauge Symmetry。
\qed



% section Lecture 5 (end)

\newpage
\section{Lecture 8: Shot Noise}\label{sec:Lecture 8: Shot Noice} % (fold)
我们前面研究了量子散射理论的set up,下面我们讨论这样的理论是怎样给出可观测量的。

\subsection{量子散射问题的set up}

在讨论之前,我们需要set up量子散射问题来说明,对于量子散射问题我们的可观测量是什么,我们怎么从经典的散射截面的定义推广到量子力学的语境下。

\bigskip
\hlr{一般in state对应out state的概率分布}

\begin{itemize}
  \item \textbf{一般in state给出out state在某个立体角的概率:} 

    我们考虑一个一般的in state $ \kin{\psi} $其在动量空间的表示为$ \psi_{\text{in}}(p) = \langle p \kin{\psi} $。如果其出射到某个out state为$ \kout{\psi} $其概率是:
    \begin{align}
      \boldsymbol{\omega}\left(d\Omega\leftarrow\psi_{\mathrm{in}}\right)=d\Omega\int p^2dp\left|\psi_{\mathrm{out}}(\mathbf{p})\right|^2.
    \end{align}
    注意到out state $ \psi_{\text{out}}(\mathbf{p}) $是和出射的大小和方向都有关系的,我们仅仅是对于出射动量的大小进行积分,所以$ \omega $实际上包含了【出射动量的方向信息】【当然也包含了入射大小和方向的信息】

\end{itemize}

\bigskip
\hlr{散射问题set up}

为了模拟经典情况,我们考虑下面的情况散射问题set up:
\begin{itemize}
  \item \textbf{入射波包:}类比经典例子我们考虑一个wave package,其:
    \begin{itemize}
      \item 其动量方向垂直与入射面;
        \item 其动量大小集中在$ p_0 $附近
          \item 其空间分布在$ 0 $这个位置附近,也就是入射面法线通过散射轴心的那个
    \end{itemize}
    我们可以先定义一个入射轴上面的态$ \ket{\phi_0} $然后使用平移算符得到所有入射平面的态:
    \begin{align}
      \ket{\phi_a} = e^{-ip_0 \cdot a}\ket{\phi_0}\mathrm{~,}
    \end{align}
    \rmk{
      这里我们平移算符认为就是一个相位,这是因为我们考虑入射态集中在$ p_0 $附近,所以我们直接使用$ p_0 $进行平移。
    }
  \item \textbf{散射截面:} 类比经典的情况,我们把散射截面理解为【把每一个波包当作in state】的情况下,out state在某个立体角概率总和!
    \begin{align}
      \frac{d\sigma}{d\Omega}d\Omega=\int d^2a\ \omega(d\boldsymbol{\Omega}\leftarrow \phi_\mathbf{a})\mathrm{~.}
    \end{align}
\end{itemize}
\begin{figure}[H]
  \centering
  \includegraphics[width=0.75\textwidth]{assets/scattering.png}
  \caption{量子散射问题set up示意图}
  \label{fig:scattering}
\end{figure}
\begin{itemize}
  \item \textbf{散射截面的依赖}

    我们分析散射截面依赖什么,会发现其取决于下面三个信息:
    \begin{itemize}
      \item 入射动量大小
      \item 入射动量方向【我们确实可以explicitly这么想,但是只有选择垂直方向才能够给出和散射振幅的关系】
      \item 出射动量方向【也就是出射的立体角】
    \end{itemize}
\end{itemize}

\subsection{Differential Cross Section and Scattering Amplitude}

\thm{微分散射截面与散射振幅

  对于一个量子理论来说,只考虑一群以$ p_0 $大小垂直入射的粒子,其出射动量在某个立体角内的微分散射截面为:
\begin{align}
  \displaystyle\frac{d \sigma}{d \Omega}(\mathbf{p}_0, \Omega)=\left|f\left(\mathbf{p}, \mathbf{p}_0\right)\right|^2\mathrm{~,}
\end{align}
我们注意,散射振幅虽然仿佛depend on出射的动量大小以及方向,其实仅仅和出射方向有关,因为我们已经默认了弹性散射了!
}

在上面的set up之下我们发现这个关系可以通过计算证明。上面我们根据散射振幅的定义给出:
\begin{align}
  \frac{d\sigma}{d\Omega}d\Omega=\int d^2a\omega(d\Omega\leftarrow\phi_{\mathbf{a}}).
\end{align}
我们的核心就是计算$ \omega(d\Omega\leftarrow\phi_{\mathbf{a}}) $是什么。下面给出计算。

\bigskip
\hlr{out put概率计算}

我们知道out put state和in state的关系是通过S矩阵给出的:
\begin{align}
  \begin{aligned}\psi_\mathrm{out}(\mathbf{p})&=\int d^3\mathbf{p}'\langle\mathbf{p}|S|\mathbf{p}'\rangle\boldsymbol{\psi}_{\mathrm{in}}(\mathbf{p}')\\&=\psi_{\mathrm{in}}(\mathbf{p})+\frac i{2\pi m}\int d^3\mathbf{p}^{\prime}\delta(E_p-E_{p^{\prime}})f(\mathbf{p}\leftarrow\mathbf{p}^{\prime})\psi_{\mathrm{in}}(\mathbf{p}^{\prime}) .\end{aligned}
\end{align}
如果我们认为考虑的出射动量和入射波包的动量大小完全不一样,那么我们有:
\begin{align}
  \psi_{\text{in}}(\mathbf{p})\approx 0\mathrm{~,}
\end{align}
因此如果考虑入射态是$ \phi_a $波包我们有:
\begin{align}
  \psi_{\text{out}}(\mathbf{p}) = \frac i{2\pi m}\int d^3\mathbf{p}^{\prime}\delta(E_p-E_{p^{\prime}})f(\mathbf{p}\leftarrow\mathbf{p}^{\prime})\phi_a(\mathbf{p}^{\prime}) 
\end{align}
然后我们将这个结果带入计算。

\bigskip
\hlr{微分散射截面与散射振幅的关系计算}

我们进行计算:
\begin{align}
  \begin{aligned}\frac{d\sigma}{d\Omega}=\frac{1}{(2\pi m)^2}\int d^2a\int p^2dp&\left[\int d^3\mathbf{p}^{\prime}\boldsymbol{\delta}\left(E_\mathbf{p}-E_{\mathbf{p}^{\prime}}\right)f\left(\mathbf{p}\leftarrow\mathbf{p}^{\prime}\right)e^{-i\mathbf{p}^{\prime}\cdot\mathbf{a}}\phi_0(\mathbf{p}^{\prime})\right.\\&\times\int d^3\mathbf{p}^{\prime\prime}\boldsymbol{\delta}\left(E_\mathbf{p}-E_{\mathbf{p}^{\prime\prime}}\right)f^*\left(\mathbf{p}\leftarrow\mathbf{p}^{\prime\prime}\right)e^{\mathrm{i}\mathbf{p}^{\prime\prime}\cdot\mathbf{a}}\boldsymbol{\phi}_0^*(\mathbf{p}^{\prime\prime})\end{aligned}
\end{align}

首先对于面积进行积分。我们意识到,这个通过下面关系会给出【垂直于入射方向的delta函数】
\begin{align}
  \int d^2ae^{-i\mathbf{a}\cdot\mathbf{p}^{\prime}+i\mathbf{a}\cdot\mathbf{p}^{\prime\prime}}=(2\pi)^2\delta^{(2)}\left(p_\perp^{\prime}-p_\perp^{\prime\prime}\right).
\end{align}
然后进行海量delta函数的化简我们有:
\begin{align}
  \frac{d\sigma}{d \Omega}=\int d^3\mathbf{p}^{\prime}\frac{p^{\prime}}{p_\parallel^{\prime}}\left|\phi(\mathbf{p}^{\prime})\right|^2\left|f\left(\mathbf{p}\leftarrow\mathbf{p}^{\prime}\right)\right|^2.
\end{align}
这个时候我们认为入射波包足够集中在$ p_0 $附近,并且入射方向垂直于入射面,所以我们有:
\begin{align}
  \frac{d\sigma}{d\Omega}=|f\left(\mathbf{p}\leftarrow\mathbf{p}_0\right)|^2,
\end{align}



\subsection{Optical Theorem}
这里我们会发现入射方向的散射振幅的虚部和总散射截面之间存在一个非常重要的关系,称为Optical Theorem。我们下面给出定理。
\thm{
  \textbf{Optical Theorem:} 
 
  对于一个入射动量为$ \mathbf{p} $的平面波入射态,设其【入射方向】散射振幅为$ f(\mathbf{p},\mathbf{p}) $【也就是出射动量和入射动量方向都一样的散射振幅】,那么其总散射截面$ \sigma $和入射方向散射振幅的虚部之间存在下面的关系:
  \begin{align}
\operatorname{Im}f(\mathbf{p}\to\mathbf{p})=\frac{p}{4\pi}\sigma(\mathbf{p})\mathrm{~.}
  \end{align}

}
我们下面逐步讨论证明

\bigskip
\hlr{S 矩阵的集团分解}

对于S矩阵我们知道没有散射的时候就是单位矩阵,所以我们不妨认为是单位矩阵加上一个散射相关的部分:
\begin{align}
  S= 1+R 
\end{align}
我们分析$ R $矩阵的性质:
\begin{itemize}
  \item \textbf{Unitary的结果:}由于我们assume S矩阵是unitary的,所以我们有:
    \begin{align}\label{eq:unitaryR}
      R+R^\dagger+R^\dagger R=0\mathrm{~.}
    \end{align}
  \item \textbf{自由粒子Hamiltonian对易}: 我们知道S矩阵和自由粒子Hamiltonian对易所以我们也有:
    \begin{align}
      [R,H_0]=0\mathrm{~.}
    \end{align}
    这意味着我们可以使用平面波本征态作为R矩阵的共同本征态。
  \item \textbf{R矩阵和散射振幅}:R矩阵也可以用平面波展开所以我们对比S矩阵用平面波展开的结果以及散射振幅的定义可以知道:
    \begin{align}\label{eq:Randf}
      \displaystyle\frac{i}{2 \pi m} \delta(E_{p'} - E_p) f(\mathbf{p}', \mathbf{p}) = \langle \mathbf{p}' | R | \mathbf{p} \rangle \mathrm{~.}
    \end{align}
\end{itemize}
我们使用平面波波展开\cref{eq:unitaryR}式子可以得到:
\begin{align}
  \left\langle\mathbf{p}^{\prime}\right|R\left|\mathbf{p}\right\rangle+\left\langle\mathbf{p}\right|R\left|\mathbf{p}^{\prime}\right\rangle^*=-\int d^3\mathbf{p}^{\prime\prime}\left\langle\mathbf{p}^{\prime}\right|R^\dagger\left|\mathbf{p}^{\prime\prime}\right\rangle\left\langle\mathbf{p}^{\prime\prime}\right|R\left|\mathbf{p}\right\rangle.
\end{align}
如果带入R矩阵和散射振幅的关系\cref{eq:Randf}我们有:
\begin{align}\label{eq:opticaltheorem1}
  \begin{aligned}\boldsymbol{\delta}\left(E_{\mathbf{p}}-E_{\mathbf{p}^{\prime}}\right)&\left[f\left(\mathbf{p}^{\prime}\leftarrow\mathbf{p}\right)-f^*\left(\mathbf{p}\leftarrow\mathbf{p}^{\prime}\right)\right]\\&=\frac{i}{2\pi m}\int d^3\mathbf{p}^{\prime\prime}\delta\left(E_{\mathbf{p}^{\prime}}-E_{\mathbf{p}^{\prime\prime}}\right)\boldsymbol{\delta}\left(E_{\mathbf{p}^{\prime\prime}}-E_{\mathbf{p}}\right)f^*\left(\mathbf{p}^{\prime\prime}\leftarrow\mathbf{p}^{\prime}\right)f\left(\mathbf{p}^{\prime\prime}\leftarrow\mathbf{p}\right).\end{aligned}
\end{align}

\bigskip
\hlr{数学公式的使用}

我们需要玩弄delta函数,下面存在两个会用到的delta函数的公式:
\begin{itemize}
  \item \textbf{Delta函数的传递}: delta函数放在一起会有传递性:
    \begin{align}
  \delta\left(E_{\mathbf{p}^{\prime}}-E_{\mathbf{p}^{\prime\prime}}\right)
  \delta\left(E_{\mathbf{p}^{\prime\prime}}-E_{\mathbf{p}}\right)
  =
  \delta\left(E_{\mathbf{p}}-E_{\mathbf{p}^{\prime}}\right)
  \delta\left(E_{\mathbf{p}}-E_{\mathbf{p}^{\prime\prime}}\right),
    \end{align}
  \item \textbf{Delta函数的复合}:如果delta函数和另一个函数复合了,我们存在关系:
    \begin{align}
      \delta\left[f(x)\right]=\delta(x-x_0)/\left|f^{\prime}(x_0)\right|
    \end{align}
    
\end{itemize}  

\bigskip
\hlr{Optical Theorem的证明}

我们将数学公式带入\cref{eq:opticaltheorem1}式子,可以把两个delta函数进行传递,然后使用delta函数的复合公式变成关于p的delta函数。我们把积分变成球坐标系的!

如果【我们只考虑入射方向$ p = p' $】,我们有:
\begin{align}
  2i\operatorname{Im}f\left(\mathbf{p}\leftarrow\mathbf{p}\right)=\frac{i}{2\pi m}\int d\Omega \ p^{\prime\prime2}dp^{\prime\prime}\frac{m}{p}\boldsymbol{\delta}\left(p-p^{\prime\prime}\right)\left|f\left(\mathbf{p}^{\prime\prime}\leftarrow\mathbf{p}\right)\right|^2,
\end{align}
最后化简积分得到:
\begin{align}
  \mathrm{Im~}f\left(\mathbf{p}\leftarrow\mathbf{p}\right)=\frac{p}{4\pi}\int d\Omega\left|f\left(\mathbf{p}\leftarrow\mathbf{p}\right)\right|^2.
\end{align}
如果使用微分散射截面与散射振幅的关系,我们最终得到Optical Theorem的结果:
\begin{align}
  \mathrm{Im~}f\left(\mathbf{p}\leftarrow\mathbf{p}\right)=\frac{p}{4\pi}\sigma(\mathbf{p})\mathrm{~.}
\end{align}



\subsection{数学技巧}

本章使用了很多数学上的特殊技巧,下面我们总结一下。

\subsubsection{Delta函数使用}
\YL{[作业8]}



\subsubsection{Green 函数使用}

我们对于分母上面的复函数可以使用一个变形技巧:
\begin{align}
  \frac{1}{x+i\varepsilon}=-i\pi\delta(x)+\mathscr{P}\frac{1}{x}\mathrm{~,~}\quad\boldsymbol{\varepsilon}\to+0\mathrm{~.}\\ 
  \frac{1}{x-i\varepsilon}=i\pi\delta(x)+\mathscr{P}\frac{1}{x}\mathrm{~,~}\quad\boldsymbol{\varepsilon}\to+0\mathrm{~.}
\end{align}
\YL{[作业8]}




\subsection{Questions and thoughts}

\question{散射振幅到底和什么有关系,为什么积分出射动量大小的时候可以不积分散射振幅里面的??}

这是因为我们在讨论最初已经假设了弹性散射。所以,我们散射振幅其实只和【出射的方向】以及【入射的动量大小和方向】有关,所以我们对于出射动量大小进行积分的时候不会影响出射动量方向的信息!!
\qed

\question{为什么散射截面会和动量有关系,到底是和动量的什么有关系大小还是方向??}
我们的set up已经固定了一堆入射动量大小以及垂直于入射面的方向。散射截面的动量依赖其实就是表示我们散射set up入射动量大小的依赖!!
\qed

\question{我们进行这个积分 $ \int d^2a\operatorname{\omega}(d\Omega\leftarrow\phi_a)\mathrm{~.} $,用它来对应一个立体角的散射截面,真的不会有归一化的问题吗?真的合法吗?}

这样积分是完全合法的,我们观察这个式子:
\begin{align}
  \frac{d\sigma}{d\Omega}d\Omega=N=\int d^2a\ \omega(d \Omega\leftarrow \phi_a)\mathrm{~.}
\end{align}
两边对于立体角进行积分之后,我们可以得到右边正好是入射面积$ \pi a^2 $,左边则是总散射截面。因此,这个式子在interpretation上是合法的!
\qed

% section Lecture 8: Shot Noice (end)

\newpage 
\section{Lecture 9: 区分量子态与矩阵模长}\label{sec:Lecture 9:} % (fold)
\subsection{区分量子态}
\hlr{量子态区分概率最大化}

给定两个量子态$ \rho_0 $以及$ \rho_1 $我们希望通过POVM对其进行区分。假设我们使用的POVM是有两个元素$ \{ M_0,M_1\} $。其中$ M_i $回答的问题是是不是$ \rho_i $态。我们给出下面定理:
\thm{
  量子态区分概率
  
  给定两个量子态$ \rho_0 $以及$ \rho_1 $,其先验概率分别假设为$ 1/2 $。我们使用POVM$ \{ M_0,M_1\} $对其进行区分,我们\textbf{【主观上认为】测量到$ M_0 $结果量子态就是$ \rho_0 $,测量到$ M_1 $结果量子态就是$ \rho_1 $}。那么我们成功区分的概率为:
  \begin{align}
    p=\frac{1}{2}+\frac{1}{2}\tr(M_0 (\rho_0 - \rho_1))\mathrm{~.}
  \end{align}
}

Proof: 我们计算成功概率:
\begin{align}
  \mathbb{P}(\mathrm{succes})&=\mathbb{P}(\rho=\rho_0)\mathbb{P}(\mathrm{measure~}0\mid\rho=\rho_0)+\mathbb{P}(\rho=\rho_1)\mathbb{P}(\mathrm{measure~}1\mid\rho=\rho_1)\\
                             &=\frac{1}{2}\operatorname{Tr}(M_0\rho_0)+\frac{1}{2}\operatorname{Tr}(M_1\rho_1).
\end{align}
根据POVM的完备性我们有$ M_1=I-M_0 $,可以直接给出上面的结论。
\rmk{
  注意,我们是主观上认为测量到$ M_0 $结果量子态就是$ \rho_0 $,测量到$ M_1 $结果量子态就是$ \rho_1 $。实际上如果不小心设计了一个$ M_0 $导致$ \rho_0 $永远测不到这个结果,我们只能说这个测量设定很不合理,但是也不是不行。
}

\thm{
量子态区分最优POVM 

给定两个量子态$ \rho_0 $以及$ \rho_1 $,其先验概率分别假设为$ 1/2 $。我们使用POVM$ \{ M_0,M_1\} $对其进行区分,我们主观上认为测量到$ M_0 $结果量子态就是$ \rho_0 $,测量到$ M_1 $结果量子态就是$ \rho_1 $。那么使得成功区分概率最大的POVM为:
\begin{align}
   p_{opt}=\max_{0\leq M_0\leq I}\frac{1+\mathrm{Tr}(M_0(\rho_0-\rho_1))}{2}
\end{align}
其中$ M_0 $是任意的正半定矩阵且$ M_0\leq I $。
}

\bigskip
\hlr{例:纯态区分}

我们考虑区分两个纯态$ |\psi_0\rangle $以及$ |\psi_1\rangle $。我们不妨构建一个POVM保证$ M_0 = \ketbra{0}{0} $ 以及$ M_1 = \ketbra{1}{1} $,其中$ |0\rangle $以及$ |1\rangle $是两个态的正交归一化基底。我们计算成功概率:
\begin{align}
  p=\frac{1+\mathrm{Tr}(M_0(\rho_0-\rho_1))}{2}=\frac{1+1}{2}=1.
\end{align}
因此我们可以完美区分两个纯态。

\bigskip
\hlr{例:混态区分}

对于single qubit的混态我们考虑区分下面两个态:
\begin{align}
  \rho_0 = \frac{1}{2}\left( I + \vec{r}_0 \cdot \vec{\sigma} \right), \quad
\rho_1 = \frac{1}{2}\left( I + \vec{r}_1 \cdot \vec{\sigma} \right).
\end{align}
我们不妨更一般的构建下面一个POVM:
\begin{align}
  M_0 = a\!\left(I + \vec{x}\!\cdot\!\vec{\sigma}\right),
\quad
M_1 = I - M_0 .
\end{align}
但是generally这两个矩阵需要满足一些约束条件才能是一个合理的POVM,我们下面给出定理:
\thm{
  Single qubit POVM的约束条件

  给定一个single qubit的POVM$ \{ M_0,M_1\} $,其中$ M_0 = a\!\left(I + \vec{x}\!\cdot\!\vec{\sigma}\right) $以及$ M_1 = I - M_0 $。那么其需要满足下面的约束条件:
  \begin{align}
    \left\|\vec{x}\right\|\leq1 \quad a\in\left[0,1/\left(1+\left\|\boldsymbol{\vec{x}}\right\|^2\right)\right]
  \end{align}
}
Proof: 我们干的事情就是计算两个的本征值,保证本征值非复即可。

\bigskip
在这个基础上我们计算这两个混态的区分概率:
\thm{
  Single qubit混态区分概率

  给定两个single qubit混态$ \rho_0 = \frac{1}{2}\left( I + \vec{r}_0 \cdot \vec{\sigma} \right) $以及$ \rho_1 = \frac{1}{2}\left( I + \vec{r}_1 \cdot \vec{\sigma} \right) $,其先验概率分别假设为$ 1/2 $。我们使用POVM$ \{ M_0,M_1\} $对其进行区分,我们主观上认为测量到$ M_0 $结果量子态就是$ \rho_0 $,测量到$ M_1 $结果量子态就是$ \rho_1 $。那么成功区分的概率为:
  \begin{align}
    p \;=\; \frac12 \;+\; \frac{a}{4}\bigl( 2\,\vec{x}\!\cdot\!(\vec{r}_0 - \vec{r}_1) \bigr).
  \end{align}
}
这个计算之中我们使用了Pauli矩阵的trace性质:
\begin{align}
  \mathrm{Tr}(\sigma_i\sigma_j)=2\delta_{ij}, \quad \tr (\sigma_i)=0.
\end{align}
我们使用Pauli Basis展开single qubit的量子态的原因正是因为这个性质。
\thm{
  Single qubit混态区分最优概率

  对于上方的set up,使得成功区分概率最大的POVM为:
  \begin{align}
    \vec{x}_{\mathrm{opt}}
= \frac{\vec{r}_0 - \vec{r}_1}{\|\vec{r}_0 - \vec{r}_1\|},
\qquad
M_{0,\mathrm{opt}}
= \frac12\!\left( I + \vec{x}_{\mathrm{opt}}\!\cdot\!\vec{\sigma} \right).
  \end{align}
  此时成功区分的最优概率为:
  \begin{align}
    p_{\mathrm{opt}}
= \frac12 + \frac14 
\left(
\frac{(\vec{r}_0 - \vec{r}_1)\!\cdot\!(\vec{r}_0 - \vec{r}_1)}
{\|\vec{r}_0 - \vec{r}_1\|}
\right)
= \frac12 + \frac14 \|\vec{r}_0 - \vec{r}_1\|.
  \end{align}
}
Proof:在上面约束条件下面进行放缩就好了!!这里我们最后一步放缩的时候会使用:
\begin{align}
  p=\frac{1}{2}+\frac{1}{2}\cdot\frac{1}{1+\left\|\boldsymbol{\vec{x}}\right\|^{2}}\|\boldsymbol{\vec{x}}\|\|\boldsymbol{\vec{r}}_{0}-\boldsymbol{\vec{r}}_{1}\|\cos(\boldsymbol{\vec{x}};\boldsymbol{\vec{r}}_{0}-\boldsymbol{\vec{r}}_{1}).
\end{align}
然后我们对此考虑到函数:
\begin{align}
  f(t)=\frac{t}{1+t^2},\quad t\geq0
\end{align}
在$ t=1 $处取得最大值$ 1/2 $。然后我们带回去就得到了最终结果!!

\bigskip
\hlr{数学工具:1-Norm}

上面讨论的都是单个bit的情况,我们考虑更高维度的量子态区分。为此我们先要定义1-norm的数学工具:
\defi{
  Matrix 1-Norm

  对于一个矩阵$ X $我们定义其1-Norm为:
  \begin{align}
    \left\|X\right\|_1=\mathrm{Tr}\left(\sqrt{X^\dagger X}\right)=\sum_i\sqrt{\lambda_i(X^\dagger X)},
  \end{align}
  其中$ \lambda_i(X^\dagger X) $是矩阵$ X^\dagger X $的本征值。
}
显然对于Unitary的矩阵$ U $我们有$ \left\|U\right\|_1=\mathrm{Tr}(I)=d $,其中$ d $是矩阵的维度。对于Hermitean矩阵$ H $我们有$ \left\|H\right\|_1=\sum_i|\lambda_i(H)| $,其中$ \lambda_i(H) $是矩阵$ H $的本征值。


\bigskip
\hlr{highier dimension state区分}

我们考虑区分两个高维度量子态$ \rho_0 $以及$ \rho_1 $。我们有下面定理:
\thm{
  Highier dimension量子态区分概率

  给定两个量子态$ \rho_0 $以及$ \rho_1 $,其先验概率分别假设为$ 1/2 $。我们使用POVM$ \{ M_0,M_1\} $对其进行区分,我们主观上认为测量到$ M_0 $结果量子态就是$ \rho_0 $,测量到$ M_1 $结果量子态就是$ \rho_1 $。那么成功区分的概率为:
  \begin{align}
    p_{opt}=\frac{1}{2}\left(1+\max_{0\leq M\leq I}\tr(M(\rho_0-\rho_1))\right)= \displaystyle\frac{1}{2}+\frac{1}{4}\|\rho_0-\rho_1\|_1.
  \end{align}
}
Proof:其中我们显然使用了一个恒成立的数学关系:
\begin{align}
  \frac{1}{2}\|\rho_0-\rho_1\|_1=\max_{0\leq M\leq I}\operatorname{Tr}(M(\rho_0-\rho_1)).
\end{align}
证明的核心是构造一个只有$ \rho_0 - \rho_1 $本征值为正的本征子空间的投影算符作为POVM的一个元素。

\subsection{矩阵模长}

\bigskip
\hlr{一般定义}

上面我们给出了1-Norm的定义,下面我们逐渐给出一些常用的定义与性质。
\defi{
  一般矩阵Norm

  对于一个矩阵如果我们希望赋予一个Norm的概念$ \|.\|:K^{m\times n}\mapsto\mathbb{R} $,那么需要满足一些必要的性质:
  \begin{itemize}
    \item 非负性:$ \|A\|\geq0 $
      \item 零点对应:$ \|A\|=0\Leftrightarrow A=0 $
      \item 三角不等式:$ \|A+B\|\leq\|A\|+\|B\| $
      \item 齐次性:$ \|\alpha A\|=|\alpha|\|A\| $
  \end{itemize}
}
辅助定义我们还需要定义一些概念:
\defi{
  Normal Matrix

  对于一个矩阵$ A\in K^{n\times n} $,如果其满足$ AA^\dagger=A^\dagger A $,那么我们称其为Normal Matrix。
}
显然Hermitean矩阵、Unitary矩阵都是Normal Matrix。我们考虑之前对于这两种normal矩阵的1-Norm的讨论可以有下面的定理:
\thm{
  Normal Matrix的1-Norm

  对于一个Normal Matrix $ A\in K^{n\times n} $,其1-Norm可以表示为:
  \begin{align}
    \left\|A\right\|_1=\sum_i|\lambda_i(A)|,
  \end{align}
  其中$ \lambda_i(A) $是矩阵$ A $的本征值。
}
我们一般都只考虑Normal Matrix的情况,因为演化算子是Unitary的量子态和测量算子是Hermitean的。所以基本上都是Normal Matrix。

\rmk{
  我们需要注意,对于Unitary矩阵来说$ \left\|U\right\|_1=d $,其中$ d $是矩阵的维度。这意味着:
  \begin{itemize}
    \item Unitary矩阵的\textbf{本征值模长}之和等于矩阵的维度。
      \item Unitary矩阵的本征值求和不一定等于矩阵的维度。
  \end{itemize}
  请区分模长求和与直接求和!!
}

\bigskip
\hlr{Schatten p-Norm}

\defi{
  Schatten p-Norm

  对于一个矩阵$ A\in K^{m\times n} $,我们定义其Schatten p-Norm为:
  \begin{align}
    \left\|A\right\|_p=\left(\sum_{i=1}^d\left|\sqrt{\lambda_i(A^\dagger A)}\right|^p\right)^{1/p}.
  \end{align}
  其中$ \lambda_i(A^\dagger A) $是矩阵$ A^\dagger A $的本征值。
}
对于一般的矩阵我们可以使用奇异值分解来分析其Schatten p-Norm的性质。我们有定理:
\thm{
  矩阵的奇异值分解与Schatten p-Norm 

  对于一个矩阵$ A\in K^{m\times n} $,其奇异值分解为$ A=U\Sigma V^\dagger $,其中$ U\in K^{m\times m} $以及$ V\in K^{n\times n} $是Unitary矩阵,$ \Sigma\in K^{m\times n} $是一个对角矩阵,其对角线元素为矩阵$ A $的奇异值。我们有:
  \begin{align}
    \left\|A\right\|_p=\left(\sum_{i=1}^d\sigma_i(A)^p\right)^{1/p},
  \end{align}
}
Proof: 这是因为矩阵的奇异值和$ A^\dagger A $的本征值有关系:
\begin{align}
  \sigma_i=\sqrt{\lambda_i(A^\dagger A)}
\end{align}
同样的考虑对于Normal Matrix的情况,我们存在一个特例:
\thm{
  Normal Matrix的Schatten p-Norm

  对于一个Normal Matrix $ A\in K^{n\times n} $,其Schatten p-Norm可以表示为:
  \begin{align}
    \left\|A\right\|_p=\left(\sum_i|\lambda_i(A)|^p\right)^{1/p},
  \end{align}
  其中$ \lambda_i(A) $是矩阵$ A $的本征值。
}
\begin{itemize}
  \item 我们从此只考虑Normal Matrix的情况。
\end{itemize}

\bigskip
\hlr{Examples of Schatten p-Norm}

如果$ p = 1 $那么Schatten 1-Norm就是我们之前定义的1-Norm。对于Normal Matrix来说:
\thm{
  Normal Matrix的Schatten 1-Norm

  对于Normal Matrix $ A\in K^{n\times n} $,其Schatten 1-Norm可以表示为:
\begin{align}
  \lVert A\rVert_1=\sum_i\lvert\lambda_i\rvert.
\end{align}
}
对于$ p = 2 $我们有Schatten 2-Norm:
\begin{align}
  \lVert A\rVert_2=\left(\sum_i\lvert\lambda_i\rvert^2\right)^{1/2}.
\end{align}
但是我们可以证明对于Normal Matrix其存在另一种更漂亮的形式
\thm{
  Normal Matrix的Schatten 2-Norm

  对于Normal Matrix $ A\in K^{n\times n} $,其Schatten 2-Norm可以表示为:
  \begin{align}
    \lVert A\rVert_2=\sqrt{\mathrm{Tr}(A^\dagger A)}.
  \end{align}
  等价的也可以表示为:
  \begin{align}
    \left\| A\right\|_2=\sqrt{\sum_{i,j}\lvert A_{ij}\rvert^2}.
  \end{align}
}
我们可以使用量子计算机来计算两个量子态区别的Schatten 2-Norm,因为数学上我们有:
\begin{align}
  \begin{aligned}\left\|\rho_0-\rho_1\right\|_2&=\sqrt{\mathrm{Tr}((\rho_0-\rho_1)(\rho_0-\rho_1))}\\&=\sqrt{\mathrm{Tr}(\rho_0^2)+\mathrm{Tr}(\rho_1^2)-2\mathrm{~Tr}(\rho_0\rho_1)}.\end{aligned}
\end{align}
其中的信息可以通过SWAP测试来获得。
\begin{align}
  \mathrm{Tr}\left(\rho_i^2\right)=\mathrm{Tr}((\rho_i\otimes\rho_i)SWAP),\quad\mathrm{Tr}(\rho_1\rho_2)=\mathrm{Tr}((\rho_1\otimes\rho_2)SWAP).
\end{align}

\bigskip
\hlr{Infinity Norm}

我们考虑Schatten p-Norm在$ p\to\infty $的极限情况,我们有下面定理:
\thm{
  Normal Matrix的Infinity Norm

  对于Normal Matrix $ A\in K^{n\times n} $,其Schatten Infinity-Norm可以表示为:
  \begin{align}
    \lVert A\rVert_\infty=\max_i|\lambda_i(A)|,
  \end{align}
  其中$ \lambda_i(A) $是矩阵$ A $的本征值。
}
也就是说这个给出了矩阵的最大本征值模长。一个很现实的意义就是他给出了一个算子的最大期望值:
\begin{align}
  \left|\mathrm{Tr}(M\rho)\right|\leq\left|\lambda_{max}\right|=\left\|M\right\|_\infty.
\end{align}

% section Lecture 9: (end)

\newpage 
\section{Lecture 10: Entropy}\label{sec:Lecture 10: Entropy} % (fold)
\subsection{Classical Entropy}
\subsubsection{Single Random Variable Entropy}


\hlr{Shannon Entropy}

首先我们定义经典信息理论之中的熵的概念:
\defi{
  Shannon Entropy

  对于一个随机变量$ X $其可能取值为$ \{x_i\} $,对应的概率分布为$ \{p_i\} $,我们定义这个随机变量的Shannon Entropy为:
  \begin{align}
    H(X) = -\sum_i p_i \log p_i.
  \end{align}
  注意!对于经典信息理论这里的对数我们一般取以2为底!
}
对于经典的Shannon Entropy我们一般有三个Interpretation:
\begin{itemize}
  \item Uncertainty of Random Variable before knowing its value
  \item Learning X所需要的「知识量」
  \item 平均的 "Surprisal" 也就是$ -\log p_i $ 的平均值
\end{itemize}
数学上我们会需要完全不可能的事件$ p_i = 0 $,我们定义$ 0 \log 0 = 0 $作为一个解析延拓。

\bigskip
\hlr{存储信息的最少bit数}

Shannon Entropy存在一个重要的interpretation就是考虑下面情况:
\begin{itemize}
  \item 假设有一组数据$ X_1, X_2, \ldots, X_n $,其中$ X_i $在某一个有限范围内取值,并且这一组数据出现每一个取值的比例是已知为$ p_a $。
\end{itemize}
下面我们的问题是:\textbf{为了储存这一组数据,我们最少需要多少bit?}

\bigskip
\hlr{1234数据例子}

假设现在存在一个存有$ n $个数的数组,其中每一个数据都是1,2,3,4中的一个,并且出现的比率分别为$ 1/2,1/4,1/8,1/8 $。那么我们需要多少bit来储存这个数据呢?我们可以设计下面的编码方法:

\textbf{普通编码方案:}我们使用2 bit来储存每一个数据,$ 1 \to 00, 2 \to 01, 3 \to 10, 4 \to 11 $。这样我们总共需要$ 2n $ bit来储存这组数据。

\textbf{优化编码方案:}我们使用1 bit来储存1,2 bit来储存2,3 bit来储存3和4。也就是$ 1 \to 0, 2 \to 10, 3 \to 110, 4 \to 111 $。我们可以证明这样的编码方法是无歧义的,并且我们总共需要的bit数为:
\begin{align}
  n\left(\frac{1}{2} \cdot 1 + \frac{1}{4} \cdot 2 + \frac{1}{8} \cdot 3 + \frac{1}{8} \cdot 3\right) = \frac{7n}{4} < 2n
\end{align}
这个例子说明了,如果我们知道一个数组之中的数据的出现的比例「一种概率分布」,我们可以通过优化编码方式来减少储存数据所需要的bit数。


\bigskip
\hlr{Shannon Entropy作为最少bit数的下界}

下面我们会给出Shannon Entropy一个终于要的性质,也就是Shannon Entropy给出了储存数据所需要的最少bit数的下界。
\thm{
  Shannon's Noiseless Coding Theorem

  假设有一个数组有$ n $个数据,数据的取值为$ \{x_i\}, i = 1,...,m $,对应的概率分布为$ \{p_i\} $。那么为了储存这个数组,我们至少需要$ n H(X) +1 $ bit,其中$ H(X) $为对应的Shannon Entropy。
}
我们下面给出证明的思路。

\textbf{Binary Case:}一个简单的情况就是整个数组只能取0和1两种值,并且0和1出现的概率分别为$ 1- p $和$ p $。因此我们知道,长成这样子的n个数据的数组只有:
\begin{align}
  \binom{n}{pn} = \frac{n!}{(pn)!(n-pn)!}
\end{align}
种可能性。我们假设一个二进制数来记录这么多个数字,我们考虑需要多少位数,也就是:
\begin{align}
  L &\geq \log_2 \binom{n}{pn} \\ 
  &=\log(n!)-\log((np)!)-\log((n(1-p))!)\\
  &\approx n\log(n)-n-(np\log(np)-np)-(n(1-p)\log\\
  &=-np\log(p)-n(1-p)\log(1-p)\\
  &=nH(p),
\end{align}
中间第二步我们使用了Stirling公式 $ \ln(n!)\approx n\ln n-n+\frac{1}{2}\ln(2\pi n) $也是一个重要的近似估计公式。

\textbf{General Case:}对于一般的情况我们可以使用类似的方法进行估计。假设数据的取值为$ \{x_i\}, i = 1,...,m $,对应的概率分布为$ \{p_i\} $。我们考虑这样的序列有多少种构型:
\begin{align}
  \# \text{config} = \frac{n!}{(p_1 n)!(p_2 n)! \cdots (p_m n)!}
\end{align}
我们同样使用Stirling公式进行近似可以得到:
\begin{align}
  L &\geq \log_2 \# \text{config} \\
  &= \log(n!) - \sum_{i=1}^m \log((p_i n)!) \\
  &\approx n\log(n) - n - \sum_{i=1}^m (p_i n \log(p_i n) - p_i n) \\
  &= -n \sum_{i=1}^m p_i \log(p_i) = n H(X).
\end{align}


\subsubsection{2 Random Variables Entropy}

更有趣的是当我们考虑两个随机变量的时候,我们可以定义更多的熵的概念来描述两个随机变量之间的关系。

\bigskip
\hlr{Joint Entropy}

我们首先定义Joint Entropy的概念:
\defi{
  Joint Entropy

  对于两个随机变量$ X $和$ Y $,其联合概率分布为$ p(x_i,y_j) $,我们定义Joint Entropy为:
  \begin{align}
    H(X,Y) = -\sum_{i,j} p(x_i,y_j) \log p(x_i,y_j).
  \end{align}
}
我们知道如果这两个随机变量完全独立,那么$ H(X,Y) = H(X) + H(Y) $。否则$ H(X,Y) \leq H(X) + H(Y) $。这个Entropy的interpretation其实就是把两个随机变量看作一个联合的随机变量,然后计算这个联合随机变量的Entropy。


\bigskip
\hlr{Conditional Entropy}

在Joint Entropy的基础上我们可以定义Conditional Entropy的概念:
\defi{
  Conditional Entropy

  对于两个随机变量$ X $和$ Y $,其联合概率分布为$ p(x_i,y_j) $,我们定义Conditional Entropy为:
  \begin{align}
    H(X|Y) = H(X,Y) - H(Y) 
  \end{align}
}
这个Entropy的意义是:
\begin{itemize}
  \item 在知道Y的情况下X的不确定性
\end{itemize}
因此这个意义已经说明了$ H(X|Y) \neq H(Y|X) $。并且满足一个不等式: $ 0\leq H(X\mid Y)\leq H(X) $

这个interpretation可以通过一个等价的定义进行理解:
\defi{
  Conditional Entropy (equivalent definition)

  对于两个随机变量$ X $和$ Y $,其联合概率分布为$ p(x_i,y_j) $,我们定义Conditional Entropy为:
  \begin{align}
    H(X|Y) = \sum_j p(y_j) H(X|Y=y_j) = -\sum_j p(y_j) \sum_i p(x_i|y_j) \log p(x_i|y_j) 
  \end{align}
  其中$ p(x_i|y_j) = \frac{p(x_i,y_j)}{p(y_j)} $被interpret为在知道$ Y=y_j $的情况下$ X $的条件概率分布。
}

\bigskip
\hlr{Mutual Information}

我们可以定义Mutual Information的概念:
\defi{
  Mutual Information

  对于两个随机变量$ X $和$ Y $,其联合概率分布为$ p(x_i,y_j) $,我们定义Mutual Information为:
  \begin{align}
    I(X:Y) = H(X) + H(Y) - H(X,Y)
  \end{align}
}
这个Entropy的意义是:
\begin{itemize}
  \item 这相当于知道Y之前的X的位置的信息量 $ H(X) $减去知道Y之后X的位置的信息量$ H(X|Y) $,也就是$ I(X:Y) = H(X) - H(X|Y) $  
    \item 可以被理解为:知道Y之后就知道了多少关于X的信息;由于其对称性也可以被理解为:知道X之后就知道了多少关于Y的信息; 也就是两者之间的共同share信息量
\end{itemize}
特别的我们会发现,如果X和Y独立,那么$ I(X:Y) = 0 $。而如果$ X= Y $,那么$ I(X:Y) = H(X) = H(Y) $。我们可以证明其满足不等式:
\begin{align}
  0 \leq I(X:Y) \leq \min\{H(X),H(Y)\}
\end{align}
上面几个entropy如果被interpret为「未知的信息量」的话,我们可以使用venn图表示如下:
\begin{figure}[H]
  \centering
  \includegraphics[width=0.55\textwidth]{assets/venndiag.png}
  \caption{Classical Entropy Venn Diagram}
  \label{fig:venndiag}
\end{figure}

\bigskip
\hlr{Relative Entropy}

我们可以定义Relative Entropy的概念:
\defi{
  Relative Entropy

我们考虑对于一个随机变量$ X $,其概率分布为$ p(x_i) $,我们考虑另一个概率分布$ q(x_i) $,我们定义Relative Entropy为:
  \begin{align}
    D(p ||q) = \sum_i p(x_i) \log \frac{p(x_i)}{q(x_i)} = -H(X) - \sum_i p(x_i) \log q(x_i)
  \end{align}
}
这个Entropy的意义是:
\begin{itemize}
  \item 表示两个概率分布之间的相似性!
\end{itemize}
我们可以证明其满足不等式:
\begin{align}
  D(p||q) \geq 0 \quad \text{with equality iff } p = q
\end{align}

\bigskip
\hlr{Shannon Entropy的另一种interpretation}

我们研究Relative Entropy,不妨选择$ q(x_i) = \frac{1}{d} $为均匀分布,那么我们发现:
\begin{align}
  D(p||q) &= \sum_i p(x_i) \log p(x_i) - \sum_i p(x_i) \log \frac{1}{d} \\
  &= -H(X) + \log d \\
  \Rightarrow H(X) &= \log d - D(p||q)
\end{align}
也就是说,Shannon Entropu表达了一个概率分布和均匀分布之间的距离!越接近均匀分布Entropy越大,越远离均匀分布Entropy越小!


\rmk{
  对于经典信息论里面所有的$ \log $我们都取以2为底。
}


\subsection{Quantum Entropy}

\subsubsection{Single Quantum System Entropy}

\bigskip
\hlr{Von Neumann Entropy}

下面我们考虑把经典信息论的内容推广到量子信息论之中。我们可以定义Von Neumann Entropy的概念:
\defi{
  Von Neumann Entropy

  对于一个量子态$ \rho $,我们定义Von Neumann Entropy为:
  \begin{align}
    S(\rho) = -\mathrm{Tr}(\rho \log \rho).
  \end{align}
  这里的对数我们一般取以2为底!如果我们可以进行谱分解$ \rho = \sum_i \lambda_i |\psi_i\rangle\langle\psi_i| $,那么我们有:
  \begin{align}
    S(\rho) = -\sum_i \lambda_i \log \lambda_i.
  \end{align}
}
一个interpretation就是,这个entropy可以描述一个量子系统存放N个量子态所需要最少的Hilbert Space的维度。如果我们希望用一个量子系统来存放$ \rho^{\otimes N} $量子态,那么这个量子系统的Hilbert Space的维度至少需要为$ 2^{N S(\rho)} $。


\bigskip
\hlr{Properties of Von Neumann Entropy}

我们可以证明Von Neumann Entropy满足下面的性质:
\begin{itemize}
  \item 对于pure state我们有$ S(\rho) = 0 $; 对于maximally mixed state我们有$ S(\rho) = \log d $,其中$ d $为Hilbert Space的维度。这是一个量子系统entropy的极限情况。
    \item 如果我们的State发生了Unitary的演化 $ \rho \to U \rho U^\dagger $,那么Entropy不变。也就说$ S(U \rho U^\dagger) = S(\rho) $
\end{itemize}

\bigskip
\hlr{Von Neumann Entropy与Measurement}

我们考虑对于一个物理量的$ M = \sum_i m_i |i\rangle\langle i| $的测量。这个测量可以为我们定义另一个随机变量$ M $,其概率分布为$ p_i = \langle i|\rho|i\rangle $给出$ m_i $的取值。那么我们可以定义这个随机变量的Shannon Entropy为:
\begin{align}
  H(M) = -\sum_i p_i \log p_i.
\end{align}
我们可以发现下面的结论:
\begin{align}
  S(\rho) = - \sum_i \lambda_i \log \lambda_i \leq H(M)
\end{align}
这个结论有两种说明的方法:
\begin{itemize}
  \item 使用Eigen Basis之外的任意basis然后decoherence(Killing the off-diagonal),则entropy增加;也可以理解为让这个state更加的mixed
    \item Measuring Uncertainty在Eigen Basis是最小的
\end{itemize}

\bigskip
\hlr{Properties of Von Neumann Entropy II}

我们研究对于多个系统以及组合系统的entropy性质:
\begin{itemize}
  \item 对于两个系统$ A $和$ B $,我们有$ S(\rho_A \otimes \rho_B) = S(\rho_A) + S(\rho_B) $
  \item Von Neumann Entropy满足三角不等式:$ S(\rho_A)+ S(\rho_B) \geq S(\rho_{AB}) \geq |S(\rho_A) - S(\rho_B)| $
    \item 对于一个系统可以写作很多state的convex combination,比如$ \rho = \sum_i p_i \rho_i $,那么我们有$ S(\rho) \geq \sum_i p_i S(\rho_i) $。也就是说entropy是一个concave函数。或者说:convex mixing会增加entropy。
\end{itemize}

\subsubsection{2 Quantum Systems Entropy}

\bigskip
\hlr{Joint Quantum Entropy}

很自然我们对于一个组合系统$ AB $定义Joint Quantum Entropy,也就是这个系统的Von Neumann Entropy:
\begin{align}
  S(\rho_{AB}) = -\mathrm{Tr}(\rho_{AB} \log \rho_{AB}).
\end{align}

\bigskip
\hlr{Conditional Quantum Entropy and Mutual Quantum Information}

我们可以定义Conditional Quantum Entropy的概念:
\defi{
  Conditional Quantum Entropy

  对于一个组合系统$ AB $,我们定义Conditional Quantum Entropy为:
  \begin{align}
    S(A|B) = S(\rho_{AB}) - S(\rho_B).
  \end{align}
}
我们可以定义Mutual Quantum Information的概念:
\defi{
  Mutual Quantum Information

  对于一个组合系统$ AB $,我们定义Mutual Quantum Information为:
  \begin{align}
    I(A:B) = S(\rho_A) + S(\rho_B) - S(\rho_{AB}).
  \end{align}
}
我们考虑一个系统,$ AB $处于一个pure state是一个bell state $ \ket{\phi^+} $我们知道纯态的entropy为0,也就是说$ S(\rho_{AB}) = 0 $。但是我们知道每一个子系统的entropy为1,也就是说$ S(\rho_A) = S(\rho_B) = 1 $。因此我们发现:$ S(A|B) = -1 $,也就是说在知道B的情况下A的entropy变成了-1!这说明量子信息论conditional entropy不一定需要满足大于0的条件!这和经典信息论有很大的不同!

\bigskip
\hlr{Relative Quantum Entropy}

\defi{
  Relative Quantum Entropy

  对于两个量子态$ \rho $和$ \sigma $,我们定义Relative Quantum Entropy为:
  \begin{align}
    S(\rho || \sigma) = \mathrm{Tr}(\rho \log \rho) - \mathrm{Tr}(\rho \log \sigma).
  \end{align}
}
如果我们使用对角化的basis进行$  \rho = \sum_i \lambda_i |\psi_i\rangle\langle \psi_i| $和$ \sigma = \sum_j \mu_j |\phi_j\rangle\langle \phi_j| $的谱分解,那么我们有:
\begin{align}
  S(\rho || \sigma) = \sum_i \lambda_i \log \lambda_i - \sum_{i,j} \langle \psi_i|\phi_j\rangle \langle \phi_j|\psi_i\rangle \lambda_i \log \mu_j,
\end{align}
我们可以证明其满足下面的性质:
\begin{itemize}
  \item $ S(\rho || \sigma) \geq 0 $,并且当且仅当$ \rho = \sigma $时取等号
    \item $ S(\rho || \sigma) \neq S(\sigma || \rho) $,也就是说这个量不是对称的
      \item Unitary Invariance:对于任意的unitary operator $ U $,我们有$ S(U \rho U^\dagger || U \sigma U^\dagger) = S(\rho || \sigma) $
\end{itemize}

\bigskip
\hlr{Data Processing Inequality}

对于Relative Entropu我们可以证明其在一个量子信道$ \mathcal{E} $下满足Data Processing Inequality:
\begin{align}
  S(\mathcal{E}(\rho) || \mathcal{E}(\sigma)) \leq S(\rho || \sigma).
\end{align}
我们回顾Matrix Norm的概念,发现1-norm也同样满足这个关系!!



\subsection{Fidelity}

\hlr{Classical Fidelity}

我们对于同一个随机变量$ X $,其概率分布分别为$ p(x_i) $和$ q(x_i) $,我们定义Classical Fidelity为:
\defi{
  Classical Fidelity

  对于同一个随机变量$ X $,其概率分布分别为$ p(x_i) $和$ q(x_i) $,我们定义Classical Fidelity为:
  \begin{align}
    F(p,q) = \left(\sum_i \sqrt{p(x_i)q(x_i)}\right).
  \end{align}
}

\bigskip
\hlr{Properties of Classical Fidelity}

我们可以证明Classical Fidelity满足下面的性质:
\begin{itemize}
  \item $ 0 \leq F(p,q) \leq 1 $,并且当且仅当$ p = q $时取$ =1 $.
    \item $ F(p,q) = 0 $当且仅当完全没有重叠也就是$ \forall i, p(x_i)q(x_i) = 0 $
\end{itemize}
注意,Fedility不能够描述两个经典vector之间的距离,因为其不满足三角不等式。


\bigskip
\hlr{Quantum Fidelity}

对于两个量子态$ \rho $和$ \sigma $,我们定义Quantum Fidelity为:
\defi{
  Quantum Fidelity  

  对于两个量子态$ \rho $和$ \sigma $,我们定义Quantum Fidelity为:
  \begin{align}
    F(\rho,\sigma)=\min_{\{M_{i}\}}\sum_{i}\sqrt{\operatorname{Tr}(M_{i}\rho)\operatorname{Tr}(\sigma M_{i})},
  \end{align}
  其中$ \{M_i\} $是一组POVM。也就是说我们选择一组测量让Quantum State给出一组概率分布,然后选择那个让这个概率分布的Classical Fidelity最小的测量给出的数值。
}
对于Quantum Fidelity我们存在另一个等价形式:
\thm{
  对于两个量子态$ \rho $和$ \sigma $,我们Quantum Fidelity等价为:
  \begin{align}
    F(\rho,\sigma) = \mathrm{Tr}\sqrt{\sqrt{\rho}\sigma\sqrt{\rho}}.
  \end{align}
}

\bigskip
\hlr{Properties of Quantum Fidelity}

我们考虑两个state$ \rho $和$ \sigma $,假设他们可以\textbf{在同一组基上}对角化,$ \rho=\sum_ir_i|i\rangle\langle i|,\quad\sigma=\sum_is_i|i\rangle\langle i|. $那么我们有:
\begin{align}
  F(\rho,\sigma)=\sum_i\sqrt{r_is_i}.
\end{align}
也就退化为这一组基上面的classical fidelity。这个可以直接通过等价形式证明!并且我们注意$ i =\ketbra{i}{i} $是一个projective operator所以$ i^2 = i = \sqrt{i} $。


\bigskip
\hlr{Pure State Fidelity}

如果$ \rho $是一个pure state $ \rho = |\psi\rangle\langle\psi| $,而$ \sigma $是任意的一个量子态,那么我们有:
\begin{align}
  F(\rho,\sigma) = \sqrt{\langle\psi|\sigma|\psi\rangle}.
\end{align}
这个可以直接通过等价形式证明!并且我们注意$ \rho^2 = \rho = \sqrt{\rho} $。

而如果$ \sigma $也是一个pure state $ \sigma = |\phi\rangle\langle\phi| $,那么我们有:
\begin{align}
  F(\rho,\sigma) = |\langle\psi|\phi\rangle|.
\end{align}
也就是说fidelity描述了两个pure state之间的overlap。

\bigskip
\hlr{Uhlmann’s theorem}

我们知道pure state的fidelity可以很好的计算并且有实际意义,因此我们希望考虑mix state纯化之后的fidelity和mix state的fidelity之间的关系。我们有下面的定理:
\thm{
  Uhlmann’s theorem

  对于两个量子态$ \rho $和$ \sigma $,我们定义他们在同一个更大的Hilbert Space上的purification分别为$ |\psi_\rho\rangle $和$ |\phi_\sigma\rangle $,那么我们有:
  \begin{align}
    F(\rho,\sigma) = \max_{|\psi_\rho\rangle,|\phi_\sigma\rangle} |\langle\psi_\rho|\phi_\sigma\rangle|,
  \end{align}
  也就是说mix state的fidelity等于所有纯化之后的pure state fidelity的最大值。
}

\bigskip
\hlr{Data Processing Inequality of Fidelity}

我们可以证明Fidelity在量子信道$ \mathcal{E} $下满足Data Processing Inequality:
\begin{align}
  F(\mathcal{E}(\rho),\mathcal{E}(\sigma)) \geq F(\rho,\sigma).
\end{align}
也就是说我们不能够通过量子信道让两个量子态变得更不相似!量子信道只会让两个态更加相似。








% section Lecture 10: Entropy (end)

\newpage
\section{Lecture 11: Resouce Theory and Entanglement Resource}\label{sec:Lecture 11: Resouce Theory and Entanglement Resource} % (fold)
\subsection{Set Up for Resource Theory}


\subsection{Resource Theory of Entanglement}




% section Lecture 11: Resouce Theory and Entanglement Resource (end)

\newpage 
\section{Lecture 12: Entanglement Measure}\label{sec:Lecture 12: Entanglement Measure} % (fold)
\subsection{Schwarzschild Metric}

\subsubsection{Schwarzschild Metric Solution}

\bigskip
\hlr{On shell求解}

我们使用上面讨论的ansatz带入真空的Einstein Field Equation进行求解:
\begin{align}
  R_{\mu\nu} = 0 \quad 
  \Rightarrow \quad \frac{2}{r}(\partial_r\alpha+\partial_r\beta) = 0 , \quad \partial_r\left(re^{2\alpha}\right)=1\mathrm{~,}
\end{align}
最后得到的解为:
\begin{align}
  e^{2 \alpha} = e^{-2 \beta} = 1-\frac{R_s}{r}
\end{align}
于是我们给出Schwarzschild Metric:
\defi{
  Schwarzschild Metric 

  对于真空Einstein Field Equation存在一系列使用$ R_s $标定的解:
  \begin{align}
    ds^2=-\left(1-\frac{R_S}{r}\right)dt^2+\left(1-\frac{R_S}{r}\right)^{-1}dr^2+r^2d\Omega^2.
  \end{align}
}
对于这个解,我们会发现其很像是描述这0点有一个质量为$ M $的球对称物体引发的引力场【虽然这个解是一个Vaccum Solution】。我们可以通过在$ r\to\infty $处的Newtonian Limit来确定$ R_S $和$ M $的关系:
\begin{align}
  R_s = 2GM.
\end{align}

\subsubsection{Schwarzschild Metric的性质}

\bigskip
\hlr{schwarzschild Metric行为}

我们研究这个metric的行为,比较trivial的会发现:
\begin{itemize}
  \item $ r >> R_s $的时候,metric的行为很像是中心有一个质量为$ M $的物体引发的Newtonian引力场;
  \item $ r \to \infty $的时候,metric的行为很像是flat Minkowski space-time;
  \item $ M \to 0 $的时候,metric的行为也很像是flat Minkowski space-time;
\end{itemize}

\bigskip
\hlr{Singularity}

下面我们进一步观察会发现:
\begin{itemize}
  \item $ r = R_s $以及$ r = 0 $的时候这个metric都存在分量会发生发散!
\end{itemize}
但是我们会发现不一定发散就意味着我们的时空存在奇异性。因为有的时候仅仅是因为我们选择了一个并不太好的coordinate导致某一些点不能覆盖到。需要有很多手段来验证一个点是不是intrisically singular。总之:
\begin{itemize}
  \item $ r = R_s $的时候仅仅是一个coordinate singularity,可以通过更换坐标系来消除这个奇异性;
  \item $ r = 0 $的时候是真正的intrinsic singularity; 而$ r = 0 $处则是时空真正的奇异点。
\end{itemize}

\bigskip
\hlr{Brikhoff's Theorem}

实际上我们可以给出一个定理:
\begin{itemize}
  \item Brikhoff's Theorem:任何stationary + spherically symmetric的vacuum solution都是static的,并且是Schwarzschild Metric。
\end{itemize}


\subsection{Schwarzschild Metric上的Geodesics}

下面我们研究一个自由例子是怎么在一个Schwarzschild 时空之中运动的。由于我们在这个常用的coordinate system下面考虑,所以我们不研究$ r < R_s $的行为,因为这个metric会发生奇异。

\bigskip
\hlr{物理粒子的Normalization条件}

我们知道Geodesic是满足Geodesic Equation的解。但是这个解在任何Affine Parameter下面都是成立的,但这不意味着所有的paramter的解都有合理的物理意义。真正能被理解为自由粒子运动轨迹的参数化 $ \lambda $ 对应的切矢量$ u^\mu = \frac{d x^\mu}{d \lambda} $需要存在如下归一化:
\begin{itemize}
  \item 对于有质量粒子,我们需要保证normalization保证:$ u^\mu u_\mu = -1 $
    等价的,我们可以选择粒子的proper time作为affine parameter;
  \item 对于无质量粒子,我们需要选择normalization保证:$ u^\mu u_\mu = 0 $。但是这还不足以确定一个粒子,我们还需要$ p^\mu = u^\mu $。其中$ p^\mu $是粒子的物理动量。
\end{itemize}
对于两者的normalization条件我们可以统一写作:
\begin{align}\label{eq:normalization}
  -g_{\mu\nu} \frac{d x^\mu}{d \lambda} \frac{d x^\nu}{d \lambda} = \epsilon \quad \text{其中} \quad \kappa = \begin{cases}
    1 & \text{有质量粒子} \\
    0 & \text{无质量粒子}
  \end{cases}
\end{align}
\rmk{
  注意!!这个normalization条件判断的是4 vector有没有物理意义!不论是自由粒子还是有外力作用的粒子都需要满足这个条件!
}

\subsubsection{Killing Vector求解Geodesic}

一般的我们可以列出Geodesic Equation进行求解。但问题是这个方程组过于复杂。于是我们回忆之前知道Geodesic在Killing Vector方向的分量必然守恒的事实,我们会发现,我们可以寻找Killing Vector,然后利用守恒量来简化方程组。我们操作的方式为:
\begin{itemize}
  \item 寻找粒子的守恒量,给出守恒方程
  \item 将守恒方程带入Normalization的条件,给出不同坐标对于$ \lambda $参数的运动方程
\end{itemize}

\bigskip
\hlr{Killing Vector Field与守恒量}

根据对称性我们可以认为粒子在某一个通过原点的平面上进行运动,我们不妨选择$ \theta = \pi/2 $的平面。
\begin{itemize}
  \item 我们仅仅研究限制在$ \theta = \pi /2 $的平面上运动的粒子。
\end{itemize}
带入Geodesic Equation我么也可以证明在这个平面内进行运动是满足方程的。为了研究这样的粒子运动,我们可以找到两个Killing Vector Field:
\begin{itemize}
  \item 时间平移对称性对应的Killing Vector Field:$ K^\mu = (1,0,0,0) , K_\mu=\left(-\left(1-\frac{2GM}{r}\right),0,0,0\right)$
  \item 转动对应的Killing Vector Field:$ R^\mu = (0,0,0,1) , R_\mu=\left(0,0,0,r^2\sin^2\theta\right) $由于我们选择了$ \theta = \pi/2 $所以$ R_\mu = (0,0,0,r^2) $
\end{itemize}
这两个Killing Vector Field分别对应两个守恒量:
\begin{align}\label{eq:conservedquantities}
  &E=-K_\mu\frac{dx^\mu}{d\lambda}=\left(1-\frac{2GM}{r}\right)\frac{dt}{d\lambda}\\
  &L=R_\mu\frac{dx^\mu}{d\lambda}=r^2\frac{d\phi}{d\lambda}.
\end{align}

\bigskip
\hlr{Normalization条件}

我们把Normalization条件\cref{eq:normalization}带入Schwarzschild Metric得到:
\begin{align}
  -\left(1-\frac{2GM}{r}\right)\left(\frac{dt}{d\lambda}\right)^2+\left(1-\frac{2GM}{r}\right)^{-1}\left(\frac{dr}{d\lambda}\right)^2+r^2\left(\frac{d\phi}{d\lambda}\right)^2=-\epsilon.
\end{align}

\bigskip
\hlr{带入Normalization条件给出径向方程}

我们固定一个粒子拥有守恒的能量$ E $和角动量$ L $,我们把上面的两个守恒量带入Normalization条件得到:
\begin{align}\label{eq:radialeq}
  \frac{1}{2}\left(\frac{dr}{d\lambda}\right)^2+V(r)=\mathcal{E}
\end{align}
其中等效势能$ V(r) $和等效能量$ \mathcal{E} $为:
\begin{align}
  V(r)=\frac{1}{2}\left(\epsilon+\frac{L^2}{r^2}\right)\left(1-\frac{2GM}{r}\right)=\frac{1}{2}\epsilon-\epsilon\frac{GM}{r}+\frac{L^2}{2r^2}-\frac{GML^2}{r^3}, \quad \mathcal{E}=\frac{E^2}{2}.
\end{align}

\bigskip
\hlr{对比在一个质量为M的Newtonian引力场中的运动}

我们对比这个方程和一个质量为$ 1 $,能量为$ E $的粒子在一个质量为$ M $的Newtonian引力场中的运动方程。会发现:引力势能的前两项和Newtonian引力场中的势能完全一样:
\begin{align}
  V_{\text{Newton}}(r) = -\frac{GM}{r} + \frac{L^2}{2r^2}.
\end{align}
但是,最后一项$ -\frac{GML^2}{r^3} $是一个完全新的项,这个项会导致一些完全不同的物理现象。首先其会作用在无质量粒子的运动上,其次也会导致有质量粒子的轨道进动现象。

\begin{itemize}
  \item 对于$ r \to \infty $的时候,这个项会变得非常小,所以在远距离上我们会发现和Newtonian引力场的行为非常类似;
  \item 对于$ r \to 0 $的时候,这个项会变得非常大。我们会发现势能永远在$ r = 2GM $的时候变为0。但是我们暂时不研究其中行为。我们知道newton引力除非直接冲着中心运动是永远不会落入中心的。但是由于这个项的存在,我们会发现如果粒子能量足够大跨越了势能的最大值,粒子就会落入中心。
\end{itemize}


\bigskip
\hlr{势能与轨道分析}

我们分析这个势能能够给出什么样子的轨道行为。下图展示了不同角动量$ L $下的势能曲线:
\begin{figure}[H]
  \centering
  \includegraphics[width=0.85\textwidth]{assets/energypoasd.png}
  \caption{不同角动量下的势能曲线}
  \label{fig:energypoasd}
\end{figure}
我们可以知道和牛顿力学一样,存在束缚的轨道以及非束缚的轨道。我们首先分析束缚轨道。我们研究可能存在的圆形轨道,也就是势能的极值点,对于势能求导为0:
\begin{align}
  \epsilon GMr_c^2-L^2r_c+3GML^2=0,
\end{align}
其中$ r_c $是圆形轨道的半径。我们发现:
\begin{itemize}
  \item \textbf{无质量粒子}:对于无质量粒子不存在stable的圆形轨道但是存在一个unstable的圆形轨道,轨道半径为$ r_c = 3GM $;
  \item \textbf{有质量粒子}:对于有质量粒子存在stable以及unstable的圆形轨道。unstable轨道的半径为:
    \begin{align}
      r_c=\frac{L^2\pm\sqrt{L^4-12G^2M^2L^2}}{2GM}.
    \end{align}
    对于很大的L我们会意识到两个轨道的半径分别趋近于$ r_c = \frac{L^2}{GM} $这个是stable的轨道以及$ r_c = 3GM $这个是unstable的轨道。我们总结如下:
    \begin{itemize}
      \item 随着角动量增大,stable轨道的半径会越来越大;
      \item 当$ L < 2\sqrt{3}GM $的时候,不存在任何圆形轨道。
      \item 当$ L = 2\sqrt{3}GM $的时候,存在唯一的unstable圆形轨道,轨道半径为$ r_c = 6GM $。这个轨道是最小可能的stable圆形轨道。
    \end{itemize}
\end{itemize}
\rmk{
  注意,我们上面讨论的都是geodesic也就是自由粒子的轨道。如果存在非引力的外力作用在粒子上,粒子是从$ r_c = 3GM $以内逃离的。
}




\subsection{GR验证I: 行星近日点进动解}

\bigskip
\hlr{守恒量给出轨道方程}

我们现在希望求解一个有质量的粒子在Schwarzschild时空之中运动轨道,径向长度和角度的关系$ r(\phi) $也就是轨道方程。我们依旧从Normalization Condition以及守恒量出发。我们使用:
\begin{align}
  L=R_\mu\frac{dx^\mu}{d\lambda}=r^2\frac{d\phi}{d\lambda}. \quad \Rightarrow \quad \frac{d\phi}{d\lambda}=\frac{L}{r^2}.
\end{align}
然后乘入径向的运动方程\cref{eq:radialeq}得到:
\begin{align}
  \left(\frac{dr}{d\phi}\right)^2+\frac{1}{L^2}r^4-\frac{2GM}{L^2}r^3+r^2-2GMr=\frac{2\mathcal{E}}{L^2}r^4.
\end{align}
  

\bigskip
\hlr{轨道方程的变量替换与微扰解法}

我们使用一个经典的中心力场的变量替换$ x=\frac{L^2}{GMr}\mathrm{~.} $带入上面运动方程并两边对$ \phi $求导,消去常数项以及$ \frac{d r}{d \phi} $一阶导数,得到:
\begin{align}
  \frac{d^2x}{d\phi^2}-1+x=\frac{3G^2M^2}{L^2}x^2.
\end{align}
可以发现这个方程相当于经典牛顿力学的情况多处了右边的一个非线性项。我们使用微扰的方法来求解这个方程,设$ x = x_0 + x_1 $,其中$ x_0 $是经典力学的解,$ x_1 $是由于广义相对论修正引入的小量。

\textbf{经典力学解}:我们先求解经典力学的解$ x_0 $,其满足方程:
\begin{align}
  \frac{d^2x_0}{d\phi^2}-1+x_0=0, \quad \Rightarrow \quad x_0 = 1 + e \cos\phi,
\end{align}

\textbf{广义相对论修正}:我们把经典力学解带入右边的非线性项,得到广义相对论修正的方程的一阶方程:
\begin{align}
  \frac{d^2x_1}{d\phi^2}+x_1=\frac{3G^2M^2}{L^2}x_0^2. \quad \Rightarrow \quad x_1=\frac{3G^2M^2}{L^2}\left[\left(1+\frac{1}{2}e^2\right)+e\phi\sin\phi-\frac{1}{6}e^2\cos2\phi\right]
\end{align}
观察这个解会发现只有$ e\phi\sin\phi $这一项是随着$ \phi $增大而增大的。这个项会导致可以累积的可观测影响。所以我们现在仅仅考虑存在这个修正项带来的影响。得到GR的修正轨道解为:
\begin{align}
  x=1+e\cos\phi+e\frac{3G^2M^2}{L^2}\phi\sin\phi\simeq1+e\cos\left[(1-\alpha)\phi\right] \quad \text{其中} \quad \alpha=\frac{3G^2M^2}{L^2}.
\end{align}

\bigskip
\hlr{轨道解的进动interpretation}

这个解我们会发现,对于经典力学我们一个椭圆是$ \phi $转动$ 2 \pi $得到的。但是现在加入修正会发现我们转动$ \phi $转动$ 2 \pi $之后径向并没有回到原来的位置而是还需要再转动$ 2 \pi \alpha $才能回到原来的位置。还需要再转的角度为:
\begin{align}
  \Delta \phi = 2 \pi \alpha = \frac{6 \pi G^2 M^2}{L^2}.\quad \Rightarrow \quad \large\Delta\phi=\frac{6\pi GM}{c^2(1-e^2)a}.
\end{align}
也就是进动角。一个方便观测的计算就是把物理的$ L $用经典轨道半长轴$ a $以及离心率$ e $表示出来。



% section Lecture 12: Entanglement Measure (end)

\newpage 
\section{Sup 1: QI之中常用的数学技巧}\label{sec:Sup 1: QI之中常用的数学技巧} % (fold)
这里我们汇总一下做题之中涉及的数学技巧以及容易混淆的常用结论捏
\subsection{基础数学技巧}

\subsubsection{张量积}

\bigskip
\hlr{张量积和矩阵加法}

张量积和矩阵加法可以互换,满足关系:
\begin{align}
  \sum_i(A_i\otimes B)=\left(\sum_iA_i\right)\otimes B.
\end{align}

\bigskip
\hlr{张量积和矩阵乘法}

张量积和矩阵乘法是可以互换的,矩阵乘法可以找到自己对应的张量积子空间对应乘起来就好:
\begin{align}
  (A\otimes B)(C\otimes D)=(AC)\otimes(BD)
\end{align}


\bigskip
\hlr{张量积和Vectorization}

\bigskip
\hlr{张量积和Tr操作}

我们知道张量积和Tr是可以互换的,并且张量积的Tr其实就是原本Tr的乘积:
\begin{align}
  \mathrm{Tr}(A\otimes B)=(\mathrm{Tr}A)\left(\mathrm{Tr}B\right)
\end{align}

\subsubsection{Unitary}

\bigskip
\hlr{Unitary变换矩阵}

对于任意算符 $A$,作酉共轭变换
\begin{equation}
    A' = U A U^\dagger,
\end{equation}
则 $A'$ 与 $A$ 具有完全相同的本征值(包括重数)。即
\begin{equation}
    \mathrm{Spec}(A') = \mathrm{Spec}(A).
\end{equation}
换言之,unitary 变换仅仅对应于基底变换,不改变物理可观测量的谱。

\bigskip

\begin{proof}[证明要点]
设 $A|\psi\rangle = \lambda|\psi\rangle$,则
\begin{equation}
    A'(U|\psi\rangle)
    = UAU^\dagger U|\psi\rangle
    = UA|\psi\rangle
    = \lambda (U|\psi\rangle),
\end{equation}
说明 $U|\psi\rangle$ 是 $A'$ 对应本征值 $\lambda$ 的本征态,从而本征值保持不变。
\end{proof}

\bigskip

因此,unitary 变换可视为``旋转基底'',保留体系的谱结构:
\begin{equation}
    \mathrm{Tr}(A') = \mathrm{Tr}(A),\qquad
    \det(A') = \det(A),\qquad
    \|A'\|_1 = \|A\|_1.
\end{equation}

\bigskip
\hlr{Unitary和奇异值}

同样的对于一般的矩阵,Unitary变换也不改变矩阵的奇异值。也就是说:
\begin{align}
  \sigma_i(UA V^\dagger)=\sigma_i(A)
\end{align}



\subsubsection{Trace}
\hlr{Trace的一个基本trick}

对于一个量子态$ \rho $写作:
\begin{align}
  \rho=\sum_{a,b}f(a,b)|a\rangle\langle b|
\end{align}
我们求Tr等价于形式化的把ket放在bra的右边:
\begin{align}
  \mathrm{Tr}(\rho)=\sum_{a,b}f(a,b)\operatorname{Tr}(|a\rangle\langle b|)=\sum_{a,b}f(a,b)\left\langle b|a\right\rangle
\end{align}

\bigskip
\hlr{Trace和Unitary演化}

如果我们有A,B两个系统,然后在其上有一个$ \rho_{AB} $。如果在B系统上面进行Unitary演化以及A系统上面进行任意演化,我们发现这两个过程是可以互换的。所以特殊的对于Partial Trace:
\begin{align}
  Tr_A ((I_A \otimes U_B) \rho_{AB} (I_A \otimes U_B^\dagger)) = U_B Tr_A(\rho_{AB}) U_B^\dagger
\end{align}

\bigskip
\hlr{SWAP operator 和 Trace}



\bigskip
\hlr{Trace of times}

两个密度矩阵的乘积的Trace满足不等式:
\begin{align}
 & |\mathrm{Tr}(\rho\sigma)|\leq\sqrt{\mathrm{Tr}(\rho^2)\operatorname{Tr}(\sigma^2)} \\ 
 &\mathrm{Tr}(\rho^2)\leq1,\quad\mathrm{Tr}(\sigma^2)\leq1 \Rightarrow \mathrm{Tr}(\rho\sigma)\leq1
\end{align}

\bigskip
\hlr{单bit measure的TR}

对于两个单比特态$ \rho_1,\rho_2 $,我们有:


\subsection{Pauli矩阵的特殊性质}

由于Pauli矩阵在计算之中太过于常见,我们下面列出计算之中常用的Pauli矩阵性质

\subsubsection{基本定义}



\subsubsection{基本乘法性质}




\subsubsection{对易反对易关系}



\subsubsection{互相作用,pauli矩阵作用在pauli基上}


\subsubsection{指数形式的Pauli矩阵}


\subsubsection{Pauli矩阵的本征值与本征基}

单比特的Pauli矩阵定义为
\[
I=\begin{pmatrix}1&0\\0&1\end{pmatrix},\quad
X=\begin{pmatrix}0&1\\1&0\end{pmatrix},\quad
Y=\begin{pmatrix}0&-i\\ i&0\end{pmatrix},\quad
Z=\begin{pmatrix}1&0\\0&-1\end{pmatrix}.
\]

其中 $I$ 是单位算符,而 $X,Y,Z$ 均为厄米且酉的算符,满足 $P^2=I$。因此,对于任意 Pauli 矩阵 $P\in\{X,Y,Z\}$ 有
\[
P^2=I \quad \Rightarrow \quad \lambda^2=1,
\]
即其本征值为
\[
\boxed{\lambda=\pm 1}.
\]

更具体地,本征值与本征态如下:
\[
X:\quad X\ket{\pm}=\pm\ket{\pm},\qquad
\ket{\pm}=\frac{1}{\sqrt{2}}(\ket{0}\pm\ket{1}),
\]
\[
Y:\quad Y\ket{\pm i}=\pm\ket{\pm i},\qquad
\ket{\pm i}=\frac{1}{\sqrt{2}}(\ket{0}\pm i\ket{1}),
\]
\[
Z:\quad Z\ket{0}=+1\ket{0},\qquad Z\ket{1}=-1\ket{1}.
\]

\paragraph{Pauli算符期望值的取值范围}

对任意量子态 $\rho$,Pauli算符 $P$ 的期望值定义为
\[
\langle P\rangle_\rho = \operatorname{Tr}(\rho P).
\]

由于 Pauli 矩阵的本征值仅为 $\pm 1$,其测量结果可以看作一个取值为 $\pm 1$ 的随机变量,因此期望值必定满足
\[
-1 \leq \langle P\rangle_\rho \leq 1.
\]

更严格地,利用谱分解 $P=\sum_i \lambda_i\ket{i}\bra{i}$,并注意 $\rho$ 为正定且 $\operatorname{Tr}\rho=1$,可得
\[
\langle P\rangle_\rho = \sum_i \lambda_i p_i,\qquad p_i\ge 0,\quad \sum_i p_i=1,
\]
即为 $\pm 1$ 的加权平均,从而有
\[
\boxed{-1 \leq \langle P\rangle_\rho \leq 1}.
\]

\subsection{常用不等式}

\bigskip
\hlr{平方和和和的平方}
对于任意实数$ a_i $,我们有下面不等式:
\begin{align}
  M\sum_{i=1}^Ma_i^2\geq\left(\sum_{i=1}^Ma_i\right)^2
\end{align}


% section Sup 1: QI之中常用的数学技巧 (end)

\newpage
\section{Scratch book}\label{sec:Scratch book} % (fold)
\input{sb.tex}
% section Scratch book (end)

\end{document}
